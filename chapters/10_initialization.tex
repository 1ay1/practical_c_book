\chapter{Initialization Patterns}

\section{Understanding Initialization}

Initialization is more than just assigning values---it's about setting up data structures in a predictable, safe state. C offers several initialization techniques, each with its own strengths and use cases.

\begin{lstlisting}
// Different initialization styles
int a = 0;                    // Simple initialization
int b = {0};                  // Brace initialization
int arr[5] = {1, 2, 3, 4, 5}; // Array initialization
struct Point p = {10, 20};    // Struct initialization

// Uninitialized (dangerous!)
int x;        // Contains garbage
int* ptr;     // Points to random memory
\end{lstlisting}

\begin{warningbox}
Uninitialized variables contain garbage values. Always initialize your variables, especially pointers. Reading uninitialized data is undefined behavior.
\end{warningbox}

\section{Zero Initialization}

The safest default---initialize everything to zero.

\begin{lstlisting}
// Zero-initialize everything
int x = 0;
int arr[100] = {0};           // All elements zero
struct Data d = {0};          // All members zero
char str[256] = {0};          // Null-terminated empty string

// Shorthand: empty braces (C23 and some compilers)
int arr2[100] = {};
struct Data d2 = {};

// Static/global variables are automatically zero-initialized
static int counter;           // Initialized to 0
static char buffer[1024];     // All bytes 0

// Local variables are NOT zero-initialized
void func(void) {
    int x;                    // GARBAGE!
    int y = 0;                // Explicitly zero
}

// Zero-initialize dynamically allocated memory
int* ptr = calloc(100, sizeof(int));  // All zeros
// vs
int* ptr2 = malloc(100 * sizeof(int)); // GARBAGE!
\end{lstlisting}

\begin{tipbox}
Use \texttt{= \{0\}} to zero-initialize any struct or array. It's simple, portable, and works everywhere.
\end{tipbox}

\section{Designated Initializers (C99)}

Initialize specific members by name---much clearer than positional initialization.

\subsection{Struct Designated Initializers}

\begin{lstlisting}
typedef struct {
    int x;
    int y;
    int z;
    char* name;
} Point3D;

// Positional initialization (old style)
Point3D p1 = {10, 20, 30, "origin"};

// Designated initializers (C99)
Point3D p2 = {
    .x = 10,
    .y = 20,
    .z = 30,
    .name = "origin"
};

// Initialize only some members (others zero)
Point3D p3 = {
    .x = 100,
    .name = "partial"
    // y and z are 0
};

// Order doesn't matter
Point3D p4 = {
    .name = "reordered",
    .z = 5,
    .x = 15
    // y is 0
};

// Mix styles (not recommended)
Point3D p5 = {
    .x = 1,
    2,          // Sets y = 2 (next in sequence)
    .z = 3
};
\end{lstlisting}

\subsection{Array Designated Initializers}

\begin{lstlisting}
// Initialize specific array elements
int sparse[100] = {
    [0] = 1,
    [10] = 2,
    [50] = 3,
    [99] = 4
    // All other elements are 0
};

// Character arrays
char vowels[26] = {
    ['a' - 'a'] = 'a',
    ['e' - 'a'] = 'e',
    ['i' - 'a'] = 'i',
    ['o' - 'a'] = 'o',
    ['u' - 'a'] = 'u'
};

// Ranges (GNU extension)
int range[100] = {
    [0 ... 9] = 1,      // First 10 elements
    [10 ... 19] = 2,    // Next 10 elements
    [90 ... 99] = 9     // Last 10 elements
};

// Lookup tables
int days_in_month[12] = {
    [0] = 31,  // January
    [1] = 28,  // February
    [2] = 31,  // March
    [3] = 30,  // April
    [4] = 31,  // May
    [5] = 30,  // June
    [6] = 31,  // July
    [7] = 31,  // August
    [8] = 30,  // September
    [9] = 31,  // October
    [10] = 30, // November
    [11] = 31  // December
};
\end{lstlisting}

\subsection{Nested Designated Initializers}

\begin{lstlisting}
typedef struct {
    int x, y;
} Point;

typedef struct {
    Point top_left;
    Point bottom_right;
    char* label;
} Rectangle;

// Initialize nested structures
Rectangle rect = {
    .top_left = {.x = 0, .y = 100},
    .bottom_right = {.x = 100, .y = 0},
    .label = "main window"
};

// Array of structs
Point points[3] = {
    [0] = {.x = 0, .y = 0},
    [1] = {.x = 10, .y = 10},
    [2] = {.x = 20, .y = 20}
};

// Struct containing array
typedef struct {
    char name[32];
    int scores[5];
} Student;

Student student = {
    .name = "Alice",
    .scores = {[0] = 95, [1] = 87, [4] = 92}
    // scores[2] and scores[3] are 0
};
\end{lstlisting}

\section{Compound Literals (C99)}

Create temporary objects without declaring variables.

\begin{lstlisting}
// Traditional: need temporary variable
Point temp = {10, 20};
draw_point(&temp);

// Compound literal: create temporary inline
draw_point(&(Point){10, 20});

// Array compound literals
process_data((int[]){1, 2, 3, 4, 5}, 5);

// String compound literal
print_string((char[]){"Hello, World!"});

// With designated initializers
configure(&(Config){
    .width = 800,
    .height = 600,
    .fullscreen = true
});

// Lifetime: until end of enclosing block
void example(void) {
    Point* p = &(Point){100, 200};  // Valid until end of function
    // Use p...
} // Compound literal destroyed here
\end{lstlisting}

\subsection{Compound Literal Patterns}

\begin{lstlisting}
// Default parameters pattern
typedef struct {
    int timeout;
    int retries;
    bool verbose;
} Options;

void connect(const char* host, const Options* opts) {
    // Use opts->timeout, etc.
}

// Call with default options
connect("localhost", &(Options){
    .timeout = 5000,
    .retries = 3,
    .verbose = false
});

// Factory function pattern
Point* create_origin(void) {
    static Point origin = {0, 0};  // Don't do this with compound literal!
    return &origin;
}

// Better: return by value or allocate
Point get_origin(void) {
    return (Point){0, 0};
}

// Initializer lists for variadic functions
void log_values(int count, ...) {
    va_list args;
    va_start(args, count);
    // Process values
    va_end(args);
}

// Use compound literal with array
int values[] = {1, 2, 3, 4, 5};
log_values(5, values[0], values[1], values[2], values[3], values[4]);

// Or pass array directly
void log_array(const int* arr, size_t count);
log_array((int[]){1, 2, 3, 4, 5}, 5);
\end{lstlisting}

\section{Static Initialization}

Initialize data at compile time---faster and safer.

\subsection{Static vs Dynamic Initialization}

\begin{lstlisting}
// Static initialization (compile time)
static const int sizes[] = {1, 2, 4, 8, 16, 32};
static const char* names[] = {"Alice", "Bob", "Charlie"};

// These are embedded in the executable, no runtime cost

// Dynamic initialization (runtime)
void init(void) {
    int* arr = malloc(6 * sizeof(int));
    arr[0] = 1;
    arr[1] = 2;
    // ... Runtime overhead
    free(arr);
}

// Static initialization wins:
// - Faster (no runtime work)
// - Safer (can't fail)
// - Simpler (no cleanup needed)
\end{lstlisting}

\subsection{Constant Tables}

\begin{lstlisting}
// Lookup table for powers of 2
static const unsigned int pow2[] = {
    1, 2, 4, 8, 16, 32, 64, 128, 256, 512, 1024
};

// Error message table
static const char* const error_messages[] = {
    [0] = "Success",
    [1] = "Invalid argument",
    [2] = "Out of memory",
    [3] = "File not found",
    [4] = "Permission denied"
};

const char* get_error_message(int code) {
    if (code >= 0 && code < sizeof(error_messages)/sizeof(error_messages[0]))
        return error_messages[code];
    return "Unknown error";
}

// State machine transition table
typedef enum { IDLE, RUNNING, PAUSED, STOPPED } State;
typedef enum { START, PAUSE, RESUME, STOP } Event;

static const State transitions[4][4] = {
    //        START     PAUSE      RESUME    STOP
    [IDLE]    = {RUNNING,  IDLE,      IDLE,     IDLE},
    [RUNNING] = {RUNNING,  PAUSED,    RUNNING,  STOPPED},
    [PAUSED]  = {PAUSED,   PAUSED,    RUNNING,  STOPPED},
    [STOPPED] = {STOPPED,  STOPPED,   STOPPED,  STOPPED}
};

State next_state(State current, Event event) {
    return transitions[current][event];
}
\end{lstlisting}

\subsection{Read-Only Data}

\begin{lstlisting}
// const ensures data can't be modified
static const int PRIMES[] = {2, 3, 5, 7, 11, 13, 17, 19, 23, 29};

// Pointer to const data
static const char* const DAYS[] = {
    "Sunday", "Monday", "Tuesday", "Wednesday",
    "Thursday", "Friday", "Saturday"
};

// Both pointer and data are const:
// - Can't modify strings
// - Can't make pointer point elsewhere

// Configuration constants
typedef struct {
    int width;
    int height;
    int bpp;
} VideoMode;

static const VideoMode VIDEO_MODES[] = {
    {.width = 640,  .height = 480,  .bpp = 32},
    {.width = 800,  .height = 600,  .bpp = 32},
    {.width = 1024, .height = 768,  .bpp = 32},
    {.width = 1920, .height = 1080, .bpp = 32}
};

static const size_t NUM_VIDEO_MODES =
    sizeof(VIDEO_MODES) / sizeof(VIDEO_MODES[0]);
\end{lstlisting}

\section{Flexible Array Members (C99)}

Structs with variable-length trailing arrays.

\begin{lstlisting}
// Flexible array member (must be last in struct)
typedef struct {
    size_t length;
    int data[];  // Flexible array (size determined at allocation)
} IntArray;

// Allocate with specific size
IntArray* create_array(size_t n) {
    IntArray* arr = malloc(sizeof(IntArray) + n * sizeof(int));
    if (arr) {
        arr->length = n;
        // Initialize data
        for (size_t i = 0; i < n; i++) {
            arr->data[i] = 0;
        }
    }
    return arr;
}

// Use like normal array
void use_array(IntArray* arr) {
    for (size_t i = 0; i < arr->length; i++) {
        printf("%d ", arr->data[i]);
    }
}

// String with flexible array
typedef struct {
    size_t length;
    char data[];
} String;

String* string_create(const char* str) {
    size_t len = strlen(str);
    String* s = malloc(sizeof(String) + len + 1);
    if (s) {
        s->length = len;
        memcpy(s->data, str, len + 1);
    }
    return s;
}
\end{lstlisting}

\begin{tipbox}
Flexible array members are perfect for variable-sized data structures where the size is known at allocation time and won't change.
\end{tipbox}

\subsection{Flexible Array Member Patterns}

\begin{lstlisting}
// Message with variable payload
typedef struct {
    int type;
    size_t payload_size;
    unsigned char payload[];
} Message;

Message* create_message(int type, const void* data, size_t size) {
    Message* msg = malloc(sizeof(Message) + size);
    if (msg) {
        msg->type = type;
        msg->payload_size = size;
        memcpy(msg->payload, data, size);
    }
    return msg;
}

// Vector implementation
typedef struct {
    size_t size;
    size_t capacity;
    int elements[];
} Vector;

Vector* vector_create(size_t capacity) {
    Vector* v = malloc(sizeof(Vector) + capacity * sizeof(int));
    if (v) {
        v->size = 0;
        v->capacity = capacity;
    }
    return v;
}

Vector* vector_push(Vector* v, int value) {
    if (v->size >= v->capacity) {
        // Reallocate with larger capacity
        size_t new_cap = v->capacity * 2;
        Vector* new_v = realloc(v, sizeof(Vector) + new_cap * sizeof(int));
        if (!new_v) return NULL;
        new_v->capacity = new_cap;
        v = new_v;
    }
    v->elements[v->size++] = value;
    return v;
}
\end{lstlisting}

\section{Initialization Functions}

When initialization is complex, use dedicated functions.

\begin{lstlisting}
// Simple initializer
typedef struct {
    int* data;
    size_t size;
    size_t capacity;
} Buffer;

void buffer_init(Buffer* buf, size_t initial_capacity) {
    buf->data = malloc(initial_capacity * sizeof(int));
    buf->size = 0;
    buf->capacity = initial_capacity;
}

void buffer_destroy(Buffer* buf) {
    free(buf->data);
    buf->data = NULL;
    buf->size = 0;
    buf->capacity = 0;
}

// Usage
Buffer buf;
buffer_init(&buf, 100);
// Use buffer...
buffer_destroy(&buf);

// Factory function (returns new object)
Buffer* buffer_create(size_t initial_capacity) {
    Buffer* buf = malloc(sizeof(Buffer));
    if (buf) {
        buffer_init(buf, initial_capacity);
    }
    return buf;
}

void buffer_free(Buffer* buf) {
    if (buf) {
        buffer_destroy(buf);
        free(buf);
    }
}

// Usage
Buffer* buf = buffer_create(100);
// Use buffer...
buffer_free(buf);
\end{lstlisting}

\subsection{Constructor/Destructor Pattern}

\begin{lstlisting}
// Database connection object
typedef struct Database Database;

// Constructor with error handling
Database* database_connect(const char* host, int port, const char* user,
                          const char* password) {
    Database* db = calloc(1, sizeof(Database));
    if (!db) return NULL;

    // Initialize members
    db->host = strdup(host);
    db->port = port;
    db->connected = false;

    // Connect
    if (!internal_connect(db, user, password)) {
        database_close(db);
        return NULL;
    }

    db->connected = true;
    return db;
}

// Destructor (cleanup)
void database_close(Database* db) {
    if (!db) return;

    if (db->connected) {
        internal_disconnect(db);
    }

    free(db->host);
    free(db->query_buffer);
    // Clean up all resources

    memset(db, 0, sizeof(*db));  // Zero for safety
    free(db);
}

// Usage with cleanup
void example(void) {
    Database* db = database_connect("localhost", 5432, "user", "pass");
    if (!db) {
        fprintf(stderr, "Connection failed\n");
        return;
    }

    // Use database...

    database_close(db);  // Always cleanup
}
\end{lstlisting}

\section{Copy Initialization}

\begin{lstlisting}
// Shallow copy (default behavior)
typedef struct {
    int x, y;
} Point;

Point p1 = {10, 20};
Point p2 = p1;  // Bitwise copy (shallow)

// Deep copy needed for pointers
typedef struct {
    char* name;
    int* data;
    size_t size;
} Object;

// Shallow copy (DANGEROUS!)
Object obj1 = {...};
Object obj2 = obj1;  // Both point to same memory!
free(obj1.data);     // obj2.data now invalid!

// Deep copy function
Object* object_copy(const Object* src) {
    Object* dst = malloc(sizeof(Object));
    if (!dst) return NULL;

    // Copy name string
    dst->name = strdup(src->name);
    if (!dst->name) {
        free(dst);
        return NULL;
    }

    // Copy data array
    dst->size = src->size;
    dst->data = malloc(dst->size * sizeof(int));
    if (!dst->data) {
        free(dst->name);
        free(dst);
        return NULL;
    }
    memcpy(dst->data, src->data, dst->size * sizeof(int));

    return dst;
}

// Copy assignment
void object_assign(Object* dst, const Object* src) {
    if (dst == src) return;  // Self-assignment check

    // Free old data
    free(dst->name);
    free(dst->data);

    // Copy new data
    dst->name = strdup(src->name);
    dst->size = src->size;
    dst->data = malloc(dst->size * sizeof(int));
    memcpy(dst->data, src->data, dst->size * sizeof(int));
}
\end{lstlisting}

\section{Global Initialization}

\begin{lstlisting}
// Global variables are zero-initialized before main()
int global_counter = 0;  // Explicit (redundant)
static char buffer[1024]; // Implicitly zero

// Complex global initialization
typedef struct {
    bool initialized;
    void* handle;
    char* config_path;
} GlobalState;

static GlobalState g_state = {0};

// One-time initialization
void ensure_initialized(void) {
    if (!g_state.initialized) {
        g_state.handle = open_handle();
        g_state.config_path = get_config_path();
        g_state.initialized = true;
        atexit(cleanup);  // Register cleanup
    }
}

void cleanup(void) {
    if (g_state.initialized) {
        close_handle(g_state.handle);
        free(g_state.config_path);
        g_state.initialized = false;
    }
}

// Thread-safe initialization (C11)
#include <threads.h>

static once_flag init_flag = ONCE_FLAG_INIT;

void do_init(void) {
    // Expensive initialization
    g_state.handle = open_handle();
    g_state.initialized = true;
}

void thread_safe_init(void) {
    call_once(&init_flag, do_init);
    // Guaranteed to run exactly once across all threads
}
\end{lstlisting}

\section{Initialization Best Practices}

\subsection{Always Initialize}

\begin{lstlisting}
// BAD: Uninitialized
void bad_example(void) {
    int x;
    int* ptr;
    char buffer[256];

    // Using these is undefined behavior!
}

// GOOD: Always initialize
void good_example(void) {
    int x = 0;
    int* ptr = NULL;
    char buffer[256] = {0};

    // Safe to use
}

// Use initializers even for complex types
typedef struct {
    int count;
    char* name;
    double value;
} Data;

// BAD
Data data;
data.count = 0;
data.name = NULL;
data.value = 0.0;

// GOOD
Data data = {0};  // Zero-initialize everything
\end{lstlisting}

\subsection{Use Designated Initializers}

\begin{lstlisting}
// BAD: Positional (fragile)
Point3D p = {10, 20, 30, "label"};
// If struct changes order, this breaks!

// GOOD: Designated (robust)
Point3D p = {
    .x = 10,
    .y = 20,
    .z = 30,
    .name = "label"
};
// Still works if struct reordered

// GREAT: Default values
Point3D p = {
    .x = 10,
    .name = "partial"
    // y and z automatically 0
};
\end{lstlisting}

\subsection{Const Correctness}

\begin{lstlisting}
// Mark read-only data as const
static const int BUFFER_SIZE = 1024;
static const char* const ERROR_MSG = "Error occurred";

// Function taking const pointer (won't modify)
void process(const Data* data) {
    // Can read, can't modify
}

// Const array
static const int LOOKUP[] = {1, 2, 3, 4, 5};

// Const protects from accidents
void example(void) {
    BUFFER_SIZE = 2048;  // Compile error!
    LOOKUP[0] = 10;      // Compile error!
}
\end{lstlisting}

\section{Summary}

Initialization patterns in C:

\begin{itemize}
    \item Always initialize variables---uninitialized data is undefined behavior
    \item Use \texttt{= \{0\}} for zero-initialization of any type
    \item Designated initializers (C99) make code clearer and more maintainable
    \item Compound literals create temporary objects inline
    \item Static initialization is faster and safer than dynamic
    \item Flexible array members handle variable-sized data elegantly
    \item Use initialization functions for complex setup
    \item Deep copy when dealing with pointers
    \item Mark const data as \texttt{const} for safety
    \item Prefer static/compile-time initialization when possible
\end{itemize}

Proper initialization prevents bugs and makes code more robust!
