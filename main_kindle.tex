\documentclass[11pt,openany]{book}

% Page setup for Kindle - optimized for readability
% Using standard page size with balanced margins
\usepackage[paperwidth=6in, paperheight=9in, margin=0.6in]{geometry}

% Essential packages
\usepackage[utf8]{inputenc}
\usepackage[T1]{fontenc}
\usepackage{lmodern}
\usepackage{inconsolata}
\usepackage{microtype}
\usepackage{xcolor}
\usepackage{listings}
\usepackage{tcolorbox}
\usepackage{enumitem}
\usepackage{hyperref}
\usepackage{fancyhdr}
\usepackage{titlesec}
\usepackage{amssymb}

% Colors for code - high contrast for better readability
\definecolor{codebg}{RGB}{248,248,250}
\definecolor{keywordcolor}{RGB}{170,13,145}
\definecolor{stringcolor}{RGB}{196,26,22}
\definecolor{commentcolor}{RGB}{87,166,74}
\definecolor{warningcolor}{RGB}{255,140,0}
\definecolor{tipcolor}{RGB}{70,130,180}

% Code listing setup - optimized for Kindle readability
\lstset{
    basicstyle=\small\ttfamily,  % Balanced size for code
    keywordstyle=\color{keywordcolor}\bfseries,
    stringstyle=\color{stringcolor},
    commentstyle=\color{commentcolor}\itshape,
    numberstyle=\tiny\color{gray},
    numbers=left,
    stepnumber=1,
    numbersep=8pt,
    backgroundcolor=\color{codebg},
    showspaces=false,
    showstringspaces=false,
    showtabs=false,
    frame=single,
    framesep=8pt,
    tabsize=4,
    captionpos=b,
    breaklines=true,
    breakatwhitespace=false,
    escapeinside={(*@}{@*)},
    language=C
}

% Custom boxes - clean and compact
\newtcolorbox{warningbox}{
    colback=warningcolor!5!white,
    colframe=warningcolor,
    title=\textbf{Warning},
    fonttitle=\bfseries,
    boxrule=0.8pt,
    arc=2pt,
    left=6pt,
    right=6pt,
    top=6pt,
    bottom=6pt
}

\newtcolorbox{tipbox}{
    colback=tipcolor!5!white,
    colframe=tipcolor,
    title=\textbf{Pro Tip},
    fonttitle=\bfseries,
    boxrule=0.8pt,
    arc=2pt,
    left=6pt,
    right=6pt,
    top=6pt,
    bottom=6pt
}

\newtcolorbox{notebox}{
    colback=yellow!10!white,
    colframe=yellow!80!black,
    title=\textbf{Note},
    fonttitle=\bfseries,
    boxrule=0.8pt,
    arc=2pt,
    left=6pt,
    right=6pt,
    top=6pt,
    bottom=6pt
}

% Header/Footer - simplified for Kindle
\pagestyle{fancy}
\fancyhf{}
\fancyhead[LE,RO]{\thepage}
\fancyhead[RE]{\leftmark}
\fancyhead[LO]{\rightmark}
\renewcommand{\headrulewidth}{0.4pt}

% Hyperref setup
\hypersetup{
    colorlinks=true,
    linkcolor=blue,
    filecolor=magenta,
    urlcolor=cyan,
    pdftitle={Practical C Programming: Idioms and Patterns (Kindle Edition)},
    pdfauthor={C Mastery Guide},
    pdfsubject={C Programming},
    pdfkeywords={C, programming, idioms, patterns, practical, Kindle}
}

% Title formatting - clean and professional
\titleformat{\chapter}[display]
{\normalfont\Large\bfseries}{\chaptertitlename\ \thechapter}{15pt}{\huge}
\titlespacing*{\chapter}{0pt}{0pt}{30pt}

% Slightly increased line spacing for Kindle readability
\linespread{1.15}

% Document info
\title{\LARGE\bfseries Practical C Programming:\\Idioms \& Patterns\\They Don't Teach in Books\\[0.8em]\large Kindle Edition}
\author{\large The Real-World Guide}
\date{\today}

\begin{document}

\frontmatter
\maketitle

% Copyright page
\thispagestyle{empty}
\vspace*{\fill}
\begin{center}
\textbf{Practical C Programming: Idioms \& Patterns They Don't Teach in Books}

\textbf{Kindle Edition}

Copyright \textcopyright\ 2025 by C Mastery Guide

All rights reserved. No part of this publication may be reproduced, distributed, or transmitted in any form or by any means, including photocopying, recording, or other electronic or mechanical methods, without the prior written permission of the publisher, except in the case of brief quotations embodied in critical reviews and certain other noncommercial uses permitted by copyright law.

\vspace{1em}

Published by C Mastery Guide

\vspace{1em}

\textbf{Disclaimer:}

The information in this book is distributed on an ``as is'' basis, without warranty. While every precaution has been taken in the preparation of this book, neither the author nor the publisher shall have any liability to any person or entity with respect to any loss or damage caused or alleged to be caused directly or indirectly by the code examples or instructions contained in this book.

All code examples are provided for educational purposes. The author and publisher are not responsible for any consequences of using this code in production environments. Please test thoroughly and use at your own risk.

\vspace{1em}

Trademarks: All terms mentioned in this book that are known to be trademarks or service marks have been appropriately capitalized. The publisher cannot attest to the accuracy of this information. Use of a term in this book should not be regarded as affecting the validity of any trademark or service mark.

\vspace{1em}

First Edition: 2025

\vspace{1em}

ISBN: 978-X-XXXX-XXXX-X (placeholder)

\end{center}
\vspace*{\fill}
\clearpage

% Dedication
\thispagestyle{empty}
\vspace*{3in}
\begin{center}
\textit{To every programmer who has debugged a segmentation fault at 3 AM,\\
spent hours hunting a missing semicolon,\\
and wondered why their carefully crafted pointer arithmetic was off by one.}

\vspace{2em}

\textit{You are not alone.}

\vspace{2em}

\textit{This book is for you.}
\end{center}
\vspace*{\fill}
\clearpage

% Acknowledgments
\chapter*{Acknowledgments}
\addcontentsline{toc}{chapter}{Acknowledgments}

This book would not exist without the contributions of countless C programmers over the past five decades. And coffee. Mostly coffee.

\textbf{To Dennis Ritchie and Brian Kernighan:} Thanks for creating C and then writing a book about it that's somehow still the best one. You set an impossibly high bar. This book doesn't reach it, but we're trying anyway.

\textbf{To the open source community:} The maintainers of SQLite, Redis, Git, the Linux kernel, FFmpeg, cURL, and countless other projects. Your code is so well-written it makes the rest of us look bad. Thanks for that. No, seriously—studying your projects is better than any CS degree.

\textbf{To Linus Torvalds:} For creating Linux, for creating Git, and for writing entertainingly angry emails about coding standards. You've taught us that good code matters and that tabs are objectively wrong. (The second part is debatable, but we're not brave enough to argue with you.)

\textbf{To Stack Overflow:} For answering the question ``Why does my C program segfault?'' approximately 47,000 times without (usually) being too sarcastic about it.

\textbf{To everyone who's ever debugged a segfault at 3 AM:} You are the true heroes. This book is for you.

\textbf{To the person who invented \texttt{valgrind}:} You saved us from ourselves. Repeatedly.

\textbf{To the readers:} Thanks for buying this book instead of just Googling everything. If it saves you even one hour of debugging time, it was worth the caffeine addiction and eye strain. If it \textit{doesn't} save you time, well, at least you learned some new ways to segfault.

\vspace{1em}

Special thanks to the creators of \LaTeX\ for making typesetting look this good, and to the maintainers of GCC, Clang, and MSVC for their compilers and their exceptionally creative error messages.

Finally, thanks to coffee. And energy drinks. And the occasional nap under the desk.

\vspace{2em}

— The Author, 2025

\textit{P.S. If you find errors in this book, they're features. But please report them anyway.}

\clearpage

% About the Author
\chapter*{About the Author}
\addcontentsline{toc}{chapter}{About the Author}

The author is a C programmer who has written enough \texttt{malloc()} calls to feel personally responsible for global memory consumption. They have debugged segmentation faults in production at 3 AM, argued about brace placement with colleagues, and once spent an entire afternoon hunting a bug that turned out to be a missing semicolon (we don't talk about that day).

This book exists because the author got tired of seeing the same question repeated: ``I know C syntax, but how do professionals actually write C?'' Textbooks teach \texttt{for} loops and \texttt{if} statements. Real codebases are full of opaque pointers, VTables, X-Macros, and other patterns that make newcomers wonder if they're reading the same language. This book fixes that.

The author believes C is like a very sharp knife: incredibly useful, potentially dangerous, and absolutely worth learning to use properly. Sure, modern languages have safety guards and garbage collectors, but C makes you understand what's actually happening in that computer sitting on your desk. That understanding is valuable regardless of what language you use day-to-day.

Also, the author has strong opinions about tabs vs spaces. Very strong opinions. (It's spaces. Don't @ me.)

\vspace{2em}

\textbf{Contact:}

For errata, questions, or feedback about this book, please visit:

\begin{itemize}
    \item \textbf{Repository}: \url{https://codeberg.org/_a/C_Idioms_And_Patterns}
    \item \textbf{Email}: See repository for contact information
    \item \textbf{Errata}: See the Errata chapter for how to report issues
\end{itemize}

\vspace{1em}

\textbf{Other Books by the Author:}

\textit{(Future titles to be announced)}

\vspace{1em}

\textbf{Stay Updated:}

\begin{itemize}
    \item Follow development on Codeberg for code examples and updates
    \item Check the repository for errata, supplementary materials, and announcements
    \item Join the community discussions on Reddit r/C\_Programming
\end{itemize}

\clearpage

\tableofcontents

\chapter{How to Use This Book}

\section*{Reading Strategies}

This book is designed to be used in multiple ways, depending on your needs and learning style.

\subsection*{Strategy 1: Cover-to-Cover (The Complete Immersion)}

\textbf{Best for:} Intermediate C programmers who want comprehensive knowledge.

\textbf{How to do it:}
\begin{enumerate}
    \item Start at Chapter 1, read sequentially through to the end
    \item Type out every code example—don't copy-paste
    \item Compile and run each example to see it work
    \item Experiment by modifying examples and observing results
    \item Take notes on patterns you find particularly useful
    \item Revisit challenging sections after completing other chapters
\end{enumerate}

\textbf{Time commitment:} 2-3 weeks of dedicated reading (2-3 hours daily)

\textbf{What you'll gain:} Complete understanding of professional C patterns and idioms

\subsection*{Strategy 2: Just-In-Time Reference (The Problem Solver)}

\textbf{Best for:} Experienced programmers solving specific problems.

\textbf{How to do it:}
\begin{enumerate}
    \item Use the detailed table of contents to find relevant chapters
    \item Jump directly to sections addressing your current problem
    \item Read the introduction and examples for that topic
    \item Implement the pattern in your code
    \item Refer to "Common Pitfalls" sections to avoid mistakes
    \item Bookmark frequently-referenced chapters for quick access
\end{enumerate}

\textbf{Time commitment:} 15-30 minutes per topic as needed

\textbf{What you'll gain:} Immediate solutions to specific challenges

\subsection*{Strategy 3: Code-First Learning (The Hands-On Approach)}

\textbf{Best for:} Developers who learn by doing.

\textbf{How to do it:}
\begin{enumerate}
    \item Scan each chapter's code examples first
    \item Try to understand what the code does before reading explanations
    \item Read the surrounding text only when confused
    \item Modify examples to test your understanding
    \item Create your own variations of the patterns
    \item Come back to read full explanations after experimenting
\end{enumerate}

\textbf{Time commitment:} Variable, depends on experimentation depth

\textbf{What you'll gain:} Deep intuitive understanding through exploration

\subsection*{Strategy 4: Reference + Study (The Professional Approach)}

\textbf{Best for:} Working professionals improving their C skills.

\textbf{How to do it:}
\begin{enumerate}
    \item Read one chapter per week during commute or lunch break
    \item Focus on chapters relevant to your current project
    \item Keep the book at your desk for quick reference
    \item Apply one new pattern per week in real code
    \item Review the glossary and quick reference regularly
    \item Discuss patterns with colleagues for deeper understanding
\end{enumerate}

\textbf{Time commitment:} 30-60 minutes per week over 4-5 months

\textbf{What you'll gain:} Gradual, sustained improvement in C skills

\section*{Chapter Dependencies}

Most chapters are self-contained, but some build on earlier concepts:

\textbf{Core Foundations (Read First):}
\begin{itemize}
    \item Chapter 1: Opaque Pointers
    \item Chapter 2: Function Pointers
    \item Chapter 7: Struct Patterns
\end{itemize}

\textbf{Can Read Independently:}
\begin{itemize}
    \item Chapter 3: Macros
    \item Chapter 4: Strings
    \item Chapter 5: Error Handling
    \item Chapter 8: Headers
    \item Chapter 9: Preprocessor
    \item Chapter 14: Testing
    \item Chapter 15: Build Patterns
\end{itemize}

\textbf{Advanced Topics (Read After Foundations):}
\begin{itemize}
    \item Chapter 11: State Machines (uses function pointers)
    \item Chapter 12: Generic Programming (uses macros and function pointers)
    \item Chapter 13: Linked Structures (uses struct patterns)
    \item Chapter 18: Advanced Patterns (uses everything)
\end{itemize}

\section*{Working with Code Examples}

\subsection*{Typing vs. Copy-Paste}

\textbf{We strongly recommend typing examples yourself.} Here's why:

\begin{itemize}
    \item \textbf{Muscle memory}: Your fingers learn the patterns
    \item \textbf{Attention to detail}: You notice every semicolon, pointer, and brace
    \item \textbf{Understanding}: You can't type what you don't understand
    \item \textbf{Debugging practice}: You'll make mistakes and learn to fix them
\end{itemize}

All code examples are available in the book's repository at:
\url{https://codeberg.org/_a/C_Idioms_And_Patterns}

Use the repository versions for:
\begin{itemize}
    \item Verifying your typed version compiles correctly
    \item Checking that you didn't miss anything
    \item Grabbing complete project structures
    \item Getting Makefiles and build scripts
\end{itemize}

\subsection*{Compiling Examples}

Most examples can be compiled with:
\begin{verbatim}
gcc -Wall -Wextra -std=c99 -o example example.c
\end{verbatim}

Some examples require additional flags or libraries—check the comments in each example.

\section*{Getting Help}

\textbf{When you're stuck:}

\begin{enumerate}
    \item \textbf{Check the code carefully}: Most issues are typos or missing semicolons
    \item \textbf{Read compiler errors slowly}: They usually tell you exactly what's wrong
    \item \textbf{Consult the glossary}: Terms you don't understand are defined there
    \item \textbf{Review related chapters}: Sometimes context from other chapters helps
    \item \textbf{Check the errata}: Known issues are documented
    \item \textbf{Ask online}: Stack Overflow, Reddit r/C\_Programming, or the book's repository
\end{enumerate}

\textbf{Resources in this book:}
\begin{itemize}
    \item \textbf{Table of Contents}: Find topics quickly
    \item \textbf{Glossary}: Define unfamiliar terms
    \item \textbf{Appendix A}: Quick reference for common idioms
    \item \textbf{Appendix B}: External resources and further reading
    \item \textbf{Bibliography}: Deep dives into specific topics
\end{itemize}

\section*{Practice Suggestions}

\textbf{After reading each chapter:}

\begin{enumerate}
    \item \textbf{Implement the patterns}: Write code using what you learned
    \item \textbf{Refactor old code}: Apply new patterns to existing projects
    \item \textbf{Study real codebases}: Find the patterns in SQLite, Redis, or Git
    \item \textbf{Teach someone}: Explaining concepts solidifies understanding
    \item \textbf{Take notes}: Write down patterns you'll use most often
\end{enumerate}

\textbf{Building a practice project:}

Consider building a small project that uses multiple patterns:
\begin{itemize}
    \item A simple database (hash tables, file I/O, error handling)
    \item A text editor (dynamic arrays, memory management, cross-platform code)
    \item A network server (sockets, state machines, circular buffers)
    \item A game (event handling, performance optimization, data structures)
\end{itemize}

\section*{Notes on Code Style}

The code in this book prioritizes \textbf{clarity over cleverness}. You'll notice:

\begin{itemize}
    \item \textbf{Explicit code}: We spell things out rather than using shortcuts
    \item \textbf{Comments}: More than you'd see in production (for teaching purposes)
    \item \textbf{Error checking}: Sometimes simplified for brevity
    \item \textbf{Naming}: Clear, descriptive names over short ones
\end{itemize}

In production code, you might:
\begin{itemize}
    \item Add more comprehensive error handling
    \item Use project-specific naming conventions
    \item Add logging and debugging support
    \item Include unit tests
    \item Follow your team's style guide
\end{itemize}

\section*{A Word on Standards}

This book primarily targets C99 and C11, with occasional C17 features. These are widely supported and represent modern C programming.

If you're working with:
\begin{itemize}
    \item \textbf{C89/C90}: Most patterns still work; avoid designated initializers and declarations in for-loops
    \item \textbf{C11}: All patterns work; you get additional features like atomics and threads
    \item \textbf{C17/C18}: Minor updates to C11; everything here applies
    \item \textbf{C2x (future)}: Check errata for updates when finalized
\end{itemize}

\vspace{2em}

\textbf{Now you're ready. Pick your reading strategy and dive in!}

\clearpage

\chapter{Preface: Read This or Suffer the Consequences}

Welcome to \textit{Practical C Programming}! Before you dive in, we need to have an honest conversation about what you're getting yourself into.

\section*{This Is NOT a Beginner Book (Seriously)}

Let's get this out of the way immediately: \textbf{If you don't already know C, close this book and run away.} Not walk. \textit{Run.}

This book assumes you already know:
\begin{itemize}
    \item What a pointer is (and that \texttt{*} and \texttt{\&} aren't just decorative symbols)
    \item How \texttt{malloc()} and \texttt{free()} work (and why forgetting \texttt{free()} is bad)
    \item Basic data structures (arrays, structs, linked lists)
    \item How to compile a C program without crying
    \item That segmentation faults are not a feature
    \item Why \texttt{char* str = "hello"; str[0] = 'H';} is a terrible idea
\end{itemize}

If you just said ``What's a segmentation fault?'' then you need a different book. May we suggest \textit{The C Programming Language} by Kernighan and Ritchie? Come back when you've read that, written a few thousand lines of C, and debugged at least one memory leak at 2 AM.

\begin{warningbox}
\textbf{Reality Check:} This book starts at intermediate and goes to advanced. We're not holding your hand. We assume you can read code, understand pointers without panicking, and have already made most of the beginner mistakes. If you're still confusing \texttt{malloc(sizeof(int*))} with \texttt{malloc(sizeof(int))}, bookmark this book and come back in six months.
\end{warningbox}

\section*{Warning: Extreme Code Density Ahead}

Fair warning: \textbf{This book is approximately 60\% code listings.}

We're not kidding. Open to a random page and you'll likely find:
\begin{itemize}
    \item At least one complete code example (15-50 lines)
    \item Multiple smaller code snippets
    \item Preprocessor macros that look like cursed incantations
    \item Function pointers doing unspeakable things
    \item Struct definitions that make you question your life choices
\end{itemize}

Why so much code? Because \textbf{this is a book about how C is actually written}, not just talked about. You can't learn C idioms from prose alone---you need to see the code, understand it, type it out, compile it, break it, fix it, and eventually internalize it.

\begin{tipbox}
\textbf{Pro Tip:} Don't just read the code---\textit{type it out}. Copy-paste is the enemy of learning. Your fingers need to feel the pain of typing \texttt{typedef struct node \{ struct node* next; \} node;} before your brain truly understands it.
\end{tipbox}

If you were hoping for a book that gently explains concepts with minimal code examples, you've come to the wrong place. This book is for people who \textit{like} reading code. If seeing a 100-line code listing makes you excited rather than nauseous, you're in the right spot.

\section*{What This Book Actually Covers}

This book fills the gap between ``I know C syntax'' and ``I can write professional C code.'' It covers the idioms, patterns, and techniques that experienced C programmers use constantly but are \textit{never} explained in university courses or beginner tutorials.

We cover things like:
\begin{itemize}
    \item \textbf{Opaque Pointers}: How to hide implementation details like a professional
    \item \textbf{Function Pointers}: Callbacks, vtables, and other ways to confuse your coworkers
    \item \textbf{Macro Magic}: The preprocessor's dark arts (use responsibly)
    \item \textbf{String Handling}: Because C strings are a nightmare and you need to know why
    \item \textbf{Error Handling}: Beyond ``just return -1 and hope for the best''
    \item \textbf{Memory Patterns}: Arena allocators, object pools, and other malloc alternatives
    \item \textbf{Struct Tricks}: Flexible arrays, inheritance without OOP, and other black magic
    \item \textbf{Header Organization}: So you stop creating circular include dependencies
    \item \textbf{Generic Programming}: Templates? We don't need no stinking templates!
    \item \textbf{Testing in C}: Yes, it's possible. No, it's not fun.
    \item \textbf{Cross-Platform Code}: Making your code work on Windows \textit{and} Unix (spoiler: it's painful)
    \item \textbf{Advanced Patterns}: X-Macros, intrusive data structures, and other party tricks
\end{itemize}

These are the patterns used by SQLite, Redis, the Linux kernel, Git, and every other serious C codebase. They're not in the C standard. They're not in K\&R. They're the accumulated wisdom of decades of C programmers solving real problems.

\section*{Who This Book Is For}

You're the target audience if:
\begin{itemize}
    \item You can write a linked list in C without consulting Stack Overflow
    \item You've debugged at least one use-after-free bug
    \item You understand why \texttt{printf("\%s", NULL)} is a bad idea
    \item You want to understand how professional C code is structured
    \item You're tired of tutorials that treat you like a child
    \item You actually \textit{enjoy} reading other people's code
    \item You have a job (or want one) where C is used in production
\end{itemize}

You're \textbf{not} the target audience if:
\begin{itemize}
    \item You're still learning what pointers do
    \item You think ``undefined behavior'' is a myth
    \item You've never heard of \texttt{valgrind}
    \item Reading code makes you anxious
    \item You want hand-holding and gentle explanations
    \item You prefer watching videos to reading dense technical content
\end{itemize}

\section*{How to Use This Book}

\textbf{Approach \#1: The Deep Dive}

Read it cover-to-cover. Type out every example. Compile everything. Break things. Fix them. This is the hard way, but you'll learn the most.

\textbf{Approach \#2: The Reference Manual}

Jump to whatever topic you need. Each chapter is relatively self-contained. Need to understand opaque pointers? Chapter 1. Function pointers confusing you? Chapter 2. Trying to make code work on Windows? Chapter 17 (and good luck).

\textbf{Approach \#3: The Code Review}

Read the code first, then read the explanations. See if you can figure out what's happening before we tell you. This trains you to read unfamiliar codebases---a critical skill.

\begin{notebox}
\textbf{Every Pattern Includes:}
\begin{itemize}
    \item Complete, working code examples (not pseudocode or fragments)
    \item Explanations of \textit{why}, not just \textit{how}
    \item Real-world use cases from actual projects
    \item Common pitfalls and gotchas
    \item Pro tips from experienced developers
\end{itemize}
\end{notebox}

\section*{A Word About the Code}

All code examples in this book:
\begin{itemize}
    \item Are complete and compilable (unless explicitly marked as pseudocode)
    \item Use C99 or later (occasionally C11 when needed)
    \item Follow common conventions (but we'll explain alternative styles)
    \item Prioritize clarity over cleverness (usually)
    \item Include error checking (when relevant to the pattern)
\end{itemize}

We assume you're using a modern C compiler (GCC, Clang, or MSVC). If you're still on Borland C++ from 1995, you have bigger problems than this book can solve.

\section*{Fair Warning About Chapter 17}

Chapter 17 is about cross-platform C development (Windows, Linux, macOS). It's \textit{dense}. It's \textit{long}. It contains approximately 847 preprocessor conditionals (we didn't count, but it feels like it).

If you've never tried to make C code work on both Windows and Unix, Chapter 17 will be enlightening. If you \textit{have} tried, Chapter 17 will feel like group therapy.

\section*{The Unspoken Promise}

By the end of this book, you'll be able to:
\begin{itemize}
    \item Read professional C codebases without feeling lost
    \item Understand \textit{why} experienced developers structure code certain ways
    \item Write C that doesn't just work, but is maintainable and robust
    \item Debug complex C programs systematically
    \item Contribute to real C projects with confidence
    \item Argue about coding style in code reviews (a critical skill)
\end{itemize}

But here's what this book \textit{won't} teach you:
\begin{itemize}
    \item The absolute basics of C syntax (go read K\&R)
    \item How to write perfect, bug-free C (nobody can)
    \item Algorithms and data structures in detail (different book)
    \item How to become a 10x programmer overnight (not possible)
\end{itemize}

\section*{Final Thoughts Before We Begin}

C is not a beginner-friendly language. It never was. It was designed by programmers, for programmers, to write operating systems. It trusts you completely---and that trust can be your downfall.

This book respects your intelligence. We won't waste time explaining what a variable is. We won't patronize you. We'll throw code at you and expect you to understand it (or at least puzzle through it).

If this sounds intimidating, good. It should be. C is a powerful tool, and powerful tools require skill to wield safely.

If this sounds \textit{exciting}, excellent. You're in the right place.

\vspace{1em}

\textbf{Ready? Let's write some damn good C code.}

\vspace{2em}

\begin{center}
\textit{``Everyone knows that debugging is twice as hard as writing a program in the first place.\\
So if you're as clever as you can be when you write it, how will you ever debug it?''}\\
\vspace{0.5em}
--- Brian Kernighan
\end{center}

\mainmatter

% Include all chapter files
\chapter{The Opaque Pointer Pattern}

\section{What Is It?}

The opaque pointer pattern (also called "pimpl" or "handle" pattern) is one of the most important idioms in professional C code. It's a way to hide the internal details of a data structure from users of your code. Think of it as the "none of your business" pattern, but polite and professional.

\subsection*{The Locked Box Analogy}

Think of it like a locked box with a claim ticket. Imagine you go to a coat check at a fancy restaurant:

\begin{enumerate}
    \item You hand your coat to the attendant
    \item The attendant puts it in a locked closet (you can't see inside)
    \item You get a ticket with a number on it (the "pointer")
    \item Later, you give back the ticket and get your coat
    \item You never saw where your coat was stored or how the closet is organized
    \item The attendant can reorganize the closet however they want---you don't care, you just want your coat back
\end{enumerate}

That's exactly how opaque pointers work! The user gets a "ticket" (pointer) that represents their data, but they can't see or access the actual storage. They must ask the library functions (the "attendants") to do things with their data.

\subsection*{Why This Matters in the Real World}

Here's what the textbooks don't tell you: this pattern is the foundation of almost every stable C API in existence. Let me give you concrete examples:

\begin{itemize}
    \item \textbf{OpenSSL} (security library): Has used this for 20+ years. They've added new features, fixed bugs, and optimized internals---all without breaking existing programs
    \item \textbf{Linux kernel}: Maintains stable interfaces so device drivers written 10 years ago still work
    \item \textbf{GTK+} (graphical toolkit): Can evolve and improve without forcing every app to recompile
    \item \textbf{SQLite} (database): Can change its internal storage format without breaking apps
\end{itemize}

It's basically the "trust me, I know what I'm doing" pattern, except done properly. You're saying to users: "You don't need to know how this works internally. Just trust that when you call my functions, I'll do the right thing." (And if they don't trust you, well, they can go write their own library. Good luck with that.)

\section{Why Use It? The Real Reasons}

\begin{enumerate}
    \item \textbf{ABI Stability}: Change internals without recompiling user code
    \item \textbf{Information Hiding}: Users can't accidentally break invariants
    \item \textbf{Reduced Coupling}: Implementation can change completely
    \item \textbf{Faster Compilation}: Users don't include implementation headers
    \item \textbf{Trade Secret Protection}: Hide proprietary algorithms
    \item \textbf{Multiple Implementations}: Same API, different backends
    \item \textbf{Stable Symbol Table}: Fewer exported symbols in shared libraries
\end{enumerate}

Let me explain the ABI stability point because it's crucial: When you ship a shared library (\texttt{.so} or \texttt{.dll}), your users compile against your headers. If you expose struct internals, adding a single field breaks binary compatibility. Every user must recompile. And they will be... unhappy. (That's putting it mildly. In reality, they'll write angry emails, create bug reports with CAPS LOCK, and possibly send you passive-aggressive tweets at 2 AM.) With opaque pointers, you can add, remove, or reorder fields freely. This is why every long-lived C library uses this pattern---survival instinct.

\section{The Basic Pattern}

\subsection{In the Header File (mylib.h)}

\begin{lstlisting}
#ifndef MYLIB_H
#define MYLIB_H

// Forward declaration - users see this
// They know the type exists but not what's inside
typedef struct MyObject MyObject;

// Constructor - returns pointer to opaque type
MyObject* myobject_create(void);

// Operations - all take opaque pointer
void myobject_do_something(MyObject* obj);
int myobject_get_value(const MyObject* obj);
void myobject_set_name(MyObject* obj, const char* name);

// Destructor - frees opaque object
void myobject_destroy(MyObject* obj);

#endif /* MYLIB_H */
\end{lstlisting}

\subsection*{Understanding This Header - Line by Line}

Let's break down what's happening here in plain English:

\textbf{Line: \texttt{typedef struct MyObject MyObject;}}

This is the magic line! This is called a "forward declaration" or "incomplete type." Think of it like this:

\begin{itemize}
    \item You're telling the compiler: "Hey, there's a thing called MyObject. I'm not telling you what's inside it, but it exists somewhere. Trust me on this."
    \item It's like saying "There's a person named Bob" without describing what Bob looks like, what Bob does for a living, or whether Bob is even human
    \item The compiler accepts this and lets users use \texttt{MyObject*} (pointer to MyObject)
    \item But the compiler won't let users do \texttt{sizeof(MyObject)} or \texttt{obj.value} because it doesn't know the contents (and that's the whole point---we're gatekeeping our struct like it's an exclusive nightclub)
\end{itemize}

\textbf{Why this works:} In C, you can have a pointer to something without knowing its size! A pointer is just a memory address (8 bytes on 64-bit systems, 4 bytes on 32-bit). The compiler doesn't need to know how big MyObject is to pass around its address.

\textbf{Function: \texttt{MyObject* myobject\_create(void);}}

This is the "constructor"---the function that creates new objects. Think of it as a factory:

\begin{itemize}
    \item User calls this function
    \item Function allocates memory (using malloc internally---users don't see this)
    \item Function returns a pointer---the "claim ticket" to the object
    \item User stores this pointer but can't peek inside
\end{itemize}

Real-world analogy: Like ordering food at a restaurant. You place an order, they give you a number, you wait. You don't go into the kitchen to cook it yourself!

\textbf{Function: \texttt{void myobject\_do\_something(MyObject* obj);}}

This is an "operation" function. The user:

\begin{itemize}
    \item Passes in their claim ticket (the pointer)
    \item The function accesses the real object
    \item Does something with it
    \item Returns (possibly with a result)
\end{itemize}

The user never touches the object directly. It's like asking a librarian to get a book from the restricted section---you hand them your library card, they get the book, they do something with it. You never enter the restricted area.

\textbf{Function: \texttt{void myobject\_destroy(MyObject* obj);}}

This is the "destructor"---cleanup function. Important points:

\begin{itemize}
    \item User MUST call this when done
    \item Function frees all memory
    \item After calling this, the pointer is invalid (ticket is voided)
    \item If user doesn't call this---memory leak! (Like never picking up your coat from coat check---it sits there forever)
\end{itemize}

\begin{notebox}
\textbf{About \texttt{const}:} Notice the \texttt{const} on \texttt{myobject\_get\_value}. This tells the compiler: "This function won't modify the object, it only reads from it." It's like the difference between a librarian who lets you \textit{read} a book vs one who lets you \textit{edit} it. Even though users can't see inside, you can still enforce const-correctness in your API! This prevents users from accidentally modifying objects when they just wanted to read them.
\end{notebox}

\subsection{In the Implementation File (mylib.c)}

\begin{lstlisting}
#include "mylib.h"
#include <stdlib.h>
#include <string.h>

// The actual definition - users NEVER see this
// You can change this freely without breaking user code
struct MyObject {
    int value;
    char* name;
    size_t ref_count;  // For reference counting
    void* internal_state;  // Internal implementation details
    // Add more fields anytime - ABI stays stable!
};

MyObject* myobject_create(void) {
    MyObject* obj = malloc(sizeof(MyObject));
    if (obj) {
        obj->value = 0;
        obj->name = NULL;
        obj->ref_count = 1;
        obj->internal_state = NULL;
    }
    return obj;
}

void myobject_do_something(MyObject* obj) {
    if (!obj) return;  // Defensive programming

    obj->value++;
    // Users can't accidentally bypass this logic
    // and corrupt obj->value
}

int myobject_get_value(const MyObject* obj) {
    return obj ? obj->value : -1;
}

void myobject_set_name(MyObject* obj, const char* name) {
    if (!obj) return;

    // Free old name
    free(obj->name);

    // Duplicate new name
    obj->name = name ? strdup(name) : NULL;
}

void myobject_destroy(MyObject* obj) {
    if (obj) {
        free(obj->name);
        free(obj->internal_state);
        free(obj);
    }
}
\end{lstlisting}

\subsection*{Understanding the Implementation - The Secret Recipe}

This is where the magic happens! This file is like the restaurant's kitchen---customers (users) never see it, but this is where the real work is done.

\textbf{The Full Struct Definition}

\begin{lstlisting}
struct MyObject {
    int value;
    char* name;
    size_t ref_count;
    void* internal_state;
};
\end{lstlisting}

This is the COMPLETE definition. Notice:

\begin{itemize}
    \item This appears ONLY in the .c file, never in the .h file
    \item Users who include your library never see this
    \item You can add fields, remove fields, reorder fields---users won't notice
    \item It's like the secret recipe---you know what's in it, customers just taste the result
\end{itemize}

\textbf{Why can you change it freely?} Because when users compile their code, they only see the header file. They never compile this .c file! You compile the .c file into a library (.so or .dll), and users link against that library. As long as function names and parameters don't change, the internals can be completely different.

\textbf{The Create Function - Step by Step}

\begin{lstlisting}
MyObject* myobject_create(void) {
    MyObject* obj = malloc(sizeof(MyObject));  // [1]
    if (obj) {                                  // [2]
        obj->value = 0;                         // [3]
        obj->name = NULL;
        obj->ref_count = 1;
        obj->internal_state = NULL;
    }
    return obj;                                 // [4]
}
\end{lstlisting}

Let's walk through this like we're explaining it to a 10-year-old:

\begin{enumerate}
    \item \textbf{[1] Allocate memory}: \texttt{malloc(sizeof(MyObject))}
    \begin{itemize}
        \item Ask the operating system: "Hey, can I have enough memory to store a MyObject?"
        \item OS responds: "Sure, here's the address: 0x5589a4f0"
        \item malloc returns this address (the pointer)
        \item Think of it like renting a storage unit---you get the unit number (address)
    \end{itemize}

    \item \textbf{[2] Check if allocation succeeded}: \texttt{if (obj)}
    \begin{itemize}
        \item Sometimes the OS says "Sorry, I'm out of memory!" and returns NULL
        \item We check if we actually got memory before using it
        \item Like checking if the ATM actually gave you money before walking away
        \item If malloc failed, we skip initialization and return NULL to user
    \end{itemize}

    \item \textbf{[3] Initialize the fields}: Set everything to safe defaults
    \begin{itemize}
        \item \texttt{obj->value = 0}: Start the value at zero (like resetting a counter)
        \item \texttt{obj->name = NULL}: No name yet (empty string would be \texttt{""})
        \item \texttt{obj->ref\_count = 1}: Someone is using this (the creator)
        \item \texttt{obj->internal\_state = NULL}: No extra state yet
        \item This is like unboxing a new phone---it comes with default settings
    \end{itemize}

    \item \textbf{[4] Return the pointer}: Give the "claim ticket" to the user
    \begin{itemize}
        \item User gets the address: 0x5589a4f0
        \item They can't see what's at that address, but they can pass it back to us
        \item Like getting a valet ticket---you don't know where they parked your car, but you can get it back with the ticket
    \end{itemize}
\end{enumerate}

\textbf{The Do Something Function - Controlled Access}

\begin{lstlisting}
void myobject_do_something(MyObject* obj) {
    if (!obj) return;  // Safety check
    obj->value++;      // Increment value
}
\end{lstlisting}

This demonstrates a key principle: \textbf{controlled modification}

\begin{itemize}
    \item User can't do \texttt{obj->value++} directly (won't compile!)
    \item User MUST call \texttt{myobject\_do\_something(obj)}
    \item We can add validation, logging, locking---whatever we want
    \item User just sees: "I called a function, something happened"
\end{itemize}

Real-world example: Imagine a bank account. You can't directly edit the balance in the bank's database. You call \texttt{withdraw()} or \texttt{deposit()}, and the bank updates it for you (with proper checks!). This is the same concept.

\textbf{The Get Value Function - Safe Reading}

\begin{lstlisting}
int myobject_get_value(const MyObject* obj) {
    return obj ? obj->value : -1;
}
\end{lstlisting}

This is a "getter" function. Breaking it down:

\begin{itemize}
    \item \texttt{const MyObject* obj}: The \texttt{const} means "I promise not to modify this object"
    \item \texttt{obj ? obj->value : -1}: This is a ternary operator (shorthand if/else)
    \begin{itemize}
        \item If obj is not NULL: return obj->value
        \item If obj is NULL: return -1 (error code)
    \end{itemize}
    \item Users get the value without touching the internals
\end{itemize}

It's like asking a store clerk "How much does this cost?" They look at the price tag (which you can't see) and tell you. You don't need to see the tag directly.

\textbf{The Destroy Function - Cleanup}

\begin{lstlisting}
void myobject_destroy(MyObject* obj) {
    if (obj) {
        free(obj->name);           // [1] Free the name string
        free(obj->internal_state); // [2] Free internal data
        free(obj);                 // [3] Free the object itself
    }
}
\end{lstlisting}

Critical cleanup sequence:

\begin{enumerate}
    \item \textbf{Free owned resources first}: Free \texttt{name} and \texttt{internal\_state}
    \begin{itemize}
        \item These were allocated separately (by strdup or other allocations)
        \item Must be freed before freeing the object
        \item Like emptying a box before throwing the box away
    \end{itemize}

    \item \textbf{Free the object last}: \texttt{free(obj)}
    \begin{itemize}
        \item This releases the memory we got from malloc
        \item After this, the pointer is invalid---it's a "dangling pointer"
        \item Using it after free = undefined behavior (crash, corruption, or worse)
        \item Like burning your valet ticket---you can't use it anymore
    \end{itemize}
\end{enumerate}

\begin{warningbox}
\textbf{Common Mistake:} New programmers often do:
\begin{lstlisting}
free(obj);           // BUG! Free the object first
free(obj->name);     // CRASH! obj->name is invalid now
\end{lstlisting}

You must free in reverse order: free the things inside, THEN free the container. It's like unpacking boxes---take items out first, then throw away the box.
\end{warningbox}

\section{What Actually Happens in Memory}

Here's what most books won't tell you: Let's examine the memory layout and how this works at the binary level.

\begin{lstlisting}
// When user code calls:
MyObject* obj = myobject_create();

// What actually happens:
// 1. malloc() allocates memory on the heap
// 2. The address is returned as a void-like pointer
// 3. User only knows it's a "MyObject*" - an address
// 4. User has NO IDEA how much memory is allocated
// 5. sizeof(MyObject) won't compile in user code!

// In memory (64-bit system):
// Address      Content
// 0x5589a4f0:  0x0000002A          // obj->value = 42
// 0x5589a4f4:  (padding)
// 0x5589a4f8:  0x5589b120          // obj->name pointer
// 0x5589a500:  0x00000001          // obj->ref_count
// 0x5589a508:  0x00000000          // obj->internal_state
// 0x5589a510:  (next allocation)

// User code only has: 0x5589a4f0 (the pointer)
// User cannot do: obj->value  (won't compile!)
// User cannot do: sizeof(*obj)  (won't compile!)
// User MUST use: myobject_get_value(obj)
\end{lstlisting}

\subsection*{Visualizing Memory - The Storage Unit Analogy}

Think of computer memory like a massive storage facility with millions of units:

\begin{itemize}
    \item Each unit has an address (like "Unit 0x5589a4f0")
    \item When you call malloc, you rent some units
    \item The pointer is the unit number---that's ALL you get
    \item You don't know what's in the units, you can't open them yourself
    \item You have to ask the storage facility staff (the library functions) to get things in/out
\end{itemize}

\textbf{What's actually stored?} Let's break down the memory layout:

\begin{enumerate}
    \item \textbf{0x5589a4f0: Value (4 bytes)}: The integer value (currently 42 or 0x2A in hex)
    \begin{itemize}
        \item Takes up 4 bytes because \texttt{int} is typically 4 bytes
        \item Stored in binary: 00000000 00000000 00000000 00101010
    \end{itemize}

    \item \textbf{0x5589a4f4: Padding (4 bytes)}: Empty space!
    \begin{itemize}
        \item Why? Because the next field is a pointer (8 bytes on 64-bit systems)
        \item Pointers must be aligned to 8-byte boundaries for performance
        \item CPU reads memory faster when data is aligned
        \item It's like leaving a gap on a shelf so the next item fits perfectly
    \end{itemize}

    \item \textbf{0x5589a4f8: Name pointer (8 bytes)}: Address pointing to the string
    \begin{itemize}
        \item Not the string itself! Just the address where the string is stored
        \item The string "hello" might be at address 0x5589b120
        \item If name is NULL, this would be 0x0000000000000000
        \item Like storing a forwarding address instead of the actual item
    \end{itemize}

    \item \textbf{0x5589a500: Reference count (8 bytes)}: How many owners
    \begin{itemize}
        \item \texttt{size\_t} is 8 bytes on 64-bit systems
        \item Tracks how many references point to this object
        \item Used for memory management (more on this later)
    \end{itemize}

    \item \textbf{0x5589a508: Internal state pointer (8 bytes)}: Extra data
    \begin{itemize}
        \item Another pointer to additional data
        \item Currently NULL (all zeros)
        \item Allows for future expansion without changing the struct
    \end{itemize}
\end{enumerate}

\textbf{Total size}: 4 + 4 + 8 + 8 + 8 = 32 bytes for one MyObject!

\textbf{What the user has:} Just the number 0x5589a4f0. That's it! They can't:
\begin{itemize}
    \item Read \texttt{obj->value} (compiler error: incomplete type)
    \item Write \texttt{obj->name = "test"} (compiler error: incomplete type)
    \item Call \texttt{sizeof(*obj)} (compiler error: incomplete type)
    \item Allocate on stack: \texttt{MyObject obj;} (compiler error: incomplete type)
\end{itemize}

They can ONLY:
\begin{itemize}
    \item Store the pointer: \texttt{MyObject* ptr = obj;}
    \item Pass it to functions: \texttt{myobject\_do\_something(obj);}
    \item Compare it: \texttt{if (obj1 == obj2)}
    \item Check for NULL: \texttt{if (obj != NULL)}
\end{itemize}

\begin{tipbox}
\textbf{Pro tip:} The incomplete type prevents users from allocating objects on the stack. This gives you control: all objects must go through your allocator, which means you can track them, pool them, or implement custom memory management.

Think of it like this: If users could create objects themselves (\texttt{MyObject obj;}), they could create them on the stack (temporary storage). When the function returns, that memory disappears---poof! Your library wouldn't know about it, couldn't track it, and couldn't clean it up properly.

By forcing users to call \texttt{myobject\_create()}, YOU control where the memory comes from. You could:
\begin{itemize}
    \item Allocate from a custom pool (faster than malloc)
    \item Keep a list of all live objects (useful for debugging)
    \item Add guards around the memory to detect corruption
    \item Use reference counting to prevent leaks
\end{itemize}

It's like being the bouncer at an exclusive club---nobody gets in without your permission, and you know exactly who's inside at all times.
\end{tipbox}

\section{Real-World Examples from Production Code}

Let's see how real, battle-tested projects use opaque pointers. These aren't toy examples---these are patterns from software running on millions of machines worldwide.

\subsection{Example 1: FILE* in the Standard Library}

This is exactly how \texttt{FILE*} works! You've been using opaque pointers all along.

\begin{lstlisting}
// In stdio.h (simplified):
typedef struct _IO_FILE FILE;  // Opaque!

FILE* fopen(const char* path, const char* mode);
int fclose(FILE* stream);

// You use it like this:
FILE* f = fopen("data.txt", "r");
if (f) {
    // You have NO IDEA what's in FILE
    // Is there a buffer? Buffer size? File descriptor?
    // Position? Error flags? You don't know and don't need to!

    fread(buffer, 1, size, f);  // Just works
    fclose(f);
}

// The actual FILE structure (glibc implementation):
struct _IO_FILE {
    int _flags;
    char* _IO_read_ptr;
    char* _IO_read_end;
    char* _IO_read_base;
    char* _IO_write_base;
    char* _IO_write_ptr;
    char* _IO_write_end;
    char* _IO_buf_base;
    char* _IO_buf_end;
    // ... many more fields

    struct _IO_FILE* _chain;
    int _fileno;
    // ... even more fields
};

// This structure has changed over 30 years of glibc evolution
// Your code from 1995? Still compiles and runs!
// That's the power of opaque pointers
// (Unlike your Pentium from 1995, which definitely does NOT still run)
\end{lstlisting}

\subsection{Example 2: Redis - String Objects (SDS)}

\textbf{What is Redis?} An in-memory data store used by Twitter, GitHub, Stack Overflow, and millions of other sites. It's famous for being blazingly fast and rock-solid.

\textbf{The Pattern:} Redis uses opaque pointers for its "Simple Dynamic String" (SDS) type.

\begin{lstlisting}
// In Redis header (sds.h) - What users see:
typedef char *sds;  // Opaque! Looks like char* but isn't

sds sdsnew(const char *init);           // Create new string
void sdsfree(sds s);                     // Free string
sds sdscat(sds s, const char *t);       // Concatenate
size_t sdslen(const sds s);             // Get length
\end{lstlisting}

\textbf{The Secret (sds.c):} The actual structure is HIDDEN BEFORE the pointer!

\begin{lstlisting}
// The real structure (users never see this):
struct sdshdr {
    unsigned int len;        // Current string length
    unsigned int free;       // Unused bytes in buffer
    char buf[];             // Flexible array member
};

// The trick: 'sds' points to buf, not to the struct!
// Memory layout:
// [len][free][b][u][f][f][e][r][\0]
//             ^
//             sds points here!

sds sdsnew(const char *init) {
    size_t initlen = strlen(init);
    // Allocate struct + buffer space
    struct sdshdr *sh = malloc(sizeof(struct sdshdr) + initlen + 1);

    sh->len = initlen;
    sh->free = 0;
    memcpy(sh->buf, init, initlen + 1);

    return sh->buf;  // Return pointer to buffer, not struct!
}

size_t sdslen(const sds s) {
    // Get the struct by backing up from the pointer!
    struct sdshdr *sh = (void*)(s - sizeof(struct sdshdr));
    return sh->len;
}
\end{lstlisting}

\textbf{Why This Is Brilliant:}

\begin{enumerate}
    \item \textbf{Compatible with C strings}: You can pass an \texttt{sds} to \texttt{printf()} or any function expecting \texttt{char*}. It works because \texttt{sds} points to a null-terminated buffer.

    \item \textbf{O(1) length}: Normal C strings require \texttt{strlen()} which scans the entire string (O(n)). Redis stores length in the hidden header, so \texttt{sdslen()} is instant.

    \item \textbf{Knows its capacity}: The \texttt{free} field tracks unused space. When concatenating, Redis can check if there's room without scanning.

    \item \textbf{Can grow efficiently}: When buffer is too small, Redis can \texttt{realloc()} the whole thing (struct + buffer) and update the pointer.
\end{enumerate}

\textbf{The Opaque Magic:} Users don't know about the hidden header. They just see a "string" type that works like \texttt{char*} but never needs \texttt{strlen()}. Redis can change the header layout (they've done this several times) without breaking user code!

\textbf{Real-world impact:} This design makes Redis strings 3-4x faster than naive C strings for common operations. When you're processing millions of requests per second, this matters!

\subsection{Example 3: PostgreSQL - Memory Contexts}

\textbf{What is PostgreSQL?} The world's most advanced open source database. Used by Apple, Instagram, Reddit, and countless enterprises.

\textbf{The Problem:} In a database, memory allocation is complicated. You might allocate thousands of small objects for a query, then need to free them all at once. Calling \texttt{free()} thousands of times is slow and error-prone.

\textbf{The Pattern:} PostgreSQL uses "memory contexts" - opaque handles to memory arenas.

\begin{lstlisting}
// In PostgreSQL header (palloc.h) - What users see:
typedef struct MemoryContextData *MemoryContext;  // Opaque!

// Create a new memory context
MemoryContext AllocSetContextCreate(
    MemoryContext parent,
    const char *name,
    Size minContextSize,
    Size initBlockSize,
    Size maxBlockSize
);

// Allocate from a context
void *MemoryContextAlloc(MemoryContext context, Size size);

// Free ALL memory in context at once!
void MemoryContextReset(MemoryContext context);

// Delete entire context
void MemoryContextDelete(MemoryContext context);
\end{lstlisting}

\textbf{The Secret (aset.c):} The implementation uses a pool allocator with multiple blocks.

\begin{lstlisting}
// Simplified version of the real structure:
typedef struct MemoryContextData {
    NodeTag type;                    // Type identifier
    MemoryContextMethods *methods;   // VTable for operations!
    MemoryContext parent;            // Parent context
    MemoryContext firstchild;        // Child contexts
    MemoryContext nextchild;         // Sibling contexts
    char *name;                      // For debugging
    bool isReset;                    // Has been reset?
    // ... more fields for tracking
} MemoryContextData;

// Different implementations use different methods:
typedef struct MemoryContextMethods {
    void *(*alloc)(MemoryContext context, Size size);
    void (*free_p)(MemoryContext context, void *pointer);
    void *(*realloc)(MemoryContext context, void *ptr, Size size);
    void (*reset)(MemoryContext context);
    void (*delete_context)(MemoryContext context);
    // ... more method pointers
} MemoryContextMethods;
\end{lstlisting}

\textbf{How It Works in Practice:}

\begin{lstlisting}
// Example: Processing a database query

// Create a context for this query
MemoryContext query_context = AllocSetContextCreate(
    CurrentMemoryContext,
    "Query Processing",
    ALLOCSET_DEFAULT_SIZES
);

// Switch to this context (all allocations go here now)
MemoryContext old_context = MemoryContextSwitchTo(query_context);

// Allocate lots of stuff (parse trees, execution plans, results)
QueryPlan *plan = palloc(sizeof(QueryPlan));
ResultSet *results = palloc(sizeof(ResultSet));
// ... thousands more allocations

// Execute the query
execute_query(plan, results);

// Send results to client
send_to_client(results);

// Clean up - ONE CALL frees EVERYTHING!
MemoryContextSwitchTo(old_context);
MemoryContextDelete(query_context);

// All those allocations? Gone. No memory leaks possible!
\end{lstlisting}

\textbf{Why This Is Brilliant:}

\begin{enumerate}
    \item \textbf{No memory leaks}: Even if query execution throws an error, you can reset the context and recover. No hunting for which allocations succeeded.

    \item \textbf{Fast cleanup}: Freeing thousands of objects = one function call instead of thousands of \texttt{free()} calls.

    \item \textbf{Hierarchical}: Contexts can have child contexts. Deleting parent deletes all children automatically.

    \item \textbf{Multiple implementations}: PostgreSQL has AllocSet (default), Slab (for fixed-size objects), and Generation (for append-only patterns). All use the same opaque \texttt{MemoryContext} type!
\end{enumerate}

\textbf{The Opaque Magic:} User code doesn't know if they're using AllocSet, Slab, or Generation allocators. The opaque pointer and VTable pattern lets PostgreSQL swap implementations transparently.

\subsection{Example 4: NGINX - Connection Objects}

\textbf{What is NGINX?} A web server powering over 400 million websites. It's famous for handling 10,000+ simultaneous connections on modest hardware.

\textbf{The Pattern:} NGINX uses opaque pointers for network connections.

\begin{lstlisting}
// In NGINX headers (ngx_connection.h):
typedef struct ngx_connection_s ngx_connection_t;  // Opaque!

// You never see the full definition unless you're in core NGINX code
struct ngx_connection_s {
    void *data;                  // User can attach custom data
    ngx_event_t *read;           // Read event handler
    ngx_event_t *write;          // Write event handler

    ngx_socket_t fd;             // File descriptor

    ngx_recv_pt recv;            // Function pointer for receiving
    ngx_send_pt send;            // Function pointer for sending
    ngx_recv_chain_pt recv_chain;
    ngx_send_chain_pt send_chain;

    ngx_listening_t *listening;  // Listening socket config

    off_t sent;                  // Bytes sent

    ngx_log_t *log;              // Connection-specific log

    ngx_pool_t *pool;            // Memory pool for this connection

    // SSL/TLS specific
    ngx_ssl_connection_t *ssl;

    // Buffering
    ngx_buf_t *buffer;

    // ... 30+ more fields for timeouts, flags, peer info, etc.
};
\end{lstlisting}

\textbf{How It's Used:}

\begin{lstlisting}
// HTTP module handling a request
static void
ngx_http_process_request(ngx_http_request_t *r)
{
    ngx_connection_t *c = r->connection;  // Get connection

    // Read from connection (opaque call)
    n = c->recv(c, buffer, size);

    // Check connection state
    if (c->read->timedout) {
        // Handle timeout
        ngx_http_close_request(r, NGX_HTTP_REQUEST_TIME_OUT);
        return;
    }

    // Write response
    c->send(c, response, length);

    // Log to connection-specific logger
    ngx_log_error(NGX_LOG_INFO, c->log, 0, "Request processed");
}
\end{lstlisting}

\textbf{Why This Is Brilliant:}

\begin{enumerate}
    \item \textbf{Polymorphic I/O}: The \texttt{recv} and \texttt{send} fields are function pointers! For regular TCP, they point to \texttt{ngx\_unix\_recv} and \texttt{ngx\_unix\_send}. For SSL connections, they point to \texttt{ngx\_ssl\_recv} and \texttt{ngx\_ssl\_send}. Same opaque connection type handles both!

    \item \textbf{Memory pool per connection}: Each connection has its own memory pool. When connection closes, one call frees all memory allocated for that connection. No leaks!

    \item \textbf{Extensible}: New connection types (HTTP/2, QUIC) can be added by creating new implementations. User code doesn't change.

    \item \textbf{Performance}: By keeping all connection state in one structure, CPU cache is happy (good locality of reference).
\end{enumerate}

\textbf{The Opaque Magic:} HTTP modules don't need to know if they're dealing with HTTP/1.1, HTTP/2, SSL, or plain TCP. They just call \texttt{c->recv()} and \texttt{c->send()}. NGINX's core sets up the right function pointers based on connection type.

\subsection{Example 5: Git - Object Database}

\textbf{What is Git?} The version control system that revolutionized software development. Every object (commit, tree, blob, tag) is stored in an object database.

\textbf{The Pattern:} Git uses opaque object IDs (SHA-1 hashes) as handles.

\begin{lstlisting}
// In Git's object.h:
struct object_id {
    unsigned char hash[GIT_MAX_RAWSZ];  // 20 bytes for SHA-1
};

// Opaque structure - details hidden
struct object {
    unsigned parsed : 1;
    unsigned type : 3;     // commit, tree, blob, or tag
    unsigned flags : 28;
    struct object_id oid;  // The hash
};

// Public API - uses opaque pointers:
struct object *parse_object(const struct object_id *oid);
struct commit *lookup_commit(const struct object_id *oid);
struct tree *lookup_tree(const struct object_id *oid);
\end{lstlisting}

\textbf{How It Works:}

\begin{lstlisting}
// Real Git code (simplified):

// User has a commit hash (like "a3b5c...")
struct object_id oid;
get_oid_hex("a3b5c...", &oid);  // Parse hex to binary

// Look up the commit - Git finds it in object database
struct commit *commit = lookup_commit(&oid);

// Git reads from .git/objects/a3/b5c...
// Parses the object
// Returns a commit struct

// Access commit data
struct commit_list *parents = commit->parents;
struct tree *tree = commit->tree;
const char *message = commit->buffer;

// But users never see how objects are stored on disk!
// Could be loose objects, packed objects, or network protocol
\end{lstlisting}

\textbf{Why This Is Brilliant:}

\begin{enumerate}
    \item \textbf{Content-addressable}: The hash IS the pointer! If you have the hash, you can find the object. No memory addresses needed.

    \item \textbf{Immutable}: Once created, objects never change (content defines hash). This makes caching and sharing trivial.

    \item \textbf{Multiple storage}: Git can store objects as:
    \begin{itemize}
        \item Loose files (.git/objects/XX/XXXXXX...)
        \item Packed files (.git/objects/pack/pack-*.pack)
        \item Network protocol (git:// or https://)
    \end{itemize}
    User code doesn't care! It just calls \texttt{lookup\_commit()} with a hash.

    \item \textbf{Deduplication}: Identical content = same hash = same object. Automatic deduplication across entire repository history!
\end{enumerate}

\textbf{The Opaque Magic:} When you do \texttt{git checkout}, Git looks up commits and trees by hash. It doesn't care if those objects are in a pack file, loose files, or being fetched from a remote server. The opaque \texttt{struct object} pointer works the same way for all cases.

\section{Production Gotcha: Incomplete Types and Linking}

Here's something that bites beginners: the incomplete type trick only works because of separate compilation.

\begin{lstlisting}
// mylib.h - Header file
typedef struct MyObject MyObject;  // Incomplete type

// mylib.c - Implementation file
struct MyObject {  // Complete type definition
    int data;
};

// When you compile user.c:
// - Compiler sees: typedef struct MyObject MyObject;
// - Compiler knows: MyObject exists, can point to it
// - Compiler doesn't know: size, layout, members
// - sizeof(MyObject) = ERROR: incomplete type
// - MyObject* ptr = OK: pointer to incomplete type

// When you LINK:
// - Linker connects myobject_create() call to implementation
// - Linker doesn't care about struct layout
// - Only function symbols need to match

// This is why you can ship:
// - mylib.h (header with forward declaration)
// - libmylib.so (compiled code with full definition)
// Users compile against .h, link against .so
// They never see the struct definition!
\end{lstlisting}

\section{Advanced: Symbol Visibility and Shared Libraries}

Professional libraries use symbol visibility to control what users can see:

\begin{lstlisting}
// In your header (public API)
#ifdef _WIN32
    #ifdef MYLIB_EXPORTS
        #define MYLIB_API __declspec(dllexport)
    #else
        #define MYLIB_API __declspec(dllimport)
    #endif
#else
    #define MYLIB_API __attribute__((visibility("default")))
#endif

// Public API - exported
MYLIB_API MyObject* myobject_create(void);
MYLIB_API void myobject_destroy(MyObject* obj);

// In your implementation file
// Private helper - NOT exported
__attribute__((visibility("hidden")))
static void internal_helper(MyObject* obj) {
    // This function doesn't appear in the shared library's
    // symbol table. Users can't accidentally call it.
}

// Check exported symbols:
// $ nm -D libmylib.so | grep " T "
// Only sees: myobject_create, myobject_destroy
// Doesn't see: internal_helper, struct definition
\end{lstlisting}

\begin{warningbox}
By default, GCC exports ALL symbols from a shared library. Use \texttt{-fvisibility=hidden} and mark only public API as visible. This reduces symbol table size, speeds up dynamic linking, and prevents symbol conflicts! Think of it as not airing your dirty laundry in public---keep your internal functions internal.
\end{warningbox}

\section{Common Mistakes and How to Avoid Them}

\subsection{Mistake 1: Forgetting NULL Checks}

\begin{lstlisting}
// BAD - crashes on NULL
void myobject_set_value(MyObject* obj, int value) {
    obj->value = value;  // SEGFAULT if obj is NULL!
}

// GOOD - defensive programming
void myobject_set_value(MyObject* obj, int value) {
    if (!obj) return;  // or assert(obj != NULL);
    obj->value = value;
}

// BETTER - return error code
int myobject_set_value(MyObject* obj, int value) {
    if (!obj) return -1;
    obj->value = value;
    return 0;
}

// In production code, NULL pointer crashes are the #1 bug
// Always validate opaque pointers at function entry
// (Your 3 AM self will thank your current self)
// (Also your users will send fewer angry emails)
\end{lstlisting}

\subsection{Mistake 2: Double-Free Bugs}

\begin{lstlisting}
// Dangerous pattern:
MyObject* obj = myobject_create();
myobject_destroy(obj);
myobject_destroy(obj);  // Double free! Undefined behavior!
                        // This is how memory corruption parties start

// Solution 1: Set to NULL after free
void myobject_destroy(MyObject** obj_ptr) {
    if (obj_ptr && *obj_ptr) {
        free(*obj_ptr);
        *obj_ptr = NULL;  // Prevent double-free
    }
}

// Usage:
MyObject* obj = myobject_create();
myobject_destroy(&obj);  // obj becomes NULL
myobject_destroy(&obj);  // Safe - does nothing

// Solution 2: Reference counting (like COM, Python)
MyObject* myobject_retain(MyObject* obj) {
    if (obj) obj->ref_count++;
    return obj;
}

void myobject_release(MyObject* obj) {
    if (obj && --obj->ref_count == 0) {
        // Actually free when ref count reaches 0
        free(obj);
    }
}
\end{lstlisting}

\subsection{Mistake 3: Memory Leaks from Exception Paths}

\begin{lstlisting}
// BAD - leaks on error
MyObject* create_and_init(const char* config) {
    MyObject* obj = myobject_create();

    if (!load_config(config)) {
        return NULL;  // LEAK! obj is never freed
    }

    return obj;
}

// GOOD - cleanup on all error paths
MyObject* create_and_init(const char* config) {
    MyObject* obj = myobject_create();
    if (!obj) return NULL;

    if (!load_config(config)) {
        myobject_destroy(obj);  // Clean up!
        return NULL;
    }

    return obj;
}

// BETTER - use goto cleanup pattern (Linux kernel style)
MyObject* create_and_init(const char* config) {
    MyObject* obj = NULL;
    char* buffer = NULL;
    FILE* f = NULL;

    obj = myobject_create();
    if (!obj) goto cleanup;

    buffer = malloc(1024);
    if (!buffer) goto cleanup;

    f = fopen(config, "r");
    if (!f) goto cleanup;

    // ... do work ...

    // Success path
    fclose(f);
    free(buffer);
    return obj;

cleanup:
    // Error path - cleanup in reverse order
    if (f) fclose(f);
    free(buffer);
    myobject_destroy(obj);
    return NULL;
}
\end{lstlisting}

\section{Advanced: Multiple Implementations}

One powerful use of opaque pointers is supporting multiple backends:

\begin{lstlisting}
// Public header - same for all implementations
typedef struct Database Database;

Database* database_create(const char* type);
int database_query(Database* db, const char* sql);
void database_close(Database* db);

// Implementation 1: SQLite (database_sqlite.c)
#include <sqlite3.h>

struct Database {
    const char* type;  // "sqlite"
    sqlite3* handle;
    // SQLite-specific fields
};

// Implementation 2: PostgreSQL (database_postgres.c)
#include <libpq-fe.h>

struct Database {
    const char* type;  // "postgres"
    PGconn* conn;
    // Postgres-specific fields
};

// Factory function chooses implementation at runtime
Database* database_create(const char* type) {
    if (strcmp(type, "sqlite") == 0) {
        return create_sqlite_database();
    } else if (strcmp(type, "postgres") == 0) {
        return create_postgres_database();
    }
    return NULL;
}

// The query function can dispatch based on type:
int database_query(Database* db, const char* sql) {
    if (!db || !sql) return -1;

    if (strcmp(db->type, "sqlite") == 0) {
        return sqlite_do_query(db, sql);
    } else if (strcmp(db->type, "postgres") == 0) {
        return postgres_do_query(db, sql);
    }

    return -1;
}

// Real-world example: OpenSSL uses this for crypto engines
// Same API, different implementations (hardware, software, etc.)
\end{lstlisting}

\section{Pro Pattern: VTable for Polymorphism}

Here's how professionals implement true polymorphism in C:

\begin{lstlisting}
// Function pointer table (VTable)
typedef struct {
    int (*query)(void* self, const char* sql);
    void (*close)(void* self);
    const char* (*get_error)(void* self);
} DatabaseVTable;

// Base "class"
struct Database {
    DatabaseVTable* vtable;  // First member!
    void* impl;  // Implementation-specific data
};

// SQLite implementation
typedef struct {
    sqlite3* handle;
    char last_error[256];
} SQLiteImpl;

int sqlite_query(void* self, const char* sql) {
    Database* db = (Database*)self;
    SQLiteImpl* impl = (SQLiteImpl*)db->impl;
    // Use impl->handle...
    return 0;
}

void sqlite_close(void* self) {
    Database* db = (Database*)self;
    SQLiteImpl* impl = (SQLiteImpl*)db->impl;
    sqlite3_close(impl->handle);
    free(impl);
    free(db);
}

const char* sqlite_get_error(void* self) {
    Database* db = (Database*)self;
    SQLiteImpl* impl = (SQLiteImpl*)db->impl;
    return impl->last_error;
}

// VTable instance
static DatabaseVTable sqlite_vtable = {
    .query = sqlite_query,
    .close = sqlite_close,
    .get_error = sqlite_get_error
};

// Constructor
Database* database_create_sqlite(const char* path) {
    Database* db = malloc(sizeof(Database));
    SQLiteImpl* impl = malloc(sizeof(SQLiteImpl));

    if (!db || !impl) {
        free(db);
        free(impl);
        return NULL;
    }

    db->vtable = &sqlite_vtable;
    db->impl = impl;

    sqlite3_open(path, &impl->handle);

    return db;
}

// Polymorphic call - works for ANY database type!
int database_query(Database* db, const char* sql) {
    if (!db || !db->vtable || !db->vtable->query) {
        return -1;
    }
    return db->vtable->query(db, sql);
}

// This is EXACTLY how GObject (GTK) works!
// Also similar to COM objects in Windows
// And C++ virtual functions under the hood
// Turns out, we were doing OOP before it was cool
\end{lstlisting}

\begin{tipbox}
The VTable must be the FIRST member of the struct. This allows safe casting between base and derived types. C guarantees that a pointer to a struct points to its first member!
\end{tipbox}

\section{Platform-Specific Considerations}

\subsection{Windows DLL Export/Import}

\begin{lstlisting}
// mylib.h
#ifdef _WIN32
    #ifdef BUILDING_MYLIB
        #define MYLIB_API __declspec(dllexport)
    #else
        #define MYLIB_API __declspec(dllimport)
    #endif
#else
    #define MYLIB_API
#endif

// Mark all public functions
MYLIB_API MyObject* myobject_create(void);
MYLIB_API void myobject_destroy(MyObject* obj);

// When building the DLL:
// cl /DBUILDING_MYLIB /LD mylib.c

// When using the DLL:
// cl user.c mylib.lib
\end{lstlisting}

\subsection{Structure Packing and Alignment}

\begin{lstlisting}
// On 64-bit systems, this struct is 24 bytes:
struct MyObject {
    int value;        // 4 bytes
    // 4 bytes padding for alignment
    char* name;       // 8 bytes
    int flags;        // 4 bytes
    // 4 bytes padding at end
};

// Reorder for better packing (16 bytes):
struct MyObject {
    char* name;       // 8 bytes
    int value;        // 4 bytes
    int flags;        // 4 bytes
};

// For network protocols, force packing:
#pragma pack(push, 1)
struct NetworkPacket {
    uint32_t magic;   // 4 bytes, no padding
    uint16_t version; // 2 bytes, no padding
    uint8_t type;     // 1 byte, no padding
};
#pragma pack(pop)

// But users never see this because it's OPAQUE!
// You can reorganize for performance anytime
\end{lstlisting}

\section{Memory Debugging with Opaque Types}

\begin{lstlisting}
// Add magic numbers for debugging
#define MYOBJECT_MAGIC 0xDEADBEEF

struct MyObject {
    uint32_t magic;  // First member
    int value;
    char* name;
    // ... rest of struct
};

MyObject* myobject_create(void) {
    MyObject* obj = malloc(sizeof(MyObject));
    if (obj) {
        obj->magic = MYOBJECT_MAGIC;
        obj->value = 0;
        obj->name = NULL;
    }
    return obj;
}

// Validate pointer in every function
static inline int myobject_is_valid(const MyObject* obj) {
    return obj && obj->magic == MYOBJECT_MAGIC;
}

void myobject_destroy(MyObject* obj) {
    if (!myobject_is_valid(obj)) {
        fprintf(stderr, "ERROR: Invalid MyObject pointer!\n");
        abort();  // Crash immediately in debug builds
    }

    obj->magic = 0;  // Clear magic before freeing
    free(obj->name);
    free(obj);
}

// Catches:
// - NULL pointers (the classic)
// - Freed objects (magic is cleared)
// - Random garbage pointers (someone's having a bad day)
// - Wrong type pointers (someone passed us a Cat when we wanted a Dog)

// Valgrind and AddressSanitizer love this pattern!
// (And so will you, when it saves you from a 6-hour debugging session)
\end{lstlisting}

\section{When NOT to Use Opaque Pointers}

Opaque pointers aren't always the answer. Sometimes they're overkill, like using a sledgehammer to crack a peanut:

\begin{itemize}
    \item \textbf{POD types}: Simple structs like \texttt{Point\{int x, y;\}} don't need hiding
    \item \textbf{Performance-critical tight loops}: Extra indirection costs CPU cycles
    \item \textbf{Stack allocation needed}: Opaque types must be heap-allocated
    \item \textbf{Embedded systems}: Limited heap, prefer stack allocation
    \item \textbf{Header-only libraries}: Convenience over encapsulation
    \item \textbf{Internal-only code}: No need for ABI stability
\end{itemize}

\begin{lstlisting}
// Good use: Public API, needs ABI stability
typedef struct Database Database;
Database* db_open(const char* path);

// Bad use: Simple 2D point
typedef struct Point Point;
Point* point_create(int x, int y);
// Just use: struct Point { int x, y; };
// Don't be that person who over-engineers everything

// Performance example:
// BAD - extra indirection in tight loop
for (int i = 0; i < 1000000; i++) {
    int x = point_get_x(points[i]);  // Function call overhead
    int y = point_get_y(points[i]);
    process(x, y);
}

// GOOD - direct access
for (int i = 0; i < 1000000; i++) {
    process(points[i].x, points[i].y);  // Inline, fast
}
\end{lstlisting}

\section{Real Production Example: OpenSSL - The 25-Year Evolution}

Let's examine how OpenSSL uses opaque pointers to maintain binary compatibility across decades. This is the ultimate test of the pattern.

\subsection{Example 6: OpenSSL - SSL/TLS Connections}

\textbf{What is OpenSSL?} The cryptographic library that secures most of the internet. Used by Apache, NGINX, Node.js, Python, Ruby, and countless other tools. If you've ever seen "https://" in a URL, OpenSSL (or a fork like BoringSSL) was probably involved.

\textbf{The Challenge:} OpenSSL was first released in 1998. It needs to support new TLS versions (TLS 1.3), new cipher suites, new features, while still working with programs compiled years ago.

\textbf{The Pattern:} Everything is opaque! \texttt{SSL}, \texttt{SSL\_CTX}, \texttt{BIO}, \texttt{X509}, etc.

\begin{lstlisting}
// In openssl/ssl.h (public header) - What users see:
typedef struct ssl_st SSL;                // Opaque!
typedef struct ssl_ctx_st SSL_CTX;        // Opaque!
typedef struct ssl_method_st SSL_METHOD;  // Opaque!

// Create SSL context (shared settings for multiple connections)
SSL_CTX* SSL_CTX_new(const SSL_METHOD* method);

// Create SSL connection object
SSL* SSL_new(SSL_CTX* ctx);

// Perform handshake
int SSL_connect(SSL* ssl);
int SSL_accept(SSL* ssl);

// Read/write encrypted data
int SSL_read(SSL* ssl, void* buf, int num);
int SSL_write(SSL* ssl, const void* buf, int num);

// Clean up
void SSL_free(SSL* ssl);
void SSL_CTX_free(SSL_CTX* ctx);

// Query functions (no direct field access!)
int SSL_version(const SSL* ssl);
const SSL_CIPHER* SSL_get_current_cipher(const SSL* ssl);
long SSL_get_verify_result(const SSL* ssl);
\end{lstlisting}

\textbf{The Secret (ssl/ssl\_local.h):} The actual structures are MASSIVE and constantly evolving.

\begin{lstlisting}
// Simplified version of the real structure:
struct ssl_st {
    // Protocol version
    int version;     // SSL 3.0, TLS 1.0, 1.1, 1.2, 1.3...

    // Method pointers (polymorphism!)
    const SSL_METHOD* method;

    // I/O abstractions
    BIO* rbio;       // Read BIO (could be socket, memory, filter...)
    BIO* wbio;       // Write BIO

    // Session information
    SSL_SESSION* session;

    // Cipher information
    STACK_OF(SSL_CIPHER)* cipher_list;
    const SSL_CIPHER* s3->tmp.new_cipher;

    // Handshake state machine
    OSSL_STATEM statem;  // State machine for handshake

    // Buffers
    struct {
        unsigned char* buf;
        size_t len;
    } s3->rrec;  // Read record

    // Security parameters
    int verify_mode;
    int (*verify_callback)(int, X509_STORE_CTX*);

    // Extensions
    TLSEXT_TYPE* extensions;

    // Threading
    CRYPTO_RWLOCK* lock;

    // ... 190+ more fields for:
    // - Certificate chains
    // - Session tickets
    // - ALPN/NPN negotiation
    // - Renegotiation state
    // - Heartbeat
    // - DTLS specifics
    // - Custom extensions
    // - Statistics
    // - Debugging info
    // And more!
};
\end{lstlisting}

\textbf{Real-World Usage Example:}

\begin{lstlisting}
// Example: HTTPS client (simplified)

// Initialize OpenSSL
SSL_library_init();
SSL_load_error_strings();

// Create SSL context (TLS 1.2 or higher)
SSL_CTX* ctx = SSL_CTX_new(TLS_client_method());

// Load CA certificates for verification
SSL_CTX_load_verify_locations(ctx, "/etc/ssl/certs/ca-certificates.crt", NULL);

// Connect to server
int sock = socket_connect("www.example.com", 443);

// Create SSL connection object
SSL* ssl = SSL_new(ctx);
SSL_set_fd(ssl, sock);  // Attach to socket

// Perform TLS handshake
if (SSL_connect(ssl) != 1) {
    ERR_print_errors_fp(stderr);
    exit(1);
}

// Verify certificate
if (SSL_get_verify_result(ssl) != X509_V_OK) {
    printf("Certificate verification failed!\n");
    exit(1);
}

// Send HTTPS request
const char* request = "GET / HTTP/1.1\r\nHost: www.example.com\r\n\r\n";
SSL_write(ssl, request, strlen(request));

// Read response
char buffer[4096];
int bytes = SSL_read(ssl, buffer, sizeof(buffer) - 1);
buffer[bytes] = '\0';
printf("Response: %s\n", buffer);

// Clean up
SSL_shutdown(ssl);
SSL_free(ssl);
close(sock);
SSL_CTX_free(ctx);
\end{lstlisting}

\textbf{Why This Is Brilliant:}

\begin{enumerate}
    \item \textbf{25 years of evolution}: The \texttt{SSL} structure has grown from ~50 fields in 1998 to over 200 fields today. Programs compiled against OpenSSL 0.9.6 (year 2000) still work with OpenSSL 3.0 (year 2021)!

    \item \textbf{Multiple protocol versions}: Same \texttt{SSL*} type handles SSL 3.0 (obsolete), TLS 1.0-1.3, and DTLS. The \texttt{method} field's function pointers implement protocol-specific behavior.

    \item \textbf{Abstraction layers}: The \texttt{BIO} (Basic I/O) abstraction means SSL can work over:
    \begin{itemize}
        \item TCP sockets
        \item UDP (for DTLS)
        \item Memory buffers (for testing)
        \item Filters (compression, encryption)
        \item Custom transports (QUIC, SCTP)
    \end{itemize}
    User code doesn't change! Just swap the BIO.

    \item \textbf{Security updates without recompilation}: When Heartbleed (2014) and other bugs were found, OpenSSL fixed them in the internal structures. Applications didn't need recompilation---just link against the new library.

    \item \textbf{New features transparently}: TLS 1.3 added in 2018. Programs written in 2010 can use TLS 1.3 by just updating the OpenSSL library, no code changes needed!
\end{enumerate}

\textbf{The Opaque Magic at Scale:}

Let's see how deep the abstraction goes:

\begin{lstlisting}
// Everything is opaque all the way down!

SSL*              // Opaque connection
 +-> SSL_CTX*          // Opaque context (shared settings)
 +-> SSL_SESSION*      // Opaque session (for resumption)
 +-> SSL_METHOD*       // Opaque method table (protocol impl)
 +-> BIO*              // Opaque I/O (network, memory, filter)
 |    +-> BIO_METHOD*       // Opaque I/O methods
 +-> X509*             // Opaque certificate
 |    +-> X509_NAME*        // Opaque subject/issuer
 |    +-> EVP_PKEY*         // Opaque public key
 |    +-> X509_STORE*       // Opaque cert store
 +-> SSL_CIPHER*       // Opaque cipher info
 +-> STACK_OF(X509)*   // Opaque stack (dynamic array)

// Users NEVER see the internals of any of these!
// All access is through functions
\end{lstlisting}

\textbf{Real-world Impact - The Heartbleed Example:}

In 2014, the Heartbleed bug was discovered in OpenSSL. The bug was in the internal heartbeat handling code. Here's what happened:

\begin{lstlisting}
// BEFORE Heartbleed fix (vulnerable):
// Inside ssl/d1_both.c (users never see this file)
int dtls1_process_heartbeat(SSL *s) {
    unsigned char *p = &s->s3->rrec.data[0];
    unsigned short hbtype;
    unsigned int payload;

    hbtype = *p++;
    n2s(p, payload);  // Read payload length from attacker

    // BUG! No validation of payload length!
    // Attacker could claim 64KB even if actual data is 1 byte

    memcpy(buffer, p, payload);  // Read beyond bounds!
    send_heartbeat_response(s, buffer, payload);
}

// AFTER Heartbleed fix:
int dtls1_process_heartbeat(SSL *s) {
    unsigned char *p = &s->s3->rrec.data[0];
    unsigned short hbtype;
    unsigned int payload, actual_length;

    hbtype = *p++;
    n2s(p, payload);

    // FIX: Validate payload length!
    actual_length = s->s3->rrec.length - 3;
    if (payload > actual_length) {
        // Attacker lying about length - reject!
        return 0;
    }

    memcpy(buffer, p, payload);  // Now safe!
    send_heartbeat_response(s, buffer, payload);
}
\end{lstlisting}

\textbf{The fix was internal.} Because \texttt{SSL} is opaque:

\begin{itemize}
    \item Users didn't need to recompile their applications
    \item Just update OpenSSL library (apt-get upgrade, yum update)
    \item All applications immediately protected
    \item No source code changes needed
    \item No ABI breakage
\end{itemize}

If \texttt{struct ssl\_st} was exposed in headers, users might have been directly accessing \texttt{s->s3->rrec.data}. Changing the internal layout would break their code!

\textbf{Lessons from OpenSSL:}

\begin{enumerate}
    \item \textbf{Opaque pointers enable security updates}: You can fix bugs in internal code without breaking users

    \item \textbf{Opaque pointers enable evolution}: Add TLS 1.3 support without changing the API

    \item \textbf{Accessor functions are mandatory}: Every query must go through a function (\texttt{SSL\_version}, \texttt{SSL\_get\_current\_cipher}, etc.)

    \item \textbf{Documentation is critical}: With 200+ internal fields, good documentation is the only way users know what's available

    \item \textbf{Versioning matters}: OpenSSL has \texttt{OPENSSL\_VERSION\_NUMBER} so code can adapt to different versions when needed
\end{enumerate}

\textbf{The Bottom Line:}

OpenSSL protects billions of connections per day. It's evolved from SSL 2.0 to TLS 1.3, from 56-bit DES to 256-bit AES, from RSA to elliptic curves, from being a few thousand lines to over 500,000 lines of code. Through all this, the same simple opaque API has remained:

\begin{lstlisting}
SSL* ssl = SSL_new(ctx);
SSL_connect(ssl);
SSL_write(ssl, data, len);
SSL_read(ssl, buffer, size);
SSL_free(ssl);
\end{lstlisting}

That's the power of opaque pointers. 25 years of evolution, billions of users, and the API still looks almost the same as it did in 1998.

\section{Best Practices from 20+ Years of C}

\begin{enumerate}
    \item \textbf{Always validate}: Check for NULL, check magic numbers (paranoia is a feature, not a bug)
    \item \textbf{Document ownership}: Who allocates? Who frees? (Avoid the "I thought YOU were freeing it" conversation)
    \item \textbf{Const correctness}: \texttt{const MyObject*} for read-only operations
    \item \textbf{Error handling}: Return status codes, set errno
    \item \textbf{Thread safety}: Document if functions are thread-safe (your users will ask at 2 AM)
    \item \textbf{Naming convention}: \texttt{prefix\_typename\_operation} (e.g., \texttt{mylib\_object\_create})
    \item \textbf{Include guards}: Always use header guards (learned this one the hard way, didn't we?)
    \item \textbf{Versioning}: Consider version numbers in struct for future compatibility
    \item \textbf{Testing}: Mock implementations for unit testing
    \item \textbf{Documentation}: Document lifetime, ownership, thread-safety
\end{enumerate}

\section{Advanced Pattern: Handle Tables (The ID System)}

Here's a pattern used by game engines, database systems, and operating systems that textbooks never mention: instead of returning raw pointers, return integer handles that index into a table. This is the "we don't trust you with real pointers" pattern, and honestly, that's probably wise.

\subsection*{The Hotel Room Analogy}

Imagine you're running a hotel with 1000 rooms:

\textbf{Bad approach (raw pointers):}
\begin{itemize}
    \item Give guests the actual physical address of their room: "123 Main St, Room 5, Bed \#2"
    \item If you renovate and move things around, all those addresses become invalid
    \item Guests could find rooms they're not supposed to access
    \item If a guest leaves but keeps their address, they might barge into the new guest's room! (Awkward.)
\end{itemize}

\textbf{Good approach (handle tables):}
\begin{itemize}
    \item Give guests a room number: "Room 42"
    \item You maintain a registry: Room 42 -> currently occupied by Guest Smith
    \item When Guest Smith checks out, you mark Room 42 as "vacant"
    \item If someone tries to use an old room number, you check your registry: "Sorry, that room is vacant now"
    \item You can move guests to different rooms without changing their room number (update your registry)
    \item Room numbers are easy to write down, remember, and verify
\end{itemize}

That's exactly what handle tables do! Instead of giving users raw memory addresses (pointers), you give them IDs (handles) that you can validate and control.

\subsection{Why Handle Tables?}

Let's break down each benefit with examples:

\begin{enumerate}
    \item \textbf{Detect stale pointers}: Can validate if handle is still valid
    \begin{itemize}
        \item Problem: User calls \texttt{destroy()}, but keeps using the pointer -> crash!
        \item Solution: Handle becomes invalid after destroy. Next use returns error instead of crashing
        \item Like: Hotel room key card stops working after checkout
    \end{itemize}

    \item \textbf{Stable references}: Handles don't change even if object moves in memory
    \begin{itemize}
        \item Problem: If you use \texttt{realloc()} to grow an array, all pointers into it become invalid
        \item Solution: Handles remain the same; you just update the table
        \item Like: Your hotel room number stays "42" even if the hotel renovates
    \end{itemize}

    \item \textbf{Serialization}: Integers are easier to save/load than pointers
    \begin{itemize}
        \item Problem: Can't save a pointer to disk (it's meaningless when you restart)
        \item Solution: Save the handle (just an integer). When loading, look it up in the table
        \item Like: Saving "Room 42" in your reservation vs saving "Third floor, second hallway, left door"
    \end{itemize}

    \item \textbf{Cross-process}: Can share handles between processes
    \begin{itemize}
        \item Problem: Pointers are only valid in one process's memory space
        \item Solution: Multiple processes can agree on handle meanings
        \item Like: Two hotels owned by the same company use the same room numbering system
    \end{itemize}

    \item \textbf{Security}: Prevents pointer arithmetic attacks
    \begin{itemize}
        \item Problem: Malicious user does \texttt{ptr + 10} to access wrong object
        \item Solution: Handles are just numbers; you can't do math to find other objects
        \item Like: Knowing Room 42 doesn't let you calculate how to access the hotel safe
    \end{itemize}

    \item \textbf{Generational IDs}: Detect use-after-free bugs
    \begin{itemize}
        \item Problem: Object is freed, new object is allocated at same address, old pointer accesses wrong object!
        \item Solution: Include a "generation" number in the handle. Each reuse increments the generation
        \item Like: Room 42, Reservation \#7 (vs Room 42, Reservation \#8). Same room, different stays
    \end{itemize}
\end{enumerate}

\begin{lstlisting}
// Handle format: [generation:16][index:16]
// Generation prevents handle reuse bugs
typedef uint32_t ObjectHandle;

#define INVALID_HANDLE 0
#define MAX_OBJECTS 4096

typedef struct {
    MyObject* object;      // Actual object pointer
    uint16_t generation;   // Incremented on free
    uint8_t is_active;     // Is this slot in use?
} ObjectSlot;

typedef struct {
    ObjectSlot slots[MAX_OBJECTS];
    uint32_t next_free;
} ObjectTable;

static ObjectTable g_table = {0};

// Pack handle from generation and index
static inline ObjectHandle make_handle(uint16_t gen, uint16_t idx) {
    return ((uint32_t)gen << 16) | idx;
}

// Unpack handle
static inline uint16_t handle_generation(ObjectHandle h) {
    return (uint16_t)(h >> 16);
}

static inline uint16_t handle_index(ObjectHandle h) {
    return (uint16_t)(h & 0xFFFF);
}

ObjectHandle myobject_create(void) {
    // Find free slot
    for (uint32_t i = 0; i < MAX_OBJECTS; i++) {
        if (!g_table.slots[i].is_active) {
            // Allocate object
            MyObject* obj = malloc(sizeof(MyObject));
            if (!obj) return INVALID_HANDLE;

            // Initialize object
            obj->value = 0;
            obj->name = NULL;

            // Setup slot
            g_table.slots[i].object = obj;
            g_table.slots[i].is_active = 1;

            // Return handle with current generation
            return make_handle(g_table.slots[i].generation, i);
        }
    }

    return INVALID_HANDLE;  // Table full
}

// Validate handle and get object
static MyObject* handle_to_object(ObjectHandle handle) {
    if (handle == INVALID_HANDLE) return NULL;

    uint16_t idx = handle_index(handle);
    uint16_t gen = handle_generation(handle);

    if (idx >= MAX_OBJECTS) return NULL;

    ObjectSlot* slot = &g_table.slots[idx];

    // Check if handle generation matches (detects use-after-free!)
    if (!slot->is_active || slot->generation != gen) {
        return NULL;  // Stale handle!
    }

    return slot->object;
}

void myobject_destroy(ObjectHandle handle) {
    MyObject* obj = handle_to_object(handle);
    if (!obj) return;

    uint16_t idx = handle_index(handle);
    ObjectSlot* slot = &g_table.slots[idx];

    // Free the object
    free(obj->name);
    free(obj);

    // Mark slot as free and increment generation
    slot->object = NULL;
    slot->is_active = 0;
    slot->generation++;  // Next handle for this slot will be different!
}

int myobject_set_value(ObjectHandle handle, int value) {
    MyObject* obj = handle_to_object(handle);
    if (!obj) return -1;  // Invalid handle

    obj->value = value;
    return 0;
}

// Example of use-after-free detection:
ObjectHandle h = myobject_create();  // gen=0, idx=5 -> handle=0x00000005
myobject_destroy(h);                 // Increments gen to 1

// Later, user tries to use old handle:
myobject_set_value(h, 42);           // FAILS! gen=0 but slot gen=1
// Returns -1 instead of crashing!

// This is how Unity game engine handles GameObjects
// And how Windows handles HWNDs (window handles)
\end{lstlisting}

\subsection*{Understanding Handle Tables - Step by Step}

Let's break down this code in simple terms:

\textbf{The Handle Format}

A handle is a 32-bit integer split into two parts:
\begin{itemize}
    \item \textbf{Upper 16 bits}: Generation number (0-65535)
    \item \textbf{Lower 16 bits}: Index into the array (0-65535)
\end{itemize}

Example: Handle value \texttt{0x00050042} means:
\begin{itemize}
    \item Generation 5 (upper 16 bits: 0x0005)
    \item Index 66 (lower 16 bits: 0x0042 = 66 in decimal)
    \item "This is the 5th time we've used slot 66"
\end{itemize}

Think of it like: "Room 66, Reservation \#5"

\textbf{Creating an Object}

When user calls \texttt{myobject\_create()}:

\begin{enumerate}
    \item \textbf{Find a free slot}: Loop through the table looking for \texttt{is\_active == 0}
    \begin{itemize}
        \item Like finding an empty hotel room
        \item If all slots are full, return INVALID\_HANDLE (hotel is fully booked!)
    \end{itemize}

    \item \textbf{Allocate the actual object}: Call malloc to get memory
    \begin{itemize}
        \item This is the actual data storage
        \item The pointer to this is stored in the slot
    \end{itemize}

    \item \textbf{Mark slot as active}: Set \texttt{is\_active = 1}
    \begin{itemize}
        \item "Room is now occupied"
    \end{itemize}

    \item \textbf{Create and return handle}: Combine generation + index
    \begin{itemize}
        \item Handle = (generation << 16) | index
        \item This is the "room number" we give to the user
    \end{itemize}
\end{enumerate}

\textbf{Using a Handle}

When user calls \texttt{myobject\_set\_value(handle, 42)}:

\begin{enumerate}
    \item \textbf{Extract index and generation from handle}:
    \begin{itemize}
        \item Index = lower 16 bits
        \item Generation = upper 16 bits
    \end{itemize}

    \item \textbf{Check if index is valid}: Is it less than MAX\_OBJECTS?
    \begin{itemize}
        \item Like: "Is room number in valid range?"
        \item If not, someone is trying to access room 10000 in a 4096-room hotel!
    \end{itemize}

    \item \textbf{Look up the slot}: Get slot from table
    \begin{itemize}
        \item Look in the hotel registry for that room number
    \end{itemize}

    \item \textbf{Validate generation}: Does handle generation match slot generation?
    \begin{itemize}
        \item This is the magic! If generations don't match, this is a stale handle
        \item Like: "You have a key for Reservation \#5, but we're on Reservation \#6 now"
        \item Someone already checked out and someone new checked in
    \end{itemize}

    \item \textbf{Check if slot is active}: Is someone currently using this slot?
    \begin{itemize}
        \item Even if generation matches, the slot might be vacant
    \end{itemize}

    \item \textbf{Return the object pointer if all checks pass}
    \begin{itemize}
        \item Only now do we give access to the actual object
        \item All those checks happened before touching any memory!
    \end{itemize}
\end{enumerate}

\textbf{Destroying an Object}

When user calls \texttt{myobject\_destroy(handle)}:

\begin{enumerate}
    \item \textbf{Validate and get object}: Use the validation above
    \begin{itemize}
        \item If handle is invalid, do nothing (safe!)
    \end{itemize}

    \item \textbf{Free the actual object}: Call free on the pointer
    \begin{itemize}
        \item Release the actual memory
    \end{itemize}

    \item \textbf{Mark slot as inactive}: Set \texttt{is\_active = 0}
    \begin{itemize}
        \item "Room is now vacant"
    \end{itemize}

    \item \textbf{Increment generation}: \texttt{generation++}
    \begin{itemize}
        \item This is crucial! Now any old handles become invalid
        \item Old handle has generation 5, but slot now has generation 6
        \item Next lookup will fail: "Your reservation \#5 is outdated"
    \end{itemize}
\end{enumerate}

\textbf{The Use-After-Free Detection Example}

\begin{lstlisting}
ObjectHandle h = myobject_create();  // Creates handle: gen=0, idx=5
// Handle value = 0x00000005

myobject_destroy(h);                 // Destroys object, generation becomes 1
// Slot 5 now has: generation=1, is_active=0

// User (wrongly) tries to use old handle:
myobject_set_value(h, 42);
// Extract: gen=0, idx=5 from handle
// Look up slot 5: generation=1, is_active=0
// Compare: handle gen (0) != slot gen (1)
// Result: FAIL! Return -1 instead of crashing!
\end{lstlisting}

Instead of a crash (accessing freed memory), user gets a clean error! This is how professional systems catch bugs early.

\begin{tipbox}
\textbf{Real-world usage:} Handle tables trade memory (fixed-size array) for safety and debuggability.

Who uses this?
\begin{itemize}
    \item \textbf{Unity game engine}: Every GameObject has a handle. When you destroy an object, all references become detectably invalid
    \item \textbf{Windows}: HWND (window handles), HANDLE (file handles). These are handles, not raw pointers!
    \item \textbf{Vulkan/DirectX}: Graphics APIs use handles for GPU resources
    \item \textbf{Databases}: Row IDs are handles into table storage
\end{itemize}

Debugging benefit: You can dump the entire handle table and see:
\begin{itemize}
    \item How many objects are alive (count active slots)
    \item Which slots have been reused most (high generation numbers)
    \item Memory leaks (slots that should be inactive but aren't)
    \item The actual objects for inspection
\end{itemize}

It's like having a complete hotel registry---you know exactly who's checked in, who's checked out, and if anyone's overstayed their welcome!
\end{tipbox}

\subsection{Optimization: Free List}

\begin{lstlisting}
// Instead of scanning for free slots, maintain a free list
typedef struct {
    ObjectSlot slots[MAX_OBJECTS];
    uint16_t free_list[MAX_OBJECTS];  // Indices of free slots
    uint32_t free_count;
} OptimizedObjectTable;

static OptimizedObjectTable g_table;

void init_object_table(void) {
    g_table.free_count = MAX_OBJECTS;
    for (uint32_t i = 0; i < MAX_OBJECTS; i++) {
        g_table.free_list[i] = i;
        g_table.slots[i].is_active = 0;
        g_table.slots[i].generation = 0;
    }
}

ObjectHandle myobject_create(void) {
    if (g_table.free_count == 0) {
        return INVALID_HANDLE;  // No free slots
    }

    // Pop from free list - O(1) instead of O(n) scan
    g_table.free_count--;
    uint16_t idx = g_table.free_list[g_table.free_count];

    MyObject* obj = malloc(sizeof(MyObject));
    if (!obj) {
        g_table.free_count++;  // Return slot to free list
        return INVALID_HANDLE;
    }

    obj->value = 0;
    obj->name = NULL;

    ObjectSlot* slot = &g_table.slots[idx];
    slot->object = obj;
    slot->is_active = 1;

    return make_handle(slot->generation, idx);
}

void myobject_destroy(ObjectHandle handle) {
    MyObject* obj = handle_to_object(handle);
    if (!obj) return;

    uint16_t idx = handle_index(handle);
    ObjectSlot* slot = &g_table.slots[idx];

    free(obj->name);
    free(obj);

    slot->object = NULL;
    slot->is_active = 0;
    slot->generation++;

    // Return to free list
    g_table.free_list[g_table.free_count] = idx;
    g_table.free_count++;
}

// Now creation/deletion is O(1) instead of O(n)!
// This is production-grade code
\end{lstlisting}

\section{Pattern: Capabilities and Permissions}

Professional APIs often need fine-grained access control. Here's how to implement it with opaque pointers:

\begin{lstlisting}
// Different capability levels
typedef enum {
    CAPABILITY_READ  = 1 << 0,  // 0x01
    CAPABILITY_WRITE = 1 << 1,  // 0x02
    CAPABILITY_EXEC  = 1 << 2,  // 0x04
    CAPABILITY_ADMIN = 1 << 3,  // 0x08
} Capability;

struct MyObject {
    int value;
    char* data;
    uint32_t capabilities;  // Bitmask of allowed operations
};

// Create object with specific capabilities
MyObject* myobject_create_with_caps(uint32_t caps) {
    MyObject* obj = malloc(sizeof(MyObject));
    if (obj) {
        obj->value = 0;
        obj->data = NULL;
        obj->capabilities = caps;
    }
    return obj;
}

// Check if operation is allowed
static inline int has_capability(const MyObject* obj, Capability cap) {
    return obj && (obj->capabilities & cap);
}

int myobject_get_value(const MyObject* obj) {
    if (!has_capability(obj, CAPABILITY_READ)) {
        errno = EACCES;  // Permission denied
        return -1;
    }
    return obj->value;
}

int myobject_set_value(MyObject* obj, int value) {
    if (!has_capability(obj, CAPABILITY_WRITE)) {
        errno = EACCES;
        return -1;
    }
    obj->value = value;
    return 0;
}

// Grant additional capability
int myobject_grant_cap(MyObject* obj, Capability cap) {
    if (!has_capability(obj, CAPABILITY_ADMIN)) {
        errno = EACCES;  // Only admins can grant capabilities
        return -1;
    }
    obj->capabilities |= cap;
    return 0;
}

// Revoke capability
int myobject_revoke_cap(MyObject* obj, Capability cap) {
    if (!has_capability(obj, CAPABILITY_ADMIN)) {
        errno = EACCES;
        return -1;
    }
    obj->capabilities &= ~cap;
    return 0;
}

// Usage example:
MyObject* obj = myobject_create_with_caps(
    CAPABILITY_READ | CAPABILITY_ADMIN
);

myobject_set_value(obj, 42);  // FAILS - no write capability

// Admin grants write
myobject_grant_cap(obj, CAPABILITY_WRITE);

myobject_set_value(obj, 42);  // SUCCESS - now has write

// This is how file descriptors work in UNIX!
// open(path, O_RDONLY) -> read capability only
// open(path, O_RDWR)   -> read + write capabilities
\end{lstlisting}

\section{Pattern: Copy-on-Write (COW) Optimization}

Here's a memory optimization pattern used by strings, arrays, and databases:

\begin{lstlisting}
struct MyObject {
    char* data;
    size_t length;
    size_t capacity;
    uint32_t* ref_count;  // Shared reference count
    uint8_t is_cow;       // Is this a COW reference?
};

// Create new object
MyObject* myobject_create(const char* str) {
    MyObject* obj = malloc(sizeof(MyObject));
    if (!obj) return NULL;

    obj->length = strlen(str);
    obj->capacity = obj->length + 1;
    obj->data = malloc(obj->capacity);
    obj->ref_count = malloc(sizeof(uint32_t));

    if (!obj->data || !obj->ref_count) {
        free(obj->data);
        free(obj->ref_count);
        free(obj);
        return NULL;
    }

    strcpy(obj->data, str);
    *obj->ref_count = 1;
    obj->is_cow = 0;

    return obj;
}

// Cheap copy - shares data buffer
MyObject* myobject_clone(const MyObject* src) {
    if (!src) return NULL;

    MyObject* obj = malloc(sizeof(MyObject));
    if (!obj) return NULL;

    // Share the data buffer!
    obj->data = src->data;
    obj->length = src->length;
    obj->capacity = src->capacity;
    obj->ref_count = src->ref_count;
    obj->is_cow = 1;

    // Increment reference count
    (*obj->ref_count)++;

    return obj;
}

// Make data buffer unique before modifying
static int cow_detach(MyObject* obj) {
    if (!obj->is_cow) return 0;  // Already unique

    if (*obj->ref_count > 1) {
        // Other objects still sharing - need to copy
        char* new_data = malloc(obj->capacity);
        if (!new_data) return -1;

        memcpy(new_data, obj->data, obj->length + 1);

        // Decrement shared ref count
        (*obj->ref_count)--;

        // Create new ref count
        obj->ref_count = malloc(sizeof(uint32_t));
        if (!obj->ref_count) {
            free(new_data);
            return -1;
        }

        *obj->ref_count = 1;
        obj->data = new_data;
    }

    obj->is_cow = 0;  // Now unique
    return 0;
}

// Modify - automatically detaches if needed
int myobject_append(MyObject* obj, const char* str) {
    if (!obj || !str) return -1;

    // Detach from shared buffer
    if (cow_detach(obj) != 0) return -1;

    size_t add_len = strlen(str);
    size_t new_len = obj->length + add_len;

    // Reallocate if needed
    if (new_len + 1 > obj->capacity) {
        size_t new_cap = (new_len + 1) * 2;
        char* new_data = realloc(obj->data, new_cap);
        if (!new_data) return -1;

        obj->data = new_data;
        obj->capacity = new_cap;
    }

    // Append
    strcpy(obj->data + obj->length, str);
    obj->length = new_len;

    return 0;
}

void myobject_destroy(MyObject* obj) {
    if (!obj) return;

    // Decrement ref count
    if (--(*obj->ref_count) == 0) {
        // Last reference - free shared resources
        free(obj->data);
        free(obj->ref_count);
    }

    free(obj);
}

// Example usage:
MyObject* obj1 = myobject_create("hello");  // Allocates buffer

MyObject* obj2 = myobject_clone(obj1);      // Shares buffer - CHEAP!
MyObject* obj3 = myobject_clone(obj1);      // Shares buffer - CHEAP!
// All three point to same "hello" buffer

myobject_append(obj2, " world");            // Detaches - obj2 gets own copy
// Now: obj1 and obj3 share "hello"
//      obj2 has unique "hello world"

// This is how Python strings work!
// And QString in Qt
// And std::string in some C++ implementations
\end{lstlisting}

\begin{tipbox}
COW is perfect for scenarios where most objects are read-only. You save memory by sharing, but pay the detach cost only when actually modifying. Redis uses this for string values!
\end{tipbox}

\section{Pattern: Object Pools (Custom Allocators)}

Professional code often implements custom memory allocators for performance:

\begin{lstlisting}
#define POOL_SIZE 1024

typedef struct {
    MyObject objects[POOL_SIZE];  // Pre-allocated objects
    uint32_t free_bitmap[POOL_SIZE / 32];  // Each bit = free/used
    uint32_t allocated_count;
} ObjectPool;

static ObjectPool g_pool = {0};

// Initialize pool
void myobject_pool_init(void) {
    // Mark all objects as free (bit = 1 means free)
    memset(g_pool.free_bitmap, 0xFF, sizeof(g_pool.free_bitmap));
    g_pool.allocated_count = 0;
}

// Find first free slot using bit operations
static int find_free_slot(void) {
    for (uint32_t i = 0; i < POOL_SIZE / 32; i++) {
        if (g_pool.free_bitmap[i] != 0) {
            // This word has free bits
            int bit = __builtin_ffs(g_pool.free_bitmap[i]) - 1;
            return i * 32 + bit;
        }
    }
    return -1;  // Pool full
}

// Allocate from pool - O(1) and NO malloc()!
MyObject* myobject_create_pooled(void) {
    int idx = find_free_slot();
    if (idx < 0) return NULL;

    // Mark as used
    uint32_t word = idx / 32;
    uint32_t bit = idx % 32;
    g_pool.free_bitmap[word] &= ~(1u << bit);

    g_pool.allocated_count++;

    // Initialize object
    MyObject* obj = &g_pool.objects[idx];
    obj->value = 0;
    obj->name = NULL;

    return obj;
}

// Free to pool - just mark bit, NO free()!
void myobject_destroy_pooled(MyObject* obj) {
    if (!obj) return;

    // Calculate index
    ptrdiff_t idx = obj - g_pool.objects;
    if (idx < 0 || idx >= POOL_SIZE) {
        fprintf(stderr, "ERROR: Object not from pool!\n");
        return;
    }

    // Clean up object
    free(obj->name);
    obj->name = NULL;

    // Mark as free
    uint32_t word = idx / 32;
    uint32_t bit = idx % 32;
    g_pool.free_bitmap[word] |= (1u << bit);

    g_pool.allocated_count--;
}

// Check if pool is from our pool
int myobject_is_pooled(const MyObject* obj) {
    ptrdiff_t idx = obj - g_pool.objects;
    return idx >= 0 && idx < POOL_SIZE;
}

// Get pool statistics
void myobject_pool_stats(void) {
    printf("Pool: %u/%u objects allocated (%.1f%% full)\n",
           g_pool.allocated_count,
           POOL_SIZE,
           100.0 * g_pool.allocated_count / POOL_SIZE);
}

// Benefits:
// - No malloc/free overhead - FAST!
// - No fragmentation
// - Cache-friendly (objects are contiguous)
// - Can iterate all objects easily
// - Easy to debug (dump entire pool)

// Drawbacks:
// - Fixed pool size
// - Wastes memory if pool is too large
// - All objects same size

// Used by:
// - Game engines (object pools everywhere!)
// - Network servers (connection pools)
// - Database systems (page pools)
// - Embedded systems (deterministic allocation)
\end{lstlisting}

\section{Pattern: Intrusive Reference Counting}

Here's how to implement automatic cleanup like COM objects or Objective-C:

\begin{lstlisting}
struct MyObject {
    uint32_t ref_count;     // Must be first for alignment
    int value;
    char* name;
    void (*destructor)(MyObject*);  // Custom cleanup
};

// Create with ref_count = 1
MyObject* myobject_create(void) {
    MyObject* obj = malloc(sizeof(MyObject));
    if (obj) {
        obj->ref_count = 1;  // Caller owns first reference
        obj->value = 0;
        obj->name = NULL;
        obj->destructor = NULL;
    }
    return obj;
}

// Retain - increment ref count
MyObject* myobject_retain(MyObject* obj) {
    if (obj) {
        obj->ref_count++;
    }
    return obj;  // Convenient for chaining
}

// Release - decrement ref count, free if zero
void myobject_release(MyObject* obj) {
    if (!obj) return;

    if (--obj->ref_count == 0) {
        // Call custom destructor if provided
        if (obj->destructor) {
            obj->destructor(obj);
        }

        // Free resources
        free(obj->name);
        free(obj);
    }
}

// Autorelease - release at scope exit (GCC/Clang)
#define myobject_autorelease(obj) \
    __attribute__((cleanup(myobject_release_cleanup))) obj

static inline void myobject_release_cleanup(MyObject** obj_ptr) {
    myobject_release(*obj_ptr);
}

// Usage with automatic cleanup:
void some_function(void) {
    MyObject* myobject_autorelease(obj) = myobject_create();

    // Use obj...

    // obj is automatically released when function returns!
    // Even if you return early or an error occurs
}

// Set custom destructor
void myobject_set_destructor(MyObject* obj, void (*destructor)(MyObject*)) {
    if (obj) {
        obj->destructor = destructor;
    }
}

// Example custom destructor
void custom_cleanup(MyObject* obj) {
    printf("Custom cleanup for object %p\n", (void*)obj);
    // Close files, network connections, etc.
}

// Practical example: shared ownership
void example_shared_ownership(void) {
    MyObject* obj = myobject_create();  // ref_count = 1

    // Thread 1 retains
    pass_to_thread1(myobject_retain(obj));  // ref_count = 2

    // Thread 2 retains
    pass_to_thread2(myobject_retain(obj));  // ref_count = 3

    // Original owner releases
    myobject_release(obj);  // ref_count = 2

    // Object still alive until both threads release!
    // Last thread to call release() will free the object
}

// This is EXACTLY how:
// - COM (Component Object Model) works in Windows
// - Objective-C ARC (Automatic Reference Counting)
// - GObject in GTK+
// - Python reference counting
// And now you understand the magic behind them all!
\end{lstlisting}

\section{Pattern: Weak References}

Sometimes you need to reference an object without keeping it alive:

\begin{lstlisting}
typedef struct WeakRef WeakRef;

struct MyObject {
    uint32_t ref_count;
    WeakRef* weak_refs;  // Linked list of weak references
    int value;
};

struct WeakRef {
    MyObject* target;    // NULL if target was destroyed
    WeakRef* next;       // Next weak ref in list
};

WeakRef* myobject_create_weak_ref(MyObject* obj) {
    if (!obj) return NULL;

    WeakRef* weak = malloc(sizeof(WeakRef));
    if (!weak) return NULL;

    weak->target = obj;

    // Add to object's weak ref list
    weak->next = obj->weak_refs;
    obj->weak_refs = weak;

    return weak;
}

// Try to get strong reference from weak reference
MyObject* weak_ref_lock(WeakRef* weak) {
    if (!weak || !weak->target) {
        return NULL;  // Target was destroyed
    }

    // Upgrade to strong reference
    return myobject_retain(weak->target);
}

void weak_ref_destroy(WeakRef* weak) {
    free(weak);
}

// When destroying object, NULL out all weak refs
void myobject_release(MyObject* obj) {
    if (!obj) return;

    if (--obj->ref_count == 0) {
        // NULL out all weak references
        WeakRef* weak = obj->weak_refs;
        while (weak) {
            weak->target = NULL;  // Now weak refs know object is gone
            weak = weak->next;
        }

        free(obj);
    }
}

// Usage:
MyObject* obj = myobject_create();
WeakRef* weak = myobject_create_weak_ref(obj);

// Later, try to access:
MyObject* strong = weak_ref_lock(weak);
if (strong) {
    // Object still alive!
    myobject_do_something(strong);
    myobject_release(strong);
} else {
    // Object was destroyed
    printf("Object no longer exists\n");
}

// This prevents circular reference problems:
// Parent -> Child (strong)
// Child -> Parent (weak)  // Won't prevent parent from being freed
\end{lstlisting}

\section{Pattern: Tagged Pointers}

On 64-bit systems, pointers only use 48 bits. We can use the unused bits for metadata:

\begin{lstlisting}
// On x86-64, pointers only use lower 48 bits
// Upper 16 bits are unused (sign-extended)
// Lower 3 bits are usually 0 due to alignment

typedef uintptr_t TaggedPtr;

// Tag in lower 3 bits (assumes 8-byte alignment)
#define TAG_MASK     0x7
#define PTR_MASK     (~TAG_MASK)

// Different object types
#define TAG_OBJECT   0
#define TAG_STRING   1
#define TAG_ARRAY    2
#define TAG_NUMBER   3

// Create tagged pointer
static inline TaggedPtr make_tagged(void* ptr, uint8_t tag) {
    uintptr_t addr = (uintptr_t)ptr;
    assert((addr & TAG_MASK) == 0);  // Must be aligned!
    return addr | (tag & TAG_MASK);
}

// Extract pointer
static inline void* tagged_ptr(TaggedPtr tagged) {
    return (void*)(tagged & PTR_MASK);
}

// Extract tag
static inline uint8_t tagged_tag(TaggedPtr tagged) {
    return tagged & TAG_MASK;
}

// Example: universal container
typedef struct {
    TaggedPtr data;  // Can hold different types
} Value;

Value value_create_object(MyObject* obj) {
    Value v;
    v.data = make_tagged(obj, TAG_OBJECT);
    return v;
}

Value value_create_string(char* str) {
    Value v;
    v.data = make_tagged(str, TAG_STRING);
    return v;
}

void value_print(Value v) {
    switch (tagged_tag(v.data)) {
        case TAG_OBJECT: {
            MyObject* obj = tagged_ptr(v.data);
            printf("Object: value=%d\n", obj->value);
            break;
        }
        case TAG_STRING: {
            char* str = tagged_ptr(v.data);
            printf("String: %s\n", str);
            break;
        }
        // ... other types
    }
}

// Save 8 bytes per value by embedding type in pointer!
// Used by:
// - JavaScript engines (V8, SpiderMonkey)
// - Lua interpreter
// - OCaml runtime
// - Many garbage collectors

// Alternative: Use upper bits for flags
#define FLAG_MARKED  (1ULL << 63)  // GC mark bit
#define FLAG_PINNED  (1ULL << 62)  // Can't move in memory
#define FLAG_SHARED  (1ULL << 61)  // Shared between threads

static inline TaggedPtr mark_object(TaggedPtr ptr) {
    return ptr | FLAG_MARKED;
}

static inline int is_marked(TaggedPtr ptr) {
    return (ptr & FLAG_MARKED) != 0;
}

static inline TaggedPtr clear_flags(TaggedPtr ptr) {
    return ptr & 0x0000FFFFFFFFFFFFULL;  // Clear upper 16 bits
}
\end{lstlisting}

\begin{warningbox}
Tagged pointers are architecture-specific and require careful alignment management. Not portable to 32-bit systems. Use with caution! (But they're incredibly powerful when you need them.)
\end{warningbox}

\section{Pattern: Intrusive Containers}

Instead of wrapping objects in container nodes, embed the links in the objects themselves:

\begin{lstlisting}
// Traditional approach (non-intrusive):
typedef struct Node {
    void* data;          // Separate allocation
    struct Node* next;
} Node;

// Intrusive approach - embed links in object
struct MyObject {
    int value;
    char* name;

    // Intrusive list links
    MyObject* next;
    MyObject* prev;
};

// List operations don't need malloc!
void list_append(MyObject** head, MyObject* obj) {
    obj->next = NULL;
    obj->prev = NULL;

    if (*head == NULL) {
        *head = obj;
    } else {
        MyObject* tail = *head;
        while (tail->next) {
            tail = tail->next;
        }
        tail->next = obj;
        obj->prev = tail;
    }
}

void list_remove(MyObject** head, MyObject* obj) {
    if (obj->prev) {
        obj->prev->next = obj->next;
    } else {
        *head = obj->next;  // Was head
    }

    if (obj->next) {
        obj->next->prev = obj->prev;
    }

    obj->next = NULL;
    obj->prev = NULL;
}

// Iterate - simple pointer chasing
void list_print(MyObject* head) {
    for (MyObject* obj = head; obj; obj = obj->next) {
        printf("Object: %d\n", obj->value);
    }
}

// Benefits:
// - No extra allocations
// - Better cache locality
// - Faster operations
// - Object can be in multiple lists!

// Multiple list support:
struct MyObject {
    int value;

    // Can be in two lists at once!
    struct {
        MyObject* next;
        MyObject* prev;
    } list1;

    struct {
        MyObject* next;
        MyObject* prev;
    } list2;
};

// This is EXACTLY how the Linux kernel list works!
// See: include/linux/list.h
// Container of is the magic macro that makes it work

#define container_of(ptr, type, member) \
    ((type *)((char *)(ptr) - offsetof(type, member)))

// Example: generic list node
typedef struct ListHead {
    struct ListHead* next;
    struct ListHead* prev;
} ListHead;

// Embed in your struct
struct MyObject {
    int value;
    ListHead list_node;  // Intrusive list node
};

// Get object from list node
#define list_entry(ptr, type, member) \
    container_of(ptr, type, member)

// Usage:
ListHead* node = /* some list node */;
MyObject* obj = list_entry(node, MyObject, list_node);
printf("Value: %d\n", obj->value);
\end{lstlisting}

\section{Summary}

The opaque pointer pattern is the cornerstone of professional C development:

\begin{itemize}
    \item Provides true encapsulation in C
    \item Enables ABI stability for shared libraries
    \item Allows multiple implementations behind single interface
    \item Prevents users from breaking invariants
    \item Reduces compilation dependencies
    \item Enables polymorphism via VTables
    \item Used by virtually every major C library
\end{itemize}

\textbf{Advanced patterns covered:}
\begin{itemize}
    \item Handle tables with generational IDs (use-after-free detection)
    \item Capability-based security (fine-grained permissions)
    \item Copy-on-write optimization (memory efficiency)
    \item Object pools (custom allocators for performance)
    \item Intrusive reference counting (automatic cleanup)
    \item Weak references (prevent circular dependencies)
    \item Tagged pointers (metadata in unused bits)
    \item Intrusive containers (Linux kernel style)
\end{itemize}

Master these patterns and you'll write C code that's maintainable, stable, and professional-grade. It's the difference between hobby code and production systems that run for decades. It's also the difference between "works on my machine" and "works on everyone's machine for the next 20 years."

\vspace{1em}

Now go forth and make your structs opaque. Your future self (and your users) will thank you. Probably with fewer bug reports and angry emails.

\begin{tipbox}
Next time you use \texttt{FILE*}, \texttt{DIR*}, \texttt{pthread\_t}, or any OpenSSL type, remember: you're using opaque pointers. This pattern has powered the world's most critical software for 50+ years. Learn it well. (It's older than most programming languages. That's not old, that's battle-tested.)
\end{tipbox}

\chapter{Function Pointers \& Callbacks}

\section{What Are Function Pointers, Really?}

In C, functions aren't just code---they're stored at memory addresses just like variables. A function pointer is a variable that stores the address of a function, allowing you to call different functions dynamically.

But here's what they don't teach in school: function pointers are how C achieves late binding without a virtual machine. They're how the Linux kernel implements system calls, how qsort can sort anything, how GUI frameworks handle events, and how game engines implement component systems. Basically, they're C's way of saying "I can be flexible too!" (Without needing a garbage collector, thank you very much.)

Think of it like a remote control. Instead of hardwiring which TV channel to display, you can change channels at runtime by pressing different buttons. Except if you press the wrong button, you get a segfault instead of infomercials. Pick your poison.

\section{What Happens at the Assembly Level}

Let's demystify what's really happening:

\begin{lstlisting}
// Simple function
int add(int a, int b) {
    return a + b;
}

// Function pointer
int (*operation)(int, int);
operation = add;
int result = operation(5, 3);

// What the compiler generates (x86-64, simplified):
//
// add function:
//   Address: 0x400500
//   Code:    mov eax, edi      ; a in edi
//            add eax, esi      ; b in esi
//            ret               ; return
//
// operation = add:
//   mov qword [rbp-8], 0x400500  ; Store address
//
// operation(5, 3):
//   mov edi, 5              ; First argument
//   mov esi, 3              ; Second argument
//   call qword [rbp-8]      ; INDIRECT call to address
//
// Direct call:    call 0x400500     ; 5 bytes, 1 cycle
// Indirect call:  call qword [mem]  ; slower, prevents inlining
//
// This is why function pointers are slightly slower!
\end{lstlisting}

\begin{notebox}
Function pointers prevent the compiler from inlining. A direct call to \texttt{add(5, 3)} can be optimized to a constant 8. A call through a function pointer cannot, because the compiler doesn't know what function will be called until runtime. It's like trying to optimize a surprise party---you can't plan if you don't know who's showing up.
\end{notebox}

\section{Basic Syntax: Reading the Declaration}

Function pointer syntax is notoriously confusing. Here's the secret:

\begin{lstlisting}
// A simple function
int add(int a, int b) {
    return a + b;
}

// Function pointer declaration - read it right-to-left, inside-out
int (*operation)(int, int);
//      ^           ^    ^
//      |           |    +--- returns int
//      |           +-------- takes (int, int)
//      +-------------------- operation is a POINTER TO function

// Common mistakes:
int *operation(int, int);    // WRONG! This is a function returning int*
int (*operation[10])(int);   // Array of 10 function pointers
int *(*operation)(int);      // Function pointer returning int*

// Assign and use
operation = add;              // Store address of add
operation = &add;             // Same thing (& is optional)
int result = operation(5, 3); // Call through pointer
result = (*operation)(5, 3);  // Same thing (* is optional)
\end{lstlisting}

\begin{notebox}
Reading function pointers: Start from the variable name and work outward. \texttt{(*operation)} means "operation is a pointer to..." and \texttt{(int, int)} means "...a function taking two ints and returning int." If this feels like reading hieroglyphics, you're not alone. Even Dennis Ritchie admitted the syntax is a bit wonky.
\end{notebox}

\section{Typedef Makes It Readable}

Function pointer syntax can get messy. Use \texttt{typedef} to make it cleaner:

\begin{lstlisting}
// Without typedef - hard to read
void register_callback(void (*callback)(int, const char*));

// With typedef - much better
typedef void (*MessageCallback)(int code, const char* msg);
void register_callback(MessageCallback callback);

// Even complex cases become readable
typedef int (*CompareFn)(const void*, const void*);
typedef void (*DestructorFn)(void*);
typedef void* (*AllocatorFn)(size_t);
typedef int (*FilterFn)(void* item, void* context);

// Real-world pattern: OpenGL callbacks
typedef void (*GLDEBUGPROC)(GLenum source, GLenum type,
                            GLuint id, GLenum severity,
                            GLsizei length, const GLchar* message,
                            const void* userParam);

// Without typedef, this would be unreadable!
\end{lstlisting}

\section{Callbacks: The Power Pattern}

Callbacks are functions you pass to other functions. This is how C achieves "customizable behavior" without objects.

\subsection{Example: Custom Sorting}

The standard library's \texttt{qsort} is a perfect example:

\begin{lstlisting}
#include <stdlib.h>
#include <stdio.h>
#include <string.h>

// Comparison function for qsort
// Must return: <0 if a<b, 0 if a==b, >0 if a>b
int compare_ints(const void* a, const void* b) {
    int arg1 = *(const int*)a;
    int arg2 = *(const int*)b;

    // Simple but has a subtle bug - can overflow!
    // return arg1 - arg2;

    // Correct implementation:
    if (arg1 < arg2) return -1;
    if (arg1 > arg2) return 1;
    return 0;
}

// Reverse comparison
int compare_ints_reverse(const void* a, const void* b) {
    return compare_ints(b, a);  // Just swap arguments
}

// Case-insensitive string comparison
int compare_strings_icase(const void* a, const void* b) {
    const char* str1 = *(const char**)a;
    const char* str2 = *(const char**)b;
    return strcasecmp(str1, str2);
}

// Sort by string length
int compare_by_length(const void* a, const void* b) {
    const char* str1 = *(const char**)a;
    const char* str2 = *(const char**)b;
    size_t len1 = strlen(str1);
    size_t len2 = strlen(str2);

    if (len1 < len2) return -1;
    if (len1 > len2) return 1;
    return strcmp(str1, str2);  // Secondary sort by content
}

int main(void) {
    int arr[] = {5, 2, 9, 1, 7};
    int n = 5;

    // Sort ascending - same qsort, different callback
    qsort(arr, n, sizeof(int), compare_ints);

    // Sort descending - same qsort, different callback
    qsort(arr, n, sizeof(int), compare_ints_reverse);

    // One qsort implementation, infinite sorting strategies!
    // This is the Strategy pattern from Gang of Four

    return 0;
}
\end{lstlisting}

\begin{warningbox}
The classic bug: \texttt{return a - b} overflows! If \texttt{a = INT\_MAX} and \texttt{b = -1}, the subtraction wraps around to negative. Always use explicit comparisons for numeric types! This bug has probably cost humanity more cumulative debugging hours than we spent building the pyramids.
\end{warningbox}

\section{The User Data Pattern (The Secret Sauce)}

A critical idiom: passing context to callbacks using \texttt{void* user\_data}:

\begin{lstlisting}
// WITHOUT user_data - limited and broken
typedef void (*SimpleCallback)(void);

int global_sum = 0;  // BAD! Global state
int global_count = 0;

void accumulate_bad(void) {
    // How do we access the current item?
    // We can't! No parameters!
    global_sum += ???;  // What value?
    global_count++;
}

// WITH user_data - powerful and correct
typedef void (*Callback)(int value, void* user_data);

void process_items(int* items, size_t count,
                   Callback handler, void* user_data) {
    for (size_t i = 0; i < count; i++) {
        handler(items[i], user_data);  // Pass context
    }
}

// Now your callback can access context
typedef struct {
    int sum;
    int count;
    int min;
    int max;
} Stats;

void accumulate(int value, void* user_data) {
    Stats* stats = (Stats*)user_data;
    stats->sum += value;
    stats->count++;

    if (value < stats->min) stats->min = value;
    if (value > stats->max) stats->max = value;
}

// Usage - no globals!
int items[] = {5, 2, 9, 1, 7};
Stats stats = {0, 0, INT_MAX, INT_MIN};
process_items(items, 5, accumulate, &stats);

printf("Sum: %d, Count: %d, Avg: %.2f\n",
       stats.sum, stats.count,
       (double)stats.sum / stats.count);
printf("Min: %d, Max: %d\n", stats.min, stats.max);
\end{lstlisting}

\begin{tipbox}
The \texttt{void* user\_data} pattern is CRUCIAL! It lets you pass context to callbacks without global variables. You'll see this in every C library that uses callbacks: GTK, libuv, libcurl, SQLite, OpenGL. This is how C does closures! (Well, "closures." We make do with what we have, okay?)
\end{tipbox}

\section{Real-World Pattern: Event Handlers}

This is how every GUI framework and event system works:

\begin{lstlisting}
#include <stdint.h>
#include <time.h>

// Event callback type
typedef void (*EventCallback)(void* sender, void* user_data);

// Event system structure
typedef struct {
    EventCallback on_click;
    EventCallback on_double_click;
    EventCallback on_hover;
    EventCallback on_release;
    void* user_data;

    // State
    int x, y;
    int width, height;
    uint32_t last_click_time;
    int enabled;
} Button;

// Initialize button with callbacks
void button_init(Button* btn,
                 int x, int y, int w, int h,
                 EventCallback click_handler,
                 void* data) {
    btn->on_click = click_handler;
    btn->on_double_click = NULL;
    btn->on_hover = NULL;
    btn->on_release = NULL;
    btn->user_data = data;

    btn->x = x;
    btn->y = y;
    btn->width = w;
    btn->height = h;
    btn->last_click_time = 0;
    btn->enabled = 1;
}

// Trigger events
void button_handle_click(Button* btn, uint32_t timestamp) {
    if (!btn || !btn->enabled) return;

    // Check for double-click (< 300ms between clicks)
    if (btn->on_double_click &&
        timestamp - btn->last_click_time < 300) {
        btn->on_double_click(btn, btn->user_data);
    } else if (btn->on_click) {
        btn->on_click(btn, btn->user_data);
    }

    btn->last_click_time = timestamp;
}

void button_handle_hover(Button* btn) {
    if (btn && btn->on_hover) {
        btn->on_hover(btn, btn->user_data);
    }
}

// User's callbacks
void submit_handler(void* sender, void* data) {
    const char* form_name = (const char*)data;
    printf("Submitting form: %s\n", form_name);

    // Sender is the button itself
    Button* btn = (Button*)sender;
    btn->enabled = 0;  // Disable after click
}

void cancel_handler(void* sender, void* data) {
    printf("Cancelled\n");
}

void hover_handler(void* sender, void* data) {
    Button* btn = (Button*)sender;
    printf("Hovering over button at (%d, %d)\n", btn->x, btn->y);
}

// Usage
int main(void) {
    Button submit_btn;
    button_init(&submit_btn, 10, 10, 100, 30,
                submit_handler, "LoginForm");
    submit_btn.on_hover = hover_handler;

    Button cancel_btn;
    button_init(&cancel_btn, 120, 10, 100, 30,
                cancel_handler, NULL);

    // Simulate events
    button_handle_hover(&submit_btn);
    button_handle_click(&submit_btn, 1000);
    button_handle_click(&submit_btn, 1100);  // Won't fire, disabled

    return 0;
}
\end{lstlisting}

\section{Function Pointer Arrays: Dispatch Tables}

Create dispatch tables for elegant control flow:

\begin{lstlisting}
typedef enum {
    CMD_READ,
    CMD_WRITE,
    CMD_DELETE,
    CMD_UPDATE,
    CMD_LIST,
    CMD_COUNT
} Command;

typedef int (*CommandHandler)(void* data);

// Handlers
int handle_read(void* data) {
    printf("Reading: %s\n", (char*)data);
    return 0;
}

int handle_write(void* data) {
    printf("Writing: %s\n", (char*)data);
    return 0;
}

int handle_delete(void* data) {
    printf("Deleting: %s\n", (char*)data);
    return 0;
}

int handle_update(void* data) {
    printf("Updating: %s\n", (char*)data);
    return 0;
}

int handle_list(void* data) {
    printf("Listing\n");
    return 0;
}

// Dispatch table - designated initializers (C99)
CommandHandler handlers[CMD_COUNT] = {
    [CMD_READ] = handle_read,
    [CMD_WRITE] = handle_write,
    [CMD_DELETE] = handle_delete,
    [CMD_UPDATE] = handle_update,
    [CMD_LIST] = handle_list
};

// Execute command - ONE LINE!
int execute_command(Command cmd, void* data) {
    if (cmd >= 0 && cmd < CMD_COUNT && handlers[cmd]) {
        return handlers[cmd](data);
    }
    return -1;
}

// Instead of this unmaintainable mess:
int execute_command_bad(Command cmd, void* data) {
    switch (cmd) {
        case CMD_READ:
            printf("Reading: %s\n", (char*)data);
            return 0;
        case CMD_WRITE:
            printf("Writing: %s\n", (char*)data);
            return 0;
        case CMD_DELETE:
            printf("Deleting: %s\n", (char*)data);
            return 0;
        case CMD_UPDATE:
            printf("Updating: %s\n", (char*)data);
            return 0;
        case CMD_LIST:
            printf("Listing\n");
            return 0;
        default:
            return -1;
    }
}

// With dispatch table:
// - Add new command: Add enum, add function, add to table
// - Much cleaner, more maintainable
// - Can be data-driven (load from config)
// - Used in Linux kernel, interpreters, state machines
\end{lstlisting}

\begin{tipbox}
Dispatch tables are faster than switch statements on some architectures. Modern CPUs have branch predictors, but a table lookup is a simple memory read with no branching at all! Plus, they look way cooler. (Yes, code aesthetics matter. Fight me.)
\end{tipbox}

\section{Polymorphism in C: The VTable Pattern}

Function pointers enable object-oriented patterns:

\begin{lstlisting}
// "Interface" - table of function pointers
typedef struct {
    void (*draw)(void* self);
    void (*move)(void* self, int x, int y);
    void (*destroy)(void* self);
    const char* (*get_type)(void* self);
} ShapeVTable;

// Base "class"
typedef struct {
    ShapeVTable* vtable;  // MUST be first member!
    int x;
    int y;
} Shape;

// Circle "subclass"
typedef struct {
    Shape base;  // MUST be first - allows casting
    int radius;
} Circle;

void circle_draw(void* self) {
    Circle* c = (Circle*)self;
    printf("Drawing circle at (%d, %d) radius %d\n",
           c->base.x, c->base.y, c->radius);
}

void circle_move(void* self, int x, int y) {
    Circle* c = (Circle*)self;
    c->base.x = x;
    c->base.y = y;
    printf("Circle moved to (%d, %d)\n", x, y);
}

void circle_destroy(void* self) {
    Circle* c = (Circle*)self;
    printf("Destroying circle\n");
    free(c);
}

const char* circle_get_type(void* self) {
    return "Circle";
}

// VTable for circles - one instance shared by all circles
static ShapeVTable circle_vtable = {
    .draw = circle_draw,
    .move = circle_move,
    .destroy = circle_destroy,
    .get_type = circle_get_type
};

// Rectangle "subclass"
typedef struct {
    Shape base;
    int width;
    int height;
} Rectangle;

void rectangle_draw(void* self) {
    Rectangle* r = (Rectangle*)self;
    printf("Drawing rectangle at (%d, %d) size %dx%d\n",
           r->base.x, r->base.y, r->width, r->height);
}

void rectangle_move(void* self, int x, int y) {
    Rectangle* r = (Rectangle*)self;
    r->base.x = x;
    r->base.y = y;
}

void rectangle_destroy(void* self) {
    free(self);
}

const char* rectangle_get_type(void* self) {
    return "Rectangle";
}

static ShapeVTable rectangle_vtable = {
    .draw = rectangle_draw,
    .move = rectangle_move,
    .destroy = rectangle_destroy,
    .get_type = rectangle_get_type
};

// Constructors
Circle* circle_create(int x, int y, int radius) {
    Circle* c = malloc(sizeof(Circle));
    if (c) {
        c->base.vtable = &circle_vtable;
        c->base.x = x;
        c->base.y = y;
        c->radius = radius;
    }
    return c;
}

Rectangle* rectangle_create(int x, int y, int w, int h) {
    Rectangle* r = malloc(sizeof(Rectangle));
    if (r) {
        r->base.vtable = &rectangle_vtable;
        r->base.x = x;
        r->base.y = y;
        r->width = w;
        r->height = h;
    }
    return r;
}

// Polymorphic functions - work with ANY shape!
void shape_draw(Shape* shape) {
    if (shape && shape->vtable && shape->vtable->draw) {
        shape->vtable->draw(shape);
    }
}

void shape_move(Shape* shape, int x, int y) {
    if (shape && shape->vtable && shape->vtable->move) {
        shape->vtable->move(shape, x, y);
    }
}

void shape_destroy(Shape* shape) {
    if (shape && shape->vtable && shape->vtable->destroy) {
        shape->vtable->destroy(shape);
    }
}

// Usage - true polymorphism!
int main(void) {
    Shape* shapes[3];

    shapes[0] = (Shape*)circle_create(10, 20, 5);
    shapes[1] = (Shape*)rectangle_create(30, 40, 15, 10);
    shapes[2] = (Shape*)circle_create(50, 60, 8);

    // Polymorphic calls - different behavior per type
    for (int i = 0; i < 3; i++) {
        shape_draw(shapes[i]);      // Calls correct draw()
        shape_move(shapes[i], i*100, i*100);

        const char* type = shapes[i]->vtable->get_type(shapes[i]);
        printf("Type: %s\n", type);
    }

    // Cleanup
    for (int i = 0; i < 3; i++) {
        shape_destroy(shapes[i]);
    }

    return 0;
}
\end{lstlisting}

\begin{notebox}
This is EXACTLY how GTK+, GObject, and many other C libraries implement object-oriented programming! The VTable pattern is fundamental to understanding large C codebases. C++ virtual functions are implemented the same way under the hood! So when C++ programmers brag about polymorphism, just smile and nod---we've been doing it since 1972.
\end{notebox}

\section{Why VTable Must Be First Member}

This is a critical detail:

\begin{lstlisting}
// C guarantees: A pointer to a struct points to its first member
struct Shape {
    ShapeVTable* vtable;  // Offset 0
    int x;                // Offset 8 (on 64-bit)
    int y;                // Offset 12
};

struct Circle {
    Shape base;           // Offset 0 (contains vtable at offset 0)
    int radius;           // Offset 16
};

// Safe cast from Circle* to Shape*
Circle* c = circle_create(10, 20, 5);
Shape* s = (Shape*)c;  // Points to same address!

// Both point to: 0x1000 (hypothetical address)
// c->base.vtable is at 0x1000
// s->vtable is at 0x1000
// Same memory location!

// This is why inheritance works in C!
// Address of Circle = Address of Shape base = Address of VTable
\end{lstlisting}

\section{Signal/Slot Pattern: Multiple Observers}

Multiple callbacks for one event (Observer pattern):

\begin{lstlisting}
#define MAX_LISTENERS 10

typedef void (*EventListener)(void* sender, void* event_data, void* user_data);

typedef struct {
    EventListener listeners[MAX_LISTENERS];
    void* user_data[MAX_LISTENERS];
    int count;
} Event;

void event_init(Event* evt) {
    evt->count = 0;
    memset(evt->listeners, 0, sizeof(evt->listeners));
    memset(evt->user_data, 0, sizeof(evt->user_data));
}

int event_connect(Event* evt, EventListener listener, void* user_data) {
    if (!evt || !listener || evt->count >= MAX_LISTENERS) {
        return -1;
    }

    evt->listeners[evt->count] = listener;
    evt->user_data[evt->count] = user_data;
    evt->count++;
    return 0;
}

int event_disconnect(Event* evt, EventListener listener) {
    if (!evt) return -1;

    for (int i = 0; i < evt->count; i++) {
        if (evt->listeners[i] == listener) {
            // Shift remaining listeners down
            for (int j = i; j < evt->count - 1; j++) {
                evt->listeners[j] = evt->listeners[j + 1];
                evt->user_data[j] = evt->user_data[j + 1];
            }
            evt->count--;
            return 0;
        }
    }
    return -1;
}

void event_emit(Event* evt, void* sender, void* event_data) {
    if (!evt) return;

    // Call all registered listeners
    for (int i = 0; i < evt->count; i++) {
        if (evt->listeners[i]) {
            evt->listeners[i](sender, event_data, evt->user_data[i]);
        }
    }
}

// Multiple handlers for same event
void log_handler(void* sender, void* data, void* user_data) {
    FILE* logfile = (FILE*)user_data;
    fprintf(logfile, "Event occurred\n");
    fflush(logfile);
}

void update_ui_handler(void* sender, void* data, void* user_data) {
    printf("UI updated with data: %s\n", (char*)data);
}

void save_handler(void* sender, void* data, void* user_data) {
    const char* filename = (const char*)user_data;
    printf("Saving to %s\n", filename);
}

void analytics_handler(void* sender, void* data, void* user_data) {
    static int event_count = 0;
    event_count++;
    printf("Event #%d tracked\n", event_count);
}

// Usage
int main(void) {
    Event on_data_changed;
    event_init(&on_data_changed);

    FILE* log = fopen("events.log", "a");
    event_connect(&on_data_changed, log_handler, log);
    event_connect(&on_data_changed, update_ui_handler, NULL);
    event_connect(&on_data_changed, save_handler, "data.txt");
    event_connect(&on_data_changed, analytics_handler, NULL);

    // All four handlers get called!
    event_emit(&on_data_changed, NULL, "new data");

    // Remove a handler
    event_disconnect(&on_data_changed, analytics_handler);

    // Now only three handlers get called
    event_emit(&on_data_changed, NULL, "updated data");

    fclose(log);
    return 0;
}
\end{lstlisting}

\section{Common Pitfalls and How to Avoid Them}

\subsection{Lifetime Issues}

\begin{warningbox}
Be careful with callback lifetimes:
\begin{lstlisting}
// DANGER: Function address becomes invalid!
void register_callback(void (*cb)(void)) {
    static void (*saved_callback)(void) = NULL;
    saved_callback = cb;
    // cb must remain valid for entire program!
}

void bad_example(void) {
    // Nested function (GCC extension, non-standard!)
    void local_callback(void) {
        printf("Callback\n");
    }

    // DANGER: local_callback dies when bad_example returns
    // The address points to freed stack memory!
    register_callback(local_callback);
}

// If someone calls saved_callback later: BOOM!
// (Not the good kind of boom, like fireworks. The bad kind.)
\end{lstlisting}
\end{warningbox}

\subsection{Type Safety Issues}

\begin{lstlisting}
// Easy to mess up types
typedef void (*Callback)(int);

void my_callback(long x) {  // WRONG TYPE!
    printf("%ld\n", x);
}

// This compiles with a cast but is undefined behavior
Callback cb = (Callback)my_callback;
cb(42);  // May crash or produce garbage!

// On 64-bit: int is 32-bit, long is 64-bit
// The calling convention is different!
// Arguments passed in wrong registers/stack locations

// SOLUTION: Match types exactly
typedef void (*Callback)(int);

void my_callback(int x) {  // Correct!
    printf("%d\n", x);
}

Callback cb = my_callback;  // No cast needed
cb(42);  // Works correctly
\end{lstlisting}

\subsection{NULL Pointer Checks}

\begin{lstlisting}
// ALWAYS check function pointers before calling
void safe_call(void (*callback)(void)) {
    if (callback) {  // Essential!
        callback();
    }
}

// Calling NULL crashes immediately
void (*null_ptr)(void) = NULL;
null_ptr();  // SEGFAULT!

// Real-world pattern: optional callbacks
typedef struct {
    void (*on_success)(void* data);
    void (*on_error)(int code);  // Optional
    void* user_data;
} Request;

void request_complete(Request* req, int success) {
    if (success && req->on_success) {
        req->on_success(req->user_data);
    } else if (!success && req->on_error) {
        req->on_error(errno);
    }
    // on_error is optional - no crash if NULL
}
\end{lstlisting}

\section{Advanced: Closures (Sort Of)}

C doesn't have real closures, but we can fake them:

\begin{lstlisting}
// Closure structure - captures context
typedef struct {
    void (*func)(void* context);
    void* context;
} Closure;

void closure_call(Closure* closure) {
    if (closure && closure->func) {
        closure->func(closure->context);
    }
}

// Context for our "closure"
typedef struct {
    int multiplier;
    int base;
} MultiplyContext;

void multiply_callback(void* context) {
    MultiplyContext* ctx = (MultiplyContext*)context;
    int result = ctx->base * ctx->multiplier;
    printf("Result: %d * %d = %d\n",
           ctx->base, ctx->multiplier, result);
}

// Usage
MultiplyContext ctx = {10, 5};
Closure closure = {
    .func = multiply_callback,
    .context = &ctx
};
closure_call(&closure);  // Prints: Result: 5 * 10 = 50

// Change context
ctx.base = 7;
closure_call(&closure);  // Prints: Result: 7 * 10 = 70

// This is how GTK+ implements callbacks with user data!
\end{lstlisting}

\section{Performance Considerations}

\begin{lstlisting}
// Benchmark: Direct vs Indirect calls
#include <time.h>

int add_direct(int a, int b) {
    return a + b;
}

void benchmark_direct() {
    clock_t start = clock();
    int sum = 0;

    for (int i = 0; i < 100000000; i++) {
        sum += add_direct(i, i);
    }

    clock_t end = clock();
    double elapsed = (double)(end - start) / CLOCKS_PER_SEC;
    printf("Direct: %.3f seconds, sum=%d\n", elapsed, sum);
}

void benchmark_indirect() {
    clock_t start = clock();
    int sum = 0;
    int (*func)(int, int) = add_direct;

    for (int i = 0; i < 100000000; i++) {
        sum += func(i, i);  // Indirect call
    }

    clock_t end = clock();
    double elapsed = (double)(end - start) / CLOCKS_PER_SEC;
    printf("Indirect: %.3f seconds, sum=%d\n", elapsed, sum);
}

// Typical results (varies by compiler/CPU):
// Direct:   0.150 seconds  (inlined, optimized)
// Indirect: 0.300 seconds  (cannot inline, branch prediction)

// Conclusion: Function pointers are ~2x slower
// But still very fast (300M calls/second)
// Use them when flexibility is worth the cost
// (If you're doing 300M calls/second, you have bigger problems)
\end{lstlisting}

\begin{tipbox}
Modern CPUs have branch predictors. If you call the same function pointer repeatedly (e.g., in a loop with qsort comparisons), the predictor learns and performance improves. Dispatch tables with consistent patterns are faster than random function pointer calls! CPUs are smart. Your function pointers? Not so much. Help them out.
\end{tipbox}

\section{Calling Conventions: What You Need to Know}

\begin{lstlisting}
// On x86-64, default calling convention (System V):
// - First 6 integer args: RDI, RSI, RDX, RCX, R8, R9
// - Return value: RAX
// - Floating point: XMM0-XMM7

// Windows x64 uses different convention:
// - First 4 args: RCX, RDX, R8, R9
// - Return value: RAX
// - Caller must reserve 32 bytes "shadow space"

// THIS MATTERS for function pointers!
// You cannot cast between different calling conventions!

#ifdef _WIN32
    // Windows calling convention
    typedef int (__stdcall *WinCallback)(int, int);
#else
    // POSIX calling convention
    typedef int (*PosixCallback)(int, int);
#endif

// Variadic functions are special
typedef int (*VariadicFunc)(const char* fmt, ...);

int my_printf(const char* fmt, ...) {
    va_list args;
    va_start(args, fmt);
    int result = vprintf(fmt, args);
    va_end(args);
    return result;
}

// You can store printf-like functions!
VariadicFunc logger = my_printf;
logger("Value: %d\n", 42);
\end{lstlisting}

\section{Real-World Example: Plugin System}

\begin{lstlisting}
// plugin.h - Plugin interface
typedef struct {
    const char* name;
    int version;

    // Function pointers for plugin methods
    int (*init)(void);
    int (*process)(void* data, size_t len);
    void (*shutdown)(void);
    const char* (*get_info)(void);
} Plugin;

// plugin_loader.c - Load plugins from shared libraries
#include <dlfcn.h>  // dlopen, dlsym (POSIX)

Plugin* load_plugin(const char* path) {
    void* handle = dlopen(path, RTLD_LAZY);
    if (!handle) {
        fprintf(stderr, "Failed to load %s: %s\n", path, dlerror());
        return NULL;
    }

    // Get plugin descriptor
    typedef Plugin* (*GetPluginFunc)(void);
    GetPluginFunc get_plugin = (GetPluginFunc)dlsym(handle, "get_plugin");

    if (!get_plugin) {
        fprintf(stderr, "No get_plugin() in %s\n", path);
        dlclose(handle);
        return NULL;
    }

    Plugin* plugin = get_plugin();
    if (!plugin) {
        dlclose(handle);
        return NULL;
    }

    printf("Loaded plugin: %s v%d\n", plugin->name, plugin->version);
    return plugin;
}

// example_plugin.c - Example plugin implementation
int plugin_init(void) {
    printf("Plugin initializing...\n");
    return 0;
}

int plugin_process(void* data, size_t len) {
    printf("Processing %zu bytes\n", len);
    return 0;
}

void plugin_shutdown(void) {
    printf("Plugin shutting down\n");
}

const char* plugin_get_info(void) {
    return "Example plugin for demonstration";
}

static Plugin example_plugin = {
    .name = "ExamplePlugin",
    .version = 1,
    .init = plugin_init,
    .process = plugin_process,
    .shutdown = plugin_shutdown,
    .get_info = plugin_get_info
};

// Export symbol
Plugin* get_plugin(void) {
    return &example_plugin;
}

// Compile plugin:
// gcc -shared -fPIC example_plugin.c -o example.so

// Main application loads and uses plugins dynamically!
// No recompilation needed to add new plugins!
\end{lstlisting}

\section{Thread Safety Considerations}

\begin{lstlisting}
// Function pointers themselves are just addresses - thread safe
// But the DATA they access must be protected!

#include <pthread.h>

typedef void (*ThreadCallback)(void* data);

// UNSAFE - race condition
int global_counter = 0;

void unsafe_callback(void* data) {
    global_counter++;  // NOT ATOMIC!
    // Thread 1: Read counter (0)
    // Thread 2: Read counter (0)
    // Thread 1: Write counter (1)
    // Thread 2: Write counter (1)
    // Result: 1, should be 2!
}

// SAFE - mutex protection
pthread_mutex_t counter_mutex = PTHREAD_MUTEX_INITIALIZER;
int safe_counter = 0;

void safe_callback(void* data) {
    pthread_mutex_lock(&counter_mutex);
    safe_counter++;
    pthread_mutex_unlock(&counter_mutex);
}

// BETTER - thread-local storage
__thread int thread_counter = 0;

void thread_local_callback(void* data) {
    thread_counter++;  // Each thread has its own copy
}

// BEST - pass data through callback parameter
void stateless_callback(void* data) {
    int* counter = (int*)data;
    __sync_fetch_and_add(counter, 1);  // Atomic increment
}
\end{lstlisting}

\section{Debugging Function Pointer Issues}

\begin{lstlisting}
// Print function pointer addresses for debugging
void debug_callback(void (*callback)(void), const char* name) {
    printf("Callback '%s' at address: %p\n", name, (void*)callback);
}

// Check if callback is NULL
#define CALL_CALLBACK(cb, ...) do { \
    if (cb) { \
        cb(__VA_ARGS__); \
    } else { \
        fprintf(stderr, "Warning: NULL callback at %s:%d\n", \
                __FILE__, __LINE__); \
    } \
} while(0)

// Validate callback before storing
int register_callback(void (*cb)(void)) {
    if (!cb) {
        fprintf(stderr, "Error: NULL callback\n");
        return -1;
    }

    // On some platforms, can check if address is valid
    // (This is platform-specific and not portable!)
    #ifdef __linux__
    if ((void*)cb < (void*)0x1000) {
        fprintf(stderr, "Error: Invalid callback address\n");
        return -1;
    }
    #endif

    // Store callback...
    return 0;
}
\end{lstlisting}

\section{Summary: When to Use Function Pointers}

Function pointers are essential for:

\begin{itemize}
    \item \textbf{Callbacks}: Event handling, async operations
    \item \textbf{Polymorphism}: Implementing OOP patterns without objects
    \item \textbf{Plugin systems}: Dynamic loading of functionality
    \item \textbf{State machines}: Function pointers as state handlers
    \item \textbf{Strategy pattern}: Swappable algorithms (qsort, filtering)
    \item \textbf{Dependency injection}: Pass behavior without globals
    \item \textbf{Dispatch tables}: Clean alternative to switch statements
    \item \textbf{Observer pattern}: Multiple callbacks for one event
\end{itemize}

Avoid function pointers when:

\begin{itemize}
    \item Performance is critical and behavior is fixed
    \item Code is simple and doesn't need flexibility
    \item You're on an embedded system with limited resources
    \item The function is called in a very tight loop
\end{itemize}

Master function pointers, and you unlock the full power of C's flexibility. They're the secret sauce that makes C suitable for everything from embedded systems to operating systems to game engines. They're how professional C developers achieve modularity, extensibility, and elegance without sacrificing performance. (Well, "elegance" might be stretching it, but you get the point.)

\begin{tipbox}
Next time you use \texttt{qsort()}, \texttt{signal()}, \texttt{atexit()}, or any GTK/Qt callback, remember: you're using function pointers. This pattern has powered systems programming for 50 years. It's the foundation of every major C library, from the Linux kernel to OpenSSL to SQLite. Learn it well, and you'll write C code that rivals modern languages in flexibility while maintaining C's legendary performance and control. (And you can tell those JavaScript developers that we had callbacks before callbacks were cool.)
\end{tipbox}

\chapter{Macro Magic \& Pitfalls}

\section{Macros: More Than You Think}

Macros are C's preprocessor magic. They're not functions—they're text substitution that happens before compilation. This makes them powerful but dangerous.

Think of macros as a find-and-replace tool that runs before your code is even seen by the compiler. This gives them unique capabilities but also unique dangers.

\section{The Parentheses Rule}

\begin{warningbox}
Always wrap macro parameters and the entire expression in parentheses!
\end{warningbox}

\begin{lstlisting}
// WRONG - breaks with complex expressions
#define SQUARE(x) x * x

int a = SQUARE(2 + 3);  // Expands to: 2 + 3 * 2 + 3 = 11!

// CORRECT
#define SQUARE(x) ((x) * (x))

int b = SQUARE(2 + 3);  // Expands to: ((2 + 3) * (2 + 3)) = 25
\end{lstlisting}

\subsection{Why This Matters}

\begin{lstlisting}
// More subtle bugs
#define DOUBLE(x) x + x

int result = DOUBLE(5) * 2;  // Expands to: 5 + 5 * 2 = 15 (not 20!)

// Always use parentheses
#define DOUBLE(x) ((x) + (x))

int result = DOUBLE(5) * 2;  // Expands to: ((5) + (5)) * 2 = 20
\end{lstlisting}

\section{Multi-Statement Macros}

\begin{lstlisting}
// WRONG - breaks in if statements
#define SWAP(a, b) \
    int temp = a; \
    a = b; \
    b = temp;

// This breaks:
if (x > y)
    SWAP(x, y);  // Only first line is in if!
// b = temp executes unconditionally!

// CORRECT - use do-while(0) idiom
#define SWAP(a, b) do { \
    int temp = a; \
    a = b; \
    b = temp; \
} while(0)

// Now this works correctly
if (x > y)
    SWAP(x, y);  // All statements in the if
\end{lstlisting}

\begin{notebox}
The \texttt{do-while(0)} trick is used everywhere in professional C. It creates a proper statement block that requires a semicolon after it, making the macro behave like a function call.
\end{notebox}

\subsection{Why do-while(0) Works}

\begin{lstlisting}
// The pattern
do {
    statement1;
    statement2;
    statement3;
} while(0);  // Always false, executes once

// Benefits:
// 1. Multiple statements act as one
// 2. Requires semicolon after macro call
// 3. Works with if/else without braces
// 4. Can use break to exit early
\end{lstlisting}

\section{Side Effects and Multiple Evaluation}

\begin{warningbox}
Macros evaluate their arguments every time they appear!
\end{warningbox}

\begin{lstlisting}
#define MAX(a, b) ((a) > (b) ? (a) : (b))

int x = 5;
int m = MAX(x++, 10);  // x gets incremented TWICE!
// Expands to: ((x++) > (10) ? (x++) : (10))

printf("x = %d\n", x);  // Could be 6 or 7!
\end{lstlisting}

\subsection{Solution: Statement Expressions (GCC/Clang)}

\begin{lstlisting}
// GCC/Clang extension
#define MAX(a, b) ({ \
    __typeof__(a) _a = (a); \
    __typeof__(b) _b = (b); \
    _a > _b ? _a : _b; \
})

// Now this works correctly
int x = 5;
int m = MAX(x++, 10);  // x incremented only once
printf("x = %d, m = %d\n", x, m);  // x = 6, m = 10
\end{lstlisting}

\section{X-Macros: The Secret Weapon}

X-Macros let you maintain a single list that generates multiple things. This is incredibly powerful!

\begin{lstlisting}
// Define your list once
#define ERROR_CODES \
    X(SUCCESS, 0, "Operation successful") \
    X(ERR_NOMEM, 1, "Out of memory") \
    X(ERR_INVALID, 2, "Invalid argument") \
    X(ERR_IO, 3, "I/O error") \
    X(ERR_TIMEOUT, 4, "Operation timed out")

// Generate enum
#define X(name, code, msg) name = code,
typedef enum {
    ERROR_CODES
} ErrorCode;
#undef X

// Generate string array
#define X(name, code, msg) msg,
static const char* error_messages[] = {
    ERROR_CODES
};
#undef X

// Generate name array
#define X(name, code, msg) #name,
static const char* error_names[] = {
    ERROR_CODES
};
#undef X

// Now you can use it:
const char* get_error_message(ErrorCode code) {
    if (code >= 0 && code < sizeof(error_messages)/sizeof(error_messages[0])) {
        return error_messages[code];
    }
    return "Unknown error";
}

const char* get_error_name(ErrorCode code) {
    if (code >= 0 && code < sizeof(error_names)/sizeof(error_names[0])) {
        return error_names[code];
    }
    return "UNKNOWN";
}
\end{lstlisting}

\begin{tipbox}
X-Macros are used in the Linux kernel and many professional projects. They eliminate duplication and keep related code in sync automatically. Add a new error? Just add one line to the X-Macro list!
\end{tipbox}

\subsection{More X-Macro Examples}

\begin{lstlisting}
// Command dispatch table
#define COMMANDS \
    X(quit, "Exit the program") \
    X(help, "Show help message") \
    X(save, "Save current state") \
    X(load, "Load saved state")

// Generate function declarations
#define X(name, desc) void cmd_##name(void);
COMMANDS
#undef X

// Generate command table
typedef struct {
    const char* name;
    const char* description;
    void (*handler)(void);
} Command;

#define X(name, desc) {#name, desc, cmd_##name},
Command commands[] = {
    COMMANDS
};
#undef X
\end{lstlisting}

\section{Stringification and Token Pasting}

\subsection{Stringification (\#)}

\begin{lstlisting}
// # makes a string literal
#define STR(x) #x

STR(hello)        // Becomes "hello"
STR(x + y)        // Becomes "x + y"
STR(123)          // Becomes "123"

// Practical use: debugging
#define PRINT_VAR(x) printf(#x " = %d\n", (x))

int age = 25;
PRINT_VAR(age);  // Prints: age = 25
\end{lstlisting}

\subsection{Token Pasting (\#\#)}

\begin{lstlisting}
// ## pastes tokens together
#define CONCAT(a, b) a##b

CONCAT(my_, function)  // Becomes my_function
CONCAT(x, 123)         // Becomes x123

// Practical use: automatic function names
#define DECLARE_GETTER_SETTER(type, name) \
    type get_##name(void) { \
        return name; \
    } \
    void set_##name(type value) { \
        name = value; \
    }

int age;
DECLARE_GETTER_SETTER(int, age)
// Generates: get_age() and set_age()
\end{lstlisting}

\subsection{Advanced Token Pasting}

\begin{lstlisting}
// Generic type-safe array
#define DEFINE_ARRAY(type) \
    typedef struct { \
        type* data; \
        size_t size; \
        size_t capacity; \
    } type##_array_t; \
    \
    type##_array_t* type##_array_create(void) { \
        type##_array_t* arr = malloc(sizeof(type##_array_t)); \
        arr->data = NULL; \
        arr->size = 0; \
        arr->capacity = 0; \
        return arr; \
    } \
    \
    void type##_array_push(type##_array_t* arr, type value) { \
        if (arr->size >= arr->capacity) { \
            arr->capacity = arr->capacity ? arr->capacity * 2 : 8; \
            arr->data = realloc(arr->data, arr->capacity * sizeof(type)); \
        } \
        arr->data[arr->size++] = value; \
    }

// Generate arrays for different types
DEFINE_ARRAY(int)
DEFINE_ARRAY(float)
DEFINE_ARRAY(double)

// Now you have:
// int_array_t, int_array_create(), int_array_push()
// float_array_t, float_array_create(), float_array_push()
// double_array_t, double_array_create(), double_array_push()
\end{lstlisting}

\section{Variadic Macros}

\begin{lstlisting}
// C99 variadic macros
#define DEBUG_PRINT(fmt, ...) \
    fprintf(stderr, "[DEBUG] " fmt "\n", ##__VA_ARGS__)

DEBUG_PRINT("Hello");              // Works with no args
DEBUG_PRINT("Value: %d", 42);      // Works with args
DEBUG_PRINT("x=%d, y=%d", 1, 2);   // Multiple args

// The ## before __VA_ARGS__ removes comma if no args
\end{lstlisting}

\subsection{Practical Logging Macro}

\begin{lstlisting}
#ifdef DEBUG
    #define LOG(level, fmt, ...) \
        fprintf(stderr, "[%s] %s:%d: " fmt "\n", \
                level, __FILE__, __LINE__, ##__VA_ARGS__)
#else
    #define LOG(level, fmt, ...) ((void)0)
#endif

#define LOG_ERROR(fmt, ...) LOG("ERROR", fmt, ##__VA_ARGS__)
#define LOG_WARN(fmt, ...)  LOG("WARN", fmt, ##__VA_ARGS__)
#define LOG_INFO(fmt, ...)  LOG("INFO", fmt, ##__VA_ARGS__)

// Usage
LOG_ERROR("Failed to open file: %s", filename);
LOG_INFO("Server started on port %d", port);
\end{lstlisting}

\section{Compile-Time Assertions}

\begin{lstlisting}
// Old-school static assert
#define STATIC_ASSERT(cond, msg) \
    typedef char static_assertion_##msg[(cond) ? 1 : -1]

// Use it
STATIC_ASSERT(sizeof(int) == 4, int_must_be_4_bytes);
STATIC_ASSERT(sizeof(void*) == 8, need_64bit_pointers);

// C11 has built-in _Static_assert
_Static_assert(sizeof(int) >= 4, "int too small");
\end{lstlisting}

\section{Macro Hygiene}

\subsection{Variable Name Collisions}

\begin{lstlisting}
// BAD - can collide with user variables
#define SWAP(a, b) do { \
    int temp = a; \
    a = b; \
    b = temp; \
} while(0)

int temp = 10;  // User's variable
int x = 5, y = 20;
SWAP(x, y);  // Collision with temp!

// BETTER - use unique names
#define SWAP(a, b) do { \
    int _swap_tmp_ = a; \
    a = b; \
    b = _swap_tmp_; \
} while(0)

// BEST - use __COUNTER__ or line number
#define SWAP(a, b) do { \
    int _tmp_##__LINE__ = a; \
    a = b; \
    b = _tmp_##__LINE__; \
} while(0)
\end{lstlisting}

\section{Conditional Compilation}

\begin{lstlisting}
// Feature flags
#ifdef FEATURE_LOGGING
    #define LOG(msg) printf("LOG: %s\n", msg)
#else
    #define LOG(msg) ((void)0)
#endif

// Platform-specific code
#if defined(_WIN32)
    #define PATH_SEPARATOR '\\'
#else
    #define PATH_SEPARATOR '/'
#endif

// Version checks
#if __STDC_VERSION__ >= 201112L
    // Use C11 features
    #define HAS_STATIC_ASSERT 1
#else
    // Fallback for older C
    #define HAS_STATIC_ASSERT 0
#endif
\end{lstlisting}

\section{Common Pitfalls}

\subsection{Semicolon Swallowing}

\begin{lstlisting}
// WRONG
#define CHECK(x) if (!(x)) return -1;

// Breaks:
if (condition)
    CHECK(something);
else  // Syntax error!
    do_other();

// RIGHT
#define CHECK(x) do { \
    if (!(x)) return -1; \
} while(0)
\end{lstlisting}

\subsection{Operator Precedence}

\begin{lstlisting}
// WRONG
#define DOUBLE(x) x * 2

int y = DOUBLE(3 + 4);  // 3 + 4 * 2 = 11, not 14!

// RIGHT
#define DOUBLE(x) ((x) * 2)

int y = DOUBLE(3 + 4);  // (3 + 4) * 2 = 14
\end{lstlisting}

\section{Useful Predefined Macros}

\begin{lstlisting}
// Standard predefined macros
__FILE__     // Current filename
__LINE__     // Current line number
__func__     // Current function name (C99)
__DATE__     // Compilation date
__TIME__     // Compilation time

// Example usage
#define LOG_LOCATION() \
    printf("At %s:%d in %s()\n", __FILE__, __LINE__, __func__)

void my_function(void) {
    LOG_LOCATION();  // Prints file, line, and function name
}

// Build info
#define VERSION_INFO() \
    printf("Built on %s at %s\n", __DATE__, __TIME__)
\end{lstlisting}

\section{Macro Best Practices}

\begin{enumerate}
    \item \textbf{Use UPPERCASE}: Makes macros obvious
    \item \textbf{Parenthesize everything}: Parameters and entire expression
    \item \textbf{Use do-while(0)}: For multi-statement macros
    \item \textbf{Avoid side effects}: Document if unavoidable
    \item \textbf{Consider inline functions}: Often better than macros
    \item \textbf{Test thoroughly}: Expand and inspect the output
\end{enumerate}

\section{When to Use Functions Instead}

\begin{lstlisting}
// Macro - no type safety
#define ADD(a, b) ((a) + (b))

// Better - inline function with type safety
static inline int add(int a, int b) {
    return a + b;
}

// Macros are better for:
// - Generic operations (works with any type)
// - Compile-time code generation
// - Conditional compilation
// - Access to __FILE__, __LINE__, etc.

// Functions are better for:
// - Type safety
// - Debugging (can step into them)
// - Complex logic
// - Avoiding multiple evaluation
\end{lstlisting}

\section{Summary}

Macros are powerful but dangerous:

\begin{itemize}
    \item Always use parentheses
    \item Use do-while(0) for multiple statements
    \item Watch out for multiple evaluation
    \item X-Macros eliminate code duplication
    \item Stringification and token pasting create code
    \item Prefer inline functions when type safety matters
\end{itemize}

Master macros, and you'll understand how professional C projects work. Just be careful—with great power comes great responsibility!

\chapter{String Handling Patterns}

\section{The Reality of C Strings}

C strings are just arrays of characters ending in \texttt{\textbackslash 0}. This simplicity is powerful but dangerous. More security vulnerabilities stem from string handling than any other source. Buffer overflows, format string attacks, SQL injection - all start with mishandled strings. If cybersecurity had a Most Wanted list, string handling bugs would be \#1 with a bullet. (A buffer overflow bullet, naturally.)

Unlike higher-level languages, C doesn't have a string "object" with methods. A string is simply a pointer to the first character, and you rely on that null terminator to know where it ends. One missing byte and your program corrupts memory, crashes, or worse - gets exploited. It's like playing Operation, except when you touch the sides, hackers get root access.

\begin{lstlisting}
// This is all a C string is:
char str[] = "Hello";
// In memory (6 bytes):
// 'H' 'e' 'l' 'l' 'o' '\0'
//  0   1   2   3   4   5

// What most programmers don't realize:
sizeof(str)    // 6 (includes null terminator)
strlen(str)    // 5 (excludes null terminator)

// The null terminator is ALWAYS there in literals
// But it's YOUR responsibility to maintain it!
\end{lstlisting}

\begin{warningbox}
The number one source of security vulnerabilities in C: forgetting the null terminator. Heartbleed (OpenSSL)? String handling. SQLSlammer worm? Buffer overflow in string code. Every major C CVE traces back to strings. If strings were a person, they'd have their own dedicated security team. And therapy sessions.
\end{warningbox}

\section{String Memory: Stack vs Heap}

\begin{lstlisting}
// Stack allocation - automatic cleanup
void func1(void) {
    char str[100];  // 100 bytes on stack
    strcpy(str, "Hello");
    // str is automatically freed when func returns
}

// Heap allocation - manual cleanup required
void func2(void) {
    char* str = malloc(100);  // 100 bytes on heap
    if (str) {
        strcpy(str, "Hello");
        free(str);  // YOU must free!
    }
}

// String literal - in read-only memory (.rodata section)
const char* func3(void) {
    return "Hello";  // OK - literal has static storage
}

// DANGEROUS - returning stack address
char* func4(void) {
    char str[100];  // On stack
    strcpy(str, "Hello");
    return str;  // BUG! Returns dangling pointer
}

// What actually happens in memory:
// Stack:   grows downward, fast, limited size (~8MB)
// Heap:    grows upward, slower, large size (GBs)
// .rodata: read-only data segment, program lifetime
// .data:   initialized data segment, program lifetime

// String literals are in .rodata:
char* s1 = "Hello";
char* s2 = "Hello";
// s1 == s2 is often TRUE! Compiler may merge identical literals
// Don't rely on this - implementation defined

// Trying to modify literal = SEGFAULT
char* s = "Hello";
s[0] = 'h';  // CRASH! Writing to read-only memory
               // The OS: "I'm gonna stop you right there"
\end{lstlisting}

\section{The Null Terminator: Source of Infinite Bugs}

\begin{lstlisting}
// Every C programmer's nightmare

// Example 1: Forgot to allocate space for null
char buf[5];
strcpy(buf, "Hello");  // BUFFER OVERFLOW!
// "Hello" is 5 chars + 1 null = 6 bytes
// buf is only 5 bytes
// Writes beyond buffer, corrupts memory
// Congratulations, you've just created a vulnerability

// Example 2: strncpy doesn't guarantee null termination
char buf[5];
strncpy(buf, "HelloWorld", 5);  // Copies "Hello"
// buf = {'H', 'e', 'l', 'l', 'o'}  NO NULL TERMINATOR!
printf("%s\n", buf);  // Undefined behavior!
// printf reads until it finds \0, could read garbage for megabytes

// Example 3: Manual null termination
char buf[6];
strncpy(buf, "HelloWorld", sizeof(buf) - 1);
buf[sizeof(buf) - 1] = '\0';  // ALWAYS do this!
// Now buf = {'H', 'e', 'l', 'l', 'o', '\0'}  SAFE!

// Example 4: Reading input
char buf[100];
fgets(buf, sizeof(buf), stdin);
// fgets DOES null-terminate, but includes newline!
// Input: "Hello\n"
// buf = {'H', 'e', 'l', 'l', 'o', '\n', '\0'}
// Need to remove \n:
buf[strcspn(buf, "\n")] = '\0';

// Example 5: Binary data (not null-terminated)
char data[100];
int n = read(fd, data, sizeof(data));
// data is NOT null-terminated!
// Don't use strlen(), strcmp() - they expect null terminator
// Use memcpy(), memcmp() with explicit length
// strlen() is the wrong tool for the job here
\end{lstlisting}

\section{String Duplication: The Right Way}

\begin{lstlisting}
#include <string.h>
#include <stdlib.h>

// WRONG - multiple bugs
char* copy_string_wrong(const char* src) {
    char* dst;              // Uninitialized pointer
    strcpy(dst, src);       // Writing to random memory!
    return dst;             // Returning garbage
}

// STILL WRONG - memory leak
char* copy_string_leak(const char* src) {
    char* dst = malloc(strlen(src) + 1);
    strcpy(dst, src);
    return dst;
    // Caller must free, but no documentation!
    // Leads to memory leaks
}

// CORRECT - with error checking
char* copy_string(const char* src) {
    if (!src) return NULL;  // Validate input

    size_t len = strlen(src);
    char* dst = malloc(len + 1);  // +1 for null terminator!

    if (!dst) return NULL;  // Check allocation

    memcpy(dst, src, len + 1);  // Copy including null
    // Or: strcpy(dst, src);

    return dst;  // Caller must free!
}

// BETTER - use POSIX strdup if available
char* str = strdup("hello");  // Allocates and copies
if (str) {
    // Use string...
    free(str);  // Must free
}

// PRO TIP: strdup implementation
char* my_strdup(const char* s) {
    if (!s) return NULL;
    size_t len = strlen(s) + 1;
    char* d = malloc(len);
    return d ? memcpy(d, s, len) : NULL;
}

// PRODUCTION: strndup for bounded copy
char* str = strndup("hello world", 5);  // Copies "hello"
// Safer than strdup for untrusted input
free(str);
\end{lstlisting}

\begin{notebox}
The \texttt{+1} for the null terminator is the most common source of string bugs. Always remember it! \texttt{strlen("Hello")} returns 5, but you need 6 bytes to store it. That one byte is like the friend you forgot to invite to your party—it WILL come back to haunt you.
\end{notebox}

\section{Safe String Operations: strncpy and Friends}

\begin{lstlisting}
#include <string.h>

char buffer[100];

// DANGEROUS - buffer overflow!
strcpy(buffer, user_input);    // What if user_input > 100 bytes?
strcat(buffer, more_input);    // Could overflow

// SAFER - bounded versions
strncpy(buffer, user_input, sizeof(buffer) - 1);
buffer[sizeof(buffer) - 1] = '\0';  // Ensure null termination

strncat(buffer, more_input, sizeof(buffer) - strlen(buffer) - 1);

// But strncpy has quirks!
char buf[10];
strncpy(buf, "Hello", 10);
// If src < n, strncpy pads with zeros
// buf = {'H','e','l','l','o','\0','\0','\0','\0','\0'}

strncpy(buf, "Hello World!", 10);
// If src >= n, NO null terminator added!
// buf = {'H','e','l','l','o',' ','W','o','r','l'}  NOT NULL-TERMINATED!

// This is why you MUST manually add null:
strncpy(buf, src, sizeof(buf) - 1);
buf[sizeof(buf) - 1] = '\0';

// strncat is safer - always null-terminates
char buf[10] = "Hello";
strncat(buf, " World", sizeof(buf) - strlen(buf) - 1);
// buf = "Hello Wor\0"  (truncated but null-terminated)
\end{lstlisting}

\subsection{Modern Safe Alternatives}

\begin{lstlisting}
// C11 bounds-checking interfaces (Annex K)
// Not widely available, but safer when they exist
#ifdef __STDC_LIB_EXT1__
    // Returns error code, always null-terminates
    errno_t strcpy_s(char* dest, rsize_t destsz, const char* src);
    errno_t strcat_s(char* dest, rsize_t destsz, const char* src);

    // Usage
    char buf[100];
    if (strcpy_s(buf, sizeof(buf), "Hello") == 0) {
        // Success - buf is guaranteed null-terminated
    }
#endif

// BSD strlcpy/strlcat (better design, widely used)
#if defined(__BSD_VISIBLE) || defined(__APPLE__)
    size_t strlcpy(char* dst, const char* src, size_t size);
    size_t strlcat(char* dst, const char* src, size_t size);

    // Always null-terminates
    // Returns strlen(src) or strlen(dst) + strlen(src)
    // Can detect truncation:
    char buf[10];
    size_t len = strlcpy(buf, "Hello World", sizeof(buf));
    if (len >= sizeof(buf)) {
        // Truncation occurred!
        // len is how much we WOULD HAVE written
    }
#endif

// Roll your own safe copy (portable)
size_t safe_strcpy(char* dst, const char* src, size_t size) {
    if (size == 0) return strlen(src);

    size_t i;
    for (i = 0; i < size - 1 && src[i]; i++) {
        dst[i] = src[i];
    }
    dst[i] = '\0';

    // Return total length of src (like strlcpy)
    while (src[i]) i++;
    return i;
}
\end{lstlisting}

\section{Buffer Overflows: How They Happen}

\begin{lstlisting}
// Classic buffer overflow vulnerability

// Vulnerable code:
void process_user_input(void) {
    char buffer[64];
    printf("Enter name: ");
    gets(buffer);  // NEVER USE gets()!
    // If user enters 100 chars, writes beyond buffer
    // Corrupts stack, can overwrite return address
    // Attacker can inject malicious code!
}

// What happens in memory (x86-64):
// Stack layout:
// [buffer 64 bytes][saved rbp 8 bytes][return address 8 bytes]
//
// User enters 80 bytes:
// [64 bytes overflow][overwrite rbp][overwrite ret address]
//
// Attacker can set return address to point to shellcode
// When function returns, executes attacker's code!

// FIX 1: Use fgets
void safe_input_fgets(void) {
    char buffer[64];
    printf("Enter name: ");
    if (fgets(buffer, sizeof(buffer), stdin)) {
        // fgets reads at most sizeof(buffer)-1 chars
        // Always null-terminates
        buffer[strcspn(buffer, "\n")] = '\0';  // Remove newline
    }
}

// FIX 2: Use scanf with width
void safe_input_scanf(void) {
    char buffer[64];
    printf("Enter name: ");
    if (scanf("%63s", buffer) == 1) {  // 63 = sizeof-1
        // Reads at most 63 chars + null
    }
}

// FIX 3: Use getline (POSIX)
void safe_input_getline(void) {
    char* buffer = NULL;
    size_t size = 0;
    printf("Enter name: ");
    ssize_t len = getline(&buffer, &size, stdin);
    if (len > 0) {
        // getline allocates buffer dynamically
        // No buffer overflow possible!
        buffer[strcspn(buffer, "\n")] = '\0';
        printf("You entered: %s\n", buffer);
        free(buffer);  // Must free!
    }
}
\end{lstlisting}

\begin{warningbox}
\texttt{gets()} is so dangerous it was REMOVED from C11 standard! Any code using it is vulnerable. Always use \texttt{fgets()} or \texttt{getline()} instead. gets() is the function equivalent of "hold my beer and watch this"—nothing good comes from it.
\end{warningbox}

\section{Format String Vulnerabilities}

\begin{lstlisting}
// Another major security issue

// VULNERABLE:
void log_message(const char* user_input) {
    printf(user_input);  // NEVER DO THIS!
    // If user_input = "%s%s%s%s%s"
    // printf reads random stack values
    // Can crash or leak sensitive data

    // If user_input = "%n"
    // Writes to arbitrary memory location
    // Can overwrite return address, execute code
}

// FIX: Use format string
void log_message_safe(const char* user_input) {
    printf("%s", user_input);  // Safe
    // printf can't interpret format specifiers in data
}

// Real-world example from actual vulnerability:
void vulnerable_logger(const char* msg) {
    fprintf(logfile, msg);  // BUG!
}

// Attacker supplies: "User %08x %08x %08x %08x %n"
// Reads stack values and writes to memory
// CVE-2000-0844, CVE-2001-0660, etc.
// (Hackers get creative when you give them a printf to play with)

// SAFE versions:
void safe_logger(const char* msg) {
    fprintf(logfile, "%s", msg);
}

void safe_logger_formatted(const char* fmt, ...) {
    // You control format string - safe
    va_list args;
    va_start(args, fmt);
    vfprintf(logfile, fmt, args);
    va_end(args);
}
\end{lstlisting}

\section{String Builder Pattern}

\begin{lstlisting}
typedef struct {
    char* buffer;
    size_t length;      // Current string length (excluding null)
    size_t capacity;    // Total buffer size
} StringBuilder;

StringBuilder* sb_create(size_t initial_capacity) {
    if (initial_capacity == 0) {
        initial_capacity = 64;  // Default
    }

    StringBuilder* sb = malloc(sizeof(StringBuilder));
    if (!sb) return NULL;

    sb->buffer = malloc(initial_capacity);
    if (!sb->buffer) {
        free(sb);
        return NULL;
    }

    sb->length = 0;
    sb->capacity = initial_capacity;
    sb->buffer[0] = '\0';

    return sb;
}

// Grow buffer to at least new_capacity
static int sb_grow(StringBuilder* sb, size_t new_capacity) {
    if (new_capacity <= sb->capacity) {
        return 0;  // Already large enough
    }

    // Grow by 1.5x or to new_capacity, whichever is larger
    size_t grow = sb->capacity + sb->capacity / 2;
    if (grow < new_capacity) {
        grow = new_capacity;
    }

    char* new_buf = realloc(sb->buffer, grow);
    if (!new_buf) return -1;

    sb->buffer = new_buf;
    sb->capacity = grow;
    return 0;
}

int sb_append(StringBuilder* sb, const char* str) {
    if (!sb || !str) return -1;

    size_t str_len = strlen(str);
    size_t needed = sb->length + str_len + 1;  // +1 for null

    if (needed > sb->capacity) {
        if (sb_grow(sb, needed) != 0) {
            return -1;
        }
    }

    // memcpy is faster than strcpy for known length
    memcpy(sb->buffer + sb->length, str, str_len + 1);
    sb->length += str_len;

    return 0;
}

int sb_append_char(StringBuilder* sb, char c) {
    if (!sb) return -1;

    if (sb->length + 2 > sb->capacity) {  // +2 for char and null
        if (sb_grow(sb, sb->length + 2) != 0) {
            return -1;
        }
    }

    sb->buffer[sb->length++] = c;
    sb->buffer[sb->length] = '\0';

    return 0;
}

int sb_append_format(StringBuilder* sb, const char* fmt, ...) {
    if (!sb || !fmt) return -1;

    va_list args;
    va_start(args, fmt);

    // Calculate needed size
    va_list args_copy;
    va_copy(args_copy, args);
    int needed = vsnprintf(NULL, 0, fmt, args_copy);
    va_end(args_copy);

    if (needed < 0) {
        va_end(args);
        return -1;
    }

    // Ensure capacity
    if (sb->length + needed + 1 > sb->capacity) {
        if (sb_grow(sb, sb->length + needed + 1) != 0) {
            va_end(args);
            return -1;
        }
    }

    // Write formatted string
    vsnprintf(sb->buffer + sb->length, needed + 1, fmt, args);
    sb->length += needed;
    va_end(args);

    return 0;
}

void sb_clear(StringBuilder* sb) {
    if (sb) {
        sb->length = 0;
        if (sb->buffer) {
            sb->buffer[0] = '\0';
        }
    }
}

char* sb_to_string(StringBuilder* sb) {
    if (!sb || !sb->buffer) return NULL;
    return strdup(sb->buffer);  // Caller must free
}

void sb_destroy(StringBuilder* sb) {
    if (sb) {
        free(sb->buffer);
        free(sb);
    }
}

// Usage example
void demo_string_builder(void) {
    StringBuilder* sb = sb_create(16);

    sb_append(sb, "Hello, ");
    sb_append(sb, "World");
    sb_append_char(sb, '!');
    sb_append_format(sb, " Number: %d", 42);

    printf("%s\n", sb->buffer);  // "Hello, World! Number: 42"

    // Efficient for building large strings
    for (int i = 0; i < 1000; i++) {
        sb_append_format(sb, " %d", i);
    }

    char* result = sb_to_string(sb);
    sb_destroy(sb);

    // Use result...
    free(result);
}
\end{lstlisting}

\section{Const Correctness for Strings}

\begin{lstlisting}
// Use const for strings you won't modify
void print_string(const char* str) {
    if (!str) return;
    printf("%s\n", str);
    // str[0] = 'X';  // Won't compile - str is const
}

// Non-const for strings you will modify
void uppercase_string(char* str) {
    if (!str) return;
    for (int i = 0; str[i]; i++) {
        str[i] = toupper((unsigned char)str[i]);
    }
}

// Return const for string literals
const char* get_error_message(int code) {
    switch (code) {
        case 0: return "Success";
        case 1: return "Error";
        case 2: return "Fatal error";
        default: return "Unknown error";
    }
    // All returns are string literals - const is correct
}

// Common mistake: discarding const
void bad_example(void) {
    const char* msg = "Hello";
    char* ptr = (char*)msg;  // Cast away const - BAD!
    ptr[0] = 'h';            // Undefined behavior! May crash
}

// Correct pattern: const input, non-const output
char* string_duplicate_upper(const char* src) {
    if (!src) return NULL;

    size_t len = strlen(src);
    char* dst = malloc(len + 1);
    if (!dst) return NULL;

    for (size_t i = 0; i <= len; i++) {
        dst[i] = toupper((unsigned char)src[i]);
    }

    return dst;  // Caller can modify returned string
}
\end{lstlisting}

\begin{tipbox}
Using \texttt{const} correctly helps catch bugs at compile time. If you try to modify a \texttt{const char*}, the compiler will warn you. This prevents accidentally modifying string literals, which is undefined behavior.
\end{tipbox}

\section{String Tokenization}

\begin{lstlisting}
// Using strtok (modifies original string - NOT THREAD-SAFE)
void demo_strtok(void) {
    char str[] = "apple,banana,cherry";  // Must be mutable
    char* token = strtok(str, ",");
    while (token != NULL) {
        printf("%s\n", token);
        token = strtok(NULL, ",");  // NULL continues previous
    }
    // str is now destroyed: "apple\0banana\0cherry"
}

// Better: strtok_r (reentrant, thread-safe)
void demo_strtok_r(void) {
    char str[] = "apple,banana,cherry";
    char* saveptr;  // Keeps state between calls
    char* token = strtok_r(str, ",", &saveptr);
    while (token != NULL) {
        printf("%s\n", token);
        token = strtok_r(NULL, ",", &saveptr);
    }
}

// Nested tokenization with strtok_r
void parse_csv(const char* data) {
    char* data_copy = strdup(data);
    char* line_save;
    char* line = strtok_r(data_copy, "\n", &line_save);

    while (line) {
        char* field_save;
        char* field = strtok_r(line, ",", &field_save);

        while (field) {
            printf("Field: %s\n", field);
            field = strtok_r(NULL, ",", &field_save);
        }

        line = strtok_r(NULL, "\n", &line_save);
    }

    free(data_copy);
}

// Custom tokenizer (non-destructive)
typedef struct {
    const char* start;
    const char* end;
} StringView;

int next_token(const char** str, const char* delim, StringView* token) {
    if (!str || !*str || !delim || !token) return 0;

    // Skip leading delimiters
    while (**str && strchr(delim, **str)) {
        (*str)++;
    }

    if (!**str) return 0;  // End of string

    token->start = *str;

    // Find end of token
    while (**str && !strchr(delim, **str)) {
        (*str)++;
    }

    token->end = *str;
    return 1;
}

// Usage - doesn't modify original
void demo_string_view(void) {
    const char* str = "apple,banana,cherry";
    StringView token;

    while (next_token(&str, ",", &token)) {
        printf("%.*s\n", (int)(token.end - token.start), token.start);
    }
    // Original string unchanged!
}
\end{lstlisting}

\section{String Comparison Patterns}

\begin{lstlisting}
// Basic comparison
int compare_strings(const char* s1, const char* s2) {
    if (!s1 && !s2) return 0;   // Both NULL - equal
    if (!s1) return -1;          // s1 NULL - less
    if (!s2) return 1;           // s2 NULL - greater

    return strcmp(s1, s2);
}

// strcmp returns:
// < 0 if s1 < s2
// = 0 if s1 == s2
// > 0 if s1 > s2

// WRONG way to use strcmp:
if (strcmp(s1, s2)) {  // BAD! Works but confusing
    // Not equal
}

// CORRECT and clear:
if (strcmp(s1, s2) == 0) {  // Equal
    // Strings match
}

// Case-insensitive (POSIX)
#include <strings.h>  // Note: strings.h, not string.h!
if (strcasecmp(s1, s2) == 0) {
    // Equal ignoring case
}

// Windows equivalent:
#ifdef _WIN32
    if (_stricmp(s1, s2) == 0) {
        // Equal ignoring case
    }
#endif

// Prefix check
if (strncmp(s1, s2, n) == 0) {
    // First n characters match
}

// Check if string starts with prefix
int starts_with(const char* str, const char* prefix) {
    if (!str || !prefix) return 0;
    size_t prefix_len = strlen(prefix);
    return strncmp(str, prefix, prefix_len) == 0;
}

// Check if string ends with suffix
int ends_with(const char* str, const char* suffix) {
    if (!str || !suffix) return 0;
    size_t str_len = strlen(str);
    size_t suffix_len = strlen(suffix);
    if (suffix_len > str_len) return 0;
    return strcmp(str + str_len - suffix_len, suffix) == 0;
}

// Contains check
if (strstr(haystack, needle) != NULL) {
    // haystack contains needle
}

// Find position
const char* pos = strstr(haystack, needle);
if (pos) {
    ptrdiff_t index = pos - haystack;
    printf("Found at index %td\n", index);
}
\end{lstlisting}

\section{String to Number Conversion}

\begin{lstlisting}
#include <stdlib.h>
#include <errno.h>
#include <limits.h>

// WRONG - no error checking
int value = atoi(str);  // Returns 0 on error AND for "0"!

// CORRECT - use strtol with error checking
int safe_atoi(const char* str, int* out) {
    if (!str || !out) return -1;

    // Skip leading whitespace
    while (isspace((unsigned char)*str)) str++;

    if (*str == '\0') return -1;  // Empty string

    char* endptr;
    errno = 0;
    long val = strtol(str, &endptr, 10);

    // Check for errors
    if (errno == ERANGE) {
        return -1;  // Overflow/underflow
    }
    if (endptr == str) {
        return -1;  // No conversion performed
    }
    if (*endptr != '\0') {
        return -1;  // Extra characters after number
    }
    if (val < INT_MIN || val > INT_MAX) {
        return -1;  // Out of int range
    }

    *out = (int)val;
    return 0;
}

// Parse with different bases
long hex_value;
char* end;
hex_value = strtol("0xFF", &end, 16);  // Hexadecimal
hex_value = strtol("0377", &end, 8);   // Octal
hex_value = strtol("1010", &end, 2);   // Binary

// Auto-detect base (0 means auto)
hex_value = strtol("0xFF", &end, 0);   // Detects hex (0x prefix)
hex_value = strtol("077", &end, 0);    // Detects octal (0 prefix)
hex_value = strtol("123", &end, 0);    // Decimal

// Floating point
double parse_double(const char* str, double* out) {
    if (!str || !out) return -1;

    char* endptr;
    errno = 0;
    double val = strtod(str, &endptr);

    if (errno == ERANGE) {
        return -1;  // Overflow/underflow
    }
    if (endptr == str) {
        return -1;  // No conversion
    }

    *out = val;
    return 0;
}

// Usage
int value;
if (safe_atoi("123", &value) == 0) {
    printf("Parsed: %d\n", value);
} else {
    printf("Parse error\n");
}

double d;
if (parse_double("3.14159", &d) == 0) {
    printf("Parsed: %f\n", d);
}
\end{lstlisting}

\section{String Searching and Manipulation}

\begin{lstlisting}
// Find character
char* pos = strchr(str, 'x');      // First occurrence
char* pos = strrchr(str, 'x');     // Last occurrence

if (pos) {
    *pos = '\0';  // Truncate at first 'x'
}

// Find any of multiple characters
char* pos = strpbrk(str, "abc");  // First of 'a', 'b', or 'c'

// Count characters not in set
size_t n = strcspn(str, " \t\n");  // Length until whitespace

// Count characters in set
size_t n = strspn(str, "0123456789");  // Length of numeric prefix

// Find substring
char* pos = strstr(haystack, needle);

// Case-insensitive search (custom implementation)
char* stristr(const char* haystack, const char* needle) {
    if (!haystack || !needle) return NULL;

    size_t needle_len = strlen(needle);
    if (needle_len == 0) return (char*)haystack;

    for (; *haystack; haystack++) {
        if (strncasecmp(haystack, needle, needle_len) == 0) {
            return (char*)haystack;
        }
    }
    return NULL;
}

// Replace all occurrences
char* str_replace_all(const char* str, const char* old, const char* new) {
    if (!str || !old || !new) return NULL;

    size_t old_len = strlen(old);
    size_t new_len = strlen(new);

    // Count occurrences
    int count = 0;
    const char* p = str;
    while ((p = strstr(p, old)) != NULL) {
        count++;
        p += old_len;
    }

    if (count == 0) return strdup(str);

    // Allocate new string
    size_t result_len = strlen(str) + count * (new_len - old_len);
    char* result = malloc(result_len + 1);
    if (!result) return NULL;

    // Copy with replacements
    char* dst = result;
    const char* src = str;
    while (*src) {
        const char* found = strstr(src, old);
        if (found) {
            size_t prefix_len = found - src;
            memcpy(dst, src, prefix_len);
            dst += prefix_len;
            memcpy(dst, new, new_len);
            dst += new_len;
            src = found + old_len;
        } else {
            strcpy(dst, src);
            break;
        }
    }

    return result;
}
\end{lstlisting}

\section{String Trimming}

\begin{lstlisting}
#include <ctype.h>

// Trim whitespace from start and end (in-place)
void str_trim(char* str) {
    if (!str) return;

    // Trim leading whitespace
    char* start = str;
    while (*start && isspace((unsigned char)*start)) {
        start++;
    }

    // If all whitespace, make empty
    if (*start == '\0') {
        str[0] = '\0';
        return;
    }

    // Trim trailing whitespace
    char* end = start + strlen(start) - 1;
    while (end > start && isspace((unsigned char)*end)) {
        end--;
    }
    end[1] = '\0';

    // Move trimmed string to beginning
    if (start != str) {
        memmove(str, start, end - start + 2);  // +2 for char and null
    }
}

// Non-destructive trim (returns view)
StringView str_trim_view(const char* str) {
    StringView view = {NULL, NULL};
    if (!str) return view;

    // Skip leading whitespace
    while (*str && isspace((unsigned char)*str)) {
        str++;
    }
    view.start = str;

    // Find end (last non-whitespace)
    const char* end = str;
    const char* last_non_space = str - 1;

    while (*end) {
        if (!isspace((unsigned char)*end)) {
            last_non_space = end;
        }
        end++;
    }

    view.end = last_non_space + 1;
    return view;
}

// Usage
char str[] = "  Hello World  ";
str_trim(str);
printf("'%s'\n", str);  // 'Hello World'
\end{lstlisting}

\section{Unicode and UTF-8}

\begin{lstlisting}
// C strings are byte arrays - encoding-agnostic
// UTF-8 is backwards-compatible with ASCII
// Multi-byte characters are common in modern applications

// ASCII:  1 byte per character, values 0-127
// Latin1: 1 byte per character, values 0-255
// UTF-8:  1-4 bytes per character, variable-width encoding

// UTF-8 encoding:
// 0xxxxxxx                           1 byte  (ASCII)
// 110xxxxx 10xxxxxx                  2 bytes
// 1110xxxx 10xxxxxx 10xxxxxx         3 bytes
// 11110xxx 10xxxxxx 10xxxxxx 10xxxxxx 4 bytes

// Example: "Hello World" in UTF-8 with Chinese characters
// 'H'     'e'     'l'     'l'     'o'     ' '
// 0x48    0x65    0x6C    0x6C    0x6F    0x20
// Chinese 'shi4' (U+4E16)   Chinese 'jie4' (U+754C)
// 0xE4 0xB8 0x96             0xE7 0x95 0x8C

const char* utf8_str = "Hello World";  // Imagine Chinese chars here
// strlen(utf8_str) = 13 bytes (not 8 characters!)

// Count UTF-8 characters (not bytes)
size_t utf8_strlen(const char* str) {
    size_t count = 0;
    while (*str) {
        if ((*str & 0xC0) != 0x80) {  // Not a continuation byte
            count++;
        }
        str++;
    }
    return count;
}

// Validate UTF-8
int is_valid_utf8(const char* str) {
    while (*str) {
        unsigned char c = *str;

        if (c <= 0x7F) {  // 1-byte (ASCII)
            str++;
        } else if ((c & 0xE0) == 0xC0) {  // 2-byte
            if ((str[1] & 0xC0) != 0x80) return 0;
            str += 2;
        } else if ((c & 0xF0) == 0xE0) {  // 3-byte
            if ((str[1] & 0xC0) != 0x80) return 0;
            if ((str[2] & 0xC0) != 0x80) return 0;
            str += 3;
        } else if ((c & 0xF8) == 0xF0) {  // 4-byte
            if ((str[1] & 0xC0) != 0x80) return 0;
            if ((str[2] & 0xC0) != 0x80) return 0;
            if ((str[3] & 0xC0) != 0x80) return 0;
            str += 4;
        } else {
            return 0;  // Invalid UTF-8
        }
    }
    return 1;
}

// Decode UTF-8 character
int utf8_decode(const char* str, uint32_t* out) {
    unsigned char c = *str;

    if (c <= 0x7F) {
        *out = c;
        return 1;
    } else if ((c & 0xE0) == 0xC0) {
        *out = ((c & 0x1F) << 6) | (str[1] & 0x3F);
        return 2;
    } else if ((c & 0xF0) == 0xE0) {
        *out = ((c & 0x0F) << 12) | ((str[1] & 0x3F) << 6) | (str[2] & 0x3F);
        return 3;
    } else if ((c & 0xF8) == 0xF0) {
        *out = ((c & 0x07) << 18) | ((str[1] & 0x3F) << 12) |
               ((str[2] & 0x3F) << 6) | (str[3] & 0x3F);
        return 4;
    }
    return -1;  // Invalid
}

// IMPORTANT: Many C string functions don't work correctly with UTF-8
// strlen() counts bytes, not characters (surprise!)
// toupper()/tolower() only work for ASCII (sorry, rest of world)
// strchr() works (searching for ASCII in UTF-8 is safe)
// strstr() works (substring search is byte-based)
// Welcome to internationalization fun times

// For proper UTF-8 handling, use a library:
// - ICU (International Components for Unicode)
// - libunistring
// - utf8proc
\end{lstlisting}

\section{Common String Bugs in Production}

\begin{lstlisting}
// Bug 1: Off-by-one errors
char buf[5];
strncpy(buf, "hello", 5);  // WRONG! No space for null
// Correct:
char buf[6];
strncpy(buf, "hello", sizeof(buf) - 1);
buf[sizeof(buf) - 1] = '\0';

// Bug 2: Returning stack addresses
char* create_greeting(void) {
    char buf[100];
    strcpy(buf, "Hello");
    return buf;  // BUG! buf is destroyed on return
}
// Fix: use malloc or static
// This is the "give them directions to a demolished building" bug

// Bug 3: Modifying string literals
char* str = "Hello";
str[0] = 'h';  // CRASH! Writing to read-only memory
// Fix: use char str[] = "Hello";

// Bug 4: Not checking for NULL
void print(const char* str) {
    printf("%s\n", str);  // CRASH if str is NULL
}
// Fix: if (!str) return;
// Optimist: "The caller will never pass NULL"
// Pessimist: "The caller WILL pass NULL"
// C programmer: "When, not if"

// Bug 5: Mixing signed/unsigned char
char c = 200;  // Negative on systems where char is signed
if (isspace(c)) { ... }  // BUG! Must cast to unsigned char
// Correct:
if (isspace((unsigned char)c)) { ... }

// Bug 6: Assuming ASCII
// strlen() works for UTF-8 (counts bytes)
// But character count != byte count

// Bug 7: Race conditions with strtok
// strtok uses static state - not thread-safe
// Use strtok_r instead

// Bug 8: Integer overflow in size calculation
size_t len = strlen(str);
char* buf = malloc(len + 1);  // What if len == SIZE_MAX?
// Check: if (len == SIZE_MAX) return NULL;
\end{lstlisting}

\section{Summary}

String handling in C requires extreme discipline:

\begin{itemize}
    \item Always allocate \texttt{strlen(s) + 1} bytes for the null terminator
    \item Use bounded functions (\texttt{strncpy}, \texttt{strncat}, \texttt{snprintf})
    \item Manually null-terminate after \texttt{strncpy}
    \item Never use \texttt{gets()} - use \texttt{fgets()} or \texttt{getline()}
    \item Never pass user input directly to \texttt{printf()} family
    \item Use \texttt{const} for read-only strings
    \item Check for NULL before using strings
    \item Prefer \texttt{strtol} over \texttt{atoi} for conversions
    \item Use \texttt{strtok\_r} instead of \texttt{strtok} for thread safety
    \item Remember UTF-8 is multi-byte - byte count != character count
    \item Validate all input strings for length and content
    \item Use string builders for efficient concatenation
    \item Cast to \texttt{unsigned char} when using \texttt{ctype.h} functions
\end{itemize}

Security implications:

\begin{itemize}
    \item Buffer overflows are the \#1 source of vulnerabilities
    \item Format string bugs can leak memory or execute code
    \item SQL injection stems from improper string escaping
    \item Path traversal attacks use string manipulation
    \item Every major CVE in C code involves strings
\end{itemize}

Master these patterns, and string bugs will become rare in your code. But remember: C strings are dangerous by design. One missing byte, one forgotten null terminator, and your program crashes or gets exploited. Stay vigilant! (And maybe keep a stress ball handy for when you're debugging string issues at 2 AM.)

\begin{tipbox}
The Heartbleed vulnerability (CVE-2014-0160) was a string handling bug. A missing bounds check in OpenSSL allowed reading 64KB of memory. This leaked passwords, private keys, and sensitive data from millions of servers. One string bug. Billions of dollars in damage. This is why string handling matters. (Also why security researchers have trust issues with memcpy.)
\end{tipbox}

\chapter{Error Handling Patterns}

\section{The Challenge of Error Handling in C}

Unlike languages with exceptions, C requires explicit error handling. Every function that can fail must communicate that failure to its caller, and callers must check for errors. This is tedious but powerful—you always know exactly where errors can occur.

But here's what 20 years of C programming teaches you: error handling is where most bugs hide. Not in algorithms, not in data structures, but in the unglamorous code that handles failures. Production systems fail not because of clever code, but because someone forgot to check a return value. (Usually at 3 AM on a Friday. Always on a Friday.)

The Linux kernel has more error handling code than any other kind. OpenSSL's worst bugs weren't in crypto algorithms, but in error paths. Every major C codebase spends 50-70\% of its code handling errors. This chapter teaches you the patterns that separate hobby code from production systems. (Spoiler: it's mostly the boring stuff that nobody wants to write but everybody needs.)

\section{The errno Pattern: How UNIX Does It}

The traditional UNIX approach: set a global error code.

\begin{lstlisting}
#include <errno.h>
#include <string.h>
#include <stdio.h>

// Return -1 on error, set errno
int my_function(const char* filename) {
    FILE* f = fopen(filename, "r");
    if (!f) {
        // errno is already set by fopen
        // Could be ENOENT, EACCES, EMFILE, etc.
        return -1;
    }

    // ... do work ...

    fclose(f);
    return 0;
}

// Usage - ALWAYS check return value
if (my_function("data.txt") == -1) {
    // strerror converts errno to human-readable string
    fprintf(stderr, "Error: %s\n", strerror(errno));
    // perror prints to stderr with prefix
    perror("my_function");
}
\end{lstlisting}

\begin{notebox}
The \texttt{errno} variable is thread-local in modern systems (POSIX.1-2001), so it's safe to use in multithreaded programs. On older systems, it was a global variable, which caused race conditions! (The '90s were a wild time for multithreading. We don't talk about it much.)
\end{notebox}

\subsection{How errno Actually Works}

Here's what textbooks don't tell you:

\begin{lstlisting}
// In older systems (pre-threading):
extern int errno;  // Global variable - NOT THREAD SAFE!

// In modern systems (glibc, etc.):
extern int *__errno_location(void);
#define errno (*__errno_location())

// Each thread has its own errno!
// __errno_location() returns pointer to thread-local storage

// In practice:
void thread_func(void* arg) {
    int fd = open("file.txt", O_RDONLY);
    if (fd == -1) {
        // This errno is THIS THREAD's errno
        // Other threads' errno is unaffected
        printf("Error: %s\n", strerror(errno));
    }
}

// You must check errno IMMEDIATELY after error
// Don't do this:
if (open("file.txt", O_RDONLY) == -1) {
    printf("Something\n");  // May call functions that change errno!
    printf("Error: %s\n", strerror(errno));  // WRONG! May be different errno
}

// Do this:
int fd = open("file.txt", O_RDONLY);
if (fd == -1) {
    int saved_errno = errno;  // Save immediately
    printf("Something\n");
    printf("Error: %s\n", strerror(saved_errno));  // Correct
}
\end{lstlisting}

\subsection{Common errno Values Every C Programmer Must Know}

\begin{lstlisting}
#include <errno.h>

// File/Directory errors
ENOENT   // No such file or directory (most common!)
EACCES   // Permission denied
EISDIR   // Is a directory (tried to open dir as file)
ENOTDIR  // Not a directory (tried to cd to file)
EEXIST   // File exists (when O_CREAT | O_EXCL)
ENAMETOOLONG  // Filename too long

// Resource errors
ENOMEM   // Out of memory (malloc failed)
EMFILE   // Too many open files (process limit)
ENFILE   // Too many open files (system limit)
ENOSPC   // No space left on device
EDQUOT   // Disk quota exceeded

// I/O errors
EAGAIN   // Resource temporarily unavailable (non-blocking I/O)
EWOULDBLOCK  // Same as EAGAIN on most systems
EINTR    // Interrupted system call (by signal)
EIO      // I/O error (hardware problem)
EPIPE    // Broken pipe (wrote to closed socket)

// Invalid input
EINVAL   // Invalid argument
EBADF    // Bad file descriptor
EFAULT   // Bad address (invalid pointer)
ERANGE   // Result too large (math functions)

// Network errors
ECONNREFUSED  // Connection refused
ETIMEDOUT     // Connection timed out
ENETUNREACH   // Network unreachable
EHOSTUNREACH  // Host unreachable

// Operation errors
EPERM    // Operation not permitted (need root)
EBUSY    // Device or resource busy
EDEADLK  // Resource deadlock avoided
ENODEV   // No such device
EXDEV    // Cross-device link (can't mv across filesystems)
\end{lstlisting}

\subsection{The EINTR Problem: Restarting System Calls}

Here's a production gotcha that bites everyone:

\begin{lstlisting}
// WRONG - doesn't handle EINTR
ssize_t n = read(fd, buffer, size);
if (n == -1) {
    fprintf(stderr, "Read failed: %s\n", strerror(errno));
    return -1;
}

// PROBLEM: If a signal arrives during read(), it returns -1
// with errno=EINTR. This is NOT an error - just retry!

// CORRECT - restart interrupted system calls
ssize_t read_restart(int fd, void* buf, size_t count) {
    ssize_t n;
    do {
        n = read(fd, buf, count);
    } while (n == -1 && errno == EINTR);
    return n;
}

// Or use SA_RESTART flag when setting up signal handlers:
struct sigaction sa;
sa.sa_handler = my_signal_handler;
sa.sa_flags = SA_RESTART;  // Automatically restart system calls
sigaction(SIGINT, &sa, NULL);

// Functions that can return EINTR:
// - read(), write(), open()
// - accept(), connect(), recv(), send()
// - wait(), waitpid()
// - sleep(), nanosleep()
// - select(), poll(), epoll_wait()

// This is REQUIRED for robust server code!
// Ignore EINTR, enjoy mysterious production failures. Your choice.
\end{lstlisting}

\section{Return Codes: The Foundation}

\subsection{Pattern 1: Return Value, Special Value for Error}

\begin{lstlisting}
// Works when you have a sentinel value
FILE* fopen(const char* path, const char* mode);
// Returns: Valid pointer or NULL on error

void* malloc(size_t size);
// Returns: Valid pointer or NULL on error

int open(const char* path, int flags);
// Returns: File descriptor (>=0) or -1 on error

// PROBLEM: What if all values are valid?
int parse_int(const char* str);
// Can't return -1 for error - might be valid input!
// Can't return 0 - might be valid input!

// SOLUTION: Use output parameter pattern (see below)
\end{lstlisting}

\subsection{Pattern 2: Return Status, Output via Pointer (The Professional Way)}

\begin{lstlisting}
// Return status code, output via pointer
// 0 = success, negative = error
int parse_int_safe(const char* str, int* result) {
    if (!str || !result) return -EINVAL;  // Invalid argument

    char* endptr;
    errno = 0;
    long val = strtol(str, &endptr, 10);

    if (errno == ERANGE) {
        return -ERANGE;  // Overflow
    }
    if (endptr == str) {
        return -EINVAL;  // No conversion
    }
    if (*endptr != '\0') {
        return -EINVAL;  // Extra characters
    }
    if (val < INT_MIN || val > INT_MAX) {
        return -ERANGE;  // Out of range
    }

    *result = (int)val;
    return 0;  // Success
}

// Usage
int value;
int ret = parse_int_safe("123", &value);
if (ret == 0) {
    printf("Parsed: %d\n", value);
} else {
    fprintf(stderr, "Parse error: %s\n", strerror(-ret));
}

// This pattern is used throughout:
// - POSIX APIs (pthread_create, etc.)
// - Linux kernel
// - Most professional C libraries
\end{lstlisting}

\begin{tipbox}
Linux kernel convention: Return negative errno values for errors (e.g., -EINVAL, -ENOMEM). This makes it easy to propagate errors while maintaining errno semantics. User space does the opposite (return -1, set errno), but the kernel way is often cleaner for library code.
\end{tipbox}

\subsection{Pattern 3: Multiple Output Parameters}

\begin{lstlisting}
// Return status, multiple outputs via pointers
int parse_url(const char* url,
              char** scheme,    // Output: "http", "https", etc.
              char** host,      // Output: "example.com"
              int* port,        // Output: 80, 443, etc.
              char** path) {    // Output: "/index.html"

    if (!url) return -EINVAL;

    // Validate outputs are provided
    if (!scheme || !host || !port || !path) {
        return -EINVAL;
    }

    // Parse URL...
    *scheme = strdup("http");
    *host = strdup("example.com");
    *port = 80;
    *path = strdup("/index.html");

    return 0;  // Success
}

// Usage
char *scheme, *host, *path;
int port;

if (parse_url("http://example.com/index.html",
              &scheme, &host, &port, &path) == 0) {
    printf("Scheme: %s, Host: %s, Port: %d, Path: %s\n",
           scheme, host, port, path);

    // Caller must free allocated strings
    free(scheme);
    free(host);
    free(path);
} else {
    fprintf(stderr, "Invalid URL\n");
}
\end{lstlisting}

\section{The Goto Cleanup Pattern (Linux Kernel Style)}

One of the few legitimate uses of \texttt{goto} in modern C:

\begin{lstlisting}
int process_file(const char* filename) {
    FILE* input = NULL;
    FILE* output = NULL;
    char* buffer = NULL;
    int result = -1;

    input = fopen(filename, "r");
    if (!input) {
        fprintf(stderr, "Cannot open input: %s\n", strerror(errno));
        goto cleanup;
    }

    output = fopen("output.txt", "w");
    if (!output) {
        fprintf(stderr, "Cannot open output: %s\n", strerror(errno));
        goto cleanup;
    }

    buffer = malloc(4096);
    if (!buffer) {
        fprintf(stderr, "Out of memory\n");
        goto cleanup;
    }

    // ... do work ...
    // If error occurs, just goto cleanup

    size_t n = fread(buffer, 1, 4096, input);
    if (ferror(input)) {
        fprintf(stderr, "Read error\n");
        goto cleanup;
    }

    if (fwrite(buffer, 1, n, output) != n) {
        fprintf(stderr, "Write error\n");
        goto cleanup;
    }

    result = 0;  // Success

cleanup:
    // Cleanup happens in REVERSE ORDER of allocation
    // This is critical! (LIFO - like stack unwinding)
    free(buffer);
    if (output) fclose(output);
    if (input) fclose(input);

    return result;
}
\end{lstlisting}

\begin{notebox}
This is the STANDARD pattern in the Linux kernel! Search the kernel source for "goto out" or "goto error". Linus Torvalds himself advocates this pattern. It ensures cleanup happens correctly and avoids deeply nested error handling. When Linus says goto is okay, goto is okay. (Though your CS professor might still have nightmares.)
\end{notebox}

\subsection{Why Goto Is Better Than Nested Ifs}

\begin{lstlisting}
// WITHOUT goto - deeply nested, hard to maintain
int process_file_nested(const char* filename) {
    FILE* input = fopen(filename, "r");
    if (input) {
        FILE* output = fopen("output.txt", "w");
        if (output) {
            char* buffer = malloc(4096);
            if (buffer) {
                // ... do work ...
                size_t n = fread(buffer, 1, 4096, input);
                if (!ferror(input)) {
                    if (fwrite(buffer, 1, n, output) == n) {
                        // Success - way down here
                        free(buffer);
                        fclose(output);
                        fclose(input);
                        return 0;
                    }
                }
                free(buffer);
            }
            fclose(output);
        }
        fclose(input);
    }
    return -1;
}

// Problems with nested approach:
// 1. Rightward drift - code disappears off screen
// 2. Hard to add new resources (good luck finding where to insert it)
// 3. Easy to mess up cleanup order (and you will)
// 4. Success path is buried deep (like treasure, but less fun)
// 5. Code duplication for cleanup (copy-paste is not a design pattern)
\end{lstlisting}

\subsection{Advanced: Multiple Cleanup Labels}

\begin{lstlisting}
// For complex cleanup with different paths
int complex_operation(void) {
    int fd = -1;
    char* buffer = NULL;
    struct data* obj = NULL;
    int result = -1;

    fd = open("file.txt", O_RDONLY);
    if (fd == -1) {
        goto out;  // Nothing to clean up
    }

    buffer = malloc(4096);
    if (!buffer) {
        goto close_fd;  // Only close fd
    }

    obj = create_object();
    if (!obj) {
        goto free_buffer;  // Free buffer and close fd
    }

    // ... do work ...

    if (some_operation(obj) != 0) {
        goto destroy_object;  // Full cleanup
    }

    result = 0;  // Success

destroy_object:
    destroy_object(obj);
free_buffer:
    free(buffer);
close_fd:
    close(fd);
out:
    return result;
}

// This is how the kernel handles complex cleanup
// Labels named by what they clean up
\end{lstlisting}

\section{Error Context Pattern: Rich Error Information}

\begin{lstlisting}
// Error structure with context
typedef struct {
    int code;           // Error code (errno-like)
    char message[256];  // Human-readable message
    const char* file;   // Source file where error occurred
    int line;           // Line number
    const char* func;   // Function name
} Error;

// Macro for setting errors with source location
#define SET_ERROR(err, code_, fmt, ...) do { \
    if (err) { \
        (err)->code = (code_); \
        snprintf((err)->message, sizeof((err)->message), \
                 fmt, ##__VA_ARGS__); \
        (err)->file = __FILE__; \
        (err)->line = __LINE__; \
        (err)->func = __func__; \
    } \
} while(0)

int risky_operation(const char* input, Error* err) {
    if (!input) {
        SET_ERROR(err, -EINVAL, "Input is NULL");
        return -1;
    }

    if (strlen(input) == 0) {
        SET_ERROR(err, -EINVAL, "Input is empty");
        return -1;
    }

    FILE* f = fopen(input, "r");
    if (!f) {
        SET_ERROR(err, -errno, "Cannot open '%s': %s",
                  input, strerror(errno));
        return -1;
    }

    // ... do work ...

    fclose(f);
    return 0;
}

// Usage with detailed error reporting
Error err;
if (risky_operation(data, &err) != 0) {
    fprintf(stderr, "Error %d: %s\n", err.code, err.message);
    fprintf(stderr, "  at %s() in %s:%d\n",
            err.func, err.file, err.line);
}
\end{lstlisting}

\subsection{Error Chains: Preserving Error Context}

\begin{lstlisting}
// Chain errors as they propagate up the stack
#define MAX_ERROR_CHAIN 10

typedef struct {
    int depth;
    struct {
        int code;
        char message[128];
        const char* file;
        int line;
    } chain[MAX_ERROR_CHAIN];
} ErrorChain;

#define ERROR_CHAIN_PUSH(ec, code_, fmt, ...) do { \
    if ((ec) && (ec)->depth < MAX_ERROR_CHAIN) { \
        int idx = (ec)->depth++; \
        (ec)->chain[idx].code = (code_); \
        snprintf((ec)->chain[idx].message, \
                 sizeof((ec)->chain[idx].message), \
                 fmt, ##__VA_ARGS__); \
        (ec)->chain[idx].file = __FILE__; \
        (ec)->chain[idx].line = __LINE__; \
    } \
} while(0)

// Low-level function
int read_config_file(const char* path, ErrorChain* ec) {
    FILE* f = fopen(path, "r");
    if (!f) {
        ERROR_CHAIN_PUSH(ec, errno, "fopen failed: %s", strerror(errno));
        return -1;
    }
    // ...
    fclose(f);
    return 0;
}

// Mid-level function
int load_config(const char* path, ErrorChain* ec) {
    if (read_config_file(path, ec) != 0) {
        ERROR_CHAIN_PUSH(ec, -1, "Failed to load config from '%s'", path);
        return -1;
    }
    return 0;
}

// High-level function
int initialize_system(ErrorChain* ec) {
    if (load_config("/etc/myapp.conf", ec) != 0) {
        ERROR_CHAIN_PUSH(ec, -1, "System initialization failed");
        return -1;
    }
    return 0;
}

// Usage - get full error trace!
ErrorChain ec = {0};
if (initialize_system(&ec) != 0) {
    fprintf(stderr, "Error trace (most recent first):\n");
    for (int i = ec.depth - 1; i >= 0; i--) {
        fprintf(stderr, "  [%d] %s (at %s:%d)\n",
                ec.chain[i].code,
                ec.chain[i].message,
                ec.chain[i].file,
                ec.chain[i].line);
    }
}

// Output:
// Error trace (most recent first):
//   [-1] System initialization failed (at main.c:123)
//   [-1] Failed to load config from '/etc/myapp.conf' (at config.c:45)
//   [2] fopen failed: No such file or directory (at config.c:12)
\end{lstlisting}

\section{Result Type Pattern}

\begin{lstlisting}
// Generic result type with status and value
typedef struct {
    int status;  // 0 = success, <0 = error code
    int value;   // Valid only if status == 0
} IntResult;

typedef struct {
    int status;
    void* ptr;
} PtrResult;

IntResult divide(int a, int b) {
    IntResult result;
    if (b == 0) {
        result.status = -EINVAL;
        result.value = 0;
        return result;
    }
    result.status = 0;
    result.value = a / b;
    return result;
}

// Usage
IntResult r = divide(10, 2);
if (r.status == 0) {
    printf("Result: %d\n", r.value);
} else {
    fprintf(stderr, "Error: %s\n", strerror(-r.status));
}
\end{lstlisting}

\subsection{Generic Result with Union}

\begin{lstlisting}
// Tagged union for different result types
typedef enum {
    RESULT_INT,
    RESULT_DOUBLE,
    RESULT_PTR,
    RESULT_STRING
} ResultType;

typedef struct {
    int status;
    ResultType type;
    union {
        int int_value;
        double double_value;
        void* ptr_value;
        char string_value[256];
    } data;
} Result;

Result read_config_int(const char* key) {
    Result r = {0};
    r.type = RESULT_INT;

    // ... read config ...
    int value;
    if (found) {
        r.status = 0;
        r.data.int_value = value;
    } else {
        r.status = -ENOENT;
    }
    return r;
}

// Usage
Result r = read_config_int("port");
if (r.status == 0) {
    printf("Port: %d\n", r.data.int_value);
}
\end{lstlisting}

\section{Error Callback Pattern: Let Users Handle Errors}

\begin{lstlisting}
// Error severity levels
typedef enum {
    ERR_DEBUG,
    ERR_INFO,
    ERR_WARN,
    ERR_ERROR,
    ERR_FATAL
} ErrorLevel;

typedef void (*ErrorHandler)(ErrorLevel level, int code,
                             const char* message,
                             void* user_data);

typedef struct {
    ErrorHandler handler;
    void* user_data;
    ErrorLevel min_level;  // Only report >= this level
} Library;

void library_init(Library* lib, ErrorHandler handler,
                  void* context, ErrorLevel min_level) {
    lib->handler = handler;
    lib->user_data = context;
    lib->min_level = min_level;
}

void library_report_error(Library* lib, ErrorLevel level,
                          int code, const char* fmt, ...) {
    if (!lib || level < lib->min_level) return;

    char message[512];
    va_list args;
    va_start(args, fmt);
    vsnprintf(message, sizeof(message), fmt, args);
    va_end(args);

    if (lib->handler) {
        lib->handler(level, code, message, lib->user_data);
    } else {
        // Default: print to stderr
        const char* level_str[] = {
            "DEBUG", "INFO", "WARN", "ERROR", "FATAL"
        };
        fprintf(stderr, "[%s] %s (code %d)\n",
                level_str[level], message, code);
    }

    if (level == ERR_FATAL) {
        abort();  // Fatal errors terminate
    }
}

// User's error handler - log to file
void my_error_handler(ErrorLevel level, int code,
                      const char* msg, void* data) {
    FILE* log = (FILE*)data;
    time_t now = time(NULL);
    fprintf(log, "[%ld] Level %d, Code %d: %s\n",
            now, level, code, msg);
    fflush(log);
}

// Usage
FILE* log = fopen("error.log", "a");
Library lib;
library_init(&lib, my_error_handler, log, ERR_WARN);

// Now all errors go to log file
library_report_error(&lib, ERR_ERROR, errno,
                     "Failed to connect: %s", strerror(errno));
\end{lstlisting}

\section{Defensive Programming: Preconditions and Postconditions}

\begin{lstlisting}
#include <assert.h>

// Use assert for programmer errors (bugs)
// Use return codes for runtime errors (user input, I/O, etc.)

void process_data(const char* data, size_t len) {
    // Preconditions - these are bugs if violated
    assert(data != NULL);  // Programmer error - should never happen
    assert(len > 0);       // Programmer error - caller's fault

    // But still validate for production
    #ifndef NDEBUG
    if (!data || len == 0) {
        fprintf(stderr, "BUG: Invalid parameters to process_data\n");
        abort();
    }
    #endif

    // Runtime errors - these CAN happen
    int fd = open("output.txt", O_WRONLY);
    if (fd == -1) {
        // This is NOT a bug - file might not exist
        fprintf(stderr, "Error: %s\n", strerror(errno));
        return;  // Handle gracefully
    }

    // ... process data ...

    close(fd);

    // Postcondition
    assert(all_data_processed);  // Verify our logic is correct
}

// Design by Contract macros
#define REQUIRE(cond) do { \
    if (!(cond)) { \
        fprintf(stderr, "Precondition failed: %s\n" \
                        "  at %s:%d in %s\n", \
                #cond, __FILE__, __LINE__, __func__); \
        abort(); \
    } \
} while(0)

#define ENSURE(cond) do { \
    if (!(cond)) { \
        fprintf(stderr, "Postcondition failed: %s\n" \
                        "  at %s:%d in %s\n", \
                #cond, __FILE__, __LINE__, __func__); \
        abort(); \
    } \
} while(0)

#define INVARIANT(cond) ENSURE(cond)

// Usage
int divide(int a, int b) {
    REQUIRE(b != 0);  // Precondition

    int result = a / b;

    ENSURE(result * b <= a);  // Postcondition (integer division)
    ENSURE(result * b + (a % b) == a);  // Exact postcondition

    return result;
}
\end{lstlisting}

\section{Retry Logic: Handling Transient Failures}

\begin{lstlisting}
#include <unistd.h>
#include <time.h>
#include <math.h>

// Simple retry with fixed delay
int retry_operation(int (*operation)(void* data), void* data,
                    int max_retries, int delay_seconds) {
    for (int i = 0; i < max_retries; i++) {
        int result = operation(data);
        if (result == 0) {
            return 0;  // Success
        }

        // Don't sleep after last attempt
        if (i < max_retries - 1) {
            fprintf(stderr, "Attempt %d/%d failed, retrying in %ds...\n",
                    i + 1, max_retries, delay_seconds);
            sleep(delay_seconds);
        }
    }

    fprintf(stderr, "Failed after %d attempts\n", max_retries);
    return -1;  // All retries failed
}

// Exponential backoff with jitter (for network operations)
int retry_with_backoff(int (*operation)(void* data), void* data,
                       int max_retries) {
    int base_delay_ms = 100;  // Start with 100ms
    int max_delay_ms = 30000; // Cap at 30 seconds

    srand(time(NULL));

    for (int i = 0; i < max_retries; i++) {
        int result = operation(data);
        if (result == 0) {
            return 0;  // Success
        }

        if (i < max_retries - 1) {
            // Exponential backoff: 100ms, 200ms, 400ms, 800ms, ...
            int delay_ms = base_delay_ms * (1 << i);
            if (delay_ms > max_delay_ms) {
                delay_ms = max_delay_ms;
            }

            // Add jitter: random +/-25% to prevent thundering herd
            // (When all servers retry at exactly the same time, nobody wins)
            int jitter = (rand() % (delay_ms / 2)) - (delay_ms / 4);
            delay_ms += jitter;

            fprintf(stderr, "Attempt %d/%d failed, waiting %dms...\n",
                    i + 1, max_retries, delay_ms);

            usleep(delay_ms * 1000);  // usleep takes microseconds
        }
    }

    return -1;
}

// Retry only on specific errors
int retry_on_error(int (*operation)(void* data), void* data,
                   int max_retries, const int* retry_errors, int num_errors) {
    for (int i = 0; i < max_retries; i++) {
        errno = 0;
        int result = operation(data);
        if (result == 0) {
            return 0;  // Success
        }

        // Check if this error is retryable
        int should_retry = 0;
        for (int j = 0; j < num_errors; j++) {
            if (errno == retry_errors[j]) {
                should_retry = 1;
                break;
            }
        }

        if (!should_retry) {
            fprintf(stderr, "Non-retryable error: %s\n", strerror(errno));
            return -1;  // Give up immediately
        }

        if (i < max_retries - 1) {
            fprintf(stderr, "Retryable error (%s), attempt %d/%d\n",
                    strerror(errno), i + 1, max_retries);
            sleep(1);
        }
    }

    return -1;
}

// Usage
int connect_to_server(void* data) {
    // ... connection logic ...
    return -1;  // Simulate failure
}

// Retry only on temporary network errors
int retryable_errors[] = {ETIMEDOUT, ECONNREFUSED, ENETUNREACH};
if (retry_on_error(connect_to_server, server_info, 5,
                   retryable_errors, 3) != 0) {
    fprintf(stderr, "Cannot connect after retries\n");
}
\end{lstlisting}

\section{Error Recovery Strategies}

\begin{lstlisting}
typedef enum {
    RECOVERY_RETRY,
    RECOVERY_USE_DEFAULT,
    RECOVERY_USE_CACHE,
    RECOVERY_SKIP,
    RECOVERY_ABORT
} RecoveryStrategy;

typedef struct {
    RecoveryStrategy strategy;
    int max_retries;
    void* default_value;
    void* cache;
} RecoveryPolicy;

int load_data_with_recovery(const char* path, Data* data,
                             RecoveryPolicy* policy) {
    int result = read_data(path, data);

    if (result == 0) {
        return 0;  // Success
    }

    // Error occurred - apply recovery strategy
    fprintf(stderr, "Error loading %s: %s\n", path, strerror(errno));

    switch (policy->strategy) {
        case RECOVERY_RETRY:
            fprintf(stderr, "Retrying...\n");
            for (int i = 0; i < policy->max_retries; i++) {
                sleep(1);
                result = read_data(path, data);
                if (result == 0) {
                    fprintf(stderr, "Retry succeeded\n");
                    return 0;
                }
            }
            fprintf(stderr, "All retries failed\n");
            return -1;

        case RECOVERY_USE_DEFAULT:
            fprintf(stderr, "Using default value\n");
            if (policy->default_value) {
                memcpy(data, policy->default_value, sizeof(Data));
                return 0;  // Treat as success
            }
            return -1;

        case RECOVERY_USE_CACHE:
            fprintf(stderr, "Using cached value\n");
            if (policy->cache) {
                memcpy(data, policy->cache, sizeof(Data));
                return 0;
            }
            return -1;

        case RECOVERY_SKIP:
            fprintf(stderr, "Skipping failed operation\n");
            memset(data, 0, sizeof(Data));
            return 0;  // Pretend success

        case RECOVERY_ABORT:
            fprintf(stderr, "Fatal error, aborting\n");
            abort();

        default:
            return -1;
    }
}

// Usage
Data data;
Data default_data = {/* defaults */};
RecoveryPolicy policy = {
    .strategy = RECOVERY_USE_DEFAULT,
    .default_value = &default_data
};

load_data_with_recovery("/etc/config.txt", &data, &policy);
\end{lstlisting}

\section{Logging Errors: Production-Grade Logging}

\begin{lstlisting}
typedef enum {
    LOG_TRACE,
    LOG_DEBUG,
    LOG_INFO,
    LOG_WARN,
    LOG_ERROR,
    LOG_FATAL
} LogLevel;

typedef struct {
    FILE* file;
    LogLevel level;
    int use_colors;     // ANSI colors for terminal
    int include_time;
    int include_location;  // File:line
    pthread_mutex_t mutex;  // Thread-safe logging
} Logger;

static Logger g_logger = {
    .file = NULL,
    .level = LOG_INFO,
    .use_colors = 0,
    .include_time = 1,
    .include_location = 1,
    .mutex = PTHREAD_MUTEX_INITIALIZER
};

void log_init(const char* path, LogLevel level, int use_colors) {
    g_logger.file = path ? fopen(path, "a") : stderr;
    g_logger.level = level;
    g_logger.use_colors = use_colors && isatty(fileno(g_logger.file));
    g_logger.include_time = 1;
    g_logger.include_location = 1;
}

void log_message(LogLevel level, const char* file, int line,
                 const char* func, const char* fmt, ...) {
    if (level < g_logger.level) return;

    pthread_mutex_lock(&g_logger.mutex);  // Because race conditions in logging are... ironic

    FILE* out = g_logger.file ? g_logger.file : stderr;

    // ANSI color codes
    const char* colors[] = {
        "\033[0;37m",  // TRACE - white
        "\033[0;36m",  // DEBUG - cyan
        "\033[0;32m",  // INFO - green
        "\033[0;33m",  // WARN - yellow
        "\033[0;31m",  // ERROR - red
        "\033[1;31m"   // FATAL - bold red
    };
    const char* reset = "\033[0m";

    const char* level_str[] = {
        "TRACE", "DEBUG", "INFO", "WARN", "ERROR", "FATAL"
    };

    // Timestamp
    if (g_logger.include_time) {
        time_t now = time(NULL);
        struct tm* tm_info = localtime(&now);
        char time_buf[64];
        strftime(time_buf, sizeof(time_buf), "%Y-%m-%d %H:%M:%S", tm_info);
        fprintf(out, "[%s] ", time_buf);
    }

    // Level with color
    if (g_logger.use_colors) {
        fprintf(out, "%s[%-5s]%s ", colors[level], level_str[level], reset);
    } else {
        fprintf(out, "[%-5s] ", level_str[level]);
    }

    // Location
    if (g_logger.include_location) {
        fprintf(out, "%s:%d in %s(): ", file, line, func);
    }

    // Message
    va_list args;
    va_start(args, fmt);
    vfprintf(out, fmt, args);
    va_end(args);

    fprintf(out, "\n");
    fflush(out);

    pthread_mutex_unlock(&g_logger.mutex);

    if (level == LOG_FATAL) {
        abort();
    }
}

// Convenient macros
#define LOG_TRACE(...) \
    log_message(LOG_TRACE, __FILE__, __LINE__, __func__, __VA_ARGS__)
#define LOG_DEBUG(...) \
    log_message(LOG_DEBUG, __FILE__, __LINE__, __func__, __VA_ARGS__)
#define LOG_INFO(...) \
    log_message(LOG_INFO, __FILE__, __LINE__, __func__, __VA_ARGS__)
#define LOG_WARN(...) \
    log_message(LOG_WARN, __FILE__, __LINE__, __func__, __VA_ARGS__)
#define LOG_ERROR(...) \
    log_message(LOG_ERROR, __FILE__, __LINE__, __func__, __VA_ARGS__)
#define LOG_FATAL(...) \
    log_message(LOG_FATAL, __FILE__, __LINE__, __func__, __VA_ARGS__)

void log_close(void) {
    if (g_logger.file && g_logger.file != stderr) {
        fclose(g_logger.file);
        g_logger.file = NULL;
    }
}

// Usage
int main(void) {
    log_init("app.log", LOG_DEBUG, 1);

    LOG_INFO("Application started");
    LOG_DEBUG("Debug value: %d", 42);

    int fd = open("missing.txt", O_RDONLY);
    if (fd == -1) {
        LOG_ERROR("Cannot open file: %s", strerror(errno));
    }

    LOG_WARN("This is a warning");

    log_close();
    return 0;
}

// Output:
// [2024-01-15 10:30:45] [INFO ] main.c:123 in main(): Application started
// [2024-01-15 10:30:45] [DEBUG] main.c:124 in main(): Debug value: 42
// [2024-01-15 10:30:45] [ERROR] main.c:128 in main(): Cannot open file: No such file or directory
\end{lstlisting}

\section{Best Practices from Production Systems}

\begin{enumerate}
    \item \textbf{Always check return values}: Every function that can fail (yes, ALL of them)
    \item \textbf{Check errno immediately}: Save it if you need to call other functions
    \item \textbf{Handle EINTR}: Restart interrupted system calls (or enjoy mysterious failures)
    \item \textbf{Document error conditions}: In comments and headers
    \item \textbf{Be consistent}: Use the same pattern throughout your codebase
    \item \textbf{Use goto for cleanup}: Don't fight it - embrace the Linux kernel way
    \item \textbf{Provide context}: Help users understand what went wrong and where
    \item \textbf{Log errors}: At minimum, log them with timestamps and context
    \item \textbf{Fail fast}: Detect errors as early as possible (before they metastasize)
    \item \textbf{Validate inputs}: Check preconditions at function entry
    \item \textbf{Test error paths}: Most bugs hide in error handling code (ironic, isn't it?)
    \item \textbf{Use different severities}: DEBUG/INFO/WARN/ERROR/FATAL
    \item \textbf{Clean up in reverse order}: LIFO - like stack unwinding
    \item \textbf{Consider retry logic}: For transient failures (network, I/O)
    \item \textbf{Thread safety matters}: Protect shared error state with mutexes
\end{enumerate}

\section{Summary}

Error handling in C requires discipline and patterns:

\begin{itemize}
    \item Use return codes consistently (0 success, negative error)
    \item Use \texttt{errno} for system call errors, check immediately
    \item Always restart interrupted system calls (EINTR)
    \item Use goto cleanup pattern for complex functions
    \item Provide error context (code, message, location)
    \item Implement retry logic with exponential backoff
    \item Log errors with timestamps and severity levels
    \item Test error paths as thoroughly as success paths
    \item Document error conditions in API
    \item Clean up resources in all paths (success and error)
\end{itemize}

Good error handling is what separates toy programs from production code. The Linux kernel has more error handling than any other kind of code. OpenSSL's worst bugs were in error paths. Redis, nginx, PostgreSQL - all spend 50-70\% of code on error handling. (The glamorous life of a C programmer: writing more cleanup code than actual features.)

Master these patterns, and your code will be robust, debuggable, and maintainable. Your future self (and your team) will thank you when things go wrong at 3 AM and the logs tell you exactly what happened and where.

\begin{tipbox}
Error handling is not glamorous. It's tedious, verbose, and feels like busywork. But it's the difference between a demo and a product. Between "works on my machine" and "runs in production for years." Learn these patterns, make them automatic, and you'll write C code that professionals respect. (And you'll sleep better at night. Probably. Maybe. At least you'll know where to look when things inevitably break.)
\end{tipbox}

\chapter{Memory Management Idioms}

\section{The Reality of Memory in C}

Memory management in C is where theory meets brutal reality. You have complete control, which means complete responsibility. One mistake---a dangling pointer, a double-free, a tiny leak---and your production server crashes at 2 AM. Or worse, it doesn't crash immediately. It corrupts data silently for weeks until someone notices the financial reports are wrong.

Think of memory management like managing a parking lot. Each malloc() is like a car entering and getting a parking spot. Each free() is like a car leaving, making that spot available again. Simple enough, right? But what if:
\begin{itemize}
    \item You give out the same spot to two different cars (double allocation)
    \item Someone tries to drive away a car that already left (use-after-free)
    \item Cars just pile up and never leave, blocking new cars (memory leak)
    \item You forget which spot a car is in and can't tell it to leave (lost pointer)
\end{itemize}

These aren't just theoretical problems---they're the daily reality of C programming. Here's what separates hobby C programmers from professionals: hobby programmers think malloc() and free() are the whole story. Professionals know that's just the beginning. This chapter covers the patterns, tools, and hard-won wisdom that keep production systems stable.

\begin{warningbox}
Memory bugs are the hardest to debug. They're non-deterministic, they manifest far from their cause, and they corrupt state in ways that make the debugger lie to you. The patterns in this chapter aren't just optimizations---they're survival techniques.
\end{warningbox}

\section{The Ownership Pattern: Who Frees What?}

The \#1 cause of memory bugs: unclear ownership. Who is responsible for freeing this pointer? If you can't answer that immediately, you have a bug waiting to happen.

Imagine you borrow a book from a friend. The ownership is clear: it's their book, you're just using it temporarily. You don't throw it away when you're done---you return it to them. But what if someone hands you a book and walks away without a word? Is it yours now? Should you keep it? Throw it away? Give it to someone else? This confusion is exactly what happens with unclear memory ownership in C.

Let's start with the most common source of memory bugs: ambiguous ownership. Look at these function signatures and ask yourself: who is responsible for freeing the returned pointer?

\begin{lstlisting}
// AMBIGUOUS - who frees the returned string?
// Does this return a pointer to a static buffer?
// Or does it allocate memory that I must free?
// Without documentation, you're guessing. And guessing wrong means leaks or crashes.
char* get_username(int user_id);

// CLEAR - caller must free
char* create_username(int user_id);

// CLEAR - function borrows, doesn't own
void print_username(const char* username);

// CLEAR - function takes ownership
void consume_username(char* username);

// CLEAR - function returns borrowed reference
const char* get_cached_username(int user_id);
\end{lstlisting}

\subsection{Naming Conventions That Save Lives}

Professional C codebases use naming conventions to communicate ownership. These aren't just style preferences---they're critical safety mechanisms. When you see a function name, you should immediately know what it does with memory.

Think of function names as instructions on a package. "create\_" is like "assembly required---you must dispose." "print\_" is like "for display only---do not consume." "destroy\_" is like "dispose of properly." These naming patterns tell you exactly what to do with the memory, without having to read documentation or guess.

\begin{lstlisting}
// Allocating functions (caller must free):
// If you see create_, new_, alloc_, or make_ - you OWN that pointer
// You allocated it, you must free it. No exceptions.
char* create_string(const char* src);
User* new_user(const char* name);
Buffer* alloc_buffer(size_t size);
Message* make_message(const char* text);

// Borrowing functions (doesn't free):
void print_user(const User* user);
int validate_buffer(const Buffer* buf);
void log_message(const Message* msg);

// Consuming functions (takes ownership, will free):
void destroy_user(User* user);
void free_buffer(Buffer* buf);
void delete_message(Message* msg);
void consume_string(char* str);  // frees str

// Returning borrowed references (don't free!):
const char* user_get_name(const User* user);
const char* get_error_string(int code);

// Real-world example - very clear ownership
FILE* fopen(const char* path, const char* mode);  // Returns owned
int fclose(FILE* stream);  // Takes ownership, frees
\end{lstlisting}

\begin{tipbox}
In professional codebases, ownership is documented in every function comment. "Caller must free", "Borrows pointer", "Takes ownership"---these phrases should be everywhere. Future you (at 3 AM debugging a customer's crash dump) will be grateful.
\end{tipbox}

\subsection{The Transfer Pattern}

Sometimes ownership needs to change hands during an operation. This is tricky because it violates the simple "who allocates, frees" rule. The key is to make transfers explicit and document them heavily.

Imagine a relay race: the first runner has the baton (ownership), then hands it off to the second runner. The first runner no longer has it---ownership transferred. Same with memory: sometimes a function takes something you own, does something with it, and gives you something else back. The original thing is gone (freed), but you now own the new thing. It's like trading in your old car for store credit---the car is gone, but now you have money that's yours to spend (or free, in programming terms).

\begin{lstlisting}
// Ownership transfer - carefully documented
// This pattern is common in parsers, compilers, and data structure libraries
typedef struct {
    char* data;
    size_t size;
} Buffer;

// Creates buffer - caller owns it
Buffer* buffer_create(size_t size) {
    Buffer* buf = malloc(sizeof(Buffer));
    if (!buf) return NULL;

    buf->data = malloc(size);
    if (!buf->data) {
        free(buf);
        return NULL;
    }

    buf->size = size;
    return buf;
}

// Takes ownership of buffer, transfers ownership of data
// This is the "take" pattern: we take the buffer (and free it),
// but we give you the data (and you must free it)
// Caller must free returned pointer, but NOT the buffer
char* buffer_take_data(Buffer* buf) {
    if (!buf) return NULL;

    char* data = buf->data;
    buf->data = NULL;  // Transfer ownership
    buf->size = 0;

    free(buf);  // Free container, but not data
    return data;  // Caller now owns data
}

// Usage
Buffer* buf = buffer_create(1024);
strcpy(buf->data, "Hello");

char* data = buffer_take_data(buf);  // buf is freed, data is ours
// buf is now invalid, don't use it
printf("%s\n", data);
free(data);  // We must free data
\end{lstlisting}

\section{RAII in C: Automatic Cleanup}

C doesn't have destructors, but GCC and Clang have a solution: the cleanup attribute. This is one of those compiler extensions that changes how you write C. Once you use it, you'll never want to go back to manual cleanup.

The idea is simple: mark a variable with a cleanup function, and the compiler automatically calls that function when the variable goes out of scope. It's like C++ RAII, but you have to opt-in per variable.

Think of it like a hotel room: when you check out, housekeeping automatically comes to clean up. You don't have to remember to call housekeeping yourself---it happens automatically when you leave. The cleanup attribute does the same thing for your variables: when the variable "checks out" (goes out of scope), cleanup happens automatically. No matter how you leave the function---return normally, return early from an if statement, whatever---cleanup always happens.

\begin{lstlisting}
// The cleanup attribute - GCC/Clang extension
// This tells the compiler: "When this variable goes out of scope,
// call this function with a pointer to the variable"
#define CLEANUP(func) __attribute__((cleanup(func)))

// Cleanup functions - note the pointer-to-pointer
// Why pointer-to-pointer? Because cleanup gets the ADDRESS of the variable
// So for "FILE* f", cleanup receives "FILE** fp"
void cleanup_file(FILE** fp) {
    if (fp && *fp) {
        fclose(*fp);
        *fp = NULL;  // Prevent double-close
    }
}

void cleanup_string(char** str) {
    if (str) {
        free(*str);
        *str = NULL;  // Prevent double-free
    }
}

void cleanup_fd(int* fd) {
    if (fd && *fd >= 0) {
        close(*fd);
        *fd = -1;  // Mark as closed
    }
}

// Usage - automatic cleanup!
void process_file(const char* path) {
    FILE* CLEANUP(cleanup_file) f = fopen(path, "r");
    if (!f) {
        return;  // cleanup_file called automatically
    }

    char* CLEANUP(cleanup_string) buffer = malloc(4096);
    if (!buffer) {
        return;  // Both f and buffer cleaned up
    }

    int CLEANUP(cleanup_fd) outfd = open("output.txt", O_WRONLY);
    if (outfd < 0) {
        return;  // All three cleaned up in reverse order
    }

    // Do work...

    // If we reach here or return early, everything is cleaned up
    // No goto cleanup needed!
}

// This is how systemd, many Linux utilities work
// Also used in kernel code (with different macros)

// The magic here: no matter how you exit this function (return, goto, exception),
// the cleanup functions are called. In reverse order of declaration.
// It's like stack unwinding, but done by the compiler at compile time.
\end{lstlisting}

\begin{notebox}
The cleanup attribute is a GCC/Clang extension, not standard C. But it's widely supported and used in production code (systemd, GNOME, many Linux projects). Variables are cleaned up in reverse order of declaration---LIFO, just like stack unwinding.
\end{notebox}

\section{Pool Allocator: Fast Allocation, Fast Deallocation}

When you're allocating thousands of small objects of similar size, malloc() becomes a bottleneck. Why? Because malloc() is general-purpose---it has to handle any size, any alignment, any pattern. That generality has cost.

Pool allocators trade generality for speed. You pre-allocate a big chunk of memory and hand out pieces of it. Allocation is a simple pointer bump---no searching free lists, no coalescing blocks, no metadata overhead. It's O(1) and cache-friendly.

Imagine a restaurant during lunch rush. Normal malloc() is like each customer ordering a custom meal---the chef has to prepare each one individually, checking ingredients, measuring portions, plating carefully. Slow! A pool allocator is like a buffet: everything is pre-made in a big batch, and people just grab what they need. Super fast! The trade-off? The buffet only works if everyone wants similar food (similar-sized allocations), and you can't take food back to the kitchen one plate at a time---you clear the whole buffet at once when lunch is over.

The trade-off? You can't free individual allocations. You free the whole pool at once. This works perfectly for request-scoped allocations (web servers), frame-scoped allocations (games), or parse-scoped allocations (compilers).

\begin{lstlisting}
// Simple bump-pointer pool allocator
// This is the simplest possible allocator, and often the fastest
typedef struct {
    void* memory;      // Pre-allocated block (malloc'd once)
    size_t size;       // Total size
    size_t used;       // Bytes used
    size_t alignment;  // Alignment requirement
} MemoryPool;

MemoryPool* pool_create(size_t size, size_t alignment) {
    if (alignment == 0) alignment = 8;  // Default

    MemoryPool* pool = malloc(sizeof(MemoryPool));
    if (!pool) return NULL;

    pool->memory = malloc(size);
    if (!pool->memory) {
        free(pool);
        return NULL;
    }

    pool->size = size;
    pool->used = 0;
    pool->alignment = alignment;

    return pool;
}

void* pool_alloc(MemoryPool* pool, size_t size) {
    if (!pool || size == 0) return NULL;

    // Align size to pool alignment
    // Why align? CPU loads/stores are faster when data is aligned to
    // natural boundaries (4-byte ints on 4-byte boundaries, etc.)
    // This bit-twiddling rounds up to the next multiple of alignment
    size_t aligned_size = (size + pool->alignment - 1) &
                          ~(pool->alignment - 1);

    // Check if we have space
    if (pool->used + aligned_size > pool->size) {
        return NULL;  // Pool exhausted
    }

    // Bump pointer allocation - super fast!
    // No searching, no bookkeeping, just arithmetic
    // This is why it's called "bump pointer" - we just bump it forward
    void* ptr = (char*)pool->memory + pool->used;
    pool->used += aligned_size;

    return ptr;
}

// Can't free individual allocations - that's the point!
// The entire design relies on not tracking individual allocations
// Free everything at once by resetting the pointer
void pool_reset(MemoryPool* pool) {
    if (pool) {
        pool->used = 0;  // Just reset the pointer
        // All allocations are now invalid
    }
}

void pool_destroy(MemoryPool* pool) {
    if (pool) {
        free(pool->memory);
        free(pool);
    }
}

// Real-world example: request handling
// This is exactly how high-performance web servers work
void handle_request(Request* req) {
    // Create pool for this request
    MemoryPool* pool = pool_create(1024 * 1024, 8);  // 1MB

    // Allocate request-scoped data
    // All these allocations are O(1) pointer bumps
    // No fragmentation, no searching, no overhead
    char* buffer = pool_alloc(pool, 4096);
    ParsedRequest* parsed = pool_alloc(pool, sizeof(ParsedRequest));
    Response* response = pool_alloc(pool, sizeof(Response));

    // ... process request ...

    // Free everything at once - O(1)
    pool_destroy(pool);
    // Much faster than freeing each allocation individually
}
\end{lstlisting}

\begin{tipbox}
Pool allocators are perfect for request-scoped allocations (web servers, game frames, parsers). Allocation is O(1) bump-pointer, deallocation is O(1) reset. nginx, Apache, game engines all use variants of this pattern. (Though be careful---accessing freed memory after pool\_reset is instant undefined behavior.)
\end{tipbox}

\subsection{Real Production Pattern: Per-Request Pools}

\begin{lstlisting}
// How web servers actually do it
typedef struct {
    MemoryPool* pool;
    // ... request data ...
} RequestContext;

// Wrapper function to make code cleaner
// Now all code just calls request_alloc() and doesn't worry about pools
void* request_alloc(RequestContext* ctx, size_t size) {
    return pool_alloc(ctx->pool, size);
}

// All allocations use request_alloc
void handle_http_request(RequestContext* ctx) {
    // Everything allocated from request pool
    char* headers = request_alloc(ctx, 2048);
    char* body = request_alloc(ctx, 8192);
    ParsedURL* url = request_alloc(ctx, sizeof(ParsedURL));

    // ... handle request ...

    // At end of request, destroy entire pool
    // No individual frees needed!
}

// This is how nginx gets such good performance
\end{lstlisting}

\section{Arena Allocator: Growing Pools}

Pool allocators have a fixed size. What if you don't know how much you'll need? Arena allocators grow automatically.

An arena is like a pool, but when it runs out of space, it allocates another block and keeps going. You get the speed of pool allocation with the flexibility of dynamic sizing. The blocks form a linked list, and you allocate from the current block until it's full, then add a new block.

Think of an arena allocator like a notebook for taking notes during a lecture. You start with one page (block), fill it up with notes (allocations), then flip to a new page when you run out of room. At the end of the lecture, you can tear out all the pages at once and recycle them---you don't erase each line individually. Fast writing, fast cleanup. The arena keeps adding new "pages" as needed, but cleans up everything at once.

This is perfect for parsers, compilers, and any code that builds large data structures during a phase, then throws them all away. Clang uses arenas for AST nodes---allocate millions of nodes during parsing, free them all at once after code generation.

\begin{lstlisting}
#define ARENA_BLOCK_SIZE (64 * 1024)  // 64KB blocks
// This size is a trade-off: too small = too many allocations
// too large = wasted space. 64KB is a common sweet spot.

typedef struct ArenaBlock {
    struct ArenaBlock* next;
    size_t used;
    size_t size;
    char data[];  // Flexible array member
} ArenaBlock;

typedef struct {
    ArenaBlock* current;
    ArenaBlock* first;
    size_t total_allocated;
} Arena;

Arena* arena_create(void) {
    Arena* arena = malloc(sizeof(Arena));
    if (!arena) return NULL;

    // Allocate first block
    ArenaBlock* block = malloc(sizeof(ArenaBlock) + ARENA_BLOCK_SIZE);
    if (!block) {
        free(arena);
        return NULL;
    }

    block->next = NULL;
    block->used = 0;
    block->size = ARENA_BLOCK_SIZE;

    arena->current = block;
    arena->first = block;
    arena->total_allocated = ARENA_BLOCK_SIZE;

    return arena;
}

void* arena_alloc(Arena* arena, size_t size) {
    if (!arena || size == 0) return NULL;

    // Align to 8 bytes
    size = (size + 7) & ~7UL;

    // Check if current block has space
    // If not, we'll allocate a new block
    if (arena->current->used + size > arena->current->size) {
        // Need a new block
        // Determine block size
        // Usually the default, but if someone requests a huge allocation,
        // give them a block exactly that size (don't waste space)
        size_t block_size = ARENA_BLOCK_SIZE;
        if (size > block_size) {
            block_size = size;  // Large allocation gets its own block
        }

        ArenaBlock* block = malloc(sizeof(ArenaBlock) + block_size);
        if (!block) return NULL;

        block->next = NULL;
        block->used = 0;
        block->size = block_size;

        // Link to chain
        arena->current->next = block;
        arena->current = block;
        arena->total_allocated += block_size;
    }

    // Allocate from current block
    void* ptr = arena->current->data + arena->current->used;
    arena->current->used += size;

    return ptr;
}

void arena_reset(Arena* arena) {
    if (!arena) return;

    // Reset all blocks but keep them allocated
    for (ArenaBlock* block = arena->first; block; block = block->next) {
        block->used = 0;
    }

    arena->current = arena->first;
}

void arena_destroy(Arena* arena) {
    if (!arena) return;

    // Free all blocks
    ArenaBlock* block = arena->first;
    while (block) {
        ArenaBlock* next = block->next;
        free(block);
        block = next;
    }

    free(arena);
}

// Statistics for debugging/profiling
void arena_stats(Arena* arena) {
    if (!arena) return;

    size_t num_blocks = 0;
    size_t total_used = 0;
    size_t total_wasted = 0;

    for (ArenaBlock* b = arena->first; b; b = b->next) {
        num_blocks++;
        total_used += b->used;
        total_wasted += (b->size - b->used);
    }

    printf("Arena: %zu blocks, %zu allocated, %zu used, %zu wasted\n",
           num_blocks, arena->total_allocated, total_used, total_wasted);
}

// Real-world usage: compiler/parser
void parse_file(const char* path) {
    Arena* arena = arena_create();

    // Parse creates AST nodes - all from arena
    ASTNode* root = parse(path, arena);

    // Process AST...
    analyze(root);
    codegen(root);

    // Destroy entire AST in one go
    arena_destroy(arena);
    // Much faster than traversing tree and freeing each node
}
\end{lstlisting}

\begin{notebox}
Arena allocators are used in compilers (LLVM, GCC), game engines, and any code that builds large temporary data structures. Clang compiles faster partly because it uses arenas for AST nodes---no individual frees during compilation. (Though memory usage can grow large---trade-off between speed and memory.)
\end{notebox}

\section{Reference Counting: Shared Ownership}

When multiple owners need the same data, reference counting solves the "who frees it?" problem. Instead of transferring ownership, we share it. Each owner increments the reference count when they take a reference, and decrements it when they're done. The last owner to decrement (reaching zero) frees the memory.

Imagine a shared apartment with roommates. There's a shared Netflix account that everyone uses. Each roommate who wants to use it "retains" it (increments the count). When someone moves out, they "release" it (decrement the count). As long as someone is still using it (count > 0), you keep paying for the subscription. When the last roommate moves out (count reaches 0), you cancel the subscription (free the memory). Nobody has to coordinate who's responsible---the last person out automatically handles cleanup.

This is how COM works on Windows, how Python's memory management works, and how Objective-C's ARC works. It's simple, deterministic, and solves a lot of problems. But it has gotchas (circular references, atomic overhead in multithreaded code).

\begin{lstlisting}
typedef struct {
    int ref_count;     // Number of owners
    size_t size;
    char data[];       // Flexible array member
} RefCountedBuffer;

RefCountedBuffer* buffer_create(size_t size) {
    RefCountedBuffer* buf = malloc(sizeof(RefCountedBuffer) + size);
    if (buf) {
        buf->ref_count = 1;  // Creator owns it
        // Important: starts at 1, not 0! The creator is the first owner.
        buf->size = size;
        memset(buf->data, 0, size);
    }
    return buf;
}

// Increment reference count - new owner
RefCountedBuffer* buffer_retain(RefCountedBuffer* buf) {
    if (buf) {
        buf->ref_count++;
    }
    return buf;
}

// Decrement reference count - owner done with it
void buffer_release(RefCountedBuffer* buf) {
    if (!buf) return;

    buf->ref_count--;
    if (buf->ref_count == 0) {
        free(buf);  // Last owner frees it
        // This is the whole magic of reference counting:
        // The last person to release it cleans it up.
        // No coordination needed, no explicit transfer of ownership.
    }
}

// Usage
void process_data(void) {
    RefCountedBuffer* buf = buffer_create(1024);

    // Share with worker thread
    worker_thread_process(buffer_retain(buf));

    // Share with another thread
    logger_thread_log(buffer_retain(buf));

    // We're done with it
    buffer_release(buf);

    // Buffer is freed when all three threads call buffer_release
}
\end{lstlisting}

\subsection{Thread-Safe Reference Counting}

The simple reference counting above has a fatal flaw in multithreaded code: ref\_count++ isn't atomic. Two threads can both read the same value, both increment it, both write back the same result---and you've lost a reference. Use atomic operations to fix this.

\begin{lstlisting}
#include <stdatomic.h>

typedef struct {
    atomic_int ref_count;  // Thread-safe counter
    // atomic_int is from C11, provides lock-free atomic operations
    size_t size;
    char data[];
} AtomicRefCountedBuffer;

AtomicRefCountedBuffer* buffer_create_atomic(size_t size) {
    AtomicRefCountedBuffer* buf =
        malloc(sizeof(AtomicRefCountedBuffer) + size);
    if (buf) {
        atomic_init(&buf->ref_count, 1);
        buf->size = size;
    }
    return buf;
}

AtomicRefCountedBuffer* buffer_retain_atomic(AtomicRefCountedBuffer* buf) {
    if (buf) {
        atomic_fetch_add(&buf->ref_count, 1);  // Thread-safe increment
    }
    return buf;
}

void buffer_release_atomic(AtomicRefCountedBuffer* buf) {
    if (!buf) return;

    // Thread-safe decrement-and-test
    if (atomic_fetch_sub(&buf->ref_count, 1) == 1) {
        // We were the last reference
        free(buf);
    }
}

// This is how COM objects work on Windows
// Also similar to reference counting in CPython, Objective-C

// Performance note: atomic operations are slower than regular operations
// (they prevent CPU reordering and ensure visibility across cores)
// But they're much faster than mutexes. Use them for ref counting.
\end{lstlisting}

\begin{warningbox}
Reference counting seems simple but has gotchas: circular references cause leaks (A references B, B references A---neither freed). Also, the atomic operations have performance cost. Use only when you truly need shared ownership. (And consider weak references to break cycles, though that's beyond basic C.)
\end{warningbox}

\section{Custom Allocators: The Strategy Pattern}

Sometimes you need to control allocation strategy at runtime. Custom allocators let you swap strategies. This is the Strategy pattern from design patterns: define an interface for allocation, and swap implementations as needed.

Think of custom allocators like choosing a payment method at checkout. The store doesn't care if you pay with cash, credit card, or mobile payment---they all work through the same interface (swipe/tap/insert). But each method works differently internally. Custom allocators are the same: they all look the same from outside (alloc/free functions), but inside they can use completely different strategies. You can swap payment methods without rewriting the whole checkout process, just like you can swap allocators without rewriting your whole program.

Why would you want this? Testing (inject a mock allocator), profiling (inject a counting allocator), performance (swap to a specialized allocator), debugging (inject a leak-detecting allocator). It's powerful because allocation strategy becomes a runtime decision, not a compile-time decision.

\begin{lstlisting}
// Allocator interface
typedef void* (*AllocFunc)(size_t size, void* ctx);
typedef void* (*ReallocFunc)(void* ptr, size_t size, void* ctx);
typedef void (*FreeFunc)(void* ptr, void* ctx);

typedef struct {
    AllocFunc alloc;
    ReallocFunc realloc;
    FreeFunc free;
    void* context;  // Allocator-specific data
    // Context is the secret sauce: different allocators need different data
    // Pool allocator: pointer to pool. Slab allocator: pointer to slab.
    // This makes allocators polymorphic.
    const char* name;  // For debugging
} Allocator;

// System allocator (default)
void* sys_alloc(size_t size, void* ctx) {
    (void)ctx;
    return malloc(size);
}

void* sys_realloc(void* ptr, size_t size, void* ctx) {
    (void)ctx;
    return realloc(ptr, size);
}

void sys_free(void* ptr, void* ctx) {
    (void)ctx;
    free(ptr);
}

Allocator system_allocator = {
    .alloc = sys_alloc,
    .realloc = sys_realloc,
    .free = sys_free,
    .context = NULL,
    .name = "system"
};

// Counting allocator (for leak detection)
typedef struct {
    size_t alloc_count;
    size_t free_count;
    size_t bytes_allocated;
} CountingContext;

void* counting_alloc(size_t size, void* ctx) {
    CountingContext* cc = (CountingContext*)ctx;
    cc->alloc_count++;
    cc->bytes_allocated += size;
    return malloc(size);
}

void counting_free(void* ptr, void* ctx) {
    CountingContext* cc = (CountingContext*)ctx;
    cc->free_count++;
    free(ptr);
}

// Usage - inject allocator
typedef struct {
    Allocator* allocator;
    // ... other fields ...
} Context;

void* context_alloc(Context* ctx, size_t size) {
    return ctx->allocator->alloc(size, ctx->allocator->context);
}

void context_free(Context* ctx, void* ptr) {
    ctx->allocator->free(ptr, ctx->allocator->context);
}

// Can swap allocators at runtime!
Context ctx;
ctx.allocator = &system_allocator;  // Use system malloc
// ... or ...
ctx.allocator = &counting_allocator;  // Track allocations
// ... or ...
ctx.allocator = &pool_allocator;  // Use pool
\end{lstlisting}

\begin{tipbox}
This pattern is used in video games (swap allocators for different game systems), databases (different allocation strategies for different query types), and any code that needs testability (inject mock allocator for tests). It's the Strategy pattern from Gang of Four, applied to memory.
\end{tipbox}

\section{Memory Debugging: Finding Leaks and Corruption}

Memory bugs are the worst kind of bugs. They're non-deterministic, they manifest far from their cause, and they corrupt state silently. You need tools to catch them. Here are patterns for building your own debugging tools.

Think of memory bugs like a silent leak in your water pipes. You don't see it immediately. The water damage shows up on the other side of the house, days later. By then, you have no idea where the leak started. That's why we need tools---like water meters that alert you immediately when something's wrong.

\subsection{Simple Leak Tracker}

This is a custom malloc/free wrapper that tracks every allocation. At program exit, report what wasn't freed. Simple, effective, and catches leaks immediately.

It's like a sign-in/sign-out sheet at a library. Every time you borrow a book (malloc), you sign your name. When you return it (free), you cross your name off. At closing time, if any names are still on the list, those books weren't returned---that's a leak. The librarian (this tool) can tell you exactly who forgot to return what.

\begin{lstlisting}
#ifdef DEBUG_MEMORY

#include <stdio.h>

typedef struct MemEntry {
    void* ptr;
    size_t size;
    const char* file;
    int line;
    struct MemEntry* next;
} MemEntry;

static MemEntry* mem_list = NULL;
static size_t total_allocated = 0;
static size_t total_freed = 0;

void* debug_malloc(size_t size, const char* file, int line) {
    void* ptr = malloc(size);
    if (ptr) {
        MemEntry* entry = malloc(sizeof(MemEntry));
        if (entry) {
            entry->ptr = ptr;
            entry->size = size;
            entry->file = file;
            entry->line = line;
            entry->next = mem_list;
            mem_list = entry;
            total_allocated += size;
        }
    }
    return ptr;
}

void debug_free(void* ptr) {
    if (!ptr) return;

    MemEntry** entry = &mem_list;
    while (*entry) {
        if ((*entry)->ptr == ptr) {
            MemEntry* to_free = *entry;
            *entry = (*entry)->next;
            total_freed += to_free->size;
            free(to_free);
            free(ptr);
            return;
        }
        entry = &(*entry)->next;
    }

    // Not in our tracking list - either:
    // 1. Freeing something we didn't allocate (bug!)
    // 2. Freeing something twice (already removed from list)
    // 3. Freeing a pointer from another allocator
    fprintf(stderr, "WARNING: freeing untracked pointer %p\n", ptr);
    free(ptr);
}

void debug_report_leaks(void) {
    int count = 0;
    size_t leaked_bytes = 0;

    for (MemEntry* e = mem_list; e; e = e->next) {
        fprintf(stderr, "LEAK: %zu bytes at %s:%d (ptr=%p)\n",
                e->size, e->file, e->line, e->ptr);
        leaked_bytes += e->size;
        count++;
    }

    if (count > 0) {
        fprintf(stderr, "\nTotal: %d leaks, %zu bytes leaked\n",
                count, leaked_bytes);
        fprintf(stderr, "Allocated: %zu, Freed: %zu\n",
                total_allocated, total_freed);
    } else {
        fprintf(stderr, "No memory leaks detected!\n");
    }
}

// Macros to wrap malloc/free
#define malloc(size) debug_malloc(size, __FILE__, __LINE__)
#define free(ptr) debug_free(ptr)

// At program exit:
// atexit(debug_report_leaks);

#endif
\end{lstlisting}

\subsection{Canary Values: Detecting Buffer Overruns}

Canaries are values placed before and after allocations. If they're changed, something wrote past the buffer. This catches buffer overflows at free() time.

The name "canary" comes from coal miners who brought canaries into mines. If toxic gas leaked, the canary died first, warning the miners. In programming, we put special values (canaries) at the edges of memory allocations. If your code writes past the buffer, it overwrites the canary. When you free the memory, we check if the canary is still alive. If it's dead (changed), we know there was a buffer overflow. The canary dies to warn you.

\begin{lstlisting}
// Add canary values around allocations to detect overwrites
// Called "canary" like the canary in a coal mine - dies first to warn you
#define CANARY 0xDEADBEEF

typedef struct {
    size_t canary_front;
    size_t size;
    char data[];
} CanaryBlock;

void* guarded_malloc(size_t size) {
    size_t total_size = sizeof(CanaryBlock) + size + sizeof(size_t);
    CanaryBlock* block = malloc(total_size);
    if (!block) return NULL;

    block->canary_front = CANARY;
    block->size = size;

    // Canary at end of allocation
    size_t* canary_back = (size_t*)(block->data + size);
    *canary_back = CANARY;

    return block->data;
}

void guarded_free(void* ptr) {
    if (!ptr) return;

    CanaryBlock* block = (CanaryBlock*)((char*)ptr -
                          offsetof(CanaryBlock, data));

    // Check front canary
    if (block->canary_front != CANARY) {
        fprintf(stderr, "CORRUPTION: front canary destroyed at %p\n", ptr);
        abort();
    }

    // Check back canary
    size_t* canary_back = (size_t*)(block->data + block->size);
    if (*canary_back != CANARY) {
        fprintf(stderr, "CORRUPTION: back canary destroyed at %p\n", ptr);
        fprintf(stderr, "Buffer overflow detected!\n");
        abort();
    }

    free(block);
}

// Catches buffer overflows immediately
// Used in debug builds
\end{lstlisting}

\section{Tools: Valgrind, AddressSanitizer, and Friends}

Professional C developers use tools. Always. These tools catch bugs that code review, testing, and careful programming miss. Use them in every build, every test run. The cost is nothing compared to debugging production memory corruption.

Think of these tools like spell-check or grammar-check for your writing. Sure, you could proofread manually, but why? The tool catches typos instantly that you'd miss. Same with memory tools---they catch bugs instantly that you'd spend hours debugging manually. Not using them is like refusing to use spell-check because "real writers don't need it." (Spoiler: real writers use spell-check.)

\begin{lstlisting}
// Use AddressSanitizer (ASan) - built into GCC/Clang
// This is the single best tool for catching memory bugs
// Compile with: gcc -fsanitize=address -g program.c

// ASan detects:
// - Buffer overflows
// - Use-after-free
// - Use-after-return
// - Double-free
// - Memory leaks

// Example that ASan will catch:
void asan_test(void) {
    int* arr = malloc(10 * sizeof(int));
    arr[10] = 42;  // Buffer overflow - ASan reports it immediately!
    free(arr);
    arr[0] = 0;    // Use-after-free - ASan catches this too!
}

// Valgrind - run without recompiling
// valgrind --leak-check=full ./program

// Electric Fence - catches errors at page boundaries
// Link with: gcc program.c -lefence

// Each tool has trade-offs:
// ASan: Fast, requires recompilation, great for testing
// Valgrind: Slow (10-50x), no recompilation, excellent for production bugs
// Electric Fence: Very slow, catches specific overruns
\end{lstlisting}

\begin{notebox}
In professional development: always run tests with AddressSanitizer enabled. Always. It catches bugs before they reach production. A 2x slowdown in tests is nothing compared to debugging a production memory corruption. (Voice of painful experience talking here.)
\end{notebox}

\section{The Slab Allocator: How the Linux Kernel Does It}

The Linux kernel allocates millions of objects of the same size: inodes, dentries, task structs, etc. The slab allocator is optimized for this pattern. Pre-allocate "slabs" (pages) of objects, and hand them out as needed. When freed, they go back to the free list. Fast allocation (pop from free list), fast deallocation (push to free list), minimal fragmentation.

Imagine an egg carton factory. Instead of making custom containers for each individual egg, you make standard 12-egg cartons. When someone needs to store eggs, you hand them a carton (allocation). When they're done, the carton goes back to the stack of empty cartons (free list), ready to be reused. Fast, efficient, no waste. That's a slab allocator: pre-made containers for same-sized objects. The Linux kernel uses this for kernel objects that get allocated and freed constantly---much faster than custom-sizing each allocation.

\begin{lstlisting}
// Simplified version of Linux slab allocator concept
// Used for objects of the same size
// Real kernel slab allocator is more complex (per-CPU caches, NUMA awareness)

typedef struct SlabNode {
    struct SlabNode* next;
} SlabNode;

typedef struct {
    size_t object_size;
    size_t objects_per_slab;
    SlabNode* free_list;
    void** slabs;
    size_t num_slabs;
    size_t slab_capacity;
} SlabAllocator;

SlabAllocator* slab_create(size_t object_size, size_t objects_per_slab) {
    SlabAllocator* slab = malloc(sizeof(SlabAllocator));
    if (!slab) return NULL;

    slab->object_size = object_size;
    slab->objects_per_slab = objects_per_slab;
    slab->free_list = NULL;
    slab->num_slabs = 0;
    slab->slab_capacity = 16;
    slab->slabs = malloc(sizeof(void*) * slab->slab_capacity);

    return slab;
}

static int slab_add_slab(SlabAllocator* slab) {
    size_t slab_size = slab->object_size * slab->objects_per_slab;
    void* new_slab = malloc(slab_size);
    if (!new_slab) return -1;

    // Add to slab list
    if (slab->num_slabs >= slab->slab_capacity) {
        size_t new_cap = slab->slab_capacity * 2;
        void** new_slabs = realloc(slab->slabs, sizeof(void*) * new_cap);
        if (!new_slabs) {
            free(new_slab);
            return -1;
        }
        slab->slabs = new_slabs;
        slab->slab_capacity = new_cap;
    }

    slab->slabs[slab->num_slabs++] = new_slab;

    // Chain objects into free list
    for (size_t i = 0; i < slab->objects_per_slab; i++) {
        SlabNode* node = (SlabNode*)((char*)new_slab +
                                     i * slab->object_size);
        node->next = slab->free_list;
        slab->free_list = node;
    }

    return 0;
}

void* slab_alloc(SlabAllocator* slab) {
    if (!slab) return NULL;

    if (!slab->free_list) {
        if (slab_add_slab(slab) != 0) {
            return NULL;
        }
    }

    SlabNode* node = slab->free_list;
    slab->free_list = node->next;
    return node;
}

void slab_free(SlabAllocator* slab, void* ptr) {
    if (!slab || !ptr) return;

    SlabNode* node = (SlabNode*)ptr;
    node->next = slab->free_list;
    slab->free_list = node;
}

void slab_destroy(SlabAllocator* slab) {
    if (!slab) return;

    for (size_t i = 0; i < slab->num_slabs; i++) {
        free(slab->slabs[i]);
    }
    free(slab->slabs);
    free(slab);
}

// Perfect for same-sized objects: network packets, AST nodes, etc.
// O(1) allocation and deallocation
// Minimal fragmentation
// Good cache locality
\end{lstlisting}

\section{Production Patterns: What the Pros Do}

Here are patterns from production systems that run 24/7 and handle millions of requests. These aren't theoretical---they're battle-tested solutions to real problems.

\begin{lstlisting}
// Pattern 1: Per-subsystem allocators
// Different parts of your program have different allocation patterns
// Give each subsystem its own allocator, tuned to its pattern
typedef struct {
    Allocator* renderer_allocator;  // Separate allocator for graphics
    Allocator* physics_allocator;   // Separate for physics
    Allocator* audio_allocator;     // Separate for audio
} GameEngine;

// Why? Each subsystem has different allocation patterns
// Renderer: lots of small temporary allocations
// Physics: fixed-size objects (slab allocator)
// Audio: streaming buffers (pool allocator)

// Pattern 2: Allocation limits per subsystem
// Prevent one runaway subsystem from eating all memory
// This is how you survive pathological inputs
typedef struct {
    Allocator* allocator;
    size_t max_bytes;
    size_t current_bytes;
} LimitedAllocator;

void* limited_alloc(LimitedAllocator* la, size_t size) {
    if (la->current_bytes + size > la->max_bytes) {
        // Budget exceeded!
        return NULL;
    }

    void* ptr = la->allocator->alloc(size, la->allocator->context);
    if (ptr) {
        la->current_bytes += size;
    }
    return ptr;
}

// Prevents one subsystem from eating all memory

// Pattern 3: Fallback allocators
// Try fast allocators first, fall back to slower ones
// This is "graceful degradation" for memory allocation
void* fallback_alloc(size_t size) {
    void* ptr = fast_pool_alloc(size);
    if (!ptr) {
        ptr = slower_heap_alloc(size);  // Fallback
    }
    if (!ptr) {
        ptr = emergency_reserve_alloc(size);  // Last resort
    }
    return ptr;
}

// Pattern 4: Allocation budgets
// Game engines: allocate no more than X per frame
// Web servers: allocate no more than Y per request
// This prevents memory growth over time ("memory leak by a thousand cuts")
#define FRAME_MEMORY_BUDGET (16 * 1024 * 1024)  // 16MB per frame

void render_frame(void) {
    Arena* frame_arena = arena_create_sized(FRAME_MEMORY_BUDGET);

    // All frame allocations from arena
    // At end of frame, destroy arena
    // Prevents memory growth over time

    arena_destroy(frame_arena);
}
\end{lstlisting}

\section{Common Memory Bugs and How to Avoid Them}

\begin{lstlisting}
// Bug 1: Use-after-free
void use_after_free_bug(void) {
    int* ptr = malloc(sizeof(int));
    *ptr = 42;
    free(ptr);
    *ptr = 0;  // BUG! Accessing freed memory
}
// Fix: Set pointer to NULL after free
void use_after_free_fix(void) {
    int* ptr = malloc(sizeof(int));
    *ptr = 42;
    free(ptr);
    ptr = NULL;  // Now any access will crash (which is better!)
}

// Bug 2: Double-free
void double_free_bug(void) {
    int* ptr = malloc(sizeof(int));
    free(ptr);
    free(ptr);  // BUG! Undefined behavior, often crashes
}
// Fix: Same as above - set to NULL

// Bug 3: Memory leak
void memory_leak_bug(void) {
    for (int i = 0; i < 1000000; i++) {
        int* ptr = malloc(sizeof(int));
        // Never freed - leaks 4MB
    }
}
// Fix: Free what you allocate

// Bug 4: Dangling pointer
int* dangling_pointer_bug(void) {
    int x = 42;
    return &x;  // BUG! Returns address of stack variable
}
// Fix: Allocate on heap or use static storage

// Bug 5: Uninitialized memory
void uninitialized_bug(void) {
    int* arr = malloc(10 * sizeof(int));
    printf("%d\n", arr[0]);  // BUG! Reading garbage
}
// Fix: Use calloc() or memset()

// Bug 6: Buffer overflow
void overflow_bug(void) {
    char* buf = malloc(10);
    strcpy(buf, "This is way too long");  // BUG! Writes past buffer
}
// Fix: Use strncpy(), check lengths
\end{lstlisting}

\section{Summary: Memory Management Wisdom}

Professional memory management isn't about malloc() and free(). It's about:

\begin{itemize}
    \item \textbf{Clear ownership}: Document who allocates, who frees
    \item \textbf{Allocator strategies}: Pools, arenas, slabs for different patterns
    \item \textbf{RAII patterns}: Automatic cleanup with GCC extensions
    \item \textbf{Reference counting}: For shared ownership
    \item \textbf{Custom allocators}: Control strategy at runtime
    \item \textbf{Debugging tools}: ASan, Valgrind, leak trackers
    \item \textbf{Production patterns}: Per-subsystem allocators, budgets, fallbacks
\end{itemize}

\begin{warningbox}
Memory bugs are subtle, non-deterministic, and hard to debug. They manifest far from their cause. They corrupt data silently. Use every tool available: static analyzers, dynamic analyzers, custom allocators, clear ownership semantics. Defense in depth is the only way to survive. (And yes, that's the voice of someone who's spent many 3 AM sessions debugging memory corruption in production.)
\end{warningbox}

\begin{tipbox}
The best memory management strategy is the one that never gives you a chance to make mistakes. Use RAII where possible. Use arenas for temporary allocations. Use reference counting for shared resources. Make it impossible to leak memory, and you won't. (Well, mostly. We're still writing C, after all.)
\end{tipbox}

\chapter{Struct Patterns \& Tricks}

\section{The Power of Structs: Beyond Simple Grouping}

Structs in C are deceptively simple---they just group data together, right? Wrong. In the hands of a professional, structs are the foundation of object-oriented patterns, memory optimization, API design, and high-performance code. This chapter covers the patterns that textbooks skip.

Think of a struct like a custom container. Just as you can organize your desk drawer with dividers for pens, papers, and clips, structs let you organize data. But professional C programmers don't just use generic containers---they design custom containers optimized for exactly what they need. That's what this chapter teaches.

\section{Understanding Struct Layout: Memory Secrets}

Structs aren't just fields in a row. The compiler adds invisible padding for CPU performance, and understanding this is crucial for memory-efficient code.

Imagine packing a suitcase. If you have big items (shoes) and small items (socks), you don't just throw them in order. You pack big items first, then fill gaps with small items. Compilers do the same with struct members---they arrange them for CPU efficiency, adding "padding" (empty space) where needed.

\begin{lstlisting}
#include <stdio.h>
#include <stddef.h>

typedef struct {
    char a;      // 1 byte
    int b;       // 4 bytes
    char c;      // 1 byte
} Example;

int main(void) {
    printf("Size: %zu\n", sizeof(Example));
    // Prints 12 on most systems, not 6!
    // Why? Padding between fields for alignment

    printf("Offset of a: %zu\n", offsetof(Example, a));  // 0
    printf("Offset of b: %zu\n", offsetof(Example, b));  // 4 (not 1!)
    printf("Offset of c: %zu\n", offsetof(Example, c));  // 8 (not 5!)

    // The actual memory layout:
    // a: 1 byte
    // padding: 3 bytes (to align b to 4-byte boundary)
    // b: 4 bytes
    // c: 1 byte
    // padding: 3 bytes (to align entire struct to 4-byte boundary)
    // Total: 12 bytes

    return 0;
}
\end{lstlisting}

\subsection{Why Padding Exists}

CPUs are faster at reading/writing data when it's aligned to natural boundaries. An \texttt{int} (4 bytes) should start at addresses divisible by 4. A \texttt{double} (8 bytes) should start at addresses divisible by 8. Misaligned access is slower on some CPUs, and crashes on others (ARM, older architectures).

The compiler adds padding to ensure each field is properly aligned. This wastes memory but gains speed---a trade-off you need to understand.

\section{Struct Padding and Alignment: Optimization Gold}

Reordering struct members can save significant memory without changing functionality. This matters in code that allocates thousands or millions of structs.

\begin{lstlisting}
// Inefficient layout - 40% wasted space!
typedef struct {
    char a;     // 1 byte
    // 3 bytes padding (to align int)
    int b;      // 4 bytes
    char c;     // 1 byte
    // 3 bytes padding (to align double)
    double d;   // 8 bytes
    char e;     // 1 byte
    // 7 bytes padding (to align entire struct to 8 bytes)
} Inefficient;  // Total: 32 bytes for 18 bytes of actual data!

// Efficient layout - reorder by size
typedef struct {
    double d;   // 8 bytes (largest first)
    int b;      // 4 bytes
    char a;     // 1 byte
    char c;     // 1 byte
    char e;     // 1 byte
    // 1 byte padding (to align to 8 bytes)
} Efficient;    // Total: 16 bytes - 50% smaller!

// Calculate savings
// If you allocate 1 million instances:
// Inefficient: 32 MB
// Efficient: 16 MB
// Savings: 16 MB per million instances!
\end{lstlisting}

\begin{tipbox}
Always order struct members from largest to smallest. Put 8-byte types first (double, int64\_t, pointers on 64-bit), then 4-byte (int, float), then 2-byte (short), then 1-byte (char, bool). This minimizes padding and can save massive amounts of memory in large-scale applications. (It's like Tetris---fit the pieces efficiently!)
\end{tipbox}

\subsection{Checking Alignment Requirements}

\begin{lstlisting}
#include <stdalign.h>  // C11

typedef struct {
    int x;
    double y;
    char z;
} MyStruct;

// Check alignment requirements
// These tell you the boundaries where data should start
printf("Alignment of int: %zu\n", alignof(int));         // Usually 4
printf("Alignment of double: %zu\n", alignof(double));   // Usually 8
printf("Alignment of char: %zu\n", alignof(char));       // Always 1
printf("Alignment of MyStruct: %zu\n", alignof(MyStruct)); // Usually 8
// The struct's alignment is the largest alignment of any member

// Force specific alignment (for special cases like SIMD)
typedef struct {
    int data[4];
} alignas(16) SIMDVector;  // Force 16-byte alignment
// Useful for SSE/AVX instructions that require aligned data
\end{lstlisting}

\subsection{Packing Structs: When You Need Exact Layout}

Sometimes you need exact layout (network protocols, file formats). Use \texttt{\#pragma pack} but understand the performance cost.

\begin{lstlisting}
// Without packing - has padding
typedef struct {
    char type;      // 1 byte
    // 3 bytes padding
    int length;     // 4 bytes
    char data[10];  // 10 bytes
    // 2 bytes padding
} NormalPacket;     // 20 bytes

// With packing - no padding
#pragma pack(push, 1)  // Pack to 1-byte boundaries
typedef struct {
    char type;      // 1 byte
    int length;     // 4 bytes (NO padding before this!)
    char data[10];  // 10 bytes
} PackedPacket;     // 15 bytes
#pragma pack(pop)   // Restore default packing

// When to use packing:
// - Network protocols (TCP/IP headers, etc.)
// - File formats (must match exact binary layout)
// - Hardware registers (embedded systems)
//
// When NOT to use:
// - Normal application structs (performance penalty)
// - Structs you'll access frequently (slower due to misalignment)
\end{lstlisting}

\begin{warningbox}
Packed structs are slower to access because of misalignment. Use them only when you must match an external binary format. For normal code, let the compiler add padding---it knows better than you do.
\end{warningbox}

\section{Flexible Array Members: Variable-Length Structs}

One of C99's best features: arrays at the end of structs without fixed size. This lets you allocate structs with variable-length data in one allocation.

Think of this like buying a magazine with variable-length articles. You don't know how many pages you need until you see the content. Flexible array members let you allocate exactly the right amount of space.

\begin{lstlisting}
// Old hack (pre-C99) - WRONG, undefined behavior
typedef struct {
    int count;
    char data[1];  // Fake array size, then over-allocate
} OldArray;        // Don't do this!

// Modern way (C99+) - CORRECT and standard
typedef struct {
    int count;
    char data[];  // Flexible array member - size determined at allocation
} ModernArray;

// Allocate with variable size
ModernArray* create_array(int count) {
    // Allocate: struct header + array data
    ModernArray* arr = malloc(sizeof(ModernArray) + count * sizeof(char));
    if (arr) {
        arr->count = count;
        // arr->data is right after count in memory, with space for count chars
    }
    return arr;
}

// Usage
ModernArray* arr = create_array(100);
if (arr) {
    arr->data[0] = 'A';
    arr->data[99] = 'Z';
    printf("Array has %d elements\n", arr->count);
    free(arr);  // One free for entire structure
}
\end{lstlisting}

\begin{notebox}
Flexible array members must be the last member of the struct, and the struct must have at least one other member. This is a language rule---the compiler needs something before the array to establish the struct's base size.
\end{notebox}

\subsection{Real-World Example: Dynamic String}

This pattern is used extensively in production code for variable-length data.

\begin{lstlisting}
typedef struct {
    size_t length;    // Current string length
    size_t capacity;  // Allocated capacity
    char data[];      // Flexible array for string content
} String;

String* string_create(size_t capacity) {
    // Allocate struct + capacity bytes for string
    String* s = malloc(sizeof(String) + capacity);
    if (s) {
        s->length = 0;
        s->capacity = capacity;
        s->data[0] = '\0';  // Empty string
    }
    return s;
}

String* string_from(const char* str) {
    size_t len = strlen(str);
    String* s = string_create(len + 1);  // +1 for null terminator
    if (s) {
        strcpy(s->data, str);
        s->length = len;
    }
    return s;
}

// Grow string if needed
String* string_append(String* s, const char* str) {
    if (!s || !str) return s;

    size_t add_len = strlen(str);
    size_t new_len = s->length + add_len;

    if (new_len + 1 > s->capacity) {
        // Need to grow - reallocate
        size_t new_capacity = (new_len + 1) * 2;  // Double capacity
        String* new_s = realloc(s, sizeof(String) + new_capacity);
        if (!new_s) return s;  // Keep old string on failure

        s = new_s;
        s->capacity = new_capacity;
    }

    strcpy(s->data + s->length, str);
    s->length = new_len;
    return s;
}

void string_destroy(String* s) {
    free(s);  // One free for everything!
}

// This pattern gives you:
// 1. One allocation instead of two (struct + data)
// 2. Better cache locality (data is right after metadata)
// 3. Simpler memory management (one malloc, one free)
// 4. Exact size (no wasted space)
\end{lstlisting}

\subsection{Generic Variable-Length Structure Pattern}

\begin{lstlisting}
// This pattern is used in Linux kernel, compilers, databases
typedef struct Message {
    uint32_t type;
    uint32_t length;
    uint8_t payload[];  // Variable-length payload
} Message;

// Create message with specific payload
Message* message_create(uint32_t type, const void* data, size_t len) {
    Message* msg = malloc(sizeof(Message) + len);
    if (msg) {
        msg->type = type;
        msg->length = len;
        if (data) {
            memcpy(msg->payload, data, len);
        }
    }
    return msg;
}

// Send over network - efficient, no extra copying
void send_message(int socket, Message* msg) {
    // Send entire message in one go
    send(socket, msg, sizeof(Message) + msg->length, 0);
}

// This is how network protocols work: header + variable payload
\end{lstlisting}

\section{Struct Inheritance (C Style): Poor Man's OOP}

C doesn't have inheritance, but we can simulate it. The secret: the first member of a struct has the same address as the struct itself. This is guaranteed by the C standard.

Imagine Russian nesting dolls. The outer doll contains the inner doll at the exact same starting point. When you open it, you can treat it as either the outer doll or the inner doll. That's struct inheritance in C.

\begin{lstlisting}
// Base "class"
typedef struct {
    int id;
    char name[50];
} Animal;

void animal_init(Animal* a, int id, const char* name) {
    a->id = id;
    strncpy(a->name, name, sizeof(a->name) - 1);
    a->name[sizeof(a->name) - 1] = '\0';
}

void animal_print(Animal* a) {
    printf("ID: %d, Name: %s\n", a->id, a->name);
}

// Derived "class" - base MUST be first member!
typedef struct {
    Animal base;  // MUST BE FIRST - this is the magic
    int num_legs;
    char breed[30];
} Dog;

void dog_init(Dog* d, int id, const char* name, int legs, const char* breed) {
    animal_init(&d->base, id, name);  // Initialize base
    d->num_legs = legs;
    strncpy(d->breed, breed, sizeof(d->breed) - 1);
}

void dog_bark(Dog* d) {
    printf("%s says: Woof! (I have %d legs)\n", d->base.name, d->num_legs);
}

// Another derived class
typedef struct {
    Animal base;  // MUST BE FIRST
    int wingspan;
    int can_fly;
} Bird;

// "Virtual" function that works with base type - polymorphism!
void print_any_animal(Animal* a) {
    printf("ID: %d, Name: %s\n", a->id, a->name);
}

// Usage - polymorphism in C!
int main(void) {
    Dog d;
    dog_init(&d, 1, "Buddy", 4, "Golden Retriever");

    Bird b;
    b.base.id = 2;
    strcpy(b.base.name, "Tweety");
    b.wingspan = 30;
    b.can_fly = 1;

    // Polymorphism: both work with Animal* functions
    print_any_animal((Animal*)&d);  // Works! Treats Dog as Animal
    print_any_animal((Animal*)&b);  // Works! Treats Bird as Animal

    // Why this works: address of Dog == address of Dog.base
    printf("Dog address: %p\n", (void*)&d);
    printf("Dog.base address: %p\n", (void*)&d.base);
    // These print the SAME address!

    return 0;
}
\end{lstlisting}

\begin{notebox}
This works because C guarantees the first member of a struct has the same address as the struct itself. This is how GTK, GObject, GStreamer, and many C libraries implement object-oriented patterns! It's not a hack---it's a fundamental C guarantee.
\end{notebox}

\subsection{Type Tags for Runtime Type Information}

In real OOP, you'd use `instanceof`. In C, we use type tags---explicit type fields that tell us what we're actually holding.

\begin{lstlisting}
typedef enum {
    ANIMAL_DOG,
    ANIMAL_CAT,
    ANIMAL_BIRD,
    ANIMAL_FISH
} AnimalType;

typedef struct {
    AnimalType type;  // Type tag - RTTI in C!
    int id;
    char name[50];
} Animal;

typedef struct {
    Animal base;  // Must be first
    int num_legs;
    char breed[30];
} Dog;

typedef struct {
    Animal base;
    int lives_remaining;  // Cats have 9 lives
} Cat;

typedef struct {
    Animal base;
    int wingspan;
    int can_fly;
} Bird;

// Type-safe casting functions
Dog* animal_as_dog(Animal* a) {
    if (a && a->type == ANIMAL_DOG) {
        return (Dog*)a;
    }
    return NULL;  // Not a dog!
}

Cat* animal_as_cat(Animal* a) {
    if (a && a->type == ANIMAL_CAT) {
        return (Cat*)a;
    }
    return NULL;
}

// Safe polymorphic operation
void feed_animal(Animal* a) {
    if (!a) return;

    switch (a->type) {
        case ANIMAL_DOG: {
            Dog* dog = (Dog*)a;
            printf("Feeding dog: %s (breed: %s)\n", a->name, dog->breed);
            break;
        }
        case ANIMAL_CAT: {
            Cat* cat = (Cat*)a;
            printf("Feeding cat: %s (%d lives left)\n",
                   a->name, cat->lives_remaining);
            break;
        }
        case ANIMAL_BIRD: {
            Bird* bird = (Bird*)a;
            printf("Feeding bird: %s (wingspan: %d cm)\n",
                   a->name, bird->wingspan);
            break;
        }
        default:
            printf("Unknown animal type\n");
    }
}

// Usage with type safety
Animal* a = get_some_animal();
Dog* d = animal_as_dog(a);
if (d) {
    // Safely use as Dog
    printf("Dog has %d legs\n", d->num_legs);
} else {
    // Not a dog - handle appropriately
    printf("Not a dog!\n");
}
\end{lstlisting}

\section{VTable Pattern: True Polymorphism in C}

This is how C++ virtual functions work under the hood, and how you achieve true polymorphism in C.

Think of a VTable like a phone directory. Instead of hardcoding which function to call, you look it up in the directory. Different objects have different directories, so calling "draw" on a Circle looks up Circle's draw function, while calling "draw" on a Rectangle looks up Rectangle's draw function. Same operation name, different implementations---that's polymorphism!

\begin{lstlisting}
// Forward declarations
typedef struct Shape Shape;

// VTable: table of function pointers
typedef struct {
    void (*draw)(Shape* self);
    void (*move)(Shape* self, int dx, int dy);
    double (*area)(Shape* self);
    void (*destroy)(Shape* self);
} ShapeVTable;

// Base "class" - VTable MUST be first!
struct Shape {
    ShapeVTable* vtable;  // MUST be first member
    int x, y;             // Common fields
    const char* name;
};

// Circle implementation
typedef struct {
    Shape base;  // Inheritance
    int radius;
} Circle;

void circle_draw(Shape* self) {
    Circle* c = (Circle*)self;
    printf("Drawing circle '%s' at (%d,%d) with radius %d\n",
           self->name, self->x, self->y, c->radius);
}

void circle_move(Shape* self, int dx, int dy) {
    self->x += dx;
    self->y += dy;
    printf("Moved circle to (%d,%d)\n", self->x, self->y);
}

double circle_area(Shape* self) {
    Circle* c = (Circle*)self;
    return 3.14159 * c->radius * c->radius;
}

void circle_destroy(Shape* self) {
    printf("Destroying circle '%s'\n", self->name);
    free(self);
}

// VTable for circles - one shared instance
static ShapeVTable circle_vtable = {
    .draw = circle_draw,
    .move = circle_move,
    .area = circle_area,
    .destroy = circle_destroy
};

// Constructor
Circle* circle_create(int x, int y, int radius, const char* name) {
    Circle* c = malloc(sizeof(Circle));
    if (c) {
        c->base.vtable = &circle_vtable;  // Link to VTable
        c->base.x = x;
        c->base.y = y;
        c->base.name = name;
        c->radius = radius;
    }
    return c;
}

// Rectangle implementation
typedef struct {
    Shape base;
    int width, height;
} Rectangle;

void rectangle_draw(Shape* self) {
    Rectangle* r = (Rectangle*)self;
    printf("Drawing rectangle '%s' at (%d,%d) size %dx%d\n",
           self->name, self->x, self->y, r->width, r->height);
}

void rectangle_move(Shape* self, int dx, int dy) {
    self->x += dx;
    self->y += dy;
}

double rectangle_area(Shape* self) {
    Rectangle* r = (Rectangle*)self;
    return r->width * r->height;
}

void rectangle_destroy(Shape* self) {
    free(self);
}

static ShapeVTable rectangle_vtable = {
    .draw = rectangle_draw,
    .move = rectangle_move,
    .area = rectangle_area,
    .destroy = rectangle_destroy
};

Rectangle* rectangle_create(int x, int y, int w, int h, const char* name) {
    Rectangle* r = malloc(sizeof(Rectangle));
    if (r) {
        r->base.vtable = &rectangle_vtable;
        r->base.x = x;
        r->base.y = y;
        r->base.name = name;
        r->width = w;
        r->height = h;
    }
    return r;
}

// Polymorphic operations - work with any Shape!
void shape_draw(Shape* s) {
    if (s && s->vtable && s->vtable->draw) {
        s->vtable->draw(s);  // Dynamic dispatch!
    }
}

void shape_move(Shape* s, int dx, int dy) {
    if (s && s->vtable && s->vtable->move) {
        s->vtable->move(s, dx, dy);
    }
}

double shape_area(Shape* s) {
    if (s && s->vtable && s->vtable->area) {
        return s->vtable->area(s);
    }
    return 0.0;
}

void shape_destroy(Shape* s) {
    if (s && s->vtable && s->vtable->destroy) {
        s->vtable->destroy(s);
    }
}

// Usage - true polymorphism!
int main(void) {
    Shape* shapes[4];

    shapes[0] = (Shape*)circle_create(10, 10, 5, "c1");
    shapes[1] = (Shape*)rectangle_create(20, 20, 10, 15, "r1");
    shapes[2] = (Shape*)circle_create(30, 30, 8, "c2");
    shapes[3] = (Shape*)rectangle_create(40, 40, 20, 25, "r2");

    // Same operation, different behavior for each type
    for (int i = 0; i < 4; i++) {
        shape_draw(shapes[i]);       // Polymorphic draw
        shape_move(shapes[i], 5, 5); // Polymorphic move
        printf("Area: %.2f\n", shape_area(shapes[i]));
        printf("\n");
    }

    // Cleanup
    for (int i = 0; i < 4; i++) {
        shape_destroy(shapes[i]);
    }

    return 0;
}

// This is EXACTLY how C++ virtual functions work!
// Also how GObject (GTK), COM (Windows), and many C APIs work.
\end{lstlisting}

\begin{tipbox}
The VTable must be the first member for safe casting between base and derived types. C guarantees that a pointer to a struct points to its first member, so Shape* and Circle* point to the same memory location when Circle starts with Shape. This is the foundation of C polymorphism.
\end{tipbox}

\section{Bit Fields: Packing Booleans and Small Integers}

When you have many boolean flags or small integers (0-7, 0-15, etc.), bit fields let you pack them into minimal space. This is crucial for embedded systems, network protocols, and memory-constrained code.

Think of bit fields like a pillbox with compartments. Instead of using a whole jar for each pill, you have one box with tiny compartments. Each compartment holds exactly what you need---no wasted space.

\begin{lstlisting}
// Without bit fields - wastes 96% of memory!
typedef struct {
    int is_valid;      // 4 bytes for 1 bit of information!
    int is_ready;      // 4 bytes for 1 bit
    int is_error;      // 4 bytes for 1 bit
    int priority;      // 4 bytes for value 0-7 (needs 3 bits)
    int type;          // 4 bytes for value 0-15 (needs 4 bits)
    int color;         // 4 bytes for value 0-7 (needs 3 bits)
} WastefulFlags;      // Total: 24 bytes for 13 bits of data!

// With bit fields - efficient
typedef struct {
    unsigned int is_valid : 1;   // 1 bit
    unsigned int is_ready : 1;   // 1 bit
    unsigned int is_error : 1;   // 1 bit
    unsigned int priority : 3;   // 3 bits (holds 0-7)
    unsigned int type : 4;       // 4 bits (holds 0-15)
    unsigned int color : 3;      // 3 bits (holds 0-7)
    unsigned int reserved : 19;  // Padding to 32 bits (good practice)
} CompactFlags;                  // Total: 4 bytes - 83% smaller!

// Usage - looks like normal struct access
CompactFlags flags = {0};
flags.is_valid = 1;
flags.is_ready = 1;
flags.is_error = 0;
flags.priority = 5;     // Can hold 0-7
flags.type = 12;        // Can hold 0-15
flags.color = 3;        // Can hold 0-7

// Read values
if (flags.is_valid && flags.priority > 3) {
    printf("High priority item: type=%u, color=%u\n",
           flags.type, flags.color);
}

// For embedded systems or arrays, this saves massive memory
CompactFlags array[1000];  // 4KB instead of 24KB!
\end{lstlisting}

\subsection{Bit Field Gotchas and Warnings}

\begin{lstlisting}
typedef struct {
    unsigned int value : 4;  // Can hold 0-15
} BitField;

BitField bf = {0};
bf.value = 15;
bf.value++;  // Wraps to 0! Overflow in 4 bits

// Can't take address of bit field
// unsigned int* p = &bf.value;  // ERROR! Won't compile

// Bit fields have implementation-defined layout
// Order in memory varies by compiler/platform
// Size and padding vary by compiler

// Portable solution: manual bit manipulation
typedef struct {
    uint32_t flags;  // Store all flags in one integer
} PortableFlags;

#define FLAG_VALID   0x01  // Bit 0
#define FLAG_READY   0x02  // Bit 1
#define FLAG_ERROR   0x04  // Bit 2
#define PRIORITY_SHIFT 3   // Bits 3-5
#define PRIORITY_MASK  0x07
#define TYPE_SHIFT     6   // Bits 6-9
#define TYPE_MASK      0x0F

// Set/get macros
#define SET_PRIORITY(f, p) \
    ((f) = ((f) & ~(PRIORITY_MASK << PRIORITY_SHIFT)) | \
           (((p) & PRIORITY_MASK) << PRIORITY_SHIFT))

#define GET_PRIORITY(f) \
    (((f) >> PRIORITY_SHIFT) & PRIORITY_MASK)

// More verbose but portable and predictable
\end{lstlisting}

\begin{warningbox}
Bit fields are NOT portable across compilers/architectures! Bit order, packing, and alignment vary. Use bit fields for memory savings within your program, never for file formats or network protocols. For external data, use explicit bit manipulation with masks and shifts.
\end{warningbox}

\section{Designated Initializers: Self-Documenting Code}

C99's designated initializers make struct initialization clear, flexible, and resistant to bugs from field reordering.

\begin{lstlisting}
typedef struct {
    int x;
    int y;
    int z;
    const char* name;
    double value;
    int flags;
} Config;

// Old way (C89) - fragile
Config c1 = {10, 20, 30, "test", 3.14, 0};
// Problems:
// 1. Must remember exact order
// 2. Easy to mix up similar types (int/int/int)
// 3. If struct changes order, this breaks silently
// 4. Unreadable - what do these numbers mean?

// New way (C99+) - robust and clear
Config c2 = {
    .x = 10,
    .y = 20,
    .z = 30,
    .name = "test",
    .value = 3.14,
    .flags = 0
};
// Benefits:
// 1. Self-documenting - clear what each value means
// 2. Order doesn't matter
// 3. Resistant to struct changes
// 4. Readable by anyone

// Can skip fields - they become 0/NULL
Config c3 = {
    .x = 5,
    .name = "partial"
    // y, z, value, flags are all zero
};

// Order doesn't matter!
Config c4 = {
    .name = "flexible",  // name first
    .z = 100,            // skip x and y
    .x = 50              // x last
    // y, value, flags are zero
};

// Arrays of structs with sparse initialization
Config configs[] = {
    [0] = {.name = "first", .x = 1},
    [5] = {.name = "sixth", .x = 6},  // Indices 1-4 are zero-initialized
    [10] = {.name = "eleventh", .x = 11}
};
\end{lstlisting}

\begin{tipbox}
Always use designated initializers for structs with more than 3 fields. Your code becomes self-documenting and immune to field reordering. This is standard practice in the Linux kernel, BSD, and professional C codebases. (And it makes code review much easier---reviewers can see what each value means without looking up the struct definition.)
\end{tipbox}

\subsection{Compound Literals: Temporary Structs}

\begin{lstlisting}
typedef struct {
    int x, y;
} Point;

void draw_line(Point start, Point end) {
    printf("Line from (%d,%d) to (%d,%d)\n",
           start.x, start.y, end.x, end.y);
}

// Without compound literals - verbose
Point p1 = {10, 20};
Point p2 = {30, 40};
draw_line(p1, p2);

// With compound literals - concise
draw_line((Point){10, 20}, (Point){30, 40});

// Great for initializing in expressions
Point* points = malloc(sizeof(Point) * 3);
points[0] = (Point){.x = 0, .y = 0};
points[1] = (Point){.x = 100, .y = 0};
points[2] = (Point){.x = 50, .y = 100};

// Reset struct to zero
Config cfg = {/* initialized */};
// Later:
cfg = (Config){0};  // Reset to all zeros!
\end{lstlisting}

\section{Anonymous Structs and Unions: Cleaner Access}

C11 allows anonymous structs and unions for more natural member access.

\begin{lstlisting}
// Without anonymous union - verbose
typedef struct {
    enum { INT, FLOAT, STRING } type;
    union {
        int int_val;
        float float_val;
        char* string_val;
    } data;  // Named union
} Value_Old;

Value_Old v1;
v1.type = INT;
v1.data.int_val = 42;  // Must go through 'data'

// With anonymous union (C11) - cleaner
typedef struct {
    enum { INT, FLOAT, STRING } type;
    union {
        int int_val;
        float float_val;
        char* string_val;
    };  // No name!
} Value;

Value v2;
v2.type = INT;
v2.int_val = 42;  // Direct access - cleaner!

// Anonymous struct example
typedef struct {
    int type;
    struct {  // Anonymous struct
        int x;
        int y;
        int z;
    };  // No name
} Entity;

Entity e;
e.x = 10;  // Direct access, not e.position.x
e.y = 20;
e.z = 30;
\end{lstlisting}

\subsection{Tagged Unions: Type-Safe Variants}

The pattern for variant types that hold different types of data.

\begin{lstlisting}
typedef enum {
    VAR_NONE,
    VAR_INT,
    VAR_DOUBLE,
    VAR_STRING,
    VAR_ARRAY
} VariantType;

typedef struct Variant Variant;

struct Variant {
    VariantType type;
    union {
        int as_int;
        double as_double;
        char* as_string;
        struct {
            Variant* items;
            size_t count;
        } as_array;
    };
};

// Constructors
Variant make_int(int value) {
    Variant v;
    v.type = VAR_INT;
    v.as_int = value;
    return v;
}

Variant make_double(double value) {
    Variant v;
    v.type = VAR_DOUBLE;
    v.as_double = value;
    return v;
}

Variant make_string(const char* str) {
    Variant v;
    v.type = VAR_STRING;
    v.as_string = strdup(str);
    return v;
}

Variant make_array(size_t capacity) {
    Variant v;
    v.type = VAR_ARRAY;
    v.as_array.items = malloc(sizeof(Variant) * capacity);
    v.as_array.count = 0;
    return v;
}

// Type-safe access
int variant_as_int(Variant* v, int* out) {
    if (v && v->type == VAR_INT) {
        *out = v->as_int;
        return 0;
    }
    return -1;  // Wrong type
}

// Print any variant
void print_variant(Variant* v) {
    if (!v) return;

    switch (v->type) {
        case VAR_NONE:
            printf("(none)");
            break;
        case VAR_INT:
            printf("%d", v->as_int);
            break;
        case VAR_DOUBLE:
            printf("%f", v->as_double);
            break;
        case VAR_STRING:
            printf("\"%s\"", v->as_string);
            break;
        case VAR_ARRAY:
            printf("[array of %zu items]", v->as_array.count);
            break;
    }
}

// Cleanup
void variant_destroy(Variant* v) {
    if (!v) return;

    if (v->type == VAR_STRING) {
        free(v->as_string);
    } else if (v->type == VAR_ARRAY) {
        for (size_t i = 0; i < v->as_array.count; i++) {
            variant_destroy(&v->as_array.items[i]);
        }
        free(v->as_array.items);
    }
}

// This pattern is used in scripting language implementations,
// JSON libraries, configuration systems, etc.
\end{lstlisting}

\section{Struct Copy: Shallow vs Deep}

Assignment operator copies structs, but watch out for pointers!

\begin{lstlisting}
typedef struct {
    int id;
    char* name;   // Pointer!
    int* data;    // Pointer!
    size_t size;
} Resource;

// Shallow copy - DANGEROUS!
Resource r1 = {
    .id = 1,
    .name = strdup("test"),
    .data = malloc(sizeof(int) * 10),
    .size = 10
};

Resource r2 = r1;  // Shallow copy - copies pointer values only!
// Now both r1 and r2 point to the SAME memory!

free(r1.name);     // r2.name is now dangling!
r2.name[0] = 'X';  // CRASH! Use-after-free

// Deep copy - SAFE
Resource resource_copy(const Resource* src) {
    Resource dst = {0};

    dst.id = src->id;
    dst.size = src->size;

    // Deep copy: allocate new memory and copy content
    dst.name = strdup(src->name);

    dst.data = malloc(sizeof(int) * src->size);
    if (dst.data) {
        memcpy(dst.data, src->data, sizeof(int) * src->size);
    }

    return dst;
}

// Usage
Resource r3 = resource_copy(&r1);
// r3 has its own separate copies of name and data
free(r1.name);     // Safe - r3.name is different pointer
free(r1.data);

// r3 still valid
printf("r3 name: %s\n", r3.name);

// Clean up r3
free(r3.name);
free(r3.data);
\end{lstlisting}

\begin{warningbox}
Assignment (\texttt{=}) only does shallow copy! If your struct contains pointers, you MUST write a custom copy function. This is a major source of bugs---two structs sharing the same pointer, leading to double-frees or use-after-free errors. (It's like copying a house address vs copying the actual house---one's just a reference, the other is a full duplicate.)
\end{warningbox}

\section{Struct Comparison: Why memcmp is Dangerous}

\begin{lstlisting}
typedef struct {
    int x;
    int y;
} Point;

Point p1 = {1, 2};
Point p2 = {1, 2};

// WRONG - unreliable due to padding!
if (memcmp(&p1, &p2, sizeof(Point)) == 0) {
    // May fail even though x and y are equal!
    // Padding bytes contain garbage and differ
}

// The actual memory:
// p1: [x=1][garbage padding][y=2]
// p2: [x=1][different garbage][y=2]
// memcmp sees different padding bytes!

// CORRECT - compare field by field
int point_equal(const Point* a, const Point* b) {
    if (!a || !b) return 0;
    return a->x == b->x && a->y == b->y;
}

// Or for many fields
typedef struct {
    int id;
    char name[50];
    double value;
} Record;

int record_equal(const Record* a, const Record* b) {
    if (!a || !b) return 0;
    return a->id == b->id &&
           strcmp(a->name, b->name) == 0 &&
           a->value == b->value;
}

// Comparison function for qsort
int point_compare(const void* a, const void* b) {
    const Point* pa = (const Point*)a;
    const Point* pb = (const Point*)b;

    // Sort by x, then by y
    if (pa->x != pb->x)
        return (pa->x > pb->x) - (pa->x < pb->x);  // Avoid overflow
    return (pa->y > pb->y) - (pa->y < pb->y);
}

// Usage with qsort
Point points[100];
// ... initialize ...
qsort(points, 100, sizeof(Point), point_compare);
\end{lstlisting}

\section{Struct Serialization: Never Write Structs Directly}

\begin{lstlisting}
typedef struct {
    uint32_t magic;     // File format identifier
    uint16_t version;   // Version number
    uint16_t flags;
    uint32_t count;
} FileHeader;

// WRONG - not portable!
void write_header_wrong(FILE* f, const FileHeader* h) {
    fwrite(h, sizeof(FileHeader), 1, f);
    // Problems:
    // 1. Padding bytes written (garbage)
    // 2. Byte order (endianness) not specified
    // 3. Struct layout varies by compiler
    // 4. Won't work on different architectures
}

// CORRECT - write field by field in specific byte order
int write_header(FILE* f, const FileHeader* h) {
    // Convert to network byte order (big-endian)
    uint32_t magic = htonl(h->magic);
    uint16_t version = htons(h->version);
    uint16_t flags = htons(h->flags);
    uint32_t count = htonl(h->count);

    // Write each field explicitly
    if (fwrite(&magic, sizeof(magic), 1, f) != 1) return -1;
    if (fwrite(&version, sizeof(version), 1, f) != 1) return -1;
    if (fwrite(&flags, sizeof(flags), 1, f) != 1) return -1;
    if (fwrite(&count, sizeof(count), 1, f) != 1) return -1;

    return 0;
}

// Read with same byte order
int read_header(FILE* f, FileHeader* h) {
    uint32_t magic;
    uint16_t version;
    uint16_t flags;
    uint32_t count;

    if (fread(&magic, sizeof(magic), 1, f) != 1) return -1;
    if (fread(&version, sizeof(version), 1, f) != 1) return -1;
    if (fread(&flags, sizeof(flags), 1, f) != 1) return -1;
    if (fread(&count, sizeof(count), 1, f) != 1) return -1;

    // Convert from network byte order
    h->magic = ntohl(magic);
    h->version = ntohs(version);
    h->flags = ntohs(flags);
    h->count = ntohl(count);

    return 0;
}

// This ensures portability across architectures and compilers
\end{lstlisting}

\section{Zero-Initialization Idioms}

\begin{lstlisting}
typedef struct {
    int x;
    char* name;
    double values[10];
    int flags;
} Data;

// Method 1: Initialize all members to zero
Data d1 = {0};  // Most common and portable

// Method 2: Empty braces (C++ style, works in C too)
Data d2 = {};

// Method 3: memset
Data d3;
memset(&d3, 0, sizeof(Data));

// Method 4: Compound literal (C99+)
Data d4;
// ... use d4 ...
d4 = (Data){0};  // Reset to zero

// Why zero-initialization matters:
// 1. Prevents uninitialized memory bugs
// 2. Sets pointers to NULL (safe)
// 3. Sets integers to 0
// 4. Sets floats to 0.0
// 5. Makes valgrind happy

// Common idiom in Linux kernel and BSD
typedef struct {
    int initialized;  // Flag
    // ... other members ...
} Module;

Module mod = {0};  // Everything zero, including initialized flag
// Later:
if (!mod.initialized) {
    initialize_module(&mod);
    mod.initialized = 1;
}
\end{lstlisting}

\section{Struct Hashing for Hash Tables}

\begin{lstlisting}
typedef struct {
    int id;
    char name[50];
    char email[100];
} Person;

// Simple hash function for structs
// This uses djb2 hash algorithm
uint32_t person_hash(const Person* p) {
    if (!p) return 0;

    uint32_t hash = 5381;  // Magic constant from djb2

    // Hash integer fields
    hash = ((hash << 5) + hash) + p->id;  // hash * 33 + id

    // Hash string fields byte by byte
    for (const char* s = p->name; *s; s++) {
        hash = ((hash << 5) + hash) + (unsigned char)*s;
    }

    for (const char* s = p->email; *s; s++) {
        hash = ((hash << 5) + hash) + (unsigned char)*s;
    }

    return hash;
}

// Use in hash table
Person person = {123, "Alice", "alice@example.com"};
uint32_t hash = person_hash(&person);
size_t index = hash % table_size;

// For more complex structs, use a better hash
uint32_t fnv1a_hash(const void* data, size_t len) {
    const uint8_t* bytes = (const uint8_t*)data;
    uint32_t hash = 2166136261u;  // FNV offset basis

    for (size_t i = 0; i < len; i++) {
        hash ^= bytes[i];
        hash *= 16777619u;  // FNV prime
    }

    return hash;
}

// Hash the person struct
uint32_t hash = fnv1a_hash(&person, sizeof(person));
\end{lstlisting}

\section{Intrusive Data Structures}

Instead of wrapping data in list nodes, embed list nodes in data. This is how the Linux kernel does it.

\begin{lstlisting}
// Traditional approach - extra allocation
typedef struct ListNode {
    void* data;               // Pointer to actual data
    struct ListNode* next;
} ListNode;

// Problems:
// 1. Extra malloc for each node
// 2. Extra pointer indirection
// 3. Cache-unfriendly

// Intrusive approach - embed link in struct
typedef struct Task Task;
struct Task {
    int pid;
    char name[64];
    int priority;
    Task* next;  // Intrusive link
};

// No extra allocation needed
Task* head = NULL;

Task* create_task(int pid, const char* name, int priority) {
    Task* t = malloc(sizeof(Task));
    if (t) {
        t->pid = pid;
        strncpy(t->name, name, sizeof(t->name) - 1);
        t->priority = priority;
        t->next = NULL;
    }
    return t;
}

// Add to list
void add_task(Task** head, Task* task) {
    task->next = *head;
    *head = task;
}

// Iterate
for (Task* t = head; t; t = t->next) {
    printf("Task %d: %s (priority %d)\n", t->pid, t->name, t->priority);
}

// Benefits:
// 1. One allocation instead of two
// 2. Better cache locality
// 3. No pointer chasing
// 4. This is how Linux kernel lists work!
\end{lstlisting}

\section{Real-World Data Structures: What Production Code Actually Uses}

Every data structure in C is built from structs. But textbooks teach linked lists and binary trees in isolation, never showing you how professionals actually implement them in production code. This section covers the data structures you'll see in real codebases---with all the practical details textbooks skip.

\subsection{Dynamic Arrays (Vectors): The Most Common Data Structure}

Let's start with the single most important data structure in C: the dynamic array, also called a vector or resizable array.

\textbf{What is a dynamic array?} Think of it like a grocery list. A regular C array is like writing your list on a sticky note---you have limited space, and once it's full, you can't add more items. A dynamic array is like having a magic notebook: when you run out of space, it automatically gives you a bigger page and copies everything over.

In C++, this is \texttt{std::vector}. In Python, it's just called a list. In Java, it's \texttt{ArrayList}. Every modern language has one because they're incredibly useful. But in C, you have to build it yourself---and understanding how it works internally makes you a better programmer in \textit{any} language.

\textbf{Why are dynamic arrays everywhere?} Three reasons:
\begin{enumerate}
    \item \textbf{Fast access}: Getting item \#5 is instant (O(1)), unlike linked lists where you have to walk through items 1, 2, 3, 4 first
    \item \textbf{Cache-friendly}: All data sits next to each other in memory, making the CPU happy
    \item \textbf{Simple}: Adding to the end is usually fast, and the implementation is straightforward
\end{enumerate}

Redis uses dynamic arrays for command arguments. Git uses them for file lists. SQLite uses them to store query results. They're the default choice for "I need to store a bunch of things" in professional C code.

\begin{lstlisting}
// Generic dynamic array (vector)
// This pattern is used by:
// - Redis for command arguments
// - Git for file lists
// - SQLite for result rows

typedef struct {
    void** data;        // Array of pointers to actual items
    size_t size;        // How many items we currently have
    size_t capacity;    // How much space we've allocated
} Vector;

// Let's understand each field:
//
// 'data': This is our actual array. It's void** (pointer to pointer)
//         because we want to store ANY type. Each element is a pointer
//         to some data. Think of it as an array of "boxes" where each
//         box can hold a pointer to anything.
//
// 'size': The number of items currently in the vector. If you add
//         3 items, size is 3. This is what users care about.
//
// 'capacity': How much space we've allocated. We might allocate space
//             for 10 items but only use 3. This lets us add more items
//             without reallocating every time. It's like buying a
//             filing cabinet with 10 slots even though you only have
//             3 folders---you have room to grow.

// Create empty vector
Vector* vector_create(void) {
    Vector* v = malloc(sizeof(Vector));
    if (!v) return NULL;

    v->data = NULL;
    v->size = 0;
    v->capacity = 0;
    return v;
}

// Add element (with automatic growth)
// This is the heart of the dynamic array---this is what makes it "dynamic"
int vector_push(Vector* v, void* item) {
    // Step 1: Check if we have room
    // If size equals capacity, we're full. Time to grow!
    if (v->size >= v->capacity) {
        // Growth strategy: DOUBLE the capacity each time
        // If capacity is 0 (new vector), start with 8
        // Otherwise, double it: 8 -> 16 -> 32 -> 64 -> 128...
        //
        // Why double? It's the magic of "amortized O(1)":
        // - If we grew by +1 each time: 1, 2, 3, 4... we'd reallocate
        //   on EVERY insertion. Terrible!
        // - If we double: 1, 2, 4, 8, 16... we reallocate rarely
        // - Total copies for n items: 1 + 2 + 4 + 8... ~= 2n
        // - Average per item: 2n/n = 2 = O(1)
        size_t new_capacity = v->capacity ? v->capacity * 2 : 8;

        // Step 2: Allocate bigger array
        // realloc is smart: it tries to grow in-place if possible,
        // otherwise it allocates new memory and copies for us
        void** new_data = realloc(v->data,
                                  new_capacity * sizeof(void*));
        if (!new_data) return -1;  // Out of memory---very rare

        // Step 3: Update our struct
        v->data = new_data;
        v->capacity = new_capacity;
        // Note: size stays the same---we didn't add items yet,
        // just made room for future items
    }

    // Step 4: Actually add the item
    // v->size++ is a post-increment: use current size as index,
    // then increment. So if size was 3, we write to index 3,
    // then size becomes 4.
    v->data[v->size++] = item;
    return 0;
}

// Get element (with bounds checking)
void* vector_get(Vector* v, size_t index) {
    if (index >= v->size) return NULL;
    return v->data[index];
}

// Remove last element
void* vector_pop(Vector* v) {
    if (v->size == 0) return NULL;
    return v->data[--v->size];
}

// Cleanup
void vector_destroy(Vector* v) {
    free(v->data);
    free(v);
}

// Usage example
Vector* files = vector_create();
vector_push(files, "main.c");    // size=1, capacity=8
vector_push(files, "utils.c");   // size=2, capacity=8
vector_push(files, "parser.c");  // size=3, capacity=8

// Iterate through all items
for (size_t i = 0; i < files->size; i++) {
    printf("%s\n", (char*)vector_get(files, i));
}

// What happened behind the scenes:
// 1. vector_create() allocated the struct but data is NULL
// 2. First push: capacity was 0, so we allocated space for 8 items
// 3. Second push: capacity is 8, size is 1, plenty of room---just add
// 4. Third push: capacity is 8, size is 2, still room---just add
// 5. If we kept pushing to the 9th item, we'd reallocate to capacity 16
\end{lstlisting}

\begin{tipbox}
\textbf{Why doubling works:} This is one of the most elegant algorithms in computer science. When capacity doubles each time (1 -> 2 -> 4 -> 8 -> 16...), the \textit{amortized} cost of insertion is O(1).

Here's why: To insert n items, we copy at most 1 + 2 + 4 + 8 + ... + n items total. That sum equals approximately 2n. So 2n copies for n insertions = 2 copies per insertion on average. That's constant time!

Growing by a fixed amount (+10 each time) would be: 10 + 20 + 30 + 40 + ... + n = O(n$^2$) total copies. Terrible!

This is why \textit{every} professional implementation doubles: Redis, Linux kernel, Git, Python, Java, C++. It's not arbitrary---it's mathematically optimal.
\end{tipbox}

\subsection{Type-Safe Vectors with Macros}

The generic vector above stores \texttt{void*} (pointers to anything), which means you lose type safety. You could accidentally store an \texttt{int} where you meant to store a string, and the compiler won't warn you. You have to remember to cast everything back to the right type.

\textbf{The problem:} \texttt{void*} is like a bag that can hold anything. That's flexible but dangerous. You can put a shoe in a bag labeled "books" and the bag doesn't care---but you'll be confused later when you pull out a shoe instead of a book.

\textbf{The solution:} Generate specialized versions for each type you need. Instead of one generic vector that stores \texttt{void*}, we create \texttt{int\_vector} that stores \texttt{int}, \texttt{string\_vector} that stores \texttt{char*}, etc. Now the compiler knows what type each vector holds and can catch mistakes.

How do we avoid writing the same code 20 times? Macros! We write the code once as a macro, then "stamp out" copies for different types. It's like a cookie cutter---one template, many cookies.

\begin{lstlisting}
// Macro to define a type-safe vector
#define DEFINE_VECTOR(T) \
    typedef struct { \
        T* data; \
        size_t size; \
        size_t capacity; \
    } T##_vector; \
    \
    T##_vector* T##_vector_create(void) { \
        T##_vector* v = malloc(sizeof(T##_vector)); \
        if (v) { \
            v->data = NULL; \
            v->size = 0; \
            v->capacity = 0; \
        } \
        return v; \
    } \
    \
    int T##_vector_push(T##_vector* v, T item) { \
        if (v->size >= v->capacity) { \
            size_t new_cap = v->capacity ? v->capacity * 2 : 8; \
            T* new_data = realloc(v->data, new_cap * sizeof(T)); \
            if (!new_data) return -1; \
            v->data = new_data; \
            v->capacity = new_cap; \
        } \
        v->data[v->size++] = item; \
        return 0; \
    } \
    \
    T T##_vector_get(T##_vector* v, size_t i) { \
        return v->data[i]; \
    } \
    \
    void T##_vector_destroy(T##_vector* v) { \
        free(v->data); \
        free(v); \
    }

// Generate int vector
// This one line expands to ~30 lines of code!
// The preprocessor copies the DEFINE_VECTOR template,
// replacing every "T" with "int"
DEFINE_VECTOR(int)

// Now we have int_vector, int_vector_create, int_vector_push, etc.

// Usage: type-safe!
int_vector* numbers = int_vector_create();
int_vector_push(numbers, 42);     // Compiler knows this is int
int_vector_push(numbers, 100);
int value = int_vector_get(numbers, 0);  // Returns int, no casting!

// If you try: int_vector_push(numbers, "hello");
// Compiler error! Can't pass char* where int is expected.
// With void*, this would silently compile and crash at runtime.

// Generate more types as needed:
// DEFINE_VECTOR(float)   -> float_vector
// DEFINE_VECTOR(char*)   -> char_ptr_vector (yes, that works!)
// DEFINE_VECTOR(MyStruct) -> MyStruct_vector
\end{lstlisting}

\subsection{Hash Tables: Fast Lookups Everywhere}

After dynamic arrays, hash tables are the second most important data structure. If you've used Python dictionaries, JavaScript objects, or Java HashMaps, you've used hash tables.

\textbf{What's the problem they solve?} Imagine you have a phone book with 1 million names. You want to find "John Smith." With an array, you'd have to check every entry until you find him---potentially 1 million checks! With a hash table, you can find him in typically just 1-2 checks. That's the magic.

\textbf{How do they work?} Think of a hash table like a filing cabinet with 128 drawers (we'll use 128 for this example). When you want to store "John Smith," you:
\begin{enumerate}
    \item Run his name through a \textit{hash function}---a math formula that converts "John Smith" into a number, say 47
    \item Put his data in drawer \#47
    \item Later, when looking up "John Smith," hash it again (gets 47), check drawer \#47---found!
\end{enumerate}

\textbf{What if two names hash to the same drawer?} This is called a \textit{collision}. Each drawer is actually a linked list, so drawer \#47 might contain chains of people: "John Smith" -> "Jane Doe" -> "Bob Jones" (if they all hashed to 47). You walk through the chain comparing names. Still way faster than searching 1 million entries!

Every language runtime, database, and compiler uses hash tables. Python dicts, Redis hashes, browser DOM lookups, Git object storage---all hash tables underneath.

\begin{lstlisting}
// Hash table with chaining (separate chaining)
// This is how Python dicts, Redis hashes, and most hash
// tables are implemented

#define TABLE_SIZE 128  // Power of 2 for fast modulo

typedef struct HashNode {
    char* key;
    void* value;
    struct HashNode* next;  // For collision handling
} HashNode;

typedef struct {
    HashNode* buckets[TABLE_SIZE];
    size_t count;
} HashTable;

// Simple hash function (djb2)
// Used by: Perl, Berkeley DB, many others
unsigned long hash_string(const char* str) {
    unsigned long hash = 5381;  // Magic starting value
    int c;

    // Process each character in the string
    while ((c = *str++)) {
        // The hash formula: hash = hash * 33 + character
        // << 5 means "multiply by 32" (shift left 5 bits)
        // So (hash << 5) + hash = hash * 32 + hash = hash * 33
        hash = ((hash << 5) + hash) + c;
    }

    return hash;

    // Why this formula? Dan Bernstein (djb) discovered that
    // multiplying by 33 and adding characters gives excellent
    // distribution---different strings rarely hash to same number.
    // Why 33 specifically? It's prime, close to a power of 2,
    // and experimentally works well. Sometimes algorithms are
    // more art than science!
}

// Create hash table
HashTable* hashtable_create(void) {
    HashTable* table = malloc(sizeof(HashTable));
    if (!table) return NULL;

    // Initialize all buckets to NULL
    for (int i = 0; i < TABLE_SIZE; i++) {
        table->buckets[i] = NULL;
    }
    table->count = 0;

    return table;
}

// Insert or update
void hashtable_set(HashTable* table, const char* key, void* value) {
    // Step 1: Hash the key to get a big number
    unsigned long hash = hash_string(key);

    // Step 2: Convert to bucket index (0 to TABLE_SIZE-1)
    // % is modulo: 9847 % 128 = 71, so use bucket 71
    int bucket = hash % TABLE_SIZE;

    // Step 3: Check if this key already exists in this bucket
    // Walk through the linked list in this bucket
    HashNode* node = table->buckets[bucket];
    while (node) {
        if (strcmp(node->key, key) == 0) {
            // Found it! Update the value and we're done
            // This is like updating an existing phone book entry
            node->value = value;
            return;
        }
        node = node->next;  // Keep looking in chain
    }

    // Step 4: Key doesn't exist, create new entry
    // Insert at HEAD of chain (faster than tail)
    HashNode* new_node = malloc(sizeof(HashNode));
    new_node->key = strdup(key);  // strdup copies the string
    new_node->value = value;
    new_node->next = table->buckets[bucket];  // Point to old head
    table->buckets[bucket] = new_node;        // Make this new head
    table->count++;

    // Why insert at head? O(1) instead of O(n) for tail.
    // We'd have to walk entire chain to find the tail.
    // Order doesn't matter in a hash table anyway.
}

// Lookup
void* hashtable_get(HashTable* table, const char* key) {
    unsigned long hash = hash_string(key);
    int bucket = hash % TABLE_SIZE;

    HashNode* node = table->buckets[bucket];
    while (node) {
        if (strcmp(node->key, key) == 0) {
            return node->value;
        }
        node = node->next;
    }

    return NULL;  // Not found
}

// Delete
int hashtable_delete(HashTable* table, const char* key) {
    unsigned long hash = hash_string(key);
    int bucket = hash % TABLE_SIZE;

    HashNode** node_ptr = &table->buckets[bucket];

    while (*node_ptr) {
        HashNode* node = *node_ptr;
        if (strcmp(node->key, key) == 0) {
            *node_ptr = node->next;  // Remove from chain
            free(node->key);
            free(node);
            table->count--;
            return 0;
        }
        node_ptr = &node->next;
    }

    return -1;  // Not found
}

// Cleanup
void hashtable_destroy(HashTable* table) {
    for (int i = 0; i < TABLE_SIZE; i++) {
        HashNode* node = table->buckets[i];
        while (node) {
            HashNode* next = node->next;
            free(node->key);
            free(node);
            node = next;
        }
    }
    free(table);
}

// Usage
HashTable* config = hashtable_create();
hashtable_set(config, "host", "localhost");
hashtable_set(config, "port", "8080");
hashtable_set(config, "debug", "true");

char* host = hashtable_get(config, "host");
printf("Host: %s\n", host);
\end{lstlisting}

\begin{notebox}
\textbf{Why separate chaining?} There are two main ways to handle collisions:

\begin{enumerate}
    \item \textbf{Separate chaining} (what we're doing): Each bucket contains a linked list. Simple, works well even when table is 75\% full.

    \item \textbf{Open addressing}: Store everything in the main array. On collision, try the next slot, then next, until you find empty space. Faster when table is mostly empty, but degrades terribly when full.
\end{enumerate}

Redis uses separate chaining. Python uses open addressing. Both work, but separate chaining is simpler and more predictable. It's the "safe" choice.

\textbf{Load factor matters:} Load factor = count / TABLE\_SIZE. If you have 100 items in 128 buckets, load factor is 0.78.

When load factor exceeds ~0.75, chains get long and lookups slow down. The fix: double the table size (128 -> 256) and rehash everything. This is expensive but happens rarely---once you hit 96 items, then not again until 192.

Professional implementations watch the load factor and automatically resize. We're keeping it simple here, but real hash tables in Redis, Python, etc. all do this.
\end{notebox}

\subsection{Circular Buffers: For Queues and Streaming}

Circular buffers (also called ring buffers) are the unsung heroes of system programming. They're used everywhere: audio processing, network packet buffers, logging systems, kernel message queues, serial port drivers---anywhere you need a fixed-size queue.

\textbf{What problem do they solve?} Imagine streaming audio from Spotify. Audio data arrives continuously, and your speakers play it continuously. You need a buffer in between---but you can't let it grow forever (you'd run out of memory), and you can't keep reallocating (too slow, causes glitches).

Solution: A circular buffer! It's a fixed-size buffer that "wraps around" like a clock. When you write past the end, you wrap back to the beginning. It's perfect for producer-consumer problems where one thread/process generates data and another consumes it.

\textbf{The mental model:} Picture a circular conveyor belt at a sushi restaurant. The chef (producer) puts plates on one side, customers (consumers) take plates from the other side. The belt is fixed-size and keeps rotating. If it's full, the chef waits. If it's empty, customers wait. Perfect!

\begin{lstlisting}
// Circular buffer - fixed size, efficient FIFO
// Used by: Linux kernel, audio systems, network stacks

typedef struct {
    char* buffer;       // The actual data storage
    size_t capacity;    // Total size (never changes)
    size_t head;        // Where we write next (producer position)
    size_t tail;        // Where we read next (consumer position)
    size_t count;       // How many items currently stored
} CircularBuffer;

// Why both head/tail AND count?
// - head and tail tell us WHERE in the buffer
// - count tells us HOW MUCH data is there
// - Without count, we can't distinguish "full" from "empty"
//   (both would have head == tail)

CircularBuffer* cbuf_create(size_t capacity) {
    CircularBuffer* cb = malloc(sizeof(CircularBuffer));
    if (!cb) return NULL;

    cb->buffer = malloc(capacity);
    if (!cb->buffer) {
        free(cb);
        return NULL;
    }

    cb->capacity = capacity;
    cb->head = 0;     // Start at beginning
    cb->tail = 0;     // Start at beginning
    cb->count = 0;    // Empty initially

    return cb;
}

// Write data (returns bytes written, which may be less than requested)
size_t cbuf_write(CircularBuffer* cb, const char* data, size_t size) {
    // Step 1: Calculate available space
    // If capacity is 100 and count is 60, space is 40
    size_t space = cb->capacity - cb->count;

    // Step 2: Write only what fits
    // If user wants to write 50 bytes but space is 40, write only 40
    size_t to_write = size < space ? size : space;

    // Step 3: Write bytes one at a time
    for (size_t i = 0; i < to_write; i++) {
        // Put byte at head position
        cb->buffer[cb->head] = data[i];

        // Move head forward, wrapping around if needed
        // The magic: (head + 1) % capacity
        // If head is 99 and capacity is 100: (99+1) % 100 = 0
        // Head wraps from end back to beginning!
        cb->head = (cb->head + 1) % cb->capacity;

        cb->count++;  // One more byte in buffer
    }

    return to_write;  // Tell caller how many we actually wrote
}

// Read data (returns bytes read)
size_t cbuf_read(CircularBuffer* cb, char* data, size_t size) {
    // Step 1: Can't read more than what's available
    size_t to_read = size < cb->count ? size : cb->count;

    // Step 2: Read bytes one at a time
    for (size_t i = 0; i < to_read; i++) {
        // Get byte from tail position
        data[i] = cb->buffer[cb->tail];

        // Move tail forward, wrapping around
        // Tail "chases" head around the circle
        cb->tail = (cb->tail + 1) % cb->capacity;

        cb->count--;  // One less byte in buffer
    }

    return to_read;

    // Example: Capacity 8, head at 3, tail at 0, count 3
    // Buffer: [A][B][C][_][_][_][_][_]
    //         ^tail       ^head
    // After read 2: tail moves to 2, count becomes 1
    // Buffer: [A][B][C][_][_][_][_][_]
    //               ^tail ^head
}

// Check if full/empty
int cbuf_is_full(CircularBuffer* cb) {
    return cb->count == cb->capacity;
}

int cbuf_is_empty(CircularBuffer* cb) {
    return cb->count == 0;
}

void cbuf_destroy(CircularBuffer* cb) {
    free(cb->buffer);
    free(cb);
}

// Example: Audio buffer
CircularBuffer* audio_buf = cbuf_create(4096);

// Producer thread writes audio samples
char samples[512];
cbuf_write(audio_buf, samples, 512);

// Consumer thread reads samples
char output[256];
cbuf_read(audio_buf, output, 256);
\end{lstlisting}

\begin{tipbox}
\textbf{Why circular buffers are amazing:}

\begin{enumerate}
    \item \textbf{No dynamic allocation}: Once created, no malloc/free during operation. Perfect for real-time systems where allocation causes unpredictable delays.

    \item \textbf{Predictable performance}: Every operation is O(1). No surprises, no slowdowns.

    \item \textbf{Cache-friendly}: Data is in one contiguous block, making the CPU happy.

    \item \textbf{Lock-free possible}: With careful design, one reader and one writer can work without locks. Blazing fast!
\end{enumerate}

The Linux kernel uses circular buffers for kernel log messages (\texttt{dmesg}). Audio systems use them to prevent buffer underruns (glitches). Network drivers use them for packet queues. Serial port drivers use them for incoming data. When you need a fixed-size queue, circular buffers are the answer.
\end{tipbox}

\subsection{Binary Trees: When You Need Ordering}

Binary search trees (BSTs) are for when you need data \textit{sorted} and you need to search efficiently. Arrays give you O(n) search. Hash tables give you O(1) search but no ordering. BSTs give you O(log n) search \textit{and} maintain sorted order.

\textbf{The mental model:} A binary tree is like a family tree or org chart, but with rules. Each node has at most two children: left and right. The rule: \textbf{left child < parent < right child}. This simple rule makes searching fast.

\textbf{Example:} Insert numbers 5, 3, 7, 1, 4, 6, 9:
\begin{verbatim}
        5
       / \
      3   7
     / \ / \
    1  4 6  9
\end{verbatim}

Want to find 6? Start at 5. Is 6 < 5? No, go right to 7. Is 6 < 7? Yes, go left to 6. Found! Only 3 comparisons instead of scanning all 7 items.

\textbf{Why O(log n)?} In a balanced tree, each level down cuts remaining nodes in half. 1000 nodes? ~10 levels. 1,000,000 nodes? ~20 levels. That's the power of logarithms!

\textbf{The catch:} Trees can become unbalanced. If you insert 1, 2, 3, 4, 5 in order, you get a "stick" (linked list in disguise) and lose all benefits. Red-black trees and AVL trees fix this by rebalancing automatically. The Linux kernel uses red-black trees everywhere.

\begin{lstlisting}
// Simple binary search tree (BST)
// Note: This is unbalanced. For production, use red-black trees
// (Linux kernel) or AVL trees

typedef struct TreeNode {
    int key;            // What we're searching on
    void* value;        // Associated data
    struct TreeNode* left;   // Left child (smaller keys)
    struct TreeNode* right;  // Right child (larger keys)
} TreeNode;

typedef struct {
    TreeNode* root;
    size_t count;
} BST;

BST* bst_create(void) {
    BST* tree = malloc(sizeof(BST));
    if (tree) {
        tree->root = NULL;
        tree->count = 0;
    }
    return tree;
}

// Insert (recursive) - this is elegant but uses call stack
TreeNode* bst_insert_node(TreeNode* node, int key, void* value) {
    // Base case: found empty spot, create new node here
    if (!node) {
        TreeNode* new_node = malloc(sizeof(TreeNode));
        new_node->key = key;
        new_node->value = value;
        new_node->left = NULL;
        new_node->right = NULL;
        return new_node;
    }

    // Recursive case: decide which direction to go
    if (key < node->key) {
        // key is smaller, belongs in left subtree
        // Recursively insert, then update left pointer
        node->left = bst_insert_node(node->left, key, value);
    } else if (key > node->key) {
        // key is larger, belongs in right subtree
        node->right = bst_insert_node(node->right, key, value);
    } else {
        // key equals node->key, already exists
        // Update the value (like hash table)
        node->value = value;
    }

    return node;

    // How recursion works here:
    // To insert 6 into tree with root 5:
    // 1. 6 > 5, so recurse on right subtree
    // 2. Right subtree might be empty -> create node 6
    // 3. Return node 6 back up
    // 4. Set node 5's right pointer to node 6
    // Done!
}

void bst_insert(BST* tree, int key, void* value) {
    tree->root = bst_insert_node(tree->root, key, value);
    tree->count++;
}

// Search (iterative - faster than recursive, no stack overhead)
void* bst_search(BST* tree, int key) {
    TreeNode* node = tree->root;  // Start at top

    // Keep going down until we find it or hit a dead end
    while (node) {
        if (key == node->key) {
            // Found it!
            return node->value;
        } else if (key < node->key) {
            // Key is smaller, search left subtree
            node = node->left;
        } else {
            // Key is larger, search right subtree
            node = node->right;
        }
    }

    return NULL;  // Reached NULL, key doesn't exist

    // Example: Search for 6 in tree with root 5
    // Start at 5: 6 > 5, go right
    // At node 7: 6 < 7, go left
    // At node 6: 6 == 6, found! Return value.
    //
    // Why iterative instead of recursive?
    // - No function call overhead
    // - No risk of stack overflow on deep trees
    // - Slightly faster in practice
}

// In-order traversal (prints keys in sorted order)
void bst_traverse_inorder(TreeNode* node,
                         void (*callback)(int key, void* value)) {
    if (!node) return;

    bst_traverse_inorder(node->left, callback);
    callback(node->key, node->value);
    bst_traverse_inorder(node->right, callback);
}

// Cleanup (post-order)
void bst_destroy_node(TreeNode* node) {
    if (!node) return;
    bst_destroy_node(node->left);
    bst_destroy_node(node->right);
    free(node);
}

void bst_destroy(BST* tree) {
    bst_destroy_node(tree->root);
    free(tree);
}

// Usage
BST* users = bst_create();
bst_insert(users, 42, "Alice");
bst_insert(users, 17, "Bob");
bst_insert(users, 99, "Charlie");

char* name = bst_search(users, 42);
printf("User 42: %s\n", name);  // Alice
\end{lstlisting}

\begin{warningbox}
\textbf{The unbalanced tree problem:} If you insert sorted data (1, 2, 3, 4, 5...), the tree degenerates into a linked list:
\begin{verbatim}
    1
     \
      2
       \
        3
         \
          4
           \
            5
\end{verbatim}

Now search is O(n) again! All benefits lost. This actually happens in practice---imagine inserting timestamps or IDs in order.

\textbf{Solutions---balanced tree variants:}
\begin{itemize}
    \item \textbf{Red-black trees}: Linux kernel's \texttt{rbtree}. Guarantees O(log n) by keeping tree "roughly" balanced. After every insert/delete, performs rotations to maintain balance. Most common in systems programming.

    \item \textbf{AVL trees}: More strictly balanced than red-black. Slightly faster search, slightly slower insert/delete. Good when reads outnumber writes.

    \item \textbf{B-trees}: Not binary (many children per node). Used by every database (SQLite, PostgreSQL, MySQL). Optimized for disk I/O---read entire disk blocks at once.

    \item \textbf{Splay trees}: Self-adjusting. Recently accessed items move to top. Good for non-uniform access patterns.
\end{itemize}

\textbf{Critical advice:} Don't implement balanced trees yourself unless you're writing a database or OS kernel. The algorithms are tricky and easy to get wrong. Use proven libraries: \texttt{<sys/tree.h>} on BSD, Linux kernel's rbtree, or just use a hash table (often good enough).
\end{warningbox}

\subsection{Skip Lists: Probabilistic Alternative to Trees}

Skip lists are the "lazy" alternative to balanced trees. Instead of carefully maintaining balance, they use randomness. Sounds crazy, but it works beautifully! Redis uses skip lists for sorted sets (ZSET), and they're simpler to implement than red-black trees.

\textbf{The idea:} A skip list is multiple linked lists stacked on top of each other. The bottom list contains all elements. Each higher list is a "fast lane" that skips elements. Like a highway system: local roads connect everything, highways skip to major cities, and interstates skip even further.

\textbf{How it works:} When you search for an element, start at the highest level. Follow pointers until you overshoot, then drop down a level. Repeat until you find the element or reach bottom. You "skip over" large sections, hence the name.

\textbf{Example:} Skip list with 3 levels for numbers 1, 3, 5, 7, 9:
\begin{verbatim}
Level 3: 1 -----------------> 9 -> NULL
Level 2: 1 -----> 5 --------> 9 -> NULL
Level 1: 1 -> 3 -> 5 -> 7 -> 9 -> NULL
\end{verbatim}

To find 7: Start at level 3 at 1. Next is 9 > 7, so drop to level 2. At 5. Next is 9 > 7, drop to level 1. At 5, 7, found! Only checked 5 nodes, not all 9.

\textbf{Why "probabilistic"?} When inserting, we flip a coin to decide the node's height. 50\% chance it's height 1, 25\% chance height 2, 12.5\% chance height 3, etc. On average, this creates a balanced structure without explicit balancing!

\begin{lstlisting}
// Skip list - used by Redis for sorted sets
// Simpler than red-black trees, similar performance

#define MAX_LEVEL 16  // Max height of any node

typedef struct SkipNode {
    int key;
    void* value;
    struct SkipNode* forward[MAX_LEVEL];  // Array of forward pointers
    // forward[0] = next node at level 0 (bottom)
    // forward[1] = next node at level 1 (one up)
    // forward[2] = next node at level 2 (two up)
    // etc.
} SkipNode;

typedef struct {
    SkipNode* header;
    int level;  // Current max level
} SkipList;

// Random level for new nodes (geometric distribution)
// This is the "magic" of skip lists---randomness creates balance!
int random_level(void) {
    int level = 1;  // Every node is at least level 1

    // Flip coins: 50% chance to go higher each time
    // Level 1: 100% (guaranteed)
    // Level 2: 50% (half the nodes)
    // Level 3: 25% (quarter of nodes)
    // Level 4: 12.5% (eighth of nodes)
    // This creates the "fast lanes" naturally!
    while (rand() < RAND_MAX / 2 && level < MAX_LEVEL) {
        level++;
    }
    return level;

    // Why does randomness work? Law of large numbers.
    // With many nodes, distribution averages out to
    // create balanced structure. No explicit rebalancing needed!
}

SkipList* skiplist_create(void) {
    SkipList* list = malloc(sizeof(SkipList));
    list->level = 1;

    // Create header node
    list->header = malloc(sizeof(SkipNode));
    list->header->key = INT_MIN;
    for (int i = 0; i < MAX_LEVEL; i++) {
        list->header->forward[i] = NULL;
    }

    return list;
}

void skiplist_insert(SkipList* list, int key, void* value) {
    SkipNode* update[MAX_LEVEL];
    SkipNode* current = list->header;

    // Find insertion point at each level
    for (int i = list->level - 1; i >= 0; i--) {
        while (current->forward[i] &&
               current->forward[i]->key < key) {
            current = current->forward[i];
        }
        update[i] = current;
    }

    // Create new node with random level
    int new_level = random_level();
    if (new_level > list->level) {
        for (int i = list->level; i < new_level; i++) {
            update[i] = list->header;
        }
        list->level = new_level;
    }

    SkipNode* new_node = malloc(sizeof(SkipNode));
    new_node->key = key;
    new_node->value = value;

    // Insert at each level
    for (int i = 0; i < new_level; i++) {
        new_node->forward[i] = update[i]->forward[i];
        update[i]->forward[i] = new_node;
    }
}

void* skiplist_search(SkipList* list, int key) {
    SkipNode* current = list->header;

    // Start from highest level, drop down when needed
    for (int i = list->level - 1; i >= 0; i--) {
        while (current->forward[i] &&
               current->forward[i]->key < key) {
            current = current->forward[i];
        }
    }

    current = current->forward[0];
    if (current && current->key == key) {
        return current->value;
    }

    return NULL;
}

// Why Redis chose skip lists over red-black trees:
//
// 1. Simpler implementation: ~200 lines vs 1000+ for red-black trees
// 2. Easier to understand: No complex rotation cases
// 3. Easier to debug: Can visualize the structure easily
// 4. Similar performance: O(log n) in practice
// 5. Better for range scans: Following level 0 gives sorted order
// 6. Lock-free variants exist: Easier than lock-free trees
// 7. Probabilistic, not deterministic: Randomness is simple
//
// The trade-off: Red-black trees have GUARANTEED O(log n).
// Skip lists have EXPECTED O(log n) (could theoretically be worse,
// but probability of bad luck decreases exponentially).
//
// In practice, skip lists perform identically to balanced trees
// and are much simpler to implement correctly.
\end{lstlisting}

\subsection{Tries (Prefix Trees): For String Lookups}

Tries are perfect for autocomplete, spell checkers, and IP routing tables. They provide O(k) lookup where k is the key length.

\begin{lstlisting}
// Trie (prefix tree) for string keys
// Used by: spell checkers, autocomplete, IP routing

#define ALPHABET_SIZE 26

typedef struct TrieNode {
    struct TrieNode* children[ALPHABET_SIZE];
    int is_end;      // Is this a complete word?
    void* value;     // Associated data
} TrieNode;

typedef struct {
    TrieNode* root;
} Trie;

TrieNode* trie_node_create(void) {
    TrieNode* node = calloc(1, sizeof(TrieNode));
    return node;  // calloc zeros all children pointers
}

Trie* trie_create(void) {
    Trie* trie = malloc(sizeof(Trie));
    trie->root = trie_node_create();
    return trie;
}

void trie_insert(Trie* trie, const char* key, void* value) {
    TrieNode* node = trie->root;

    for (const char* p = key; *p; p++) {
        int index = tolower(*p) - 'a';  // Assume lowercase a-z

        if (!node->children[index]) {
            node->children[index] = trie_node_create();
        }

        node = node->children[index];
    }

    node->is_end = 1;
    node->value = value;
}

void* trie_search(Trie* trie, const char* key) {
    TrieNode* node = trie->root;

    for (const char* p = key; *p; p++) {
        int index = tolower(*p) - 'a';

        if (!node->children[index]) {
            return NULL;  // Key not found
        }

        node = node->children[index];
    }

    return node->is_end ? node->value : NULL;
}

// Check if prefix exists
int trie_has_prefix(Trie* trie, const char* prefix) {
    TrieNode* node = trie->root;

    for (const char* p = prefix; *p; p++) {
        int index = tolower(*p) - 'a';

        if (!node->children[index]) {
            return 0;
        }

        node = node->children[index];
    }

    return 1;
}

// Usage: Dictionary
Trie* dict = trie_create();
trie_insert(dict, "cat", "a feline animal");
trie_insert(dict, "car", "a vehicle");
trie_insert(dict, "card", "a piece of paper");

void* def = trie_search(dict, "cat");
printf("Cat: %s\n", (char*)def);

// Check prefix
if (trie_has_prefix(dict, "ca")) {
    printf("Words starting with 'ca' exist\n");
}
\end{lstlisting}

\begin{tipbox}
\textbf{Tries are memory-hungry but fast:} Each node needs ALPHABET\_SIZE pointers. For large alphabets (Unicode), use compressed tries (radix trees) like Git uses for file paths. Redis uses radix trees for key lookups.
\end{tipbox}

\subsection{Bloom Filters: Probabilistic Set Membership}

Bloom filters answer "Is X in the set?" with:
\begin{itemize}
    \item \textbf{Maybe yes} (with controllable false positive rate)
    \item \textbf{Definitely no} (no false negatives)
\end{itemize}

Used by: Chrome (malicious URLs), Cassandra (disk reads), Bitcoin (transaction filtering).

\begin{lstlisting}
// Bloom filter - space-efficient probabilistic set
// Perfect for "probably contains" checks

#define BLOOM_SIZE 1024  // Bit array size
#define NUM_HASHES 3     // Number of hash functions

typedef struct {
    unsigned char bits[BLOOM_SIZE / 8];  // Bit array
    size_t count;
} BloomFilter;

// Hash functions (simple for demo, use better in production)
unsigned int hash1(const char* str) {
    unsigned int hash = 0;
    while (*str) hash = hash * 31 + *str++;
    return hash % BLOOM_SIZE;
}

unsigned int hash2(const char* str) {
    unsigned int hash = 5381;
    while (*str) hash = hash * 33 + *str++;
    return hash % BLOOM_SIZE;
}

unsigned int hash3(const char* str) {
    unsigned int hash = 0;
    while (*str) hash = hash * 65599 + *str++;
    return hash % BLOOM_SIZE;
}

BloomFilter* bloom_create(void) {
    BloomFilter* bf = calloc(1, sizeof(BloomFilter));
    return bf;
}

// Set bit at position
void bloom_set_bit(BloomFilter* bf, unsigned int pos) {
    bf->bits[pos / 8] |= (1 << (pos % 8));
}

// Get bit at position
int bloom_get_bit(BloomFilter* bf, unsigned int pos) {
    return (bf->bits[pos / 8] & (1 << (pos % 8))) != 0;
}

// Add element
void bloom_add(BloomFilter* bf, const char* str) {
    bloom_set_bit(bf, hash1(str));
    bloom_set_bit(bf, hash2(str));
    bloom_set_bit(bf, hash3(str));
    bf->count++;
}

// Check membership (may have false positives)
int bloom_maybe_contains(BloomFilter* bf, const char* str) {
    return bloom_get_bit(bf, hash1(str)) &&
           bloom_get_bit(bf, hash2(str)) &&
           bloom_get_bit(bf, hash3(str));
}

// Example: Spam filter
BloomFilter* spam_filter = bloom_create();
bloom_add(spam_filter, "viagra");
bloom_add(spam_filter, "casino");
bloom_add(spam_filter, "lottery");

if (bloom_maybe_contains(spam_filter, "viagra")) {
    // Maybe spam (could be false positive)
    // Do expensive check
}

if (!bloom_maybe_contains(spam_filter, "hello")) {
    // Definitely not spam
}
\end{lstlisting}

\begin{notebox}
\textbf{Why Bloom filters?} Tiny memory footprint. A Bloom filter with 1\% false positive rate needs only ~10 bits per element. Compare that to a hash table which needs ~100 bits per element. Chrome uses Bloom filters to check billions of URLs against a malicious URL database---it would be impossible with hash tables.
\end{notebox}

\subsection{Comparison: When to Use Which Structure}

\begin{center}
\begin{tabular}{|l|l|l|l|}
\hline
\textbf{Structure} & \textbf{Best For} & \textbf{Time} & \textbf{Space} \\
\hline
Dynamic Array & Sequential access, append & O(1) avg & O(n) \\
Hash Table & Key-value, fast lookup & O(1) avg & O(n) \\
Circular Buffer & FIFO queue, streaming & O(1) & O(capacity) \\
BST (balanced) & Sorted data, range queries & O(log n) & O(n) \\
Skip List & Simpler than trees & O(log n) avg & O(n) \\
Trie & String prefixes, autocomplete & O(k) & O(alphabet \times n) \\
Bloom Filter & Set membership, huge sets & O(k) & O(bits) \\
\hline
\end{tabular}
\end{center}

\begin{tipbox}
\textbf{Real-world advice:}
\begin{itemize}
    \item \textbf{Start with arrays}: 90\% of the time, a dynamic array is enough
    \item \textbf{Hash tables for lookups}: When you need fast key-based access
    \item \textbf{Don't write balanced trees}: Use libraries (BSD tree.h, Linux rbtree)
    \item \textbf{Circular buffers for streaming}: Audio, network, logs
    \item \textbf{Tries for strings}: If you have many string keys with common prefixes
    \item \textbf{Bloom filters when space matters}: Billion-element sets in megabytes
\end{itemize}
\end{tipbox}

\section{Summary: Struct Mastery}

You've now learned everything professionals know about structs and data structures in C. This chapter covered:

\subsection{Struct Fundamentals}

\begin{itemize}
    \item \textbf{Memory layout}: Order members largest-to-smallest to minimize padding (can save 50\% memory)
    \item \textbf{Alignment}: CPUs require data aligned to natural boundaries for performance
    \item \textbf{Padding}: Compiler adds invisible gaps---understand them to optimize
    \item \textbf{Packing}: Use \texttt{\#pragma pack} only for external formats, not normal code
\end{itemize}

\subsection{Advanced Struct Patterns}

\begin{itemize}
    \item \textbf{Flexible arrays}: C99 flexible array members for variable-length data
    \item \textbf{Inheritance}: First member = base struct for OOP-style inheritance
    \item \textbf{VTables}: Function pointer tables for true polymorphism
    \item \textbf{Type tags}: Runtime type information for safe variant types
    \item \textbf{Bit fields}: Pack booleans and small ints (but watch portability)
    \item \textbf{Designated initializers}: Self-documenting, order-independent initialization
    \item \textbf{Anonymous unions}: Cleaner access to variant data
    \item \textbf{Intrusive lists}: Embed links in structs for zero-overhead containers
\end{itemize}

\subsection{Essential Data Structures}

\begin{itemize}
    \item \textbf{Dynamic arrays}: The most common structure---use for 80\% of cases
    \item \textbf{Hash tables}: O(1) lookups for key-value pairs
    \item \textbf{Circular buffers}: Perfect for queues, streaming, real-time systems
    \item \textbf{Binary trees}: Use balanced variants (red-black, AVL) in production
    \item \textbf{Skip lists}: Simpler than trees, used by Redis
    \item \textbf{Tries}: String prefixes, autocomplete, routing tables
    \item \textbf{Bloom filters}: Space-efficient probabilistic membership testing
\end{itemize}

\subsection{Critical Best Practices}

\begin{itemize}
    \item \textbf{Deep copy}: Write custom functions for structs with pointers
    \item \textbf{Never use memcmp}: Padding bytes have undefined values
    \item \textbf{Explicit serialization}: Write fields individually, never dump raw structs
    \item \textbf{Always zero-initialize}: Prevents undefined behavior bugs
    \item \textbf{Choose the right structure}: Arrays for 90\%, hash tables for lookups, specialized for specific needs
\end{itemize}

\begin{tipbox}
Structs are the foundation of C programming---every data structure, every abstraction, every pattern is built from them. Master struct layout and you'll write memory-efficient code. Master struct patterns and you'll write maintainable code. Master data structures and you'll write professional code.

These techniques power:
\begin{itemize}
    \item \textbf{Linux kernel}: Intrusive lists, red-black trees, circular buffers
    \item \textbf{Redis}: Skip lists, hash tables, dynamic arrays
    \item \textbf{SQLite}: B-trees, hash tables, flexible arrays
    \item \textbf{Git}: Tries (radix trees), hash tables, packed objects
    \item \textbf{Chrome}: Bloom filters for malicious URL checks
\end{itemize}

You now have the complete professional toolkit. Start with simple arrays, graduate to hash tables when needed, and use specialized structures when they genuinely solve your problem better. And remember: the best data structure is the simplest one that meets your requirements.
\end{tipbox}

\chapter{Header File Organization}

\section{The Purpose of Header Files}

Header files are C's way of declaring interfaces. They tell the compiler what exists without showing how it's implemented. Think of them as a contract between different parts of your program.

\begin{lstlisting}
// mylib.h - The interface (what users see)
#ifndef MYLIB_H
#define MYLIB_H

int add(int a, int b);
void process_data(const char* data);

#endif

// mylib.c - The implementation (how it works)
#include "mylib.h"

int add(int a, int b) {
    return a + b;
}

void process_data(const char* data) {
    // Implementation details
}
\end{lstlisting}

\section{Include Guards}

The most fundamental pattern - preventing multiple inclusion:

\begin{lstlisting}
// Traditional include guards
#ifndef MYHEADER_H
#define MYHEADER_H

// Header content here

#endif // MYHEADER_H
\end{lstlisting}

\begin{notebox}
The \texttt{\#ifndef} guard prevents the header from being included twice in the same translation unit, which would cause redefinition errors.
\end{notebox}

\subsection{Naming Include Guards}

\begin{lstlisting}
// Bad - generic names can collide
#ifndef UTILS_H
#define UTILS_H

// Better - project prefix
#ifndef MYPROJECT_UTILS_H
#define MYPROJECT_UTILS_H

// Best - full path encoding
#ifndef MYPROJECT_INCLUDE_UTILS_H
#define MYPROJECT_INCLUDE_UTILS_H

// Alternative - use pragma once
#pragma once
// Not standard but widely supported (GCC, Clang, MSVC)
\end{lstlisting}

\begin{tipbox}
Use \texttt{\#pragma once} for simplicity if you're targeting modern compilers. It's cleaner and can be faster to compile. Otherwise, use include guards with project-prefixed names.
\end{tipbox}

\section{Header File Anatomy}

A well-structured header follows this order:

\begin{lstlisting}
// mylib.h

// 1. Include guard / pragma once
#ifndef MYPROJECT_MYLIB_H
#define MYPROJECT_MYLIB_H

// 2. Feature test macros (if needed)
#define _POSIX_C_SOURCE 200809L

// 3. System includes
#include <stddef.h>
#include <stdint.h>

// 4. Project includes
#include "myproject/common.h"

// 5. C++ compatibility
#ifdef __cplusplus
extern "C" {
#endif

// 6. Preprocessor defines
#define MYLIB_VERSION_MAJOR 1
#define MYLIB_VERSION_MINOR 0

// 7. Type definitions and forward declarations
typedef struct MyObject MyObject;
typedef enum { SUCCESS, ERROR } Status;

// 8. Function declarations
MyObject* myobject_create(void);
Status myobject_process(MyObject* obj);
void myobject_destroy(MyObject* obj);

// 9. Inline functions (if any)
static inline int mylib_is_valid(MyObject* obj) {
    return obj != NULL;
}

// 10. Close C++ compatibility
#ifdef __cplusplus
}
#endif

// 11. Close include guard
#endif // MYPROJECT_MYLIB_H
\end{lstlisting}

\section{What Goes in Headers}

\subsection{YES - Put These in Headers}

\begin{lstlisting}
// Function declarations
int calculate(int x);

// Type definitions
typedef struct Point Point;
typedef int (*Callback)(void* data);

// Enums
typedef enum {
    STATE_IDLE,
    STATE_RUNNING,
    STATE_DONE
} State;

// Macros
#define MAX(a, b) ((a) > (b) ? (a) : (b))

// Inline functions (small, frequently used)
static inline int square(int x) {
    return x * x;
}

// External variable declarations
extern int global_counter;

// Constants
#define BUFFER_SIZE 1024
extern const char* VERSION_STRING;
\end{lstlisting}

\subsection{NO - Don't Put These in Headers}

\begin{lstlisting}
// Function implementations (unless inline/static)
// WRONG in header:
int calculate(int x) {
    return x * x;
}

// Variable definitions (only declarations)
// WRONG in header:
int global_counter = 0;

// Use extern instead:
extern int global_counter;

// Non-const data
// WRONG in header:
char buffer[1024];

// Large inline functions
// WRONG - makes compile slow:
static inline void huge_function(void) {
    // 100 lines of code...
}
\end{lstlisting}

\begin{warningbox}
Never define variables in headers (except \texttt{static inline} functions). This causes multiple definition errors when the header is included in multiple source files.
\end{warningbox}

\section{Forward Declarations}

Avoid including headers when a forward declaration suffices:

\begin{lstlisting}
// Instead of including the full header
// #include "widget.h"  // Full definition

// Use forward declaration
typedef struct Widget Widget;

// Now you can use pointers
Widget* get_widget(void);
void process_widget(Widget* w);

// You can't do this without full definition:
// Widget w;  // ERROR - incomplete type
// w.x = 10;  // ERROR - don't know struct layout
\end{lstlisting}

\subsection{Why Forward Declarations Matter}

\begin{lstlisting}
// window.h
#include "widget.h"  // Includes everything from widget.h

// widget.h
#include "window.h"  // Circular dependency!

// Solution: Use forward declarations
// window.h
typedef struct Widget Widget;  // Forward declaration
Widget* window_get_widget(void);

// widget.h
typedef struct Window Window;  // Forward declaration
Window* widget_get_window(void);
\end{lstlisting}

\section{Public vs Private Headers}

Professional projects separate public and private interfaces:

\begin{lstlisting}
// Public header (installed with library)
// include/mylib/mylib.h
#ifndef MYLIB_H
#define MYLIB_H

typedef struct MyObject MyObject;

MyObject* myobject_create(void);
void myobject_destroy(MyObject* obj);

#endif

// Private header (internal use only)
// src/mylib_internal.h
#ifndef MYLIB_INTERNAL_H
#define MYLIB_INTERNAL_H

#include "mylib/mylib.h"

// Full definition - only implementation sees this
struct MyObject {
    int value;
    char* name;
    void* internal_data;
};

// Internal helper functions
void internal_helper(MyObject* obj);
void internal_cleanup(void);

#endif
\end{lstlisting}

\section{C++ Compatibility}

Make your C headers usable from C++:

\begin{lstlisting}
#ifndef MYLIB_H
#define MYLIB_H

#ifdef __cplusplus
extern "C" {
#endif

// Your C declarations here
void my_function(int x);

#ifdef __cplusplus
}
#endif

#endif

// Why this matters:
// C++ mangles function names: my_function -> _Z11my_functioni
// extern "C" tells C++ to use C naming: my_function
\end{lstlisting}

\section{Platform-Specific Headers}

Handle platform differences cleanly:

\begin{lstlisting}
// platform.h
#ifndef PLATFORM_H
#define PLATFORM_H

// Detect platform
#if defined(_WIN32) || defined(_WIN64)
    #define PLATFORM_WINDOWS
    #include <windows.h>
#elif defined(__linux__)
    #define PLATFORM_LINUX
    #include <unistd.h>
#elif defined(__APPLE__)
    #define PLATFORM_MACOS
    #include <unistd.h>
#else
    #error "Unsupported platform"
#endif

// Platform-specific types
#ifdef PLATFORM_WINDOWS
    typedef HANDLE ThreadHandle;
    typedef DWORD ThreadId;
#else
    typedef pthread_t ThreadHandle;
    typedef pthread_t ThreadId;
#endif

// Platform-independent API
ThreadHandle thread_create(void (*func)(void*), void* arg);
void thread_join(ThreadHandle handle);

#endif
\end{lstlisting}

\section{Configuration Headers}

Generate configuration at build time:

\begin{lstlisting}
// config.h.in (template processed by build system)
#ifndef CONFIG_H
#define CONFIG_H

// Version information
#define VERSION_MAJOR @VERSION_MAJOR@
#define VERSION_MINOR @VERSION_MINOR@
#define VERSION_PATCH @VERSION_PATCH@
#define VERSION_STRING "@VERSION_STRING@"

// Feature detection
#cmakedefine HAVE_PTHREAD
#cmakedefine HAVE_OPENSSL
#cmakedefine HAVE_ZLIB

// Platform-specific
#cmakedefine WORDS_BIGENDIAN

// Sizes
#define SIZEOF_INT @SIZEOF_INT@
#define SIZEOF_LONG @SIZEOF_LONG@
#define SIZEOF_POINTER @SIZEOF_POINTER@

#endif

// Usage in code
#ifdef HAVE_PTHREAD
    #include <pthread.h>
    // Use threading
#else
    // Single-threaded fallback
#endif
\end{lstlisting}

\section{Minimizing Dependencies}

\begin{lstlisting}
// Bad - includes everything
// graphics.h
#include <stdio.h>      // Only need for implementation
#include <stdlib.h>     // Only need for implementation
#include <string.h>     // Only need for implementation
#include <math.h>       // Only need for implementation

typedef struct {
    double x, y;
} Point;

// Good - minimal includes
// graphics.h
// No includes needed!

typedef struct {
    double x, y;
} Point;

// graphics.c includes what it needs
#include "graphics.h"
#include <stdio.h>
#include <stdlib.h>
#include <string.h>
#include <math.h>
\end{lstlisting}

\begin{tipbox}
Only include headers in your header file if you need the complete type definition. Use forward declarations whenever possible.
\end{tipbox}

\section{Documentation in Headers}

Headers are the perfect place for API documentation:

\begin{lstlisting}
/**
 * @file mylib.h
 * @brief Public API for mylib
 * @author Your Name
 * @version 1.0
 */

#ifndef MYLIB_H
#define MYLIB_H

#include <stddef.h>

/**
 * @brief Create a new object
 *
 * Allocates and initializes a new object with default values.
 * The caller is responsible for freeing the object with
 * myobject_destroy().
 *
 * @return Pointer to new object, or NULL on failure
 *
 * @see myobject_destroy()
 *
 * Example:
 * @code
 * MyObject* obj = myobject_create();
 * if (obj) {
 *     // Use object
 *     myobject_destroy(obj);
 * }
 * @endcode
 */
MyObject* myobject_create(void);

/**
 * @brief Destroy an object
 *
 * Frees all resources associated with the object.
 * After calling this, the object pointer is invalid.
 *
 * @param obj Object to destroy (may be NULL)
 *
 * @note It's safe to pass NULL
 */
void myobject_destroy(MyObject* obj);

#endif
\end{lstlisting}

\section{Header Organization Patterns}

\subsection{Umbrella Headers}

\begin{lstlisting}
// myproject.h - One header includes all modules
#ifndef MYPROJECT_H
#define MYPROJECT_H

#include "myproject/core.h"
#include "myproject/utils.h"
#include "myproject/network.h"
#include "myproject/graphics.h"

#endif

// Users can include just one header:
#include <myproject.h>
\end{lstlisting}

\subsection{Layered Headers}

\begin{lstlisting}
// Layer 1: Platform abstraction
#include "platform/types.h"
#include "platform/threads.h"

// Layer 2: Core utilities
#include "core/memory.h"
#include "core/string.h"

// Layer 3: Domain logic
#include "domain/model.h"
#include "domain/logic.h"

// Each layer only depends on lower layers
\end{lstlisting}

\section{Header-Only Libraries}

Some libraries live entirely in headers:

\begin{lstlisting}
// stb-style header-only library
// mylib.h
#ifndef MYLIB_H
#define MYLIB_H

// Declarations visible to everyone
void mylib_function(void);

// Implementation only compiled once
#ifdef MYLIB_IMPLEMENTATION

void mylib_function(void) {
    // Implementation here
}

#endif // MYLIB_IMPLEMENTATION
#endif // MYLIB_H

// Usage:
// In one .c file:
#define MYLIB_IMPLEMENTATION
#include "mylib.h"

// In other files:
#include "mylib.h"
\end{lstlisting}

\section{Version Guards}

Detect and require minimum versions:

\begin{lstlisting}
// mylib.h
#ifndef MYLIB_H
#define MYLIB_H

#define MYLIB_VERSION_MAJOR 2
#define MYLIB_VERSION_MINOR 1
#define MYLIB_VERSION_PATCH 0

#define MYLIB_VERSION \
    ((MYLIB_VERSION_MAJOR * 10000) + \
     (MYLIB_VERSION_MINOR * 100) + \
     MYLIB_VERSION_PATCH)

// Check minimum required version
#ifdef MYLIB_REQUIRE_VERSION
    #if MYLIB_VERSION < MYLIB_REQUIRE_VERSION
        #error "mylib version too old"
    #endif
#endif

// Usage in user code:
#define MYLIB_REQUIRE_VERSION 20100  // Require 2.1.0
#include <mylib.h>
\end{lstlisting}

\section{Common Header Mistakes}

\subsection{Missing Include Guards}

\begin{lstlisting}
// WRONG - no include guard
// mylib.h
typedef struct Point {
    int x, y;
} Point;

// If included twice, compiler sees:
// typedef struct Point { int x, y; } Point;
// typedef struct Point { int x, y; } Point;
// ERROR: redefinition of 'Point'
\end{lstlisting}

\subsection{Using "using" in Headers}

\begin{lstlisting}
// C++ code - NEVER do this in headers!
// mylib.hpp
#include <string>
using namespace std;  // Pollutes all includers!

// Now anyone who includes this header has
// 'using namespace std' forced on them
\end{lstlisting}

\subsection{Including <windows.h> Carelessly}

\begin{lstlisting}
// WRONG - windows.h pollutes namespace
#include <windows.h>

// windows.h defines macros that break code:
// #define min(a,b) ...
// #define max(a,b) ...
// Now your min/max functions don't work!

// BETTER - define before including
#define WIN32_LEAN_AND_MEAN
#define NOMINMAX
#include <windows.h>
\end{lstlisting}

\section{Summary}

Header file best practices:

\begin{itemize}
    \item Always use include guards or \texttt{\#pragma once}
    \item Only declare, never define (except inline/static)
    \item Minimize includes - use forward declarations
    \item Separate public and private interfaces
    \item Add \texttt{extern "C"} for C++ compatibility
    \item Document your API in headers
    \item Never use \texttt{using namespace} in headers
    \item Keep headers minimal and focused
\end{itemize}

Well-organized headers make your API a joy to use!

\chapter{Preprocessor Directives and Techniques}

\section{Understanding the Preprocessor}

The C preprocessor is a text manipulation tool that runs before compilation. It doesn't understand C syntax—it just processes text based on directives that start with \texttt{\#}.

\begin{lstlisting}
// The preprocessor flow:
// 1. source.c --> [Preprocessor] --> expanded.i
// 2. expanded.i --> [Compiler] --> object.o
// 3. object.o --> [Linker] --> executable

// See preprocessor output:
// gcc -E source.c -o expanded.i
\end{lstlisting}

\begin{tipbox}
Use \texttt{gcc -E} to see exactly what the preprocessor does to your code. This is invaluable for debugging complex macros and understanding header inclusion.
\end{tipbox}

\section{Conditional Compilation}

The most fundamental preprocessor feature—compile different code for different scenarios.

\subsection{Basic Conditionals}

\begin{lstlisting}
// Check if defined
#ifdef DEBUG
    printf("Debug: x = %d\n", x);
#endif

// Check if not defined
#ifndef NDEBUG
    assert(x > 0);
#endif

// Check specific value
#if MAX_BUFFER_SIZE > 1024
    // Use optimized algorithm for large buffers
#else
    // Use simple algorithm
#endif

// Logical operators
#if defined(LINUX) || defined(MACOS)
    // Unix-like systems
#elif defined(WINDOWS)
    // Windows-specific code
#else
    #error "Unsupported platform"
#endif
\end{lstlisting}

\subsection{Feature Detection}

\begin{lstlisting}
// Check compiler
#if defined(__GNUC__)
    // GCC-specific code
    #define PACKED __attribute__((packed))
#elif defined(_MSC_VER)
    // MSVC-specific code
    #define PACKED __pragma(pack(push, 1))
#else
    #define PACKED
    #warning "Unknown compiler, packing not supported"
#endif

// Check C standard version
#if __STDC_VERSION__ >= 201112L
    // C11 or later - can use _Generic
    #define typename(x) _Generic((x), \
        int: "int", \
        float: "float", \
        default: "unknown")
#else
    // C99 fallback
    #define typename(x) "unknown"
#endif

// Check if specific features available
#ifdef __STDC_NO_THREADS__
    #error "C11 threads not available"
#endif
\end{lstlisting}

\subsection{Build Configuration}

\begin{lstlisting}
// Debug vs Release
#ifdef NDEBUG
    #define DEBUG_PRINT(...)
    #define ASSERT(x)
#else
    #define DEBUG_PRINT(...) fprintf(stderr, __VA_ARGS__)
    #define ASSERT(x) assert(x)
#endif

// Feature flags
#ifdef ENABLE_LOGGING
    #define LOG(level, ...) log_message(level, __VA_ARGS__)
#else
    #define LOG(level, ...)
#endif

#ifdef ENABLE_PROFILING
    #define PROFILE_START(name) Timer _t_ ## name = timer_start()
    #define PROFILE_END(name) timer_end(_t_ ## name, #name)
#else
    #define PROFILE_START(name)
    #define PROFILE_END(name)
#endif

// Usage
void process_data(void) {
    PROFILE_START(processing);
    LOG(INFO, "Starting data processing\n");

    // Do work

    LOG(INFO, "Processing complete\n");
    PROFILE_END(processing);
}
\end{lstlisting}

\section{Macro Definitions}

\subsection{Object-Like Macros}

\begin{lstlisting}
// Constants
#define MAX_SIZE 1024
#define PI 3.14159265359
#define VERSION "1.2.3"

// Use const instead when possible (type-safe)
static const int MAX_SIZE = 1024;
static const double PI = 3.14159265359;
static const char VERSION[] = "1.2.3";

// Multi-line macros with backslash
#define INIT_ARRAY \
    { \
        1, 2, 3, \
        4, 5, 6, \
        7, 8, 9  \
    }

int arr[] = INIT_ARRAY;
\end{lstlisting}

\subsection{Function-Like Macros}

\begin{lstlisting}
// Simple macro
#define SQUARE(x) ((x) * (x))

// Always parenthesize arguments!
#define BAD_SQUARE(x) x * x
int result = BAD_SQUARE(2 + 3);  // Expands to: 2 + 3 * 2 + 3 = 11
int result = SQUARE(2 + 3);       // Expands to: ((2 + 3) * (2 + 3)) = 25

// Parenthesize the whole expression too
#define BAD_DOUBLE(x) (x) + (x)
int result = 10 * BAD_DOUBLE(5);  // 10 * 5 + 5 = 55
#define GOOD_DOUBLE(x) ((x) + (x))
int result = 10 * GOOD_DOUBLE(5); // 10 * (5 + 5) = 100

// Multiple arguments
#define MAX(a, b) ((a) > (b) ? (a) : (b))
#define MIN(a, b) ((a) < (b) ? (a) : (b))
#define CLAMP(x, low, high) (MIN(MAX(x, low), high))
\end{lstlisting}

\begin{warningbox}
Function-like macros evaluate arguments multiple times! \texttt{MAX(x++, y++)} will increment variables multiple times, leading to bugs.
\end{warningbox}

\subsection{Do-While(0) Trick}

\begin{lstlisting}
// Problem: Multi-statement macro
#define LOG_ERROR(msg) \
    fprintf(stderr, "ERROR: %s\n", msg); \
    error_count++

// Breaks with if-statement:
if (failed)
    LOG_ERROR("Operation failed");  // Only fprintf is in if!
// error_count++ always executes!

// Solution: do-while(0)
#define LOG_ERROR(msg) \
    do { \
        fprintf(stderr, "ERROR: %s\n", msg); \
        error_count++; \
    } while(0)

// Now works correctly
if (failed)
    LOG_ERROR("Operation failed");  // Both statements in block

// Why while(0)? Requires semicolon at call site
LOG_ERROR("test");  // Must have semicolon, looks like function call
\end{lstlisting}

\subsection{Variadic Macros}

\begin{lstlisting}
// C99 variadic macros
#define DEBUG_PRINT(fmt, ...) \
    fprintf(stderr, "[%s:%d] " fmt, __FILE__, __LINE__, __VA_ARGS__)

DEBUG_PRINT("Value is %d\n", x);
// Expands to:
// fprintf(stderr, "[%s:%d] Value is %d\n", "main.c", 42, x);

// Problem: requires at least one argument
DEBUG_PRINT("Hello\n");  // ERROR: missing arguments

// Solution: GNU extension
#define DEBUG_PRINT(fmt, ...) \
    fprintf(stderr, "[%s:%d] " fmt, __FILE__, __LINE__, ##__VA_ARGS__)

DEBUG_PRINT("Hello\n");  // Works! ## removes comma if __VA_ARGS__ empty

// C++20 and later: __VA_OPT__
#define DEBUG_PRINT(fmt, ...) \
    fprintf(stderr, "[%s:%d] " fmt, \
        __FILE__, __LINE__ __VA_OPT__(,) __VA_ARGS__)
\end{lstlisting}

\section{Stringification and Token Pasting}

\subsection{Stringification (\#)}

\begin{lstlisting}
// Convert macro argument to string
#define STRINGIFY(x) #x
#define TO_STRING(x) STRINGIFY(x)

// Usage
printf("%s\n", STRINGIFY(hello));     // "hello"
printf("%s\n", STRINGIFY(123));       // "123"
printf("%s\n", STRINGIFY(a + b));     // "a + b"

// Indirect stringification (for macro expansion)
#define VERSION_MAJOR 1
#define VERSION_MINOR 2

printf("%s\n", STRINGIFY(VERSION_MAJOR));  // "VERSION_MAJOR" (not expanded!)
printf("%s\n", TO_STRING(VERSION_MAJOR));  // "1" (expanded!)

// Practical example: variable name debugging
#define DEBUG_VAR(var) \
    printf("%s = %d\n", #var, var)

int count = 42;
DEBUG_VAR(count);  // Prints: count = 42
\end{lstlisting}

\subsection{Token Pasting (\#\#)}

\begin{lstlisting}
// Concatenate tokens
#define CONCAT(a, b) a ## b

// Usage
int CONCAT(var, 123) = 0;  // Creates: int var123 = 0;

// Generate function names
#define DEFINE_GETTER(type, name) \
    type get_ ## name(void) { \
        return name; \
    }

int count;
DEFINE_GETTER(int, count)
// Expands to:
// int get_count(void) { return count; }

// Enum to string converter
#define ENUM_CASE(name) case name: return #name

const char* error_to_string(ErrorCode err) {
    switch(err) {
        ENUM_CASE(SUCCESS);
        ENUM_CASE(ERR_INVALID);
        ENUM_CASE(ERR_MEMORY);
        ENUM_CASE(ERR_IO);
        default: return "UNKNOWN";
    }
}
\end{lstlisting}

\subsection{Advanced Token Manipulation}

\begin{lstlisting}
// X-Macros: Define list once, use multiple times
#define ERROR_LIST \
    X(SUCCESS,     0, "Success") \
    X(ERR_INVALID, 1, "Invalid argument") \
    X(ERR_MEMORY,  2, "Out of memory") \
    X(ERR_IO,      3, "I/O error")

// Generate enum
typedef enum {
#define X(name, code, desc) name = code,
    ERROR_LIST
#undef X
} ErrorCode;

// Generate string table
static const char* error_strings[] = {
#define X(name, code, desc) [code] = desc,
    ERROR_LIST
#undef X
};

// Generate conversion function
const char* error_to_string(ErrorCode err) {
    if (err >= 0 && err < sizeof(error_strings)/sizeof(error_strings[0]))
        return error_strings[err];
    return "Unknown error";
}
\end{lstlisting}

\section{Predefined Macros}

\subsection{Standard Predefined Macros}

\begin{lstlisting}
// File and line information
printf("Error at %s:%d\n", __FILE__, __LINE__);

// Function name (C99)
void my_function(void) {
    printf("In function: %s\n", __func__);
}

// Date and time of compilation
printf("Compiled on %s at %s\n", __DATE__, __TIME__);

// C standard version
#if __STDC_VERSION__ >= 201112L
    printf("Using C11 or later\n");
#elif __STDC_VERSION__ >= 199901L
    printf("Using C99\n");
#else
    printf("Using C90 or earlier\n");
#endif

// Practical logging macro
#define LOG(level, fmt, ...) \
    do { \
        fprintf(stderr, "[%s] %s:%d:%s(): " fmt "\n", \
            level, __FILE__, __LINE__, __func__, ##__VA_ARGS__); \
    } while(0)

LOG("ERROR", "Failed to open file: %s", filename);
// Output: [ERROR] main.c:42:process_file(): Failed to open file: data.txt
\end{lstlisting}

\subsection{Compiler-Specific Macros}

\begin{lstlisting}
// Detect compiler
#if defined(__GNUC__)
    const char* compiler = "GCC";
    int version = __GNUC__ * 10000 + __GNUC_MINOR__ * 100 + __GNUC_PATCHLEVEL__;
#elif defined(__clang__)
    const char* compiler = "Clang";
    int version = __clang_major__ * 10000 + __clang_minor__ * 100;
#elif defined(_MSC_VER)
    const char* compiler = "MSVC";
    int version = _MSC_VER;
#else
    const char* compiler = "Unknown";
    int version = 0;
#endif

// Detect platform
#if defined(_WIN32) || defined(_WIN64)
    #define PLATFORM "Windows"
#elif defined(__linux__)
    #define PLATFORM "Linux"
#elif defined(__APPLE__) && defined(__MACH__)
    #define PLATFORM "macOS"
#elif defined(__unix__)
    #define PLATFORM "Unix"
#else
    #define PLATFORM "Unknown"
#endif

// Detect architecture
#if defined(__x86_64__) || defined(_M_X64)
    #define ARCH "x86_64"
#elif defined(__i386__) || defined(_M_IX86)
    #define ARCH "x86"
#elif defined(__aarch64__) || defined(_M_ARM64)
    #define ARCH "ARM64"
#elif defined(__arm__) || defined(_M_ARM)
    #define ARCH "ARM"
#else
    #define ARCH "Unknown"
#endif
\end{lstlisting}

\section{Include Directives}

\subsection{Include Paths}

\begin{lstlisting}
// System headers (search in system directories)
#include <stdio.h>
#include <stdlib.h>

// Local headers (search in current directory first)
#include "myheader.h"
#include "utils/helper.h"

// Absolute path (not recommended)
#include "/usr/local/include/mylib.h"

// Computed includes (rare, avoid)
#define HEADER_NAME "config.h"
#include HEADER_NAME
\end{lstlisting}

\subsection{Conditional Includes}

\begin{lstlisting}
// Include based on platform
#ifdef _WIN32
    #include <windows.h>
#else
    #include <unistd.h>
    #include <pthread.h>
#endif

// Include optional dependencies
#ifdef HAVE_OPENSSL
    #include <openssl/ssl.h>
#endif

// Version-specific includes
#if MYLIB_VERSION >= 20000
    #include "mylib/v2/api.h"
#else
    #include "mylib/v1/api.h"
#endif
\end{lstlisting}

\section{Advanced Preprocessor Techniques}

\subsection{Macro Overloading by Argument Count}

\begin{lstlisting}
// Count arguments (up to 5 for this example)
#define GET_MACRO(_1, _2, _3, _4, _5, NAME, ...) NAME

// Define overloaded versions
#define PRINT_1(a)           printf("%d\n", a)
#define PRINT_2(a, b)        printf("%d %d\n", a, b)
#define PRINT_3(a, b, c)     printf("%d %d %d\n", a, b, c)
#define PRINT_4(a, b, c, d)  printf("%d %d %d %d\n", a, b, c, d)
#define PRINT_5(a, b, c, d, e) printf("%d %d %d %d %d\n", a, b, c, d, e)

// Dispatch to correct version
#define PRINT(...) \
    GET_MACRO(__VA_ARGS__, PRINT_5, PRINT_4, PRINT_3, PRINT_2, PRINT_1)(__VA_ARGS__)

// Usage
PRINT(1);           // Calls PRINT_1
PRINT(1, 2);        // Calls PRINT_2
PRINT(1, 2, 3);     // Calls PRINT_3
\end{lstlisting}

\subsection{Compile-Time Assertions}

\begin{lstlisting}
// C11 static_assert
_Static_assert(sizeof(int) == 4, "int must be 4 bytes");

// Pre-C11 compile-time assert
#define STATIC_ASSERT(cond, msg) \
    typedef char static_assertion_##msg[(cond) ? 1 : -1]

STATIC_ASSERT(sizeof(int) == 4, int_size_check);

// Negative array size causes compile error if condition false

// Check at compile time
STATIC_ASSERT(MAX_BUFFER >= 1024, buffer_too_small);
STATIC_ASSERT(sizeof(MyStruct) == 64, wrong_struct_size);
\end{lstlisting}

\subsection{Defer Macro Expansion}

\begin{lstlisting}
// Sometimes you need to control when macros expand
#define EMPTY()
#define DEFER(id) id EMPTY()

#define A() 123
DEFER(A)()  // Defers expansion of A

// Recursive macro (limited depth)
#define REPEAT_0(m, x)
#define REPEAT_1(m, x) m(x)
#define REPEAT_2(m, x) m(x) REPEAT_1(m, x)
#define REPEAT_3(m, x) m(x) REPEAT_2(m, x)
#define REPEAT_4(m, x) m(x) REPEAT_3(m, x)

#define INC(x) x++

REPEAT_4(INC, counter);
// Expands to: counter++ counter++ counter++ counter++
\end{lstlisting}

\subsection{Type-Generic Macros}

\begin{lstlisting}
// C11 _Generic selection
#define print_any(x) _Generic((x), \
    int: printf("%d", x), \
    long: printf("%ld", x), \
    float: printf("%f", (double)x), \
    double: printf("%f", x), \
    char*: printf("%s", x), \
    default: printf("%p", (void*)&x))

// Usage
print_any(42);           // Prints int
print_any(3.14);         // Prints double
print_any("hello");      // Prints string

// Type-generic absolute value
#define abs_generic(x) _Generic((x), \
    int: abs(x), \
    long: labs(x), \
    long long: llabs(x), \
    float: fabsf(x), \
    double: fabs(x), \
    long double: fabsl(x))
\end{lstlisting}

\section{Debugging Macros}

\subsection{Macro Expansion Debugging}

\begin{lstlisting}
// Show what preprocessor does
// Compile with: gcc -E source.c

// Add debug prints in macros
#define DEBUG_MACRO(x) \
    do { \
        printf("Macro called with: %s\n", #x); \
        printf("Value: %d\n", x); \
    } while(0)

// Trace macro expansion
#define TRACE_EXPAND(x) TRACE_EXPAND_IMPL(x)
#define TRACE_EXPAND_IMPL(x) #x

#define VALUE 42
printf("VALUE expands to: %s\n", TRACE_EXPAND(VALUE));
// Prints: VALUE expands to: 42
\end{lstlisting}

\subsection{Assertion Macros}

\begin{lstlisting}
// Enhanced assert with message
#define ASSERT_MSG(cond, msg) \
    do { \
        if (!(cond)) { \
            fprintf(stderr, "Assertion failed: %s\n", #cond); \
            fprintf(stderr, "Message: %s\n", msg); \
            fprintf(stderr, "File: %s, Line: %d\n", __FILE__, __LINE__); \
            abort(); \
        } \
    } while(0)

// Runtime verification (always enabled)
#define VERIFY(cond) \
    do { \
        if (!(cond)) { \
            fprintf(stderr, "Verification failed: %s at %s:%d\n", \
                #cond, __FILE__, __LINE__); \
            abort(); \
        } \
    } while(0)

// Check preconditions
#define REQUIRE(cond) VERIFY(cond)
// Check postconditions
#define ENSURE(cond) VERIFY(cond)

void process(int* data, size_t size) {
    REQUIRE(data != NULL);
    REQUIRE(size > 0);

    // Process data

    ENSURE(result >= 0);
}
\end{lstlisting}

\section{Macro Pitfalls and Solutions}

\subsection{Common Problems}

\begin{lstlisting}
// Problem 1: Double evaluation
#define MAX(a, b) ((a) > (b) ? (a) : (b))
int x = 5;
int result = MAX(x++, 10);  // x incremented twice!

// Solution: Use inline functions (C99)
static inline int max_int(int a, int b) {
    return (a > b) ? a : b;
}

// Problem 2: Semicolon swallowing
#define SWAP(a, b) { int tmp = a; a = b; b = tmp; }
if (x > y)
    SWAP(x, y);  // Extra semicolon breaks else
else
    printf("ok\n");

// Solution: do-while(0)
#define SWAP(a, b) \
    do { int tmp = a; a = b; b = tmp; } while(0)

// Problem 3: Macro shadowing
#define BEGIN {
#define END }
// Looks nice but breaks code that uses begin/end variables

// Problem 4: Operator precedence
#define DOUBLE(x) x + x
int result = 10 * DOUBLE(5);  // 10 * 5 + 5 = 55, not 100!

// Solution: Always parenthesize
#define DOUBLE(x) ((x) + (x))
\end{lstlisting}

\subsection{When NOT to Use Macros}

\begin{lstlisting}
// BAD: Complex logic in macros
#define PROCESS_DATA(data, size) \
    do { \
        for (int i = 0; i < size; i++) { \
            if (data[i] < 0) data[i] = 0; \
            data[i] *= 2; \
        } \
    } while(0)

// GOOD: Use a function instead
static inline void process_data(int* data, size_t size) {
    for (size_t i = 0; i < size; i++) {
        if (data[i] < 0) data[i] = 0;
        data[i] *= 2;
    }
}

// BAD: Type-unsafe operations
#define SWAP(a, b) \
    do { typeof(a) tmp = a; a = b; b = tmp; } while(0)
// typeof is GNU extension, not standard

// GOOD: Type-safe generic (C11)
#define swap(a, b) \
    do { \
        _Generic((a), \
            int: swap_int, \
            double: swap_double)(&(a), &(b)); \
    } while(0)
\end{lstlisting}

\section{Preprocessor Best Practices}

\begin{lstlisting}
// 1. Use UPPERCASE for macros
#define MAX_SIZE 1024  // Clear it's a macro
static const int max_size = 1024;  // Clear it's not

// 2. Prefix macros with project name
#define MYLIB_MAX(a, b) ((a) > (b) ? (a) : (b))
// Avoids conflicts with other libraries

// 3. Parenthesize everything
#define BAD(x) x * 2
#define GOOD(x) ((x) * 2)

// 4. Document complex macros
/**
 * FOR_EACH - Iterate over array elements
 * @type: Element type
 * @var: Loop variable name
 * @array: Array to iterate
 * @count: Number of elements
 *
 * Usage:
 *   FOR_EACH(int, x, array, 10) {
 *       printf("%d\n", x);
 *   }
 */
#define FOR_EACH(type, var, array, count) \
    for (type var, *_arr_ = (array), \
         *_end_ = _arr_ + (count); \
         _arr_ < _end_ && (var = *_arr_, 1); \
         _arr_++)

// 5. Undefine temporary macros
#define X(a, b) a + b
// Use X
#undef X  // Clean up namespace
\end{lstlisting}

\section{Summary}

The preprocessor is a powerful text manipulation tool:

\begin{itemize}
    \item Use conditional compilation for platform/feature handling
    \item Always parenthesize macro arguments and expressions
    \item Use do-while(0) for multi-statement macros
    \item Beware of double evaluation in function-like macros
    \item Prefer inline functions for type safety when possible
    \item Use \texttt{\#} for stringification, \texttt{\#\#} for token pasting
    \item X-macros reduce code duplication elegantly
    \item Use predefined macros for debugging and logging
    \item Document complex macros thoroughly
    \item Know when NOT to use macros
\end{itemize}

Master the preprocessor, but use it judiciously. Modern C features like inline functions and \texttt{\_Generic} often provide safer alternatives!

\chapter{Initialization Patterns}

\section{Understanding Initialization}

Initialization is more than just assigning values—it's about setting up data structures in a predictable, safe state. C offers several initialization techniques, each with its own strengths and use cases.

\begin{lstlisting}
// Different initialization styles
int a = 0;                    // Simple initialization
int b = {0};                  // Brace initialization
int arr[5] = {1, 2, 3, 4, 5}; // Array initialization
struct Point p = {10, 20};    // Struct initialization

// Uninitialized (dangerous!)
int x;        // Contains garbage
int* ptr;     // Points to random memory
\end{lstlisting}

\begin{warningbox}
Uninitialized variables contain garbage values. Always initialize your variables, especially pointers. Reading uninitialized data is undefined behavior.
\end{warningbox}

\section{Zero Initialization}

The safest default—initialize everything to zero.

\begin{lstlisting}
// Zero-initialize everything
int x = 0;
int arr[100] = {0};           // All elements zero
struct Data d = {0};          // All members zero
char str[256] = {0};          // Null-terminated empty string

// Shorthand: empty braces (C23 and some compilers)
int arr2[100] = {};
struct Data d2 = {};

// Static/global variables are automatically zero-initialized
static int counter;           // Initialized to 0
static char buffer[1024];     // All bytes 0

// Local variables are NOT zero-initialized
void func(void) {
    int x;                    // GARBAGE!
    int y = 0;                // Explicitly zero
}

// Zero-initialize dynamically allocated memory
int* ptr = calloc(100, sizeof(int));  // All zeros
// vs
int* ptr2 = malloc(100 * sizeof(int)); // GARBAGE!
\end{lstlisting}

\begin{tipbox}
Use \texttt{= \{0\}} to zero-initialize any struct or array. It's simple, portable, and works everywhere.
\end{tipbox}

\section{Designated Initializers (C99)}

Initialize specific members by name—much clearer than positional initialization.

\subsection{Struct Designated Initializers}

\begin{lstlisting}
typedef struct {
    int x;
    int y;
    int z;
    char* name;
} Point3D;

// Positional initialization (old style)
Point3D p1 = {10, 20, 30, "origin"};

// Designated initializers (C99)
Point3D p2 = {
    .x = 10,
    .y = 20,
    .z = 30,
    .name = "origin"
};

// Initialize only some members (others zero)
Point3D p3 = {
    .x = 100,
    .name = "partial"
    // y and z are 0
};

// Order doesn't matter
Point3D p4 = {
    .name = "reordered",
    .z = 5,
    .x = 15
    // y is 0
};

// Mix styles (not recommended)
Point3D p5 = {
    .x = 1,
    2,          // Sets y = 2 (next in sequence)
    .z = 3
};
\end{lstlisting}

\subsection{Array Designated Initializers}

\begin{lstlisting}
// Initialize specific array elements
int sparse[100] = {
    [0] = 1,
    [10] = 2,
    [50] = 3,
    [99] = 4
    // All other elements are 0
};

// Character arrays
char vowels[26] = {
    ['a' - 'a'] = 'a',
    ['e' - 'a'] = 'e',
    ['i' - 'a'] = 'i',
    ['o' - 'a'] = 'o',
    ['u' - 'a'] = 'u'
};

// Ranges (GNU extension)
int range[100] = {
    [0 ... 9] = 1,      // First 10 elements
    [10 ... 19] = 2,    // Next 10 elements
    [90 ... 99] = 9     // Last 10 elements
};

// Lookup tables
int days_in_month[12] = {
    [0] = 31,  // January
    [1] = 28,  // February
    [2] = 31,  // March
    [3] = 30,  // April
    [4] = 31,  // May
    [5] = 30,  // June
    [6] = 31,  // July
    [7] = 31,  // August
    [8] = 30,  // September
    [9] = 31,  // October
    [10] = 30, // November
    [11] = 31  // December
};
\end{lstlisting}

\subsection{Nested Designated Initializers}

\begin{lstlisting}
typedef struct {
    int x, y;
} Point;

typedef struct {
    Point top_left;
    Point bottom_right;
    char* label;
} Rectangle;

// Initialize nested structures
Rectangle rect = {
    .top_left = {.x = 0, .y = 100},
    .bottom_right = {.x = 100, .y = 0},
    .label = "main window"
};

// Array of structs
Point points[3] = {
    [0] = {.x = 0, .y = 0},
    [1] = {.x = 10, .y = 10},
    [2] = {.x = 20, .y = 20}
};

// Struct containing array
typedef struct {
    char name[32];
    int scores[5];
} Student;

Student student = {
    .name = "Alice",
    .scores = {[0] = 95, [1] = 87, [4] = 92}
    // scores[2] and scores[3] are 0
};
\end{lstlisting}

\section{Compound Literals (C99)}

Create temporary objects without declaring variables.

\begin{lstlisting}
// Traditional: need temporary variable
Point temp = {10, 20};
draw_point(&temp);

// Compound literal: create temporary inline
draw_point(&(Point){10, 20});

// Array compound literals
process_data((int[]){1, 2, 3, 4, 5}, 5);

// String compound literal
print_string((char[]){"Hello, World!"});

// With designated initializers
configure(&(Config){
    .width = 800,
    .height = 600,
    .fullscreen = true
});

// Lifetime: until end of enclosing block
void example(void) {
    Point* p = &(Point){100, 200};  // Valid until end of function
    // Use p...
} // Compound literal destroyed here
\end{lstlisting}

\subsection{Compound Literal Patterns}

\begin{lstlisting}
// Default parameters pattern
typedef struct {
    int timeout;
    int retries;
    bool verbose;
} Options;

void connect(const char* host, const Options* opts) {
    // Use opts->timeout, etc.
}

// Call with default options
connect("localhost", &(Options){
    .timeout = 5000,
    .retries = 3,
    .verbose = false
});

// Factory function pattern
Point* create_origin(void) {
    static Point origin = {0, 0};  // Don't do this with compound literal!
    return &origin;
}

// Better: return by value or allocate
Point get_origin(void) {
    return (Point){0, 0};
}

// Initializer lists for variadic functions
void log_values(int count, ...) {
    va_list args;
    va_start(args, count);
    // Process values
    va_end(args);
}

// Use compound literal with array
int values[] = {1, 2, 3, 4, 5};
log_values(5, values[0], values[1], values[2], values[3], values[4]);

// Or pass array directly
void log_array(const int* arr, size_t count);
log_array((int[]){1, 2, 3, 4, 5}, 5);
\end{lstlisting}

\section{Static Initialization}

Initialize data at compile time—faster and safer.

\subsection{Static vs Dynamic Initialization}

\begin{lstlisting}
// Static initialization (compile time)
static const int sizes[] = {1, 2, 4, 8, 16, 32};
static const char* names[] = {"Alice", "Bob", "Charlie"};

// These are embedded in the executable, no runtime cost

// Dynamic initialization (runtime)
void init(void) {
    int* arr = malloc(6 * sizeof(int));
    arr[0] = 1;
    arr[1] = 2;
    // ... Runtime overhead
    free(arr);
}

// Static initialization wins:
// - Faster (no runtime work)
// - Safer (can't fail)
// - Simpler (no cleanup needed)
\end{lstlisting}

\subsection{Constant Tables}

\begin{lstlisting}
// Lookup table for powers of 2
static const unsigned int pow2[] = {
    1, 2, 4, 8, 16, 32, 64, 128, 256, 512, 1024
};

// Error message table
static const char* const error_messages[] = {
    [0] = "Success",
    [1] = "Invalid argument",
    [2] = "Out of memory",
    [3] = "File not found",
    [4] = "Permission denied"
};

const char* get_error_message(int code) {
    if (code >= 0 && code < sizeof(error_messages)/sizeof(error_messages[0]))
        return error_messages[code];
    return "Unknown error";
}

// State machine transition table
typedef enum { IDLE, RUNNING, PAUSED, STOPPED } State;
typedef enum { START, PAUSE, RESUME, STOP } Event;

static const State transitions[4][4] = {
    //        START     PAUSE      RESUME    STOP
    [IDLE]    = {RUNNING,  IDLE,      IDLE,     IDLE},
    [RUNNING] = {RUNNING,  PAUSED,    RUNNING,  STOPPED},
    [PAUSED]  = {PAUSED,   PAUSED,    RUNNING,  STOPPED},
    [STOPPED] = {STOPPED,  STOPPED,   STOPPED,  STOPPED}
};

State next_state(State current, Event event) {
    return transitions[current][event];
}
\end{lstlisting}

\subsection{Read-Only Data}

\begin{lstlisting}
// const ensures data can't be modified
static const int PRIMES[] = {2, 3, 5, 7, 11, 13, 17, 19, 23, 29};

// Pointer to const data
static const char* const DAYS[] = {
    "Sunday", "Monday", "Tuesday", "Wednesday",
    "Thursday", "Friday", "Saturday"
};

// Both pointer and data are const:
// - Can't modify strings
// - Can't make pointer point elsewhere

// Configuration constants
typedef struct {
    int width;
    int height;
    int bpp;
} VideoMode;

static const VideoMode VIDEO_MODES[] = {
    {.width = 640,  .height = 480,  .bpp = 32},
    {.width = 800,  .height = 600,  .bpp = 32},
    {.width = 1024, .height = 768,  .bpp = 32},
    {.width = 1920, .height = 1080, .bpp = 32}
};

static const size_t NUM_VIDEO_MODES =
    sizeof(VIDEO_MODES) / sizeof(VIDEO_MODES[0]);
\end{lstlisting}

\section{Flexible Array Members (C99)}

Structs with variable-length trailing arrays.

\begin{lstlisting}
// Flexible array member (must be last in struct)
typedef struct {
    size_t length;
    int data[];  // Flexible array (size determined at allocation)
} IntArray;

// Allocate with specific size
IntArray* create_array(size_t n) {
    IntArray* arr = malloc(sizeof(IntArray) + n * sizeof(int));
    if (arr) {
        arr->length = n;
        // Initialize data
        for (size_t i = 0; i < n; i++) {
            arr->data[i] = 0;
        }
    }
    return arr;
}

// Use like normal array
void use_array(IntArray* arr) {
    for (size_t i = 0; i < arr->length; i++) {
        printf("%d ", arr->data[i]);
    }
}

// String with flexible array
typedef struct {
    size_t length;
    char data[];
} String;

String* string_create(const char* str) {
    size_t len = strlen(str);
    String* s = malloc(sizeof(String) + len + 1);
    if (s) {
        s->length = len;
        memcpy(s->data, str, len + 1);
    }
    return s;
}
\end{lstlisting}

\begin{tipbox}
Flexible array members are perfect for variable-sized data structures where the size is known at allocation time and won't change.
\end{tipbox}

\subsection{Flexible Array Member Patterns}

\begin{lstlisting}
// Message with variable payload
typedef struct {
    int type;
    size_t payload_size;
    unsigned char payload[];
} Message;

Message* create_message(int type, const void* data, size_t size) {
    Message* msg = malloc(sizeof(Message) + size);
    if (msg) {
        msg->type = type;
        msg->payload_size = size;
        memcpy(msg->payload, data, size);
    }
    return msg;
}

// Vector implementation
typedef struct {
    size_t size;
    size_t capacity;
    int elements[];
} Vector;

Vector* vector_create(size_t capacity) {
    Vector* v = malloc(sizeof(Vector) + capacity * sizeof(int));
    if (v) {
        v->size = 0;
        v->capacity = capacity;
    }
    return v;
}

Vector* vector_push(Vector* v, int value) {
    if (v->size >= v->capacity) {
        // Reallocate with larger capacity
        size_t new_cap = v->capacity * 2;
        Vector* new_v = realloc(v, sizeof(Vector) + new_cap * sizeof(int));
        if (!new_v) return NULL;
        new_v->capacity = new_cap;
        v = new_v;
    }
    v->elements[v->size++] = value;
    return v;
}
\end{lstlisting}

\section{Initialization Functions}

When initialization is complex, use dedicated functions.

\begin{lstlisting}
// Simple initializer
typedef struct {
    int* data;
    size_t size;
    size_t capacity;
} Buffer;

void buffer_init(Buffer* buf, size_t initial_capacity) {
    buf->data = malloc(initial_capacity * sizeof(int));
    buf->size = 0;
    buf->capacity = initial_capacity;
}

void buffer_destroy(Buffer* buf) {
    free(buf->data);
    buf->data = NULL;
    buf->size = 0;
    buf->capacity = 0;
}

// Usage
Buffer buf;
buffer_init(&buf, 100);
// Use buffer...
buffer_destroy(&buf);

// Factory function (returns new object)
Buffer* buffer_create(size_t initial_capacity) {
    Buffer* buf = malloc(sizeof(Buffer));
    if (buf) {
        buffer_init(buf, initial_capacity);
    }
    return buf;
}

void buffer_free(Buffer* buf) {
    if (buf) {
        buffer_destroy(buf);
        free(buf);
    }
}

// Usage
Buffer* buf = buffer_create(100);
// Use buffer...
buffer_free(buf);
\end{lstlisting}

\subsection{Constructor/Destructor Pattern}

\begin{lstlisting}
// Database connection object
typedef struct Database Database;

// Constructor with error handling
Database* database_connect(const char* host, int port, const char* user,
                          const char* password) {
    Database* db = calloc(1, sizeof(Database));
    if (!db) return NULL;

    // Initialize members
    db->host = strdup(host);
    db->port = port;
    db->connected = false;

    // Connect
    if (!internal_connect(db, user, password)) {
        database_close(db);
        return NULL;
    }

    db->connected = true;
    return db;
}

// Destructor (cleanup)
void database_close(Database* db) {
    if (!db) return;

    if (db->connected) {
        internal_disconnect(db);
    }

    free(db->host);
    free(db->query_buffer);
    // Clean up all resources

    memset(db, 0, sizeof(*db));  // Zero for safety
    free(db);
}

// Usage with cleanup
void example(void) {
    Database* db = database_connect("localhost", 5432, "user", "pass");
    if (!db) {
        fprintf(stderr, "Connection failed\n");
        return;
    }

    // Use database...

    database_close(db);  // Always cleanup
}
\end{lstlisting}

\section{Copy Initialization}

\begin{lstlisting}
// Shallow copy (default behavior)
typedef struct {
    int x, y;
} Point;

Point p1 = {10, 20};
Point p2 = p1;  // Bitwise copy (shallow)

// Deep copy needed for pointers
typedef struct {
    char* name;
    int* data;
    size_t size;
} Object;

// Shallow copy (DANGEROUS!)
Object obj1 = {...};
Object obj2 = obj1;  // Both point to same memory!
free(obj1.data);     // obj2.data now invalid!

// Deep copy function
Object* object_copy(const Object* src) {
    Object* dst = malloc(sizeof(Object));
    if (!dst) return NULL;

    // Copy name string
    dst->name = strdup(src->name);
    if (!dst->name) {
        free(dst);
        return NULL;
    }

    // Copy data array
    dst->size = src->size;
    dst->data = malloc(dst->size * sizeof(int));
    if (!dst->data) {
        free(dst->name);
        free(dst);
        return NULL;
    }
    memcpy(dst->data, src->data, dst->size * sizeof(int));

    return dst;
}

// Copy assignment
void object_assign(Object* dst, const Object* src) {
    if (dst == src) return;  // Self-assignment check

    // Free old data
    free(dst->name);
    free(dst->data);

    // Copy new data
    dst->name = strdup(src->name);
    dst->size = src->size;
    dst->data = malloc(dst->size * sizeof(int));
    memcpy(dst->data, src->data, dst->size * sizeof(int));
}
\end{lstlisting}

\section{Global Initialization}

\begin{lstlisting}
// Global variables are zero-initialized before main()
int global_counter = 0;  // Explicit (redundant)
static char buffer[1024]; // Implicitly zero

// Complex global initialization
typedef struct {
    bool initialized;
    void* handle;
    char* config_path;
} GlobalState;

static GlobalState g_state = {0};

// One-time initialization
void ensure_initialized(void) {
    if (!g_state.initialized) {
        g_state.handle = open_handle();
        g_state.config_path = get_config_path();
        g_state.initialized = true;
        atexit(cleanup);  // Register cleanup
    }
}

void cleanup(void) {
    if (g_state.initialized) {
        close_handle(g_state.handle);
        free(g_state.config_path);
        g_state.initialized = false;
    }
}

// Thread-safe initialization (C11)
#include <threads.h>

static once_flag init_flag = ONCE_FLAG_INIT;

void do_init(void) {
    // Expensive initialization
    g_state.handle = open_handle();
    g_state.initialized = true;
}

void thread_safe_init(void) {
    call_once(&init_flag, do_init);
    // Guaranteed to run exactly once across all threads
}
\end{lstlisting}

\section{Initialization Best Practices}

\subsection{Always Initialize}

\begin{lstlisting}
// BAD: Uninitialized
void bad_example(void) {
    int x;
    int* ptr;
    char buffer[256];

    // Using these is undefined behavior!
}

// GOOD: Always initialize
void good_example(void) {
    int x = 0;
    int* ptr = NULL;
    char buffer[256] = {0};

    // Safe to use
}

// Use initializers even for complex types
typedef struct {
    int count;
    char* name;
    double value;
} Data;

// BAD
Data data;
data.count = 0;
data.name = NULL;
data.value = 0.0;

// GOOD
Data data = {0};  // Zero-initialize everything
\end{lstlisting}

\subsection{Use Designated Initializers}

\begin{lstlisting}
// BAD: Positional (fragile)
Point3D p = {10, 20, 30, "label"};
// If struct changes order, this breaks!

// GOOD: Designated (robust)
Point3D p = {
    .x = 10,
    .y = 20,
    .z = 30,
    .name = "label"
};
// Still works if struct reordered

// GREAT: Default values
Point3D p = {
    .x = 10,
    .name = "partial"
    // y and z automatically 0
};
\end{lstlisting}

\subsection{Const Correctness}

\begin{lstlisting}
// Mark read-only data as const
static const int BUFFER_SIZE = 1024;
static const char* const ERROR_MSG = "Error occurred";

// Function taking const pointer (won't modify)
void process(const Data* data) {
    // Can read, can't modify
}

// Const array
static const int LOOKUP[] = {1, 2, 3, 4, 5};

// Const protects from accidents
void example(void) {
    BUFFER_SIZE = 2048;  // Compile error!
    LOOKUP[0] = 10;      // Compile error!
}
\end{lstlisting}

\section{Summary}

Initialization patterns in C:

\begin{itemize}
    \item Always initialize variables—uninitialized data is undefined behavior
    \item Use \texttt{= \{0\}} for zero-initialization of any type
    \item Designated initializers (C99) make code clearer and more maintainable
    \item Compound literals create temporary objects inline
    \item Static initialization is faster and safer than dynamic
    \item Flexible array members handle variable-sized data elegantly
    \item Use initialization functions for complex setup
    \item Deep copy when dealing with pointers
    \item Mark const data as \texttt{const} for safety
    \item Prefer static/compile-time initialization when possible
\end{itemize}

Proper initialization prevents bugs and makes code more robust!

\chapter{State Machine Patterns}

\section{Why State Machines?}

State machines are one of the most practical patterns in C. They help manage complex behavior by breaking it down into discrete states and transitions. Think of a vending machine, network connection, or game character---all are state machines.

A \textbf{state machine} (also called a finite state machine or FSM) is a mathematical model of computation that can be in exactly one state at any given time. The machine changes from one state to another in response to external inputs called \textbf{events} or \textbf{transitions}.

Why use state machines?

\begin{itemize}
    \item \textbf{Clarity}: Complex logic becomes easy to understand
    \item \textbf{Maintainability}: Adding new states or transitions is straightforward
    \item \textbf{Testability}: Each state can be tested independently
    \item \textbf{Bug Prevention}: Invalid state transitions are impossible
    \item \textbf{Documentation}: The state diagram IS the documentation
\end{itemize}

Consider a door: it can be \textit{open}, \textit{closed}, or \textit{locked}. You can't lock an open door, and you can't open a locked door. State machines enforce these rules naturally.

\begin{lstlisting}
// Without state machine - messy conditionals
void handle_input(char c) {
    if (connected) {
        if (authenticated) {
            if (in_transaction) {
                // Handle transaction input
            } else {
                // Handle command input
            }
        } else {
            // Handle authentication input
        }
    } else {
        // Handle connection input
    }
}

// With state machine - clear and organized
void handle_input(char c) {
    switch (current_state) {
        case STATE_CONNECTING:
            handle_connecting(c);
            break;
        case STATE_AUTHENTICATING:
            handle_authenticating(c);
            break;
        case STATE_READY:
            handle_ready(c);
            break;
        case STATE_IN_TRANSACTION:
            handle_transaction(c);
            break;
    }
}
\end{lstlisting}

\section{Enum-Based State Machines}

The simplest and most common pattern---using enums to represent states and a simple variable to track the current state.

This is the most straightforward implementation and should be your default choice for simple state machines. The state is just an enum value, and state transitions are simple assignments. This approach is fast, type-safe, and easy to understand.

\begin{lstlisting}
typedef enum {
    STATE_IDLE,
    STATE_CONNECTING,
    STATE_CONNECTED,
    STATE_DISCONNECTING,
    STATE_ERROR
} ConnectionState;

typedef struct {
    ConnectionState state;
    int socket;
    char error_msg[256];
    time_t state_entered;
} Connection;

// State transition function
void connection_set_state(Connection* conn, ConnectionState new_state) {
    printf("Transition: %d -> %d\n", conn->state, new_state);
    conn->state = new_state;
    conn->state_entered = time(NULL);
}

// State-based behavior
int connection_send(Connection* conn, const char* data) {
    if (conn->state != STATE_CONNECTED) {
        return -1;  // Invalid state
    }
    return send(conn->socket, data, strlen(data), 0);
}
\end{lstlisting}

\begin{tipbox}
Always validate state before performing actions. This prevents bugs where operations are attempted in invalid states.
\end{tipbox}

\section{Switch-Based State Machine}

The classic approach using switch statements to handle different behavior for each state.

This pattern is extremely common in parsers, protocol handlers, and character-by-character processing. Each \texttt{case} in the switch represents a state, and within each case, you process inputs and potentially transition to other states.

The key advantage is that all state-specific logic is grouped together, making it easy to see what happens in each state. This is particularly useful for \textbf{event-driven} systems where you need to react differently to the same input depending on the current state.

\begin{lstlisting}
typedef enum {
    PARSE_START,
    PARSE_TAG_OPEN,
    PARSE_TAG_NAME,
    PARSE_TAG_CLOSE,
    PARSE_TEXT,
    PARSE_ERROR
} ParseState;

typedef struct {
    ParseState state;
    char buffer[1024];
    size_t buffer_pos;
} Parser;

void parser_init(Parser* p) {
    p->state = PARSE_START;
    p->buffer_pos = 0;
}

void parser_process_char(Parser* p, char c) {
    switch (p->state) {
        case PARSE_START:
            if (c == '<') {
                p->state = PARSE_TAG_OPEN;
            } else {
                p->buffer[p->buffer_pos++] = c;
                p->state = PARSE_TEXT;
            }
            break;

        case PARSE_TAG_OPEN:
            if (c == '/') {
                p->state = PARSE_TAG_CLOSE;
            } else if (isalpha(c)) {
                p->buffer[0] = c;
                p->buffer_pos = 1;
                p->state = PARSE_TAG_NAME;
            } else {
                p->state = PARSE_ERROR;
            }
            break;

        case PARSE_TAG_NAME:
            if (c == '>') {
                p->buffer[p->buffer_pos] = '\0';
                printf("Found tag: %s\n", p->buffer);
                p->buffer_pos = 0;
                p->state = PARSE_START;
            } else if (isalnum(c)) {
                p->buffer[p->buffer_pos++] = c;
            } else {
                p->state = PARSE_ERROR;
            }
            break;

        case PARSE_TAG_CLOSE:
            if (c == '>') {
                p->state = PARSE_START;
            } else if (!isalnum(c)) {
                p->state = PARSE_ERROR;
            }
            break;

        case PARSE_TEXT:
            if (c == '<') {
                p->buffer[p->buffer_pos] = '\0';
                printf("Text: %s\n", p->buffer);
                p->buffer_pos = 0;
                p->state = PARSE_TAG_OPEN;
            } else {
                p->buffer[p->buffer_pos++] = c;
            }
            break;

        case PARSE_ERROR:
            // Stay in error state
            break;
    }
}
\end{lstlisting}

\section{Function Pointer State Machine}

More flexible---each state is a function rather than a case in a switch statement.

This pattern provides several advantages over switch-based machines:

\begin{itemize}
    \item \textbf{Encapsulation}: Each state's logic is in its own function
    \item \textbf{Extensibility}: Adding states doesn't require modifying a central switch
    \item \textbf{Runtime Flexibility}: States can be changed or added at runtime
    \item \textbf{Polymorphism}: Different objects can have different state functions
\end{itemize}

The trade-off is slightly more complexity and an indirect function call overhead (usually negligible). This approach shines when you have many states or when state behavior needs to be determined dynamically.

\begin{lstlisting}
// Forward declaration
typedef struct StateMachine StateMachine;

// State function type
typedef void (*StateFunc)(StateMachine* sm, int event);

struct StateMachine {
    StateFunc current_state;
    void* context;  // User data
};

// State functions
void state_idle(StateMachine* sm, int event) {
    printf("Idle state, event: %d\n", event);

    if (event == EVENT_START) {
        sm->current_state = state_running;
    }
}

void state_running(StateMachine* sm, int event) {
    printf("Running state, event: %d\n", event);

    if (event == EVENT_STOP) {
        sm->current_state = state_idle;
    } else if (event == EVENT_PAUSE) {
        sm->current_state = state_paused;
    }
}

void state_paused(StateMachine* sm, int event) {
    printf("Paused state, event: %d\n", event);

    if (event == EVENT_RESUME) {
        sm->current_state = state_running;
    } else if (event == EVENT_STOP) {
        sm->current_state = state_idle;
    }
}

// Process event
void sm_handle_event(StateMachine* sm, int event) {
    if (sm->current_state) {
        sm->current_state(sm, event);
    }
}

// Usage
StateMachine sm = {
    .current_state = state_idle,
    .context = NULL
};

sm_handle_event(&sm, EVENT_START);
sm_handle_event(&sm, EVENT_PAUSE);
sm_handle_event(&sm, EVENT_RESUME);
\end{lstlisting}

\begin{notebox}
Function pointer state machines are more flexible than switch-based ones. They allow runtime state addition and are easier to extend.
\end{notebox}

\section{Hierarchical State Machines}

States within states for complex behavior---also known as \textbf{nested states} or \textbf{substates}.

In real systems, states often have substates. For example, a character might be "alive" with substates "idle", "moving", or "attacking". When the character dies, all these substates become irrelevant.

Hierarchical state machines allow you to:
\begin{itemize}
    \item Share behavior across related states
    \item Reduce code duplication
    \item Model complex systems more naturally
    \item Handle events at the appropriate level
\end{itemize}

Think of it as inheritance for states: substates inherit the behavior of their parent state, but can override specific behaviors.

\begin{lstlisting}
typedef enum {
    STATE_ALIVE,
    STATE_ALIVE_IDLE,
    STATE_ALIVE_MOVING,
    STATE_ALIVE_ATTACKING,
    STATE_DEAD
} CharacterState;

typedef struct {
    CharacterState state;
    CharacterState parent_state;
} Character;

// Check if in a parent state
int character_is_alive(Character* c) {
    return c->state == STATE_ALIVE ||
           c->state == STATE_ALIVE_IDLE ||
           c->state == STATE_ALIVE_MOVING ||
           c->state == STATE_ALIVE_ATTACKING;
}

void character_take_damage(Character* c, int damage) {
    if (character_is_alive(c)) {
        // All "alive" substates can take damage
        c->health -= damage;
        if (c->health <= 0) {
            c->state = STATE_DEAD;
        }
    }
}
\end{lstlisting}

\section{State Machine with Entry/Exit Actions}

Execute code when entering or leaving states---a critical pattern for resource management and initialization.

Many state machines need to perform actions when transitioning between states:
\begin{itemize}
    \item \textbf{Entry actions}: Run when entering a state (initialization, resource allocation)
    \item \textbf{Exit actions}: Run when leaving a state (cleanup, resource deallocation)
    \item \textbf{Transition actions}: Run during the transition itself
\end{itemize}

This pattern is essential for:
\begin{itemize}
    \item Starting/stopping timers
    \item Acquiring/releasing locks
    \item Opening/closing files or connections
    \item Playing sounds or animations
    \item Logging state changes
\end{itemize}

Without entry/exit actions, you'd have to remember to run setup/cleanup code every time you transition, leading to bugs and duplicated code.

\begin{lstlisting}
typedef enum {
    STATE_OFF,
    STATE_STARTING,
    STATE_ON,
    STATE_STOPPING
} MotorState;

typedef struct {
    MotorState state;
} Motor;

// Entry actions
void motor_enter_starting(Motor* m) {
    printf("Motor starting up...\n");
    // Initialize hardware
}

void motor_enter_on(Motor* m) {
    printf("Motor running\n");
    // Enable monitoring
}

void motor_enter_stopping(Motor* m) {
    printf("Motor shutting down...\n");
    // Cleanup
}

// Exit actions
void motor_exit_on(Motor* m) {
    printf("Leaving ON state\n");
    // Disable monitoring
}

// State transition with actions
void motor_change_state(Motor* m, MotorState new_state) {
    // Exit current state
    switch (m->state) {
        case STATE_ON:
            motor_exit_on(m);
            break;
        default:
            break;
    }

    // Enter new state
    m->state = new_state;

    switch (new_state) {
        case STATE_STARTING:
            motor_enter_starting(m);
            break;
        case STATE_ON:
            motor_enter_on(m);
            break;
        case STATE_STOPPING:
            motor_enter_stopping(m);
            break;
        default:
            break;
    }
}
\end{lstlisting}

\section{Table-Driven State Machine}

Use a table for complex state transitions---separating the transition logic from the implementation.

Table-driven state machines represent transitions as data rather than code. This is powerful because:

\begin{itemize}
    \item \textbf{Clarity}: The transition table is easy to visualize
    \item \textbf{Validation}: You can verify all transitions are defined
    \item \textbf{Generation}: Tables can be generated from diagrams or specifications
    \item \textbf{Configuration}: Behavior can be changed without recompiling
    \item \textbf{Testing}: You can systematically test all transitions
\end{itemize}

This approach is ideal for complex state machines with many states and transitions. The table can even be loaded from a file, making the state machine behavior configurable at runtime. Protocol implementations often use this pattern.

\begin{lstlisting}
typedef enum {
    STATE_LOCKED,
    STATE_UNLOCKED,
    STATE_OPEN
} DoorState;

typedef enum {
    EVENT_LOCK,
    EVENT_UNLOCK,
    EVENT_OPEN,
    EVENT_CLOSE
} DoorEvent;

typedef struct {
    DoorState from_state;
    DoorEvent event;
    DoorState to_state;
    void (*action)(void*);  // Optional action
} Transition;

// Transition table
Transition door_transitions[] = {
    {STATE_LOCKED,   EVENT_UNLOCK, STATE_UNLOCKED, NULL},
    {STATE_UNLOCKED, EVENT_LOCK,   STATE_LOCKED,   NULL},
    {STATE_UNLOCKED, EVENT_OPEN,   STATE_OPEN,     open_door},
    {STATE_OPEN,     EVENT_CLOSE,  STATE_UNLOCKED, close_door},
    {STATE_OPEN,     EVENT_LOCK,   STATE_OPEN,     NULL},  // Invalid
};

typedef struct {
    DoorState state;
} Door;

int door_handle_event(Door* door, DoorEvent event) {
    // Search transition table
    for (size_t i = 0; i < sizeof(door_transitions) / sizeof(Transition); i++) {
        Transition* t = &door_transitions[i];

        if (t->from_state == door->state && t->event == event) {
            printf("Transition: %d -> %d\n", t->from_state, t->to_state);

            door->state = t->to_state;

            if (t->action) {
                t->action(door);
            }

            return 0;  // Success
        }
    }

    printf("Invalid transition from state %d with event %d\n",
           door->state, event);
    return -1;  // Invalid transition
}
\end{lstlisting}

\begin{tipbox}
Table-driven state machines separate data from code. You can load transitions from files, making behavior easily configurable!
\end{tipbox}

\section{Timeout and Timed States}

States that automatically transition after a timeout---essential for real-time and reactive systems.

Many systems need states that expire or time out:
\begin{itemize}
    \item Network connections that timeout if no response
    \item User interfaces with automatic dismiss
    \item Game states with time limits
    \item Safety systems that require periodic "heartbeat"
\end{itemize}

The key is tracking when each state was entered and checking elapsed time. This requires a main loop or periodic update function that calls your state machine's update method.

\textbf{Important}: Avoid busy-waiting in states. Instead, check timeout conditions in an update loop that's called regularly (e.g., 60 times per second in a game, or in your event loop).

\begin{lstlisting}
#include <time.h>

typedef struct {
    State state;
    time_t state_entered;
    double timeout_seconds;
} TimedStateMachine;

void sm_set_state_with_timeout(TimedStateMachine* sm,
                                State new_state,
                                double timeout) {
    sm->state = new_state;
    sm->state_entered = time(NULL);
    sm->timeout_seconds = timeout;
}

void sm_update(TimedStateMachine* sm) {
    double elapsed = difftime(time(NULL), sm->state_entered);

    if (sm->timeout_seconds > 0 && elapsed >= sm->timeout_seconds) {
        // Timeout occurred
        printf("State timeout!\n");

        switch (sm->state) {
            case STATE_WAITING:
                sm_set_state_with_timeout(sm, STATE_TIMEOUT, 0);
                break;
            // Handle other timeout transitions
        }
    }
}

// Call sm_update() regularly in your main loop
\end{lstlisting}

\section{State History}

Remember previous states for "back" functionality---implementing undo/redo or navigation history.

State history is crucial for:
\begin{itemize}
    \item UI navigation (back button)
    \item Undo/redo functionality
    \item Debugging (trace how you got to current state)
    \item Context preservation (return to where you were)
\end{itemize}

This is essentially a stack of states. Each time you transition to a new state, you push the old state onto the stack. Going "back" pops from the stack and returns to the previous state.

\textbf{Design consideration}: Decide whether history should be limited (circular buffer) or unlimited (dynamic array). Limited history prevents memory growth but loses old history.

\begin{lstlisting}
#define HISTORY_SIZE 10

typedef struct {
    State current_state;
    State history[HISTORY_SIZE];
    int history_pos;
} StateMachineWithHistory;

void sm_push_state(StateMachineWithHistory* sm, State new_state) {
    // Save current state to history
    sm->history[sm->history_pos] = sm->current_state;
    sm->history_pos = (sm->history_pos + 1) % HISTORY_SIZE;

    // Change to new state
    sm->current_state = new_state;
}

State sm_pop_state(StateMachineWithHistory* sm) {
    if (sm->history_pos == 0) {
        return sm->current_state;  // No history
    }

    // Go back to previous state
    sm->history_pos = (sm->history_pos - 1 + HISTORY_SIZE) % HISTORY_SIZE;
    sm->current_state = sm->history[sm->history_pos];

    return sm->current_state;
}
\end{lstlisting}

\section{Mealy vs Moore Machines}

Two fundamental types of state machines, differing in how they produce output.

\textbf{Moore Machine}: Output depends only on the current state. The output is determined by which state you're in, not how you got there. This makes Moore machines easier to reason about and debug.

\textbf{Mealy Machine}: Output depends on both the current state and the input event. This can make Mealy machines more compact (fewer states), but also more complex to understand.

In practice, most real-world state machines are hybrids---some outputs depend only on state, others on state and input. Choose the style that makes your code clearest.

\subsection{Moore Machine (Output depends on state)}

\begin{lstlisting}
typedef enum {
    STATE_GREEN,
    STATE_YELLOW,
    STATE_RED
} TrafficLightState;

// Output is determined by state
const char* get_light_color(TrafficLightState state) {
    switch (state) {
        case STATE_GREEN:  return "GREEN";
        case STATE_YELLOW: return "YELLOW";
        case STATE_RED:    return "RED";
        default:           return "UNKNOWN";
    }
}
\end{lstlisting}

\subsection{Mealy Machine (Output depends on state and input)}

Notice how the same state can produce different outputs depending on the input. This is more flexible but requires careful design to avoid confusion. The output is part of the transition, not just the state.

\begin{lstlisting}
typedef enum {
    VENDING_IDLE,
    VENDING_HAS_25,
    VENDING_HAS_50
} VendingState;

// Output depends on both state and input
const char* vending_insert_coin(VendingState* state, int cents) {
    switch (*state) {
        case VENDING_IDLE:
            if (cents == 25) {
                *state = VENDING_HAS_25;
                return "Insert 50 more cents";
            } else if (cents == 50) {
                *state = VENDING_HAS_50;
                return "Insert 25 more cents";
            }
            break;

        case VENDING_HAS_25:
            if (cents == 50) {
                *state = VENDING_IDLE;
                return "DISPENSING ITEM";
            }
            break;

        case VENDING_HAS_50:
            if (cents == 25) {
                *state = VENDING_IDLE;
                return "DISPENSING ITEM";
            }
            break;
    }
    return "Invalid coin";
}
\end{lstlisting}

\section{Real-World Example: TCP Connection}

TCP (Transmission Control Protocol) is one of the most famous real-world state machines. Understanding it helps you see how state machines model real protocols.

The TCP connection lifecycle involves 11 states. Each state represents a specific phase of connection establishment, data transfer, or connection termination. The beauty of the state machine model is that it precisely defines what to do with each packet type in each state.

For example, receiving a SYN (synchronize) packet in the CLOSED state means someone wants to connect, so you send SYN+ACK and move to SYN\_RECEIVED. But receiving SYN in the ESTABLISHED state is an error---connections are already established.

This example shows how state machines are essential for implementing network protocols correctly. Without a clear state machine, protocol implementations become bug-ridden tangles of conditionals.

\begin{lstlisting}
typedef enum {
    TCP_CLOSED,
    TCP_LISTEN,
    TCP_SYN_SENT,
    TCP_SYN_RECEIVED,
    TCP_ESTABLISHED,
    TCP_FIN_WAIT_1,
    TCP_FIN_WAIT_2,
    TCP_CLOSE_WAIT,
    TCP_CLOSING,
    TCP_LAST_ACK,
    TCP_TIME_WAIT
} TCPState;

typedef struct {
    TCPState state;
    int socket;
} TCPConnection;

void tcp_handle_packet(TCPConnection* conn, int flags) {
    switch (conn->state) {
        case TCP_CLOSED:
            if (flags & SYN) {
                send_syn_ack(conn->socket);
                conn->state = TCP_SYN_RECEIVED;
            }
            break;

        case TCP_SYN_SENT:
            if (flags & (SYN | ACK)) {
                send_ack(conn->socket);
                conn->state = TCP_ESTABLISHED;
            }
            break;

        case TCP_ESTABLISHED:
            if (flags & FIN) {
                send_ack(conn->socket);
                conn->state = TCP_CLOSE_WAIT;
            } else if (flags & ACK) {
                // Handle data...
            }
            break;

        case TCP_CLOSE_WAIT:
            // User calls close()
            send_fin(conn->socket);
            conn->state = TCP_LAST_ACK;
            break;

        // ... more states
    }
}
\end{lstlisting}

\section{State Machine Debugging}

Debugging state machines requires visibility into state transitions and the ability to validate transitions.

Common debugging challenges:
\begin{itemize}
    \item \textbf{Invalid transitions}: How did we get to this impossible state?
    \item \textbf{Missing transitions}: This event should do something but doesn't
    \item \textbf{Race conditions}: Multiple threads changing state simultaneously
    \item \textbf{Timing issues}: State changed too quickly or too slowly
\end{itemize}

The solutions are logging, validation, and visualization. Log every state transition with timestamps. Validate that transitions are legal. Generate diagrams showing the state machine structure.

\textbf{Pro tip}: Add a "trace mode" that logs every event and transition. When a bug occurs, you can replay the trace to see exactly how the machine got into that state.

\begin{lstlisting}
// State name lookup for debugging
const char* state_name(State s) {
    static const char* names[] = {
        "IDLE", "RUNNING", "PAUSED", "STOPPED"
    };
    return names[s];
}

// Logging state transitions
void sm_set_state_debug(StateMachine* sm, State new_state) {
    printf("[SM] %s -> %s\n",
           state_name(sm->state),
           state_name(new_state));
    sm->state = new_state;
}

// Validate transitions
int is_valid_transition(State from, State to) {
    // Define valid transitions
    static const int valid[][2] = {
        {STATE_IDLE, STATE_RUNNING},
        {STATE_RUNNING, STATE_PAUSED},
        {STATE_PAUSED, STATE_RUNNING},
        {STATE_RUNNING, STATE_STOPPED},
        {STATE_PAUSED, STATE_STOPPED},
    };

    for (size_t i = 0; i < sizeof(valid) / sizeof(valid[0]); i++) {
        if (valid[i][0] == from && valid[i][1] == to) {
            return 1;
        }
    }
    return 0;
}
\end{lstlisting}

\section{Concurrent State Machines}

Multiple state machines running in parallel---modeling objects with independent behaviors.

Complex systems often have multiple orthogonal (independent) aspects of state. A game character can be walking AND attacking simultaneously---movement and combat are independent state machines.

\textbf{Orthogonal states} mean that changes in one state machine don't affect the other. The character's movement state (idle, walking, running, jumping) is independent of their combat state (idle, attacking, blocking, stunned).

However, you often need coordination between state machines. In the example below, the animation state machine depends on both movement and combat states, giving priority to combat animations.

This pattern is common in:
\begin{itemize}
    \item Game engines (animation, physics, AI all have separate state)
    \item UI systems (focus state, hover state, drag state are independent)
    \item Embedded systems (sensor reading, motor control, communication are separate)
\end{itemize}

\begin{lstlisting}
// Game character with independent state machines
typedef struct {
    // Movement state machine
    enum {
        MOVE_IDLE,
        MOVE_WALKING,
        MOVE_RUNNING,
        MOVE_JUMPING
    } movement_state;

    // Combat state machine
    enum {
        COMBAT_IDLE,
        COMBAT_ATTACKING,
        COMBAT_BLOCKING,
        COMBAT_STUNNED
    } combat_state;

    // Animation state machine
    enum {
        ANIM_IDLE,
        ANIM_WALK,
        ANIM_RUN,
        ANIM_JUMP,
        ANIM_ATTACK,
        ANIM_BLOCK
    } anim_state;
} Character;

// Update all state machines
void character_update(Character* c, float dt) {
    // Update movement
    update_movement_sm(c, dt);

    // Update combat (independent of movement)
    update_combat_sm(c, dt);

    // Animation depends on both movement and combat
    update_animation_sm(c);
}

// Animation selects based on priority
void update_animation_sm(Character* c) {
    // Combat animations have priority
    if (c->combat_state == COMBAT_ATTACKING) {
        c->anim_state = ANIM_ATTACK;
    } else if (c->combat_state == COMBAT_BLOCKING) {
        c->anim_state = ANIM_BLOCK;
    }
    // Then movement animations
    else if (c->movement_state == MOVE_RUNNING) {
        c->anim_state = ANIM_RUN;
    } else if (c->movement_state == MOVE_WALKING) {
        c->anim_state = ANIM_WALK;
    } else if (c->movement_state == MOVE_JUMPING) {
        c->anim_state = ANIM_JUMP;
    } else {
        c->anim_state = ANIM_IDLE;
    }
}
\end{lstlisting}

\section{Pushdown Automaton}

State machines with a stack for nested states---a more powerful computational model.

A \textbf{pushdown automaton} (PDA) is a state machine augmented with a stack. This allows it to remember an arbitrary amount of information, making it more powerful than a finite state machine.

PDAs are perfect for:
\begin{itemize}
    \item Menu systems (main -> options -> graphics -> advanced, then back out)
    \item Expression parsing (matching parentheses)
    \item Function call stacks
    \item Undo/redo that preserves full state
    \item Hierarchical navigation
\end{itemize}

The stack remembers "where you came from," allowing you to return to previous contexts. This is more powerful than simple history because each state can be entered from multiple previous states.

Think of it like a web browser's back button---it remembers the full navigation path, not just the previous page.

\begin{lstlisting}
#define STATE_STACK_SIZE 16

typedef struct {
    State stack[STATE_STACK_SIZE];
    int top;
} StateStack;

void stack_push(StateStack* s, State state) {
    if (s->top < STATE_STACK_SIZE - 1) {
        s->stack[++s->top] = state;
    }
}

State stack_pop(StateStack* s) {
    if (s->top >= 0) {
        return s->stack[s->top--];
    }
    return STATE_INVALID;
}

State stack_peek(StateStack* s) {
    if (s->top >= 0) {
        return s->stack[s->top];
    }
    return STATE_INVALID;
}

// Menu system with state stack
typedef enum {
    MENU_MAIN,
    MENU_OPTIONS,
    MENU_GRAPHICS,
    MENU_AUDIO,
    MENU_CONTROLS,
    MENU_CONFIRM_EXIT
} MenuState;

typedef struct {
    StateStack states;
} MenuSystem;

void menu_init(MenuSystem* menu) {
    menu->states.top = -1;
    stack_push(&menu->states, MENU_MAIN);
}

void menu_enter_submenu(MenuSystem* menu, MenuState state) {
    stack_push(&menu->states, state);
    printf("Entering menu: %d\n", state);
}

void menu_go_back(MenuSystem* menu) {
    if (menu->states.top > 0) {  // Keep at least one state
        State old = stack_pop(&menu->states);
        State current = stack_peek(&menu->states);
        printf("Back from %d to %d\n", old, current);
    }
}

MenuState menu_current(MenuSystem* menu) {
    return stack_peek(&menu->states);
}

// Usage
// Main Menu -> Options -> Graphics -> (back) -> Options -> (back) -> Main
\end{lstlisting}

\section{Event Queue State Machine}

Process events from a queue for better control---decoupling event generation from event processing.

\textbf{Why use an event queue?}

\begin{itemize}
    \item \textbf{Decoupling}: Event producers don't need to know about the state machine
    \item \textbf{Ordering}: Events are processed in a predictable order
    \item \textbf{Rate limiting}: Control how many events to process per frame
    \item \textbf{Replay}: Save and replay event sequences for testing
    \item \textbf{Thread safety}: Only one thread processes the queue
\end{itemize}

This pattern is essential in event-driven architectures like GUI systems, game engines, and embedded systems. Events are posted to the queue from anywhere (user input, network, timers), and the state machine processes them at a controlled rate.

\textbf{Important}: Decide what happens when the queue fills up. Drop old events? Drop new events? Block the producer? The right answer depends on your application.

\begin{lstlisting}
#define EVENT_QUEUE_SIZE 64

typedef struct {
    int type;
    void* data;
} Event;

typedef struct {
    Event events[EVENT_QUEUE_SIZE];
    int read_pos;
    int write_pos;
    int count;
} EventQueue;

void event_queue_init(EventQueue* q) {
    q->read_pos = 0;
    q->write_pos = 0;
    q->count = 0;
}

int event_queue_push(EventQueue* q, Event event) {
    if (q->count >= EVENT_QUEUE_SIZE) {
        return -1;  // Queue full
    }

    q->events[q->write_pos] = event;
    q->write_pos = (q->write_pos + 1) % EVENT_QUEUE_SIZE;
    q->count++;
    return 0;
}

int event_queue_pop(EventQueue* q, Event* event) {
    if (q->count == 0) {
        return -1;  // Queue empty
    }

    *event = q->events[q->read_pos];
    q->read_pos = (q->read_pos + 1) % EVENT_QUEUE_SIZE;
    q->count--;
    return 0;
}

// State machine with event queue
typedef struct {
    State state;
    EventQueue queue;
} QueuedStateMachine;

void sm_post_event(QueuedStateMachine* sm, int type, void* data) {
    Event e = {.type = type, .data = data};
    event_queue_push(&sm->queue, e);
}

void sm_process_events(QueuedStateMachine* sm) {
    Event e;
    while (event_queue_pop(&sm->queue, &e) == 0) {
        // Process event based on current state
        switch (sm->state) {
            case STATE_IDLE:
                if (e.type == EVENT_START) {
                    sm->state = STATE_RUNNING;
                }
                break;

            case STATE_RUNNING:
                if (e.type == EVENT_STOP) {
                    sm->state = STATE_IDLE;
                }
                break;
        }

        // Free event data if needed
        if (e.data) {
            free(e.data);
        }
    }
}
\end{lstlisting}

\section{Guard Conditions}

Add conditions to state transitions---making transitions conditional on runtime state.

Sometimes a transition should only occur if certain conditions are met. For example:
\begin{itemize}
    \item Only allow checkout if cart has items and user has payment method
    \item Only allow file deletion if user has permission
    \item Only allow engine start if safety checks pass
\end{itemize}

\textbf{Guard conditions} are boolean functions evaluated before a transition. If the guard returns false, the transition is blocked, and the state machine remains in its current state.

This keeps your state machine declarative---the transition table says "when X happens IF condition Y, go to state Z." Without guards, you'd need separate states for every combination of conditions, leading to state explosion.

Guards vs. events: Events are external stimuli. Guards are internal conditions. Both are needed for flexible, real-world state machines.

\begin{lstlisting}
typedef struct {
    State from_state;
    int event;
    State to_state;
    int (*guard)(void* context);  // Condition function
    void (*action)(void* context);
} GuardedTransition;

// Guard functions
int has_permission(void* context) {
    User* user = (User*)context;
    return user->is_admin;
}

int has_enough_money(void* context) {
    Account* acc = (Account*)context;
    return acc->balance >= 100;
}

// Transition table with guards
GuardedTransition transitions[] = {
    {STATE_MENU, EVENT_ADMIN, STATE_ADMIN_PANEL, has_permission, NULL},
    {STATE_CART, EVENT_CHECKOUT, STATE_PAYMENT, has_enough_money, NULL},
    {STATE_IDLE, EVENT_START, STATE_RUNNING, NULL, start_engine},
};

int sm_handle_guarded_event(StateMachine* sm, int event, void* context) {
    for (size_t i = 0; i < sizeof(transitions)/sizeof(GuardedTransition); i++) {
        GuardedTransition* t = &transitions[i];

        if (t->from_state == sm->state && t->event == event) {
            // Check guard condition
            if (t->guard == NULL || t->guard(context)) {
                sm->state = t->to_state;

                if (t->action) {
                    t->action(context);
                }

                return 0;  // Transition succeeded
            } else {
                return -1;  // Guard failed
            }
        }
    }

    return -2;  // No matching transition
}
\end{lstlisting}

\section{State Machine Code Generation}

Generate state machine code from a table---reducing boilerplate and ensuring consistency.

Writing state machine code by hand is tedious and error-prone. You need:
\begin{itemize}
    \item State enum definitions
    \item State name strings for debugging
    \item Entry action function declarations
    \item Exit action function declarations
    \item Action function arrays
    \item Transition logic
\end{itemize}

The X-macro technique lets you define your state machine once and generate all this boilerplate automatically. Change the state list in one place, and all the generated code updates automatically.

This is similar to how parser generators (like yacc/bison) generate code from grammar specifications. You describe \textit{what} the state machine is, and the macro system generates \textit{how} to implement it.

\textbf{Bonus}: With external tools, you can generate state machines from visual diagrams or XML specifications, making them accessible to non-programmers.

\begin{lstlisting}
// State machine description (could be from a file)
#define STATE_MACHINE_DEF \
    X(IDLE,      "Idle",      on_enter_idle,   on_exit_idle) \
    X(STARTING,  "Starting",  on_enter_starting, NULL) \
    X(RUNNING,   "Running",   on_enter_running, on_exit_running) \
    X(STOPPING,  "Stopping",  on_enter_stopping, NULL) \
    X(ERROR,     "Error",     on_enter_error,  NULL)

// Generate enum
typedef enum {
#define X(name, str, enter, exit) STATE_##name,
    STATE_MACHINE_DEF
#undef X
    STATE_COUNT
} State;

// Generate string table
static const char* state_names[] = {
#define X(name, str, enter, exit) str,
    STATE_MACHINE_DEF
#undef X
};

// Forward declare action functions
#define X(name, str, enter, exit) \
    void enter(void* ctx); \
    void exit(void* ctx);
STATE_MACHINE_DEF
#undef X

// Entry action table
typedef void (*StateAction)(void* ctx);

static StateAction entry_actions[] = {
#define X(name, str, enter, exit) enter,
    STATE_MACHINE_DEF
#undef X
};

static StateAction exit_actions[] = {
#define X(name, str, enter, exit) exit,
    STATE_MACHINE_DEF
#undef X
};

// Transition with actions
void sm_transition(StateMachine* sm, State new_state, void* ctx) {
    // Exit current state
    if (exit_actions[sm->state]) {
        exit_actions[sm->state](ctx);
    }

    printf("Transition: %s -> %s\n",
           state_names[sm->state],
           state_names[new_state]);

    sm->state = new_state;

    // Enter new state
    if (entry_actions[new_state]) {
        entry_actions[new_state](ctx);
    }
}
\end{lstlisting}

\section{Real-World Example: Protocol Parser}

HTTP request parser as a state machine---a practical example of character-by-character parsing.

Parsing text-based protocols like HTTP is a perfect application for state machines. Each character advances the machine through states representing different parts of the request:

\begin{enumerate}
    \item START -> METHOD (reading "GET", "POST", etc.)
    \item METHOD -> URI (reading "/path/to/resource")
    \item URI -> VERSION (reading "HTTP/1.1")
    \item VERSION -> HEADER\_NAME (reading "Content-Type")
    \item HEADER\_NAME -> HEADER\_VALUE (reading "application/json")
    \item Repeat headers until blank line
    \item HEADER\_NAME -> BODY (blank line signals end of headers)
    \item BODY -> DONE (message complete)
\end{enumerate}

This approach is extremely efficient---each character is examined once, no backtracking, no complex string operations. The state machine enforces the protocol grammar, automatically rejecting malformed requests.

Many high-performance servers (nginx, nodejs http-parser) use state machine parsers for this reason.

\begin{lstlisting}
typedef enum {
    HTTP_START,
    HTTP_METHOD,
    HTTP_URI,
    HTTP_VERSION,
    HTTP_HEADER_NAME,
    HTTP_HEADER_VALUE,
    HTTP_BODY,
    HTTP_DONE,
    HTTP_ERROR
} HTTPParseState;

typedef struct {
    HTTPParseState state;

    // Parsed data
    char method[16];
    char uri[256];
    char version[16];
    char headers[32][2][256];  // name/value pairs
    int header_count;
    char* body;
    size_t body_length;

    // Parsing buffers
    char buffer[1024];
    size_t buffer_pos;
} HTTPParser;

void http_parser_init(HTTPParser* p) {
    memset(p, 0, sizeof(*p));
    p->state = HTTP_START;
}

int http_parser_feed(HTTPParser* p, char c) {
    switch (p->state) {
        case HTTP_START:
            if (isalpha(c)) {
                p->buffer[0] = c;
                p->buffer_pos = 1;
                p->state = HTTP_METHOD;
            } else if (!isspace(c)) {
                p->state = HTTP_ERROR;
                return -1;
            }
            break;

        case HTTP_METHOD:
            if (isalpha(c)) {
                p->buffer[p->buffer_pos++] = c;
            } else if (c == ' ') {
                p->buffer[p->buffer_pos] = '\0';
                strcpy(p->method, p->buffer);
                p->buffer_pos = 0;
                p->state = HTTP_URI;
            } else {
                p->state = HTTP_ERROR;
                return -1;
            }
            break;

        case HTTP_URI:
            if (c == ' ') {
                p->buffer[p->buffer_pos] = '\0';
                strcpy(p->uri, p->buffer);
                p->buffer_pos = 0;
                p->state = HTTP_VERSION;
            } else if (!iscntrl(c)) {
                p->buffer[p->buffer_pos++] = c;
            } else {
                p->state = HTTP_ERROR;
                return -1;
            }
            break;

        case HTTP_VERSION:
            if (c == '\r') {
                // Ignore
            } else if (c == '\n') {
                p->buffer[p->buffer_pos] = '\0';
                strcpy(p->version, p->buffer);
                p->buffer_pos = 0;
                p->state = HTTP_HEADER_NAME;
            } else {
                p->buffer[p->buffer_pos++] = c;
            }
            break;

        case HTTP_HEADER_NAME:
            if (c == '\r') {
                // Ignore
            } else if (c == '\n') {
                if (p->buffer_pos == 0) {
                    // Empty line - end of headers
                    p->state = HTTP_BODY;
                } else {
                    p->state = HTTP_ERROR;
                    return -1;
                }
            } else if (c == ':') {
                p->buffer[p->buffer_pos] = '\0';
                strcpy(p->headers[p->header_count][0], p->buffer);
                p->buffer_pos = 0;
                p->state = HTTP_HEADER_VALUE;
            } else {
                p->buffer[p->buffer_pos++] = c;
            }
            break;

        case HTTP_HEADER_VALUE:
            if (c == '\r') {
                // Ignore
            } else if (c == '\n') {
                p->buffer[p->buffer_pos] = '\0';
                strcpy(p->headers[p->header_count][1], p->buffer);
                p->header_count++;
                p->buffer_pos = 0;
                p->state = HTTP_HEADER_NAME;
            } else if (c == ' ' && p->buffer_pos == 0) {
                // Skip leading space
            } else {
                p->buffer[p->buffer_pos++] = c;
            }
            break;

        case HTTP_BODY:
            // Body parsing depends on Content-Length header
            p->state = HTTP_DONE;
            break;

        case HTTP_DONE:
            return 1;  // Complete

        case HTTP_ERROR:
            return -1;
    }

    return 0;  // Continue
}

// Usage
HTTPParser parser;
http_parser_init(&parser);

const char* request = "GET /index.html HTTP/1.1\r\n"
                     "Host: example.com\r\n"
                     "User-Agent: MyClient/1.0\r\n"
                     "\r\n";

for (size_t i = 0; request[i]; i++) {
    int result = http_parser_feed(&parser, request[i]);
    if (result != 0) break;
}

printf("Method: %s\n", parser.method);
printf("URI: %s\n", parser.uri);
\end{lstlisting}

\section{Real-World Example: Game AI}

Enemy AI with behavior states---implementing intelligent NPC behavior.

Game AI is often implemented as state machines where each state represents a different behavior:

\begin{itemize}
    \item \textbf{PATROL}: Default behavior, following waypoints
    \item \textbf{INVESTIGATE}: Heard something, checking it out
    \item \textbf{CHASE}: Found the player, pursuing
    \item \textbf{ATTACK}: Close enough, attacking
    \item \textbf{FLEE}: Low health, running away
    \item \textbf{SEARCH}: Lost track of player, searching area
\end{itemize}

The beauty is in the transitions. The AI feels intelligent because it reacts appropriately to stimuli:
\begin{itemize}
    \item Spots player while patrolling -> investigate
    \item Gets close while investigating -> chase
    \item Gets very close while chasing -> attack
    \item Health drops too low -> flee
    \item Loses sight of player -> search
    \item Can't find player after searching -> give up, resume patrol
\end{itemize}

This creates believable, fun-to-fight enemies without complex algorithms. Add some randomness to transitions, and each encounter feels different.

\begin{lstlisting}
typedef enum {
    AI_PATROL,
    AI_INVESTIGATE,
    AI_CHASE,
    AI_ATTACK,
    AI_FLEE,
    AI_SEARCH
} AIState;

typedef struct {
    AIState state;
    float state_timer;

    // AI memory
    Vector3 last_known_player_pos;
    float time_since_seen_player;
    float health;
    Vector3 patrol_points[4];
    int current_patrol_point;
} EnemyAI;

void ai_update(EnemyAI* ai, Player* player, float dt) {
    ai->state_timer += dt;

    float distance = vector3_distance(ai->position, player->position);
    bool can_see_player = line_of_sight(ai->position, player->position);

    switch (ai->state) {
        case AI_PATROL:
            // Move to patrol points
            move_towards(ai, ai->patrol_points[ai->current_patrol_point]);

            if (reached_point(ai)) {
                ai->current_patrol_point =
                    (ai->current_patrol_point + 1) % 4;
            }

            // Spot player
            if (can_see_player && distance < 30.0f) {
                ai->last_known_player_pos = player->position;
                ai->state = AI_INVESTIGATE;
                ai->state_timer = 0;
            }
            break;

        case AI_INVESTIGATE:
            // Move to last known position
            move_towards(ai, ai->last_known_player_pos);

            if (can_see_player) {
                ai->last_known_player_pos = player->position;

                if (distance < 10.0f) {
                    ai->state = AI_CHASE;
                    ai->state_timer = 0;
                }
            } else if (reached_point(ai) || ai->state_timer > 5.0f) {
                // Lost track
                ai->state = AI_SEARCH;
                ai->state_timer = 0;
            }
            break;

        case AI_CHASE:
            if (can_see_player) {
                ai->last_known_player_pos = player->position;
                move_towards(ai, player->position);

                if (distance < 2.0f) {
                    ai->state = AI_ATTACK;
                    ai->state_timer = 0;
                }

                // Low health - flee
                if (ai->health < 30.0f) {
                    ai->state = AI_FLEE;
                    ai->state_timer = 0;
                }
            } else {
                // Lost sight
                ai->state = AI_INVESTIGATE;
                ai->state_timer = 0;
            }
            break;

        case AI_ATTACK:
            look_at(ai, player->position);

            if (ai->state_timer > 1.0f) {  // Attack cooldown
                perform_attack(ai);
                ai->state_timer = 0;
            }

            if (distance > 3.0f) {
                ai->state = AI_CHASE;
            }

            if (ai->health < 30.0f) {
                ai->state = AI_FLEE;
                ai->state_timer = 0;
            }
            break;

        case AI_FLEE:
            // Run away from player
            Vector3 flee_dir = vector3_sub(ai->position, player->position);
            flee_dir = vector3_normalize(flee_dir);
            move_in_direction(ai, flee_dir);

            if (distance > 50.0f) {
                // Escaped
                ai->state = AI_SEARCH;
                ai->state_timer = 0;
            }
            break;

        case AI_SEARCH:
            // Search area for player
            wander(ai);

            if (can_see_player) {
                ai->last_known_player_pos = player->position;
                ai->state = AI_INVESTIGATE;
                ai->state_timer = 0;
            } else if (ai->state_timer > 10.0f) {
                // Give up
                ai->state = AI_PATROL;
                ai->state_timer = 0;
            }
            break;
    }
}
\end{lstlisting}

\section{State Machine Testing}

Testing state machines systematically---ensuring all transitions work correctly.

State machines are highly testable because their behavior is deterministic: given a starting state and a sequence of events, the final state is predictable.

\textbf{Testing strategies}:
\begin{itemize}
    \item \textbf{Transition coverage}: Test every valid transition at least once
    \item \textbf{Invalid transition testing}: Verify invalid transitions are rejected
    \item \textbf{State coverage}: Exercise all states
    \item \textbf{Sequence testing}: Test common event sequences
    \item \textbf{Stress testing}: Rapid transitions, deep nesting, long-running states
\end{itemize}

The test framework shown here lets you define test cases as data: initial state, event sequence, expected final state. Run all tests automatically and catch regressions.

\textbf{Advanced testing}: Generate random valid event sequences and verify the state machine never enters an invalid state. This is called \textbf{fuzzing} and catches edge cases you didn't think of.

\begin{lstlisting}
// Test framework for state machines
typedef struct {
    const char* name;
    State initial_state;
    int events[10];
    int num_events;
    State expected_final_state;
} StateTest;

int run_state_test(StateTest* test) {
    StateMachine sm = {.state = test->initial_state};

    printf("Running test: %s\n", test->name);

    for (int i = 0; i < test->num_events; i++) {
        printf("  Event %d: %d\n", i, test->events[i]);
        sm_handle_event(&sm, test->events[i]);
        printf("  State: %s\n", state_name(sm.state));
    }

    if (sm.state == test->expected_final_state) {
        printf("  PASS\n");
        return 0;
    } else {
        printf("  FAIL: Expected %s, got %s\n",
               state_name(test->expected_final_state),
               state_name(sm.state));
        return -1;
    }
}

// Test cases
StateTest tests[] = {
    {
        .name = "Normal startup",
        .initial_state = STATE_OFF,
        .events = {EVENT_POWER_ON, EVENT_START},
        .num_events = 2,
        .expected_final_state = STATE_RUNNING
    },
    {
        .name = "Emergency stop",
        .initial_state = STATE_RUNNING,
        .events = {EVENT_EMERGENCY_STOP},
        .num_events = 1,
        .expected_final_state = STATE_ERROR
    },
    // More tests...
};

void run_all_tests(void) {
    int passed = 0;
    int total = sizeof(tests) / sizeof(StateTest);

    for (int i = 0; i < total; i++) {
        if (run_state_test(&tests[i]) == 0) {
            passed++;
        }
    }

    printf("\nTests: %d/%d passed\n", passed, total);
}
\end{lstlisting}

\section{State Machine Visualization}

Generate GraphViz diagrams---making state machines visible and understandable.

A picture is worth a thousand lines of code. State diagrams show:
\begin{itemize}
    \item All states (nodes)
    \item All transitions (edges)
    \item Transition triggers (edge labels)
    \item Overall structure at a glance
\end{itemize}

GraphViz is perfect for this---it automatically layouts the diagram so you don't have to position nodes manually. The DOT language is simple: define nodes, define edges with labels, done.

\textbf{Use cases}:
\begin{itemize}
    \item \textbf{Documentation}: Include diagrams in your manual
    \item \textbf{Code review}: Visualize before implementing
    \item \textbf{Debugging}: See if actual transitions match expected
    \item \textbf{Communication}: Show designers/stakeholders how it works
\end{itemize}

You can even generate diagrams automatically from your code's transition table, ensuring documentation never gets out of sync with implementation.

\begin{lstlisting}
void sm_generate_dot(FILE* f) {
    fprintf(f, "digraph StateMachine {\n");
    fprintf(f, "  rankdir=LR;\n");
    fprintf(f, "  node [shape=circle];\n\n");

    // Define states
    for (int i = 0; i < STATE_COUNT; i++) {
        fprintf(f, "  %s [label=\"%s\"];\n",
                state_name(i), state_name(i));
    }

    fprintf(f, "\n");

    // Define transitions
    for (size_t i = 0; i < num_transitions; i++) {
        fprintf(f, "  %s -> %s [label=\"%s\"];\n",
                state_name(transitions[i].from),
                state_name(transitions[i].to),
                event_name(transitions[i].event));
    }

    fprintf(f, "}\n");
}

// Usage:
// FILE* f = fopen("statemachine.dot", "w");
// sm_generate_dot(f);
// fclose(f);
// system("dot -Tpng statemachine.dot -o statemachine.png");
\end{lstlisting}

\section{Performance Considerations}

Optimizing state machines for high-performance applications---when millions of transitions per second matter.

Most state machines are fast enough without optimization. But in hot paths (game loops, packet processing, real-time systems), performance matters.

\textbf{Optimization techniques}:
\begin{itemize}
    \item \textbf{Direct table lookup}: O(1) transitions instead of searching
    \item \textbf{Perfect hashing}: Hash state+event to array index
    \item \textbf{Jump tables}: Compiler generates efficient code for switch statements
    \item \textbf{Cache-friendly layout}: Keep state data in one cache line
    \item \textbf{Avoid function pointers}: Direct calls are faster than indirect
\end{itemize}

The table-based approach shown here trades memory for speed. If you have 32 states and 32 events, you need a 32\times32 table (1KB). But each lookup is just two array accesses---no searching, no function calls.

\textbf{Measurement matters}: Profile before optimizing. Most state machines aren't bottlenecks. But when they are, these techniques can speed them up 10-100\times.

\begin{lstlisting}
// Optimize state transitions with perfect hashing
typedef struct {
    State state;
    int event;
} StateEventKey;

// Hash function for state/event pair
static inline unsigned int hash_state_event(State s, int e) {
    return (s << 8) | e;
}

// Direct lookup table (if state/event space is small)
#define MAX_STATES 32
#define MAX_EVENTS 32

static State transition_table[MAX_STATES][MAX_EVENTS];

void init_transition_table(void) {
    // Initialize with invalid transitions
    for (int i = 0; i < MAX_STATES; i++) {
        for (int j = 0; j < MAX_EVENTS; j++) {
            transition_table[i][j] = STATE_INVALID;
        }
    }

    // Fill in valid transitions
    transition_table[STATE_IDLE][EVENT_START] = STATE_RUNNING;
    transition_table[STATE_RUNNING][EVENT_STOP] = STATE_IDLE;
    // etc...
}

// O(1) transition lookup
State fast_transition(State current, int event) {
    if (current < MAX_STATES && event < MAX_EVENTS) {
        return transition_table[current][event];
    }
    return STATE_INVALID;
}

// Cache-friendly state machine for high-performance
typedef struct {
    State state;
    uint32_t padding;  // Align to cache line
} __attribute__((aligned(64))) CacheFriendlySM;
\end{lstlisting}

\section{Summary}

State machines in C:

\begin{itemize}
    \item Use enums for states---clear and type-safe
    \item Switch-based for simple state machines
    \item Function pointers for flexible behavior
    \item Tables for complex transition logic
    \item Add entry/exit actions for cleaner code
    \item Track state history for undo/back functionality
    \item Use guard conditions for conditional transitions
    \item Event queues for better event handling
    \item Concurrent state machines for complex objects
    \item Pushdown automata for hierarchical states
    \item Always validate state transitions
    \item Log transitions for debugging
    \item Generate visualizations for documentation
    \item Test thoroughly with automated test cases
    \item Optimize hot paths with direct lookup tables
\end{itemize}

State machines turn complex behavior into manageable, testable code. Master them and your code will be more reliable and easier to understand!

\chapter{Generic Programming in C}

\section{The Challenge of Generic Code}

C is a strongly-typed language without built-in generics. Yet real-world code constantly needs generic data structures and algorithms. How do you write a linked list that works with any type? How do you implement qsort that can sort anything?

The answer: C provides several mechanisms for generic programming, each with trade-offs:

\begin{itemize}
    \item \textbf{Void pointers}: Type-erased generic code (runtime flexibility)
    \item \textbf{Macros}: Code generation at compile-time (zero overhead)
    \item \textbf{Function pointers}: Parameterize behavior (callbacks, comparators)
    \item \textbf{\_Generic (C11)}: Type-based dispatch at compile-time
    \item \textbf{Code generation}: External tools generate typed code
\end{itemize}

\textbf{The fundamental tension}: Type safety vs. genericity. C makes you choose. Void pointers sacrifice type safety for flexibility. Macros provide type safety but can be complex. Understanding when to use each approach is the mark of an experienced C programmer.

\section{Void Pointer Basics}

The foundation of generic programming in C---\texttt{void*} is a pointer to "unknown type" that can point to anything.

\textbf{Key properties}:
\begin{itemize}
    \item Any pointer can be assigned to \texttt{void*} without casting
    \item \texttt{void*} must be cast before dereferencing
    \item Cannot do pointer arithmetic on \texttt{void*} (no size information)
    \item Perfect for APIs that don't need to know the data type
\end{itemize}

\begin{lstlisting}
// Basic void pointer usage
void print_bytes(const void* data, size_t size) {
    const unsigned char* bytes = (const unsigned char*)data;

    for (size_t i = 0; i < size; i++) {
        printf("%02x ", bytes[i]);
    }
    printf("\n");
}

// Works with any type
int x = 0x12345678;
print_bytes(&x, sizeof(x));

float f = 3.14f;
print_bytes(&f, sizeof(f));

char str[] = "Hello";
print_bytes(str, sizeof(str));
\end{lstlisting}

\begin{warningbox}
Never dereference \texttt{void*} directly! Always cast to the correct type first. The compiler cannot help you if you cast incorrectly---this is where bugs hide.
\end{warningbox}

\subsection{Generic Swap Function}

The classic example of generic programming:

\begin{lstlisting}
// Generic swap - works with any type
void swap(void* a, void* b, size_t size) {
    // Allocate temporary buffer
    unsigned char temp[size];

    // Copy a to temp
    memcpy(temp, a, size);

    // Copy b to a
    memcpy(a, b, size);

    // Copy temp to b
    memcpy(b, temp, size);
}

// Usage with different types
int x = 10, y = 20;
swap(&x, &y, sizeof(int));
printf("x=%d, y=%d\n", x, y);  // x=20, y=10

double d1 = 3.14, d2 = 2.71;
swap(&d1, &d2, sizeof(double));

struct Point { int x, y; } p1 = {1, 2}, p2 = {3, 4};
swap(&p1, &p2, sizeof(struct Point));
\end{lstlisting}

\textbf{Why this works}: We don't need to know the type, just its size. \texttt{memcpy} treats memory as raw bytes. This is the essence of type erasure.

\section{Generic Comparison Functions}

The standard library's \texttt{qsort} and \texttt{bsearch} use function pointers for generic comparison---a pattern you'll use constantly.

\begin{lstlisting}
// Standard comparator signature
typedef int (*CompareFn)(const void* a, const void* b);

// Comparator returns:
//   < 0  if a < b
//   = 0  if a == b
//   > 0  if a > b

// Integer comparator
int compare_int(const void* a, const void* b) {
    int x = *(const int*)a;
    int y = *(const int*)b;
    return (x > y) - (x < y);  // Branchless comparison
}

// String comparator
int compare_string(const void* a, const void* b) {
    const char* str_a = *(const char**)a;  // Double pointer!
    const char* str_b = *(const char**)b;
    return strcmp(str_a, str_b);
}

// Struct comparator
typedef struct {
    char name[32];
    int age;
    double salary;
} Employee;

int compare_employee_by_salary(const void* a, const void* b) {
    const Employee* emp_a = (const Employee*)a;
    const Employee* emp_b = (const Employee*)b;

    if (emp_a->salary < emp_b->salary) return -1;
    if (emp_a->salary > emp_b->salary) return 1;
    return 0;
}

// Usage
int numbers[] = {5, 2, 8, 1, 9};
qsort(numbers, 5, sizeof(int), compare_int);

const char* names[] = {"Charlie", "Alice", "Bob"};
qsort(names, 3, sizeof(char*), compare_string);

Employee employees[100];
qsort(employees, 100, sizeof(Employee), compare_employee_by_salary);
\end{lstlisting}

\begin{tipbox}
When comparing strings in arrays, remember you have an array of pointers, so you need \texttt{const char**} after casting. This trips up beginners constantly!
\end{tipbox}

\section{Generic Data Structures: Dynamic Array}

Let's build a real generic dynamic array (like C++'s vector) using void pointers.

\begin{lstlisting}
typedef struct {
    void* data;           // Pointer to array data
    size_t element_size;  // Size of each element
    size_t size;          // Number of elements
    size_t capacity;      // Allocated capacity
} Vector;

// Create vector for any type
Vector* vector_create(size_t element_size, size_t initial_capacity) {
    Vector* vec = malloc(sizeof(Vector));
    if (!vec) return NULL;

    vec->element_size = element_size;
    vec->size = 0;
    vec->capacity = initial_capacity;
    vec->data = malloc(element_size * initial_capacity);

    if (!vec->data) {
        free(vec);
        return NULL;
    }

    return vec;
}

// Grow capacity when needed
static int vector_grow(Vector* vec) {
    size_t new_capacity = vec->capacity * 2;
    void* new_data = realloc(vec->data,
                             vec->element_size * new_capacity);
    if (!new_data) return -1;

    vec->data = new_data;
    vec->capacity = new_capacity;
    return 0;
}

// Push element to end
int vector_push(Vector* vec, const void* element) {
    if (vec->size >= vec->capacity) {
        if (vector_grow(vec) < 0) return -1;
    }

    // Calculate destination address
    void* dest = (char*)vec->data + (vec->size * vec->element_size);

    // Copy element
    memcpy(dest, element, vec->element_size);
    vec->size++;

    return 0;
}

// Get element by index (returns pointer)
void* vector_get(Vector* vec, size_t index) {
    if (index >= vec->size) return NULL;

    return (char*)vec->data + (index * vec->element_size);
}

// Set element by index
int vector_set(Vector* vec, size_t index, const void* element) {
    if (index >= vec->size) return -1;

    void* dest = (char*)vec->data + (index * vec->element_size);
    memcpy(dest, element, vec->element_size);

    return 0;
}

void vector_destroy(Vector* vec) {
    if (vec) {
        free(vec->data);
        free(vec);
    }
}
\end{lstlisting}

\textbf{Usage with different types}:

\begin{lstlisting}
// Vector of integers
Vector* int_vec = vector_create(sizeof(int), 10);

int values[] = {10, 20, 30, 40, 50};
for (int i = 0; i < 5; i++) {
    vector_push(int_vec, &values[i]);
}

// Access elements
for (size_t i = 0; i < int_vec->size; i++) {
    int* val = (int*)vector_get(int_vec, i);
    printf("%d ", *val);
}

vector_destroy(int_vec);

// Vector of structs
typedef struct { double x, y; } Point;

Vector* point_vec = vector_create(sizeof(Point), 10);

Point p = {3.14, 2.71};
vector_push(point_vec, &p);

Point* retrieved = (Point*)vector_get(point_vec, 0);
printf("Point: (%.2f, %.2f)\n", retrieved->x, retrieved->y);

vector_destroy(point_vec);
\end{lstlisting}

\textbf{Critical pointer arithmetic}: When working with \texttt{void*}, cast to \texttt{char*} for arithmetic. Each \texttt{char} is 1 byte, so \texttt{(char*)data + offset} works correctly regardless of element size.

\section{Generic Linked List}

A linked list that can store any type---used everywhere in systems programming.

\begin{lstlisting}
typedef struct Node {
    void* data;
    struct Node* next;
} Node;

typedef struct {
    Node* head;
    Node* tail;
    size_t element_size;
    size_t size;
} LinkedList;

LinkedList* list_create(size_t element_size) {
    LinkedList* list = malloc(sizeof(LinkedList));
    if (!list) return NULL;

    list->head = NULL;
    list->tail = NULL;
    list->element_size = element_size;
    list->size = 0;

    return list;
}

// Append to end
int list_append(LinkedList* list, const void* data) {
    Node* node = malloc(sizeof(Node));
    if (!node) return -1;

    // Allocate and copy data
    node->data = malloc(list->element_size);
    if (!node->data) {
        free(node);
        return -1;
    }

    memcpy(node->data, data, list->element_size);
    node->next = NULL;

    // Add to list
    if (list->tail) {
        list->tail->next = node;
        list->tail = node;
    } else {
        list->head = list->tail = node;
    }

    list->size++;
    return 0;
}

// Prepend to front
int list_prepend(LinkedList* list, const void* data) {
    Node* node = malloc(sizeof(Node));
    if (!node) return -1;

    node->data = malloc(list->element_size);
    if (!node->data) {
        free(node);
        return -1;
    }

    memcpy(node->data, data, list->element_size);
    node->next = list->head;

    list->head = node;
    if (!list->tail) {
        list->tail = node;
    }

    list->size++;
    return 0;
}

// Find element using custom comparator
Node* list_find(LinkedList* list, const void* key,
                CompareFn compare) {
    Node* current = list->head;

    while (current) {
        if (compare(current->data, key) == 0) {
            return current;
        }
        current = current->next;
    }

    return NULL;
}

// Remove element
int list_remove(LinkedList* list, const void* key,
                CompareFn compare) {
    Node* prev = NULL;
    Node* current = list->head;

    while (current) {
        if (compare(current->data, key) == 0) {
            // Found it - remove
            if (prev) {
                prev->next = current->next;
            } else {
                list->head = current->next;
            }

            if (current == list->tail) {
                list->tail = prev;
            }

            free(current->data);
            free(current);
            list->size--;
            return 0;
        }

        prev = current;
        current = current->next;
    }

    return -1;  // Not found
}

// Destroy entire list
void list_destroy(LinkedList* list) {
    if (!list) return;

    Node* current = list->head;
    while (current) {
        Node* next = current->next;
        free(current->data);
        free(current);
        current = next;
    }

    free(list);
}

// Iterate over list
void list_foreach(LinkedList* list, void (*callback)(void* data)) {
    Node* current = list->head;

    while (current) {
        callback(current->data);
        current = current->next;
    }
}
\end{lstlisting}

\textbf{Usage}:

\begin{lstlisting}
// List of integers
LinkedList* int_list = list_create(sizeof(int));

int values[] = {10, 20, 30, 40, 50};
for (int i = 0; i < 5; i++) {
    list_append(int_list, &values[i]);
}

// Find an element
int search = 30;
Node* found = list_find(int_list, &search, compare_int);
if (found) {
    printf("Found: %d\n", *(int*)found->data);
}

// Iterate
void print_int(void* data) {
    printf("%d ", *(int*)data);
}
list_foreach(int_list, print_int);

list_destroy(int_list);
\end{lstlisting}

\section{Generic Hash Table}

The workhorse of generic programming---hash tables store key-value pairs of any type.

\begin{lstlisting}
#define HASH_TABLE_INITIAL_SIZE 16
#define HASH_TABLE_LOAD_FACTOR 0.75

typedef struct HashEntry {
    void* key;
    void* value;
    struct HashEntry* next;  // Chaining for collisions
} HashEntry;

typedef struct {
    HashEntry** buckets;
    size_t bucket_count;
    size_t size;
    size_t key_size;
    size_t value_size;

    // Function pointers for key operations
    unsigned int (*hash)(const void* key);
    int (*compare)(const void* a, const void* b);
} HashTable;

// Create hash table
HashTable* ht_create(size_t key_size, size_t value_size,
                     unsigned int (*hash)(const void*),
                     int (*compare)(const void*, const void*)) {
    HashTable* ht = malloc(sizeof(HashTable));
    if (!ht) return NULL;

    ht->bucket_count = HASH_TABLE_INITIAL_SIZE;
    ht->buckets = calloc(ht->bucket_count, sizeof(HashEntry*));
    if (!ht->buckets) {
        free(ht);
        return NULL;
    }

    ht->size = 0;
    ht->key_size = key_size;
    ht->value_size = value_size;
    ht->hash = hash;
    ht->compare = compare;

    return ht;
}

// Insert or update
int ht_set(HashTable* ht, const void* key, const void* value) {
    unsigned int hash_val = ht->hash(key);
    size_t index = hash_val % ht->bucket_count;

    // Check if key exists
    HashEntry* entry = ht->buckets[index];
    while (entry) {
        if (ht->compare(entry->key, key) == 0) {
            // Update existing
            memcpy(entry->value, value, ht->value_size);
            return 0;
        }
        entry = entry->next;
    }

    // Create new entry
    entry = malloc(sizeof(HashEntry));
    if (!entry) return -1;

    entry->key = malloc(ht->key_size);
    entry->value = malloc(ht->value_size);

    if (!entry->key || !entry->value) {
        free(entry->key);
        free(entry->value);
        free(entry);
        return -1;
    }

    memcpy(entry->key, key, ht->key_size);
    memcpy(entry->value, value, ht->value_size);

    // Insert at head of chain
    entry->next = ht->buckets[index];
    ht->buckets[index] = entry;
    ht->size++;

    return 0;
}

// Get value by key
void* ht_get(HashTable* ht, const void* key) {
    unsigned int hash_val = ht->hash(key);
    size_t index = hash_val % ht->bucket_count;

    HashEntry* entry = ht->buckets[index];
    while (entry) {
        if (ht->compare(entry->key, key) == 0) {
            return entry->value;
        }
        entry = entry->next;
    }

    return NULL;  // Not found
}

// Remove entry
int ht_remove(HashTable* ht, const void* key) {
    unsigned int hash_val = ht->hash(key);
    size_t index = hash_val % ht->bucket_count;

    HashEntry* prev = NULL;
    HashEntry* entry = ht->buckets[index];

    while (entry) {
        if (ht->compare(entry->key, key) == 0) {
            if (prev) {
                prev->next = entry->next;
            } else {
                ht->buckets[index] = entry->next;
            }

            free(entry->key);
            free(entry->value);
            free(entry);
            ht->size--;
            return 0;
        }

        prev = entry;
        entry = entry->next;
    }

    return -1;  // Not found
}

void ht_destroy(HashTable* ht) {
    if (!ht) return;

    for (size_t i = 0; i < ht->bucket_count; i++) {
        HashEntry* entry = ht->buckets[i];
        while (entry) {
            HashEntry* next = entry->next;
            free(entry->key);
            free(entry->value);
            free(entry);
            entry = next;
        }
    }

    free(ht->buckets);
    free(ht);
}
\end{lstlisting}

\textbf{Hash functions for common types}:

\begin{lstlisting}
// Integer hash
unsigned int hash_int(const void* key) {
    int k = *(const int*)key;
    // Knuth's multiplicative hash
    return (unsigned int)k * 2654435761u;
}

// String hash (djb2 algorithm)
unsigned int hash_string(const void* key) {
    const char* str = *(const char**)key;
    unsigned int hash = 5381;
    int c;

    while ((c = *str++)) {
        hash = ((hash << 5) + hash) + c;  // hash * 33 + c
    }

    return hash;
}

// Generic hash (hash any bytes)
unsigned int hash_bytes(const void* data, size_t len) {
    const unsigned char* bytes = data;
    unsigned int hash = 0;

    for (size_t i = 0; i < len; i++) {
        hash = hash * 31 + bytes[i];
    }

    return hash;
}
\end{lstlisting}

\textbf{Usage}:

\begin{lstlisting}
// String -> Integer mapping
HashTable* word_count = ht_create(
    sizeof(char*),
    sizeof(int),
    hash_string,
    compare_string
);

// Count word frequencies
const char* words[] = {"hello", "world", "hello", "C"};
for (int i = 0; i < 4; i++) {
    int* count = ht_get(word_count, &words[i]);
    if (count) {
        (*count)++;
        ht_set(word_count, &words[i], count);
    } else {
        int one = 1;
        ht_set(word_count, &words[i], &one);
    }
}

// Lookup
const char* query = "hello";
int* result = ht_get(word_count, &query);
if (result) {
    printf("'%s' appears %d times\n", query, *result);
}

ht_destroy(word_count);
\end{lstlisting}

\section{Macro-Based Generic Programming}

Macros can generate type-specific code at compile-time---zero runtime overhead, full type safety.

\subsection{Simple Generic Macros}

\begin{lstlisting}
// Generic min/max with type safety
#define MIN(a, b) ({ \
    __typeof__(a) _a = (a); \
    __typeof__(b) _b = (b); \
    _a < _b ? _a : _b; \
})

#define MAX(a, b) ({ \
    __typeof__(a) _a = (a); \
    __typeof__(b) _b = (b); \
    _a > _b ? _a : _b; \
})

// Usage
int x = MIN(10, 20);          // Type: int
double d = MAX(3.14, 2.71);   // Type: double
\end{lstlisting}

\textbf{Note}: \texttt{\_\_typeof\_\_} is a GCC extension. The compound statement ensures arguments are evaluated once.

\subsection{Container Generation Macros}

Generate type-specific containers at compile-time:

\begin{lstlisting}
// Define a typed dynamic array
#define DEFINE_VECTOR(T) \
    typedef struct { \
        T* data; \
        size_t size; \
        size_t capacity; \
    } T##_vector; \
    \
    static inline T##_vector* T##_vector_create(size_t cap) { \
        T##_vector* vec = malloc(sizeof(T##_vector)); \
        if (!vec) return NULL; \
        vec->data = malloc(sizeof(T) * cap); \
        if (!vec->data) { free(vec); return NULL; } \
        vec->size = 0; \
        vec->capacity = cap; \
        return vec; \
    } \
    \
    static inline int T##_vector_push(T##_vector* vec, T elem) { \
        if (vec->size >= vec->capacity) { \
            size_t new_cap = vec->capacity * 2; \
            T* new_data = realloc(vec->data, sizeof(T) * new_cap); \
            if (!new_data) return -1; \
            vec->data = new_data; \
            vec->capacity = new_cap; \
        } \
        vec->data[vec->size++] = elem; \
        return 0; \
    } \
    \
    static inline T T##_vector_get(T##_vector* vec, size_t idx) { \
        return vec->data[idx]; \
    } \
    \
    static inline void T##_vector_destroy(T##_vector* vec) { \
        if (vec) { free(vec->data); free(vec); } \
    }

// Generate int vector
DEFINE_VECTOR(int)

// Generate double vector
DEFINE_VECTOR(double)

// Generate Point vector
typedef struct { int x, y; } Point;
DEFINE_VECTOR(Point)

// Usage - fully type-safe!
int_vector* iv = int_vector_create(10);
int_vector_push(iv, 42);
int_vector_push(iv, 100);
int value = int_vector_get(iv, 0);  // Returns int, not void*
int_vector_destroy(iv);

Point_vector* pv = Point_vector_create(10);
Point p = {10, 20};
Point_vector_push(pv, p);
Point retrieved = Point_vector_get(pv, 0);  // Type-safe!
Point_vector_destroy(pv);
\end{lstlisting}

\textbf{Advantages}:
\begin{itemize}
    \item Full type safety---compiler catches type errors
    \item No void pointer overhead---direct memory access
    \item Inlined functions---maximum performance
    \item No heap allocation for elements (stored inline)
\end{itemize}

\textbf{Disadvantages}:
\begin{itemize}
    \item Code bloat---separate code for each type
    \item Difficult debugging---macro expansion can be cryptic
    \item Must include in header---increases compile time
\end{itemize}

\section{C11 \_Generic Type Selection}

C11 added \texttt{\_Generic} for compile-time type dispatch---the closest C gets to function overloading.

\begin{lstlisting}
// Generic print function
#define print(x) _Generic((x), \
    int: print_int, \
    double: print_double, \
    char*: print_string, \
    default: print_generic)(x)

void print_int(int x) {
    printf("int: %d\n", x);
}

void print_double(double x) {
    printf("double: %.2f\n", x);
}

void print_string(char* x) {
    printf("string: %s\n", x);
}

void print_generic(void* x) {
    printf("generic: %p\n", x);
}

// Usage - looks like function overloading!
print(42);           // Calls print_int
print(3.14);         // Calls print_double
print("hello");      // Calls print_string
\end{lstlisting}

\subsection{Generic Math Functions}

\begin{lstlisting}
// Generic absolute value
#define abs_value(x) _Generic((x), \
    int: abs, \
    long: labs, \
    long long: llabs, \
    float: fabsf, \
    double: fabs, \
    long double: fabsl)(x)

// Generic square root
#define sqrt_generic(x) _Generic((x), \
    float: sqrtf, \
    double: sqrt, \
    long double: sqrtl, \
    default: sqrt)(x)

// Usage
int i = abs_value(-10);              // Calls abs()
double d = abs_value(-3.14);         // Calls fabs()
float f = sqrt_generic(16.0f);       // Calls sqrtf()
\end{lstlisting}

\subsection{Generic Container Operations}

\begin{lstlisting}
// Generic size macro
#define container_size(c) _Generic((c), \
    Vector*: vector_size, \
    LinkedList*: list_size, \
    HashTable*: ht_size)(c)

size_t vector_size(Vector* v) { return v->size; }
size_t list_size(LinkedList* l) { return l->size; }
size_t ht_size(HashTable* ht) { return ht->size; }

// Usage
Vector* vec = vector_create(sizeof(int), 10);
LinkedList* list = list_create(sizeof(int));

printf("Vector size: %zu\n", container_size(vec));
printf("List size: %zu\n", container_size(list));
\end{lstlisting}

\section{Intrusive Data Structures}

A powerful pattern used in Linux kernel and many high-performance systems---embed list/tree nodes inside your structs instead of storing pointers.

\begin{lstlisting}
// Intrusive list node
typedef struct ListNode {
    struct ListNode* next;
    struct ListNode* prev;
} ListNode;

// Your struct embeds the list node
typedef struct {
    char name[32];
    int age;
    ListNode node;  // Embedded list node
} Person;

// Generic list operations work on ListNode
void list_add(ListNode* head, ListNode* node) {
    node->next = head->next;
    node->prev = head;
    head->next->prev = node;
    head->next = node;
}

void list_remove(ListNode* node) {
    node->prev->next = node->next;
    node->next->prev = node->prev;
}

// Get container struct from node (Linux kernel pattern)
#define container_of(ptr, type, member) \
    ((type*)((char*)(ptr) - offsetof(type, member)))

// Usage
ListNode person_list = {&person_list, &person_list};  // Sentinel

Person alice = {.name = "Alice", .age = 30};
list_add(&person_list, &alice.node);

Person bob = {.name = "Bob", .age = 25};
list_add(&person_list, &bob.node);

// Iterate
ListNode* curr = person_list.next;
while (curr != &person_list) {
    Person* p = container_of(curr, Person, node);
    printf("%s, %d\n", p->name, p->age);
    curr = curr->next;
}
\end{lstlisting}

\textbf{Why intrusive structures?}
\begin{itemize}
    \item No separate allocation for list nodes
    \item Better cache locality---data and links together
    \item One object can be in multiple lists simultaneously
    \item Zero memory overhead beyond the links
\end{itemize}

\textbf{Trade-off}: Less generic (must embed node in struct), but much more efficient.

\section{Function Pointer Tables for Polymorphism}

Implement polymorphism using function pointer tables---the foundation of object-oriented C.

\begin{lstlisting}
// Interface (vtable)
typedef struct {
    void (*draw)(void* self);
    void (*move)(void* self, int dx, int dy);
    void (*destroy)(void* self);
} ShapeVTable;

// Base shape type
typedef struct {
    const ShapeVTable* vtable;
    int x, y;
} Shape;

// Circle implementation
typedef struct {
    Shape base;  // Inherit from Shape
    int radius;
} Circle;

void circle_draw(void* self) {
    Circle* c = (Circle*)self;
    printf("Drawing circle at (%d,%d) radius %d\n",
           c->base.x, c->base.y, c->radius);
}

void circle_move(void* self, int dx, int dy) {
    Circle* c = (Circle*)self;
    c->base.x += dx;
    c->base.y += dy;
}

void circle_destroy(void* self) {
    free(self);
}

static const ShapeVTable circle_vtable = {
    .draw = circle_draw,
    .move = circle_move,
    .destroy = circle_destroy
};

// Rectangle implementation
typedef struct {
    Shape base;
    int width, height;
} Rectangle;

void rectangle_draw(void* self) {
    Rectangle* r = (Rectangle*)self;
    printf("Drawing rectangle at (%d,%d) size %dx%d\n",
           r->base.x, r->base.y, r->width, r->height);
}

void rectangle_move(void* self, int dx, int dy) {
    Rectangle* r = (Rectangle*)self;
    r->base.x += dx;
    r->base.y += dy;
}

void rectangle_destroy(void* self) {
    free(self);
}

static const ShapeVTable rectangle_vtable = {
    .draw = rectangle_draw,
    .move = rectangle_move,
    .destroy = rectangle_destroy
};

// Constructors
Circle* circle_create(int x, int y, int radius) {
    Circle* c = malloc(sizeof(Circle));
    if (!c) return NULL;

    c->base.vtable = &circle_vtable;
    c->base.x = x;
    c->base.y = y;
    c->radius = radius;

    return c;
}

Rectangle* rectangle_create(int x, int y, int w, int h) {
    Rectangle* r = malloc(sizeof(Rectangle));
    if (!r) return NULL;

    r->base.vtable = &rectangle_vtable;
    r->base.x = x;
    r->base.y = y;
    r->width = w;
    r->height = h;

    return r;
}

// Generic shape operations (polymorphic)
void shape_draw(Shape* shape) {
    shape->vtable->draw(shape);
}

void shape_move(Shape* shape, int dx, int dy) {
    shape->vtable->move(shape, dx, dy);
}

void shape_destroy(Shape* shape) {
    shape->vtable->destroy(shape);
}

// Usage - polymorphic behavior
Shape* shapes[10];
shapes[0] = (Shape*)circle_create(0, 0, 10);
shapes[1] = (Shape*)rectangle_create(5, 5, 20, 30);
shapes[2] = (Shape*)circle_create(10, 10, 5);

for (int i = 0; i < 3; i++) {
    shape_draw(shapes[i]);    // Calls correct draw function
    shape_move(shapes[i], 1, 1);
    shape_destroy(shapes[i]);
}
\end{lstlisting}

This is how GLib, GTK+, and many other C libraries implement object-oriented programming.

\section{Iterator Pattern}

Generic iteration over any container type.

\begin{lstlisting}
// Generic iterator interface
typedef struct {
    void* container;
    void* current;

    void* (*next)(void* iterator);
    int (*has_next)(void* iterator);
    void* (*get)(void* iterator);
} Iterator;

// Vector iterator implementation
typedef struct {
    Vector* vec;
    size_t index;
} VectorIterator;

void* vector_iter_next(void* it) {
    VectorIterator* iter = (VectorIterator*)it;
    if (iter->index < iter->vec->size) {
        iter->index++;
    }
    return it;
}

int vector_iter_has_next(void* it) {
    VectorIterator* iter = (VectorIterator*)it;
    return iter->index < iter->vec->size;
}

void* vector_iter_get(void* it) {
    VectorIterator* iter = (VectorIterator*)it;
    return vector_get(iter->vec, iter->index);
}

Iterator vector_iterator(Vector* vec) {
    VectorIterator* viter = malloc(sizeof(VectorIterator));
    viter->vec = vec;
    viter->index = 0;

    Iterator iter = {
        .container = vec,
        .current = viter,
        .next = vector_iter_next,
        .has_next = vector_iter_has_next,
        .get = vector_iter_get
    };

    return iter;
}

// Usage - works with any container
void print_all(Iterator iter) {
    while (iter.has_next(iter.current)) {
        void* elem = iter.get(iter.current);
        // Process element
        iter.next(iter.current);
    }
}
\end{lstlisting}

\section{Real-World Generic Patterns}

\subsection{Plugin System}

Load and use plugins at runtime:

\begin{lstlisting}
typedef struct {
    const char* name;
    const char* version;

    int (*init)(void);
    void (*shutdown)(void);
    void (*process)(void* data);
} Plugin;

// Plugin registry
#define MAX_PLUGINS 32
static Plugin* plugins[MAX_PLUGINS];
static int plugin_count = 0;

int register_plugin(Plugin* plugin) {
    if (plugin_count >= MAX_PLUGINS) return -1;

    printf("Registering plugin: %s v%s\n",
           plugin->name, plugin->version);

    if (plugin->init && plugin->init() < 0) {
        return -1;
    }

    plugins[plugin_count++] = plugin;
    return 0;
}

void process_all_plugins(void* data) {
    for (int i = 0; i < plugin_count; i++) {
        if (plugins[i]->process) {
            plugins[i]->process(data);
        }
    }
}

// Example plugin
int json_plugin_init(void) {
    printf("JSON plugin initialized\n");
    return 0;
}

void json_plugin_process(void* data) {
    printf("Processing JSON data\n");
}

Plugin json_plugin = {
    .name = "JSON Parser",
    .version = "1.0",
    .init = json_plugin_init,
    .process = json_plugin_process
};

// Usage
register_plugin(&json_plugin);
\end{lstlisting}

\subsection{Allocator Interface}

Generic memory allocator pattern:

\begin{lstlisting}
typedef struct {
    void* (*alloc)(void* ctx, size_t size);
    void (*free)(void* ctx, void* ptr);
    void* context;
} Allocator;

// Default system allocator
void* system_alloc(void* ctx, size_t size) {
    (void)ctx;
    return malloc(size);
}

void system_free(void* ctx, void* ptr) {
    (void)ctx;
    free(ptr);
}

Allocator system_allocator = {
    .alloc = system_alloc,
    .free = system_free,
    .context = NULL
};

// Arena allocator
typedef struct {
    char* buffer;
    size_t size;
    size_t used;
} Arena;

void* arena_alloc(void* ctx, size_t size) {
    Arena* arena = (Arena*)ctx;

    if (arena->used + size > arena->size) {
        return NULL;  // Out of memory
    }

    void* ptr = arena->buffer + arena->used;
    arena->used += size;
    return ptr;
}

void arena_free(void* ctx, void* ptr) {
    // Arena doesn't free individual allocations
    (void)ctx;
    (void)ptr;
}

Allocator create_arena_allocator(void* buffer, size_t size) {
    Arena* arena = (Arena*)buffer;
    arena->buffer = (char*)buffer + sizeof(Arena);
    arena->size = size - sizeof(Arena);
    arena->used = 0;

    Allocator alloc = {
        .alloc = arena_alloc,
        .free = arena_free,
        .context = arena
    };

    return alloc;
}

// Use any allocator
void* generic_alloc(Allocator* alloc, size_t size) {
    return alloc->alloc(alloc->context, size);
}

void generic_free(Allocator* alloc, void* ptr) {
    alloc->free(alloc->context, ptr);
}

// Data structures can use any allocator
Vector* vector_create_with_allocator(size_t elem_size,
                                     Allocator* alloc) {
    Vector* vec = generic_alloc(alloc, sizeof(Vector));
    // ... initialize with custom allocator
    return vec;
}
\end{lstlisting}

\section{Performance Considerations}

\subsection{Void Pointer Overhead}

\begin{lstlisting}
// Void pointer version - indirection
void* get_element(Vector* v, size_t index) {
    return (char*)v->data + (index * v->element_size);
}
int x = *(int*)get_element(vec, i);  // Load, cast, dereference

// Typed version - direct access
int* data = (int*)vec->data;
int x = data[i];  // Single array access

// The difference:
// Void pointer: 3-5 instructions
// Typed: 1 instruction
\end{lstlisting}

\textbf{When void pointers matter}: In tight loops processing millions of elements, the overhead adds up. Use macros or code generation for hot paths.

\subsection{Function Pointer Overhead}

\begin{lstlisting}
// Function pointer call
void (*func)(void*) = get_function();
func(data);  // Indirect call - can't be inlined

// Direct call
process_data(data);  // Can be inlined

// Benchmark: function pointers are ~2-5x slower
\end{lstlisting}

\textbf{Optimization}: Use function pointers for configuration and rare paths. Use direct calls for hot paths.

\section{Summary}

Generic programming in C requires understanding multiple techniques:

\begin{itemize}
    \item \textbf{Void pointers}: Runtime genericity, loss of type safety
    \item \textbf{Function pointers}: Generic behavior, callback patterns
    \item \textbf{Macros}: Compile-time code generation, full type safety
    \item \textbf{\_Generic (C11)}: Type-based dispatch, pseudo-overloading
    \item \textbf{Intrusive structures}: Maximum performance, less generic
    \item \textbf{VTables}: Polymorphism and OOP in C
\end{itemize}

\textbf{Choose based on needs}:
\begin{itemize}
    \item Need runtime flexibility? -> Void pointers
    \item Need maximum performance? -> Macros or intrusive structures
    \item Need type safety? -> Macros or \_Generic
    \item Need polymorphism? -> Function pointer tables
\end{itemize}

Real C code uses all these techniques. Libraries like GLib, SQLite, and the Linux kernel demonstrate that C can handle complex generic programming when you master these patterns!

\chapter{Linked Structures}

\section{Beyond Basic Linked Lists}

Most textbooks stop at singly-linked lists with toy examples. Real code uses doubly-linked lists, circular lists, skip lists, and sophisticated pointer manipulation. This chapter covers the linked structures you'll actually encounter in production systems—from the Linux kernel's intrusive lists to Redis's skip lists.

\textbf{The linked structure family tree}:

Linked structures are all about managing relationships between data through pointers. Unlike arrays where elements sit contiguously in memory, linked structures scatter data across the heap and connect the pieces with pointers. This fundamental difference drives everything: the performance characteristics, the memory patterns, the algorithms, and the bugs you'll encounter.

\textbf{Why linked structures?}

\begin{itemize}
    \item \textbf{Dynamic size}: Grow and shrink without reallocation
    \item \textbf{Constant-time insertion/deletion}: O(1) at known positions
    \item \textbf{No contiguous memory}: Work with fragmented memory
    \item \textbf{Natural recursion}: Many algorithms are naturally recursive
\end{itemize}

\textbf{Trade-offs}:

\begin{itemize}
    \item Poor cache locality—each node is a separate allocation
    \item Memory overhead—pointers take space
    \item No random access—must traverse from head
    \item More complex memory management
\end{itemize}

Understanding these trade-offs helps you choose the right data structure. Arrays beat linked lists for most use cases, but when you need dynamic insertion/deletion at arbitrary positions, linked structures shine.

\section{Singly-Linked List Deep Dive}

The simplest linked structure, but with many subtle details.

\subsection{Robust Implementation}

A truly robust linked list implementation needs more than just next pointers. You need proper memory management, error handling, and utility functions. Let's build a production-quality singly-linked list from scratch.

\textbf{Design decisions}:
\begin{itemize}
    \item Store both head and tail for O(1) append (most implementations forget this!)
    \item Track size for O(1) length queries
    \item Accept function pointer for custom data cleanup
    \item Return error codes for allocation failures
\end{itemize}

\begin{lstlisting}
typedef struct Node {
    void* data;
    struct Node* next;
} Node;

typedef struct {
    Node* head;
    Node* tail;  // Keep tail pointer for O(1) append
    size_t size;

    // Function pointers for memory management
    void (*free_data)(void* data);
} LinkedList;

// Create list
LinkedList* list_create(void (*free_fn)(void*)) {
    LinkedList* list = malloc(sizeof(LinkedList));
    if (!list) return NULL;

    list->head = NULL;
    list->tail = NULL;
    list->size = 0;
    list->free_data = free_fn;

    return list;
}

// Prepend - O(1)
int list_prepend(LinkedList* list, void* data) {
    Node* node = malloc(sizeof(Node));
    if (!node) return -1;

    node->data = data;
    node->next = list->head;

    list->head = node;

    // Update tail if this is first element
    if (!list->tail) {
        list->tail = node;
    }

    list->size++;
    return 0;
}

// Append - O(1) with tail pointer
int list_append(LinkedList* list, void* data) {
    Node* node = malloc(sizeof(Node));
    if (!node) return -1;

    node->data = data;
    node->next = NULL;

    if (list->tail) {
        list->tail->next = node;
        list->tail = node;
    } else {
        // Empty list
        list->head = list->tail = node;
    }

    list->size++;
    return 0;
}

// Remove first occurrence
int list_remove(LinkedList* list, const void* data,
                int (*compare)(const void*, const void*)) {
    Node* prev = NULL;
    Node* curr = list->head;

    while (curr) {
        if (compare(curr->data, data) == 0) {
            // Found it - unlink
            if (prev) {
                prev->next = curr->next;
            } else {
                list->head = curr->next;
            }

            // Update tail if we removed last element
            if (curr == list->tail) {
                list->tail = prev;
            }

            if (list->free_data) {
                list->free_data(curr->data);
            }
            free(curr);
            list->size--;
            return 0;
        }

        prev = curr;
        curr = curr->next;
    }

    return -1;  // Not found
}

// Reverse the list - O(n)
void list_reverse(LinkedList* list) {
    Node* prev = NULL;
    Node* curr = list->head;
    Node* next = NULL;

    // Swap head and tail
    list->tail = list->head;

    while (curr) {
        next = curr->next;
        curr->next = prev;
        prev = curr;
        curr = next;
    }

    list->head = prev;
}

// Find middle element (tortoise and hare algorithm)
Node* list_find_middle(LinkedList* list) {
    if (!list->head) return NULL;

    Node* slow = list->head;
    Node* fast = list->head;

    while (fast->next && fast->next->next) {
        slow = slow->next;
        fast = fast->next->next;
    }

    return slow;
}

// Detect cycle (Floyd's algorithm)
int list_has_cycle(LinkedList* list) {
    if (!list->head) return 0;

    Node* slow = list->head;
    Node* fast = list->head;

    while (fast && fast->next) {
        slow = slow->next;
        fast = fast->next->next;

        if (slow == fast) {
            return 1;  // Cycle detected
        }
    }

    return 0;
}

// Destroy list
void list_destroy(LinkedList* list) {
    if (!list) return;

    Node* curr = list->head;
    while (curr) {
        Node* next = curr->next;
        if (list->free_data) {
            list->free_data(curr->data);
        }
        free(curr);
        curr = next;
    }

    free(list);
}
\end{lstlisting}

\begin{tipbox}
Always maintain a tail pointer for O(1) append operations. Without it, appending becomes O(n) because you must traverse the entire list. This single optimization makes linked lists practical for many more use cases.
\end{tipbox}

\section{Doubly-Linked Lists}

Bidirectional traversal and O(1) deletion at any node—the workhorse of many systems.

\textbf{Why doubly-linked lists dominate in practice}:

Singly-linked lists have a fatal flaw: to delete a node, you need the previous node. This means deletion is O(n) unless you already have the previous pointer. Doubly-linked lists solve this by storing both next and prev pointers, enabling O(1) deletion anywhere.

The extra pointer costs memory (8 bytes per node on 64-bit systems), but the algorithmic improvement is worth it. That's why most real systems (databases, operating systems, GUI frameworks) use doubly-linked lists.

\subsection{Linux Kernel Style Doubly-Linked List}

The Linux kernel uses an elegant intrusive list pattern—arguably the most clever linked list design ever created. Instead of the list containing pointers to your data, your data structures embed the list nodes directly. This inverts the normal relationship and provides massive benefits.

\textbf{Why this pattern is revolutionary}:
\begin{itemize}
    \item No separate allocation for nodes—eliminates allocation overhead
    \item One struct can be in multiple lists simultaneously
    \item Type-agnostic operations—same code works for all types
    \item Cache-friendly—data and links stored together in memory
    \item Zero memory overhead beyond the link pointers themselves
\end{itemize}

This pattern appears in Linux, FreeBSD, GLib, and countless other production systems. Master it.

\begin{lstlisting}
// The list node - embedded in your structs
typedef struct list_head {
    struct list_head* next;
    struct list_head* prev;
} list_head;

// Initialize a list head (circular sentinel)
#define LIST_HEAD_INIT(name) { &(name), &(name) }
#define LIST_HEAD(name) \
    list_head name = LIST_HEAD_INIT(name)

static inline void INIT_LIST_HEAD(list_head* list) {
    list->next = list;
    list->prev = list;
}

// Insert between prev and next
static inline void __list_add(list_head* new_node,
                              list_head* prev,
                              list_head* next) {
    next->prev = new_node;
    new_node->next = next;
    new_node->prev = prev;
    prev->next = new_node;
}

// Add to front of list
static inline void list_add(list_head* new_node, list_head* head) {
    __list_add(new_node, head, head->next);
}

// Add to end of list
static inline void list_add_tail(list_head* new_node, list_head* head) {
    __list_add(new_node, head->prev, head);
}

// Delete a node
static inline void __list_del(list_head* prev, list_head* next) {
    next->prev = prev;
    prev->next = next;
}

static inline void list_del(list_head* entry) {
    __list_del(entry->prev, entry->next);
    entry->next = NULL;
    entry->prev = NULL;
}

// Check if list is empty
static inline int list_empty(const list_head* head) {
    return head->next == head;
}

// Get container struct from list_head pointer
#define list_entry(ptr, type, member) \
    container_of(ptr, type, member)

// Iterate over list
#define list_for_each(pos, head) \
    for (pos = (head)->next; pos != (head); pos = pos->next)

// Iterate over list safely (allows deletion during iteration)
#define list_for_each_safe(pos, n, head) \
    for (pos = (head)->next, n = pos->next; pos != (head); \
         pos = n, n = pos->next)

// Iterate over entries (structs)
#define list_for_each_entry(pos, head, member) \
    for (pos = list_entry((head)->next, typeof(*pos), member); \
         &pos->member != (head); \
         pos = list_entry(pos->member.next, typeof(*pos), member))
\end{lstlisting}

\textbf{Usage example}:

\begin{lstlisting}
// Your data structure embeds list_head
typedef struct {
    int pid;
    char name[32];
    int priority;
    list_head list;  // Embedded list node
} Task;

// Create list head
LIST_HEAD(task_list);

// Add tasks
Task* task1 = malloc(sizeof(Task));
task1->pid = 1;
strcpy(task1->name, "init");
task1->priority = 10;
INIT_LIST_HEAD(&task1->list);
list_add(&task1->list, &task_list);

Task* task2 = malloc(sizeof(Task));
task2->pid = 2;
strcpy(task2->name, "worker");
task2->priority = 5;
INIT_LIST_HEAD(&task2->list);
list_add_tail(&task2->list, &task_list);

// Iterate over tasks
list_head* pos;
list_for_each(pos, &task_list) {
    Task* t = list_entry(pos, Task, list);
    printf("Task: %d %s (priority %d)\n",
           t->pid, t->name, t->priority);
}

// Safe deletion during iteration
list_head* pos_safe;
list_head* n;
list_for_each_safe(pos_safe, n, &task_list) {
    Task* t = list_entry(pos_safe, Task, list);
    if (t->priority < 5) {
        list_del(&t->list);
        free(t);
    }
}
\end{lstlisting}

\textbf{Why this pattern is brilliant}:

\begin{itemize}
    \item No separate node allocation—nodes embedded in your structs
    \item One struct can be in multiple lists (embed multiple list\_head members)
    \item Type-agnostic list operations—same code works for all types
    \item Cache-friendly—data and links stored together
    \item O(1) deletion when you have the node pointer
\end{itemize}

\section{Circular Linked Lists}

Lists where the last node points back to the first—useful for round-robin scheduling and ring buffers.

\textbf{What makes circular lists special}:

In a circular list, there's no "end"—you can keep traversing forever. This property is perfect for modeling cyclic processes: round-robin schedulers, circular buffers, token passing protocols, and game turn systems.

The key insight: you don't need separate head and tail pointers. Just maintain a "current" pointer and you can access the entire circle. To traverse the whole list, just stop when you return to your starting point.

\textbf{Practical applications}:
\begin{itemize}
    \item \textbf{Round-robin scheduler}: Each process gets CPU time, then move to next
    \item \textbf{Circular buffer}: Efficient FIFO with wrap-around
    \item \textbf{Music playlist}: Keep looping through songs
    \item \textbf{Network token ring}: Pass token around the network
    \item \textbf{Josephus problem}: Classic algorithmic puzzle
\end{itemize}

\begin{lstlisting}
typedef struct Node {
    void* data;
    struct Node* next;
} Node;

typedef struct {
    Node* current;  // Current position in circle
    size_t size;
} CircularList;

// Create circular list
CircularList* clist_create(void) {
    CircularList* list = malloc(sizeof(CircularList));
    if (!list) return NULL;

    list->current = NULL;
    list->size = 0;
    return list;
}

// Insert after current
int clist_insert(CircularList* list, void* data) {
    Node* node = malloc(sizeof(Node));
    if (!node) return -1;

    node->data = data;

    if (!list->current) {
        // First element - points to itself
        node->next = node;
        list->current = node;
    } else {
        // Insert after current
        node->next = list->current->next;
        list->current->next = node;
    }

    list->size++;
    return 0;
}

// Advance to next element (round-robin)
void* clist_next(CircularList* list) {
    if (!list->current) return NULL;

    list->current = list->current->next;
    return list->current->data;
}

// Remove current element
int clist_remove_current(CircularList* list) {
    if (!list->current) return -1;

    if (list->size == 1) {
        // Last element
        free(list->current);
        list->current = NULL;
    } else {
        // Find previous node
        Node* prev = list->current;
        while (prev->next != list->current) {
            prev = prev->next;
        }

        // Remove current
        prev->next = list->current->next;
        Node* to_free = list->current;
        list->current = list->current->next;
        free(to_free);
    }

    list->size--;
    return 0;
}

// Josephus problem solver using circular list
int josephus(int n, int k) {
    CircularList* list = clist_create();

    // Add n people
    for (int i = 0; i < n; i++) {
        int* person = malloc(sizeof(int));
        *person = i + 1;
        clist_insert(list, person);
    }

    // Eliminate every kth person
    while (list->size > 1) {
        for (int i = 0; i < k - 1; i++) {
            clist_next(list);
        }

        int* eliminated = (int*)list->current->data;
        printf("Eliminated: %d\n", *eliminated);
        free(eliminated);
        clist_remove_current(list);
    }

    // Return survivor
    int survivor = *(int*)list->current->data;
    return survivor;
}
\end{lstlisting}

\textbf{Real-world use}: Round-robin schedulers, circular buffers, Josephus problem, token ring networks.

\section{Skip Lists}

Probabilistic data structure providing O(log n) search, insert, and delete—simpler than balanced trees.

\textbf{The genius of skip lists}:

Balanced trees (AVL, Red-Black) guarantee O(log n) operations but require complex rotation algorithms. Skip lists achieve the same performance with a brilliantly simple idea: maintain multiple "express lanes" that skip over elements.

Imagine a linked list where:
\begin{itemize}
    \item Level 0: Every element (the full list)
    \item Level 1: Every other element (skip 1)
    \item Level 2: Every fourth element (skip 3)
    \item Level 3: Every eighth element (skip 7)
\end{itemize}

To search, start at the highest level and move forward until you overshoot, then drop down a level. This gives you binary search performance on a linked structure!

\textbf{How randomness helps}:

Rather than rigidly maintaining perfect skip patterns (which would be complex), we use randomness. When inserting a node, flip a coin to decide how many levels it participates in. On average, this creates the express lane structure we want.

\textbf{Why skip lists are popular}:
\begin{itemize}
    \item Much simpler than balanced tree algorithms
    \item Lock-free implementations are possible (huge for concurrent systems)
    \item Used in Redis (sorted sets), LevelDB, and many databases
    \item Expected O(log n) operations with low constant factors
    \item Easy to understand and debug
\end{itemize}

\begin{lstlisting}
#define MAX_LEVEL 16
#define SKIP_LIST_P 0.5

typedef struct SkipNode {
    int key;
    void* value;
    struct SkipNode* forward[MAX_LEVEL];
} SkipNode;

typedef struct {
    int level;
    SkipNode* header;
} SkipList;

// Random level generator
static int random_level(void) {
    int level = 1;
    while ((rand() / (double)RAND_MAX) < SKIP_LIST_P &&
           level < MAX_LEVEL) {
        level++;
    }
    return level;
}

// Create skip list
SkipList* skiplist_create(void) {
    SkipList* list = malloc(sizeof(SkipList));
    if (!list) return NULL;

    list->level = 1;
    list->header = malloc(sizeof(SkipNode));
    if (!list->header) {
        free(list);
        return NULL;
    }

    for (int i = 0; i < MAX_LEVEL; i++) {
        list->header->forward[i] = NULL;
    }

    return list;
}

// Search
void* skiplist_search(SkipList* list, int key) {
    SkipNode* curr = list->header;

    // Start from top level, move down
    for (int i = list->level - 1; i >= 0; i--) {
        while (curr->forward[i] && curr->forward[i]->key < key) {
            curr = curr->forward[i];
        }
    }

    // Move to next node at level 0
    curr = curr->forward[0];

    if (curr && curr->key == key) {
        return curr->value;
    }

    return NULL;
}

// Insert
int skiplist_insert(SkipList* list, int key, void* value) {
    SkipNode* update[MAX_LEVEL];
    SkipNode* curr = list->header;

    // Find insert position at each level
    for (int i = list->level - 1; i >= 0; i--) {
        while (curr->forward[i] && curr->forward[i]->key < key) {
            curr = curr->forward[i];
        }
        update[i] = curr;
    }

    curr = curr->forward[0];

    // Key already exists - update value
    if (curr && curr->key == key) {
        curr->value = value;
        return 0;
    }

    // Create new node with random level
    int new_level = random_level();

    // Update list level if necessary
    if (new_level > list->level) {
        for (int i = list->level; i < new_level; i++) {
            update[i] = list->header;
        }
        list->level = new_level;
    }

    // Create and insert node
    SkipNode* node = malloc(sizeof(SkipNode));
    if (!node) return -1;

    node->key = key;
    node->value = value;

    for (int i = 0; i < new_level; i++) {
        node->forward[i] = update[i]->forward[i];
        update[i]->forward[i] = node;
    }

    return 0;
}

// Delete
int skiplist_delete(SkipList* list, int key) {
    SkipNode* update[MAX_LEVEL];
    SkipNode* curr = list->header;

    // Find node at each level
    for (int i = list->level - 1; i >= 0; i--) {
        while (curr->forward[i] && curr->forward[i]->key < key) {
            curr = curr->forward[i];
        }
        update[i] = curr;
    }

    curr = curr->forward[0];

    if (!curr || curr->key != key) {
        return -1;  // Not found
    }

    // Remove node from all levels
    for (int i = 0; i < list->level; i++) {
        if (update[i]->forward[i] != curr) {
            break;
        }
        update[i]->forward[i] = curr->forward[i];
    }

    free(curr);

    // Update list level
    while (list->level > 1 && !list->header->forward[list->level - 1]) {
        list->level--;
    }

    return 0;
}
\end{lstlisting}

\textbf{Why skip lists?}

\begin{itemize}
    \item Simpler than balanced trees (AVL, Red-Black)
    \item O(log n) operations on average
    \item Lock-free implementations possible (great for concurrent systems)
    \item Easy to implement and understand
    \item Used in Redis, LevelDB, and many databases
\end{itemize}

\section{Memory Pool for Linked Structures}

Allocating one node at a time is slow. Memory pools batch allocations for massive speedup.

\textbf{The malloc problem}:

Every time you insert into a linked list, you call malloc(). Every deletion calls free(). For small objects like list nodes (16-32 bytes), this overhead dominates:
\begin{itemize}
    \item malloc() is slow—must find free block, update metadata, handle alignment
    \item Memory fragmentation—lots of small allocations fragment the heap
    \item Cache misses—malloc'd memory scattered across address space
    \item Allocator contention—in multi-threaded code, malloc() uses locks
\end{itemize}

\textbf{Memory pool solution}:

Allocate a large block once, then carve out small pieces as needed. When nodes are freed, return them to the pool instead of calling free(). This is 10-100x faster than malloc/free for small objects.

\textbf{How the pool works}:
\begin{enumerate}
    \item Allocate chunks of N nodes at a time (e.g., 64 nodes)
    \item Maintain a free list of returned nodes
    \item On allocation: return from free list, or carve from current chunk
    \item On free: add node to free list (don't actually free memory)
    \item On pool destruction: free all chunks at once
\end{enumerate}

This is exactly how high-performance allocators like jemalloc work internally. You're building a specialized allocator for one object size.

\begin{lstlisting}
#define POOL_CHUNK_SIZE 64

typedef struct PoolChunk {
    void* memory;
    size_t used;
    struct PoolChunk* next;
} PoolChunk;

typedef struct {
    size_t node_size;
    PoolChunk* chunks;
    void** free_list;  // Stack of freed nodes
} NodePool;

// Create memory pool
NodePool* pool_create(size_t node_size) {
    NodePool* pool = malloc(sizeof(NodePool));
    if (!pool) return NULL;

    pool->node_size = node_size;
    pool->chunks = NULL;
    pool->free_list = NULL;

    return pool;
}

// Allocate new chunk
static PoolChunk* pool_add_chunk(NodePool* pool) {
    PoolChunk* chunk = malloc(sizeof(PoolChunk));
    if (!chunk) return NULL;

    chunk->memory = malloc(pool->node_size * POOL_CHUNK_SIZE);
    if (!chunk->memory) {
        free(chunk);
        return NULL;
    }

    chunk->used = 0;
    chunk->next = pool->chunks;
    pool->chunks = chunk;

    return chunk;
}

// Allocate node from pool
void* pool_alloc(NodePool* pool) {
    // Check free list first
    if (pool->free_list) {
        void* node = pool->free_list;
        pool->free_list = *(void**)node;
        return node;
    }

    // Need new memory
    PoolChunk* chunk = pool->chunks;
    if (!chunk || chunk->used >= POOL_CHUNK_SIZE) {
        chunk = pool_add_chunk(pool);
        if (!chunk) return NULL;
    }

    void* node = (char*)chunk->memory +
                 (chunk->used * pool->node_size);
    chunk->used++;

    return node;
}

// Free node back to pool (add to free list)
void pool_free(NodePool* pool, void* node) {
    *(void**)node = pool->free_list;
    pool->free_list = node;
}

// Destroy entire pool
void pool_destroy(NodePool* pool) {
    if (!pool) return;

    PoolChunk* chunk = pool->chunks;
    while (chunk) {
        PoolChunk* next = chunk->next;
        free(chunk->memory);
        free(chunk);
        chunk = next;
    }

    free(pool);
}

// Fast linked list using pool
typedef struct PoolNode {
    void* data;
    struct PoolNode* next;
} PoolNode;

typedef struct {
    PoolNode* head;
    NodePool* pool;
    size_t size;
} PoolList;

PoolList* poollist_create(void) {
    PoolList* list = malloc(sizeof(PoolList));
    if (!list) return NULL;

    list->pool = pool_create(sizeof(PoolNode));
    if (!list->pool) {
        free(list);
        return NULL;
    }

    list->head = NULL;
    list->size = 0;

    return list;
}

int poollist_prepend(PoolList* list, void* data) {
    PoolNode* node = pool_alloc(list->pool);
    if (!node) return -1;

    node->data = data;
    node->next = list->head;
    list->head = node;
    list->size++;

    return 0;
}

void poollist_remove(PoolList* list, PoolNode* node) {
    // Unlink and return to pool
    pool_free(list->pool, node);
    list->size--;
}
\end{lstlisting}

\textbf{Performance impact}: Pool allocation can be 10-100x faster than malloc/free for small objects. Critical for high-performance linked structures.

\section{XOR Linked List}

Space-efficient doubly-linked list using XOR trick—stores only one pointer per node instead of two.

\textbf{The XOR linked list hack}:

Normal doubly-linked lists store two pointers per node: prev and next. But you only ever use them together: to move forward, you need current and next; to move backward, you need current and prev. What if we stored XOR(prev, next) instead?

\textbf{The math behind it}:
\begin{itemize}
    \item Node stores: both = prev XOR next
    \item To get next: next = prev XOR both (since prev XOR prev XOR next = next)
    \item To get prev: prev = next XOR both (since next XOR prev XOR next = prev)
    \item XOR is its own inverse: A XOR B XOR B = A
\end{itemize}

\textbf{Why it works}:

When traversing forward, you always know the previous node (you just came from there). Use it to extract the next node: next = prev XOR node->both. Similarly for backward traversal.

\textbf{Space savings}:

Save one pointer per node. For a million-node list on 64-bit systems, that's 8MB saved. Sounds great, right?

\begin{lstlisting}
typedef struct XORNode {
    void* data;
    struct XORNode* both;  // XOR of prev and next
} XORNode;

typedef struct {
    XORNode* head;
    XORNode* tail;
    size_t size;
} XORList;

// XOR two pointers
static inline XORNode* xor_ptrs(XORNode* a, XORNode* b) {
    return (XORNode*)((uintptr_t)a ^ (uintptr_t)b);
}

// Create XOR list
XORList* xorlist_create(void) {
    XORList* list = malloc(sizeof(XORList));
    if (!list) return NULL;

    list->head = NULL;
    list->tail = NULL;
    list->size = 0;

    return list;
}

// Add to front
int xorlist_prepend(XORList* list, void* data) {
    XORNode* node = malloc(sizeof(XORNode));
    if (!node) return -1;

    node->data = data;
    node->both = xor_ptrs(NULL, list->head);

    if (list->head) {
        // Update old head's both pointer
        list->head->both = xor_ptrs(node,
                           xor_ptrs(NULL, list->head->both));
    }

    list->head = node;

    if (!list->tail) {
        list->tail = node;
    }

    list->size++;
    return 0;
}

// Add to end
int xorlist_append(XORList* list, void* data) {
    XORNode* node = malloc(sizeof(XORNode));
    if (!node) return -1;

    node->data = data;
    node->both = xor_ptrs(list->tail, NULL);

    if (list->tail) {
        list->tail->both = xor_ptrs(
            xor_ptrs(list->tail->both, NULL),
            node
        );
    }

    list->tail = node;

    if (!list->head) {
        list->head = node;
    }

    list->size++;
    return 0;
}

// Forward traversal
void xorlist_traverse_forward(XORList* list,
                               void (*visit)(void* data)) {
    XORNode* curr = list->head;
    XORNode* prev = NULL;
    XORNode* next;

    while (curr) {
        visit(curr->data);

        // Get next: next = prev XOR curr->both
        next = xor_ptrs(prev, curr->both);

        prev = curr;
        curr = next;
    }
}

// Backward traversal
void xorlist_traverse_backward(XORList* list,
                                void (*visit)(void* data)) {
    XORNode* curr = list->tail;
    XORNode* next = NULL;
    XORNode* prev;

    while (curr) {
        visit(curr->data);

        // Get prev: prev = next XOR curr->both
        prev = xor_ptrs(curr->both, next);

        next = curr;
        curr = prev;
    }
}
\end{lstlisting}

\textbf{Caveat}: XOR lists are clever but rarely used in practice because:

\begin{itemize}
    \item Can't traverse from arbitrary node (need prev or next)
    \item Pointer arithmetic with XOR is non-standard
    \item Not compatible with garbage collectors
    \item Negligible space savings on 64-bit systems (8 bytes per node vs. total memory)
    \item Hard to debug—can't inspect pointers in debugger
    \item Breaks pointer provenance rules in modern C
    \item No real-world performance benefit (the extra pointer is usually cached)
\end{itemize}

\textbf{When to use}: Embedded systems with severe memory constraints, or when you need to impress interviewers! In 30+ years of C programming, I've never seen XOR lists in production code outside of academic papers and interview questions. It's a neat trick, but optimizing pointer count without considering cache effects is premature optimization.

\textbf{The cache reality}:

Modern CPUs fetch entire cache lines (64 bytes). Your node with two pointers (16 bytes) fits in the same cache line as a node with one pointer (8 bytes). You're not saving cache bandwidth, just heap space. And heap space is cheap—developer time debugging pointer bugs is expensive.

\section{Self-Organizing Lists}

Lists that reorganize based on access patterns—improve performance for non-uniform access.

\textbf{The 80/20 rule applied to data structures}:

Most data access follows a power law: 20\% of items receive 80\% of accesses. If your linked list puts frequently-accessed items at the front, searches become much faster on average. Self-organizing lists do this automatically.

\textbf{Three classic heuristics}:

\begin{enumerate}
    \item \textbf{Move-to-Front (MTF)}: When you access an item, move it to the front. Simple and aggressive.
    \item \textbf{Transpose}: When you access an item, swap it with the previous item. Conservative, gradually bubbles up popular items.
    \item \textbf{Count}: Track access frequency, periodically reorder by count. Most accurate but requires storage and maintenance.
\end{enumerate}

\textbf{When self-organizing lists excel}:
\begin{itemize}
    \item Cache implementations (LRU-like behavior)
    \item Symbol tables (frequently-used identifiers accessed often)
    \item Network routing tables (popular routes accessed constantly)
    \item Spell checkers (common words checked often)
    \item Any scenario with locality of reference
\end{itemize}

\textbf{Performance analysis}:

For random access: self-organizing lists perform worse (O(n) with reordering overhead).
For skewed access: self-organizing lists approach O(1) for popular items.
The more skewed your access pattern, the bigger the win.

\begin{lstlisting}
// Move-to-front heuristic
typedef struct MoveToFrontNode {
    void* data;
    struct MoveToFrontNode* next;
    int access_count;
} MTFNode;

typedef struct {
    MTFNode* head;
    int (*compare)(const void*, const void*);
} MoveToFrontList;

// Search and move to front
void* mtf_search(MoveToFrontList* list, const void* key) {
    MTFNode* prev = NULL;
    MTFNode* curr = list->head;

    while (curr) {
        if (list->compare(curr->data, key) == 0) {
            curr->access_count++;

            // Move to front if not already there
            if (prev) {
                prev->next = curr->next;
                curr->next = list->head;
                list->head = curr;
            }

            return curr->data;
        }

        prev = curr;
        curr = curr->next;
    }

    return NULL;
}

// Transpose heuristic - swap with previous
void* transpose_search(MoveToFrontList* list, const void* key) {
    MTFNode* prev_prev = NULL;
    MTFNode* prev = NULL;
    MTFNode* curr = list->head;

    while (curr) {
        if (list->compare(curr->data, key) == 0) {
            curr->access_count++;

            // Swap with previous element
            if (prev) {
                prev->next = curr->next;
                curr->next = prev;

                if (prev_prev) {
                    prev_prev->next = curr;
                } else {
                    list->head = curr;
                }
            }

            return curr->data;
        }

        prev_prev = prev;
        prev = curr;
        curr = curr->next;
    }

    return NULL;
}

// Count heuristic - sort by access frequency
void mtf_reorder_by_frequency(MoveToFrontList* list) {
    // Insertion sort by access_count
    MTFNode sorted_head = {.next = NULL};
    MTFNode* curr = list->head;

    while (curr) {
        MTFNode* next = curr->next;

        // Find position in sorted list
        MTFNode* sorted_prev = &sorted_head;
        while (sorted_prev->next &&
               sorted_prev->next->access_count > curr->access_count) {
            sorted_prev = sorted_prev->next;
        }

        // Insert
        curr->next = sorted_prev->next;
        sorted_prev->next = curr;

        curr = next;
    }

    list->head = sorted_head.next;
}
\end{lstlisting}

\textbf{Real-world use}: Cache implementations, frequency-based optimization, adaptive data structures.

\section{Unrolled Linked List}

Hybrid of array and linked list—multiple elements per node for better cache performance.

\textbf{The best of both worlds}:

Regular linked lists have terrible cache performance—each node is a separate allocation scattered across memory. Every traversal incurs a cache miss. Unrolled linked lists fix this by storing multiple elements per node.

Instead of:
\begin{verbatim}
Node -> [data] -> [data] -> [data] -> [data]
\end{verbatim}

You get:
\begin{verbatim}
Node -> [data, data, data, data] -> [data, data, data, data]
\end{verbatim}

\textbf{Performance implications}:

\begin{itemize}
    \item \textbf{Fewer allocations}: 64 elements needs 64 nodes normally, only 4 with UNROLL\_SIZE=16
    \item \textbf{Better cache utilization}: Sequential elements in same node are cached together
    \item \textbf{Lower memory overhead}: One pointer per 16 elements instead of per element
    \item \textbf{Still dynamic}: Can grow and insert like regular linked list
\end{itemize}

\textbf{The trade-off}:

Unrolled lists are more complex than either arrays or linked lists. Insertion might require shifting elements within a node, or splitting a full node. But for large collections with sequential access patterns, the 2-10x speedup is worth it.

\textbf{Real-world usage}:

Database B-tree implementations use this idea—each tree node contains multiple keys. This reduces tree height and improves cache performance. You're applying the same principle to linked lists.

\begin{lstlisting}
#define UNROLL_SIZE 16

typedef struct UnrolledNode {
    void* data[UNROLL_SIZE];
    int count;  // Number of elements in this node
    struct UnrolledNode* next;
} UnrolledNode;

typedef struct {
    UnrolledNode* head;
    size_t size;
    size_t node_count;
} UnrolledList;

// Create unrolled list
UnrolledList* unrolled_create(void) {
    UnrolledList* list = malloc(sizeof(UnrolledList));
    if (!list) return NULL;

    list->head = NULL;
    list->size = 0;
    list->node_count = 0;

    return list;
}

// Insert element
int unrolled_insert(UnrolledList* list, void* data) {
    // Find node with space or create new one
    UnrolledNode* node = list->head;

    if (!node || node->count >= UNROLL_SIZE) {
        // Need new node
        node = malloc(sizeof(UnrolledNode));
        if (!node) return -1;

        node->count = 0;
        node->next = list->head;
        list->head = node;
        list->node_count++;
    }

    node->data[node->count++] = data;
    list->size++;

    return 0;
}

// Get element by index
void* unrolled_get(UnrolledList* list, size_t index) {
    if (index >= list->size) return NULL;

    UnrolledNode* node = list->head;
    size_t count = 0;

    while (node) {
        if (index < count + node->count) {
            return node->data[index - count];
        }
        count += node->count;
        node = node->next;
    }

    return NULL;
}

// Iterate
void unrolled_foreach(UnrolledList* list, void (*visit)(void*)) {
    UnrolledNode* node = list->head;

    while (node) {
        for (int i = 0; i < node->count; i++) {
            visit(node->data[i]);
        }
        node = node->next;
    }
}
\end{lstlisting}

\textbf{Benefits}:

\begin{itemize}
    \item Better cache locality than regular linked list
    \item Fewer allocations (fewer nodes)
    \item Less memory overhead (fewer pointers per element)
    \item Still dynamic and allows fast insertion
\end{itemize}

\textbf{Trade-off}: More complex than simple linked list, slower than pure arrays for sequential access.

\section{Lock-Free Linked Lists}

Thread-safe linked structures without locks—using atomic operations.

\textbf{The lock-free promise}:

Locks have problems: they block threads, can deadlock, suffer from contention, and kill performance under high concurrency. Lock-free data structures use atomic compare-and-swap operations instead—threads never block, guaranteed progress, no deadlocks.

\textbf{The core technique: Compare-And-Swap (CAS)}:

\begin{verbatim}
bool CAS(pointer* location, old_value, new_value) {
    atomically {
        if (*location == old_value) {
            *location = new_value;
            return true;
        }
        return false;
    }
}
\end{verbatim}

If the value at location hasn't changed since we read it (still equals old\_value), update it to new\_value. If another thread modified it, CAS fails and we retry.

\textbf{Lock-free push algorithm}:
\begin{enumerate}
    \item Read current head
    \item Create new node pointing to current head
    \item CAS: if head unchanged, swap to new node
    \item If CAS failed (another thread modified head), retry
\end{enumerate}

\textbf{Why this is tricky}:

The real challenge isn't insertion—it's memory reclamation. You can't just free() a node after removal because another thread might still be accessing it! Solutions include hazard pointers, epoch-based reclamation, or reference counting. All are complex.

\begin{lstlisting}
#include <stdatomic.h>

typedef struct LockFreeNode {
    void* data;
    _Atomic(struct LockFreeNode*) next;
} LockFreeNode;

typedef struct {
    _Atomic(LockFreeNode*) head;
} LockFreeList;

// Create lock-free list
LockFreeList* lockfree_create(void) {
    LockFreeList* list = malloc(sizeof(LockFreeList));
    if (!list) return NULL;

    atomic_init(&list->head, NULL);
    return list;
}

// Push to front (lock-free)
int lockfree_push(LockFreeList* list, void* data) {
    LockFreeNode* node = malloc(sizeof(LockFreeNode));
    if (!node) return -1;

    node->data = data;

    // Compare-and-swap loop
    LockFreeNode* old_head = atomic_load(&list->head);
    do {
        atomic_store(&node->next, old_head);
    } while (!atomic_compare_exchange_weak(&list->head,
                                           &old_head,
                                           node));

    return 0;
}

// Pop from front (lock-free)
void* lockfree_pop(LockFreeList* list) {
    LockFreeNode* old_head;
    LockFreeNode* new_head;

    do {
        old_head = atomic_load(&list->head);
        if (!old_head) return NULL;

        new_head = atomic_load(&old_head->next);
    } while (!atomic_compare_exchange_weak(&list->head,
                                           &old_head,
                                           new_head));

    void* data = old_head->data;
    // Note: Can't immediately free old_head - ABA problem!
    // Need hazard pointers or epoch-based reclamation

    return data;
}
\end{lstlisting}

\textbf{Challenge}: Memory reclamation is hard in lock-free structures. You can't just free a node—another thread might still be accessing it. Solutions include:

\begin{itemize}
    \item \textbf{Hazard pointers}: Threads announce which pointers they're using
    \item \textbf{Epoch-based reclamation}: Track epochs, free memory from old epochs
    \item \textbf{Reference counting}: Atomic reference count, free at zero
    \item \textbf{Garbage collection}: Let GC handle it (not available in C)
\end{itemize}

\textbf{The ABA problem}:

A classic lock-free bug: Thread 1 reads head=A, gets preempted. Thread 2 removes A, removes B, adds A back. Thread 1 resumes, sees head still equals A, assumes nothing changed, does CAS. But everything changed! A might point to freed memory now.

Solution: tagged pointers or version numbers. Store a counter with the pointer, increment on each change. CAS checks both pointer and counter.

\textbf{When to use lock-free structures}:

Lock-free programming is extremely difficult to get right. Use it only when:
\begin{itemize}
    \item Profiling shows locks are a bottleneck
    \item You understand memory ordering and the memory model
    \item You have comprehensive tests and formal verification
    \item You're willing to debug race conditions
\end{itemize}

For most applications, a simple mutex is faster, simpler, and correct. Lock-free is not inherently faster—it's about avoiding blocking, not raw speed.

\section{Common Linked List Algorithms}

These algorithms appear constantly in interviews and real code. Master them.

\subsection{Reverse a Linked List}

Reversing a linked list is the "Hello World" of linked list algorithms. It tests your understanding of pointer manipulation and is a building block for more complex algorithms.

\begin{lstlisting}
// Iterative reverse
Node* reverse_iterative(Node* head) {
    Node* prev = NULL;
    Node* curr = head;

    while (curr) {
        Node* next = curr->next;
        curr->next = prev;
        prev = curr;
        curr = next;
    }

    return prev;
}

// Recursive reverse
Node* reverse_recursive(Node* head) {
    if (!head || !head->next) {
        return head;
    }

    Node* new_head = reverse_recursive(head->next);
    head->next->next = head;
    head->next = NULL;

    return new_head;
}
\end{lstlisting}

\subsection{Merge Two Sorted Lists}

The merge operation is fundamental to merge sort. Given two sorted lists, produce one sorted list. The trick: use a dummy head node to simplify edge cases.

\begin{lstlisting}
Node* merge_sorted(Node* l1, Node* l2,
                   int (*compare)(const void*, const void*)) {
    Node dummy = {0};
    Node* tail = &dummy;

    while (l1 && l2) {
        if (compare(l1->data, l2->data) <= 0) {
            tail->next = l1;
            l1 = l1->next;
        } else {
            tail->next = l2;
            l2 = l2->next;
        }
        tail = tail->next;
    }

    tail->next = l1 ? l1 : l2;

    return dummy.next;
}
\end{lstlisting}

\subsection{Merge Sort for Linked Lists}

Merge sort is the best sorting algorithm for linked lists. Unlike quicksort (which needs random access) or heapsort (which needs arrays), merge sort only needs sequential access—perfect for linked lists.

\textbf{Why merge sort for linked lists?}
\begin{itemize}
    \item O(n log n) time complexity
    \item O(1) space complexity (in-place, unlike array merge sort)
    \item Stable sort (preserves order of equal elements)
    \item No random access needed
\end{itemize}

\textbf{Algorithm}:
\begin{enumerate}
    \item Find middle using slow/fast pointers
    \item Split list in half
    \item Recursively sort both halves
    \item Merge sorted halves
\end{enumerate}

\begin{lstlisting}
// Find middle using slow/fast pointers
Node* find_middle(Node* head) {
    Node* slow = head;
    Node* fast = head->next;

    while (fast && fast->next) {
        slow = slow->next;
        fast = fast->next->next;
    }

    return slow;
}

// Merge sort
Node* merge_sort(Node* head,
                 int (*compare)(const void*, const void*)) {
    if (!head || !head->next) {
        return head;
    }

    // Split list
    Node* middle = find_middle(head);
    Node* right = middle->next;
    middle->next = NULL;

    // Recursively sort both halves
    Node* left = merge_sort(head, compare);
    right = merge_sort(right, compare);

    // Merge sorted halves
    return merge_sorted(left, right, compare);
}
\end{lstlisting}

\subsection{Remove Duplicates from Sorted List}

A common operation: given a sorted list, remove duplicate values. For sorted lists, duplicates are adjacent, making this a simple single-pass algorithm.

\begin{lstlisting}
void remove_duplicates(Node* head,
                       int (*compare)(const void*, const void*)) {
    Node* curr = head;

    while (curr && curr->next) {
        if (compare(curr->data, curr->next->data) == 0) {
            Node* dup = curr->next;
            curr->next = dup->next;
            free(dup);
        } else {
            curr = curr->next;
        }
    }
}
\end{lstlisting}

\section{Summary}

Linked structures are fundamental to systems programming. Understanding them deeply separates competent C programmers from experts.

\textbf{The key insight}: Linked structures trade memory efficiency and cache performance for flexibility and algorithmic efficiency. Arrays are faster for sequential access and random access. Linked structures win when you need:
\begin{itemize}
    \item Frequent insertions/deletions at arbitrary positions
    \item Dynamic size without reallocation
    \item No contiguous memory requirement
    \item Ability to split/merge collections in O(1)
\end{itemize}

\textbf{Choosing the right linked structure}:

\begin{itemize}
    \item \textbf{Singly-linked}: Simple, forward traversal only
    \item \textbf{Doubly-linked}: Bidirectional, O(1) deletion
    \item \textbf{Circular}: Round-robin, no end
    \item \textbf{Skip lists}: O(log n) operations, simpler than trees
    \item \textbf{Intrusive lists}: Embed nodes in structs (Linux kernel style)
    \item \textbf{Memory pools}: Batch allocation for performance
    \item \textbf{Unrolled lists}: Hybrid array/list for cache locality
    \item \textbf{Self-organizing}: Adapt to access patterns
\end{itemize}

\textbf{Key techniques}:

\begin{itemize}
    \item Always maintain tail pointers for O(1) append
    \item Use sentinel nodes to simplify edge cases
    \item Master pointer manipulation—draw pictures!
    \item Use memory pools for high-performance code
    \item Know common algorithms (reverse, merge, detect cycle)
\end{itemize}

Linked structures sacrifice cache locality and memory efficiency for flexibility. Choose them when you need dynamic size and frequent insertions/deletions at arbitrary positions. For most other cases, arrays are faster!

\chapter{Testing \& Debugging Idioms}

\section{Why Testing Matters in C}

C doesn't have built-in testing frameworks, exception handling, or memory safety. This makes testing absolutely critical. One small bug can corrupt memory, crash your program, or create security vulnerabilities.

\begin{lstlisting}
// A simple bug with devastating consequences
char buffer[10];
strcpy(buffer, user_input);  // Buffer overflow!
// Could overwrite return address, function pointers, etc.
\end{lstlisting}

\section{Simple Unit Test Framework}

You don't need fancy frameworks. Here's a minimal but effective test system:

\begin{lstlisting}
#include <stdio.h>
#include <stdlib.h>

// Global test counters
static int tests_run = 0;
static int tests_passed = 0;
static int tests_failed = 0;

// Test macros
#define TEST(name) \
    static void test_##name(void); \
    static void test_##name##_wrapper(void) { \
        printf("Running %s...", #name); \
        test_##name(); \
        tests_run++; \
        printf(" PASSED\n"); \
        tests_passed++; \
    } \
    static void test_##name(void)

#define RUN_TEST(name) test_##name##_wrapper()

#define ASSERT(condition) do { \
    if (!(condition)) { \
        fprintf(stderr, "\n  FAILED: %s\n", #condition); \
        fprintf(stderr, "  at %s:%d\n", __FILE__, __LINE__); \
        tests_failed++; \
        return; \
    } \
} while(0)

#define ASSERT_EQ(a, b) do { \
    if ((a) != (b)) { \
        fprintf(stderr, "\n  FAILED: %s == %s\n", #a, #b); \
        fprintf(stderr, "  Expected: %d, Got: %d\n", (int)(b), (int)(a)); \
        fprintf(stderr, "  at %s:%d\n", __FILE__, __LINE__); \
        tests_failed++; \
        return; \
    } \
} while(0)

#define ASSERT_STR_EQ(a, b) do { \
    if (strcmp((a), (b)) != 0) { \
        fprintf(stderr, "\n  FAILED: %s == %s\n", #a, #b); \
        fprintf(stderr, "  Expected: \"%s\", Got: \"%s\"\n", (b), (a)); \
        fprintf(stderr, "  at %s:%d\n", __FILE__, __LINE__); \
        tests_failed++; \
        return; \
    } \
} while(0)

// Define tests
TEST(addition) {
    ASSERT_EQ(2 + 2, 4);
    ASSERT_EQ(10 + 5, 15);
    ASSERT_EQ(-5 + 5, 0);
}

TEST(string_operations) {
    char str[] = "hello";
    ASSERT_STR_EQ(str, "hello");
    ASSERT_EQ(strlen(str), 5);
}

// Main test runner
int main(void) {
    printf("Running tests...\n\n");

    RUN_TEST(addition);
    RUN_TEST(string_operations);

    printf("\n=== Test Results ===\n");
    printf("Passed: %d\n", tests_passed);
    printf("Failed: %d\n", tests_failed);
    printf("Total:  %d\n", tests_run);

    return tests_failed > 0 ? 1 : 0;
}
\end{lstlisting}

\begin{tipbox}
This simple framework is enough for most C projects. It's self-contained, requires no external dependencies, and gives clear output.
\end{tipbox}

\section{Testing Memory Allocations}

Memory bugs are C's biggest problem. Test them explicitly:

\begin{lstlisting}
// Test that function handles allocation failure
TEST(handles_allocation_failure) {
    // Save original malloc
    void* (*old_malloc)(size_t) = malloc;

    // Inject failure (using macro or function wrapper)
    // This example assumes you have a test malloc wrapper
    set_malloc_failure_mode(1);

    char* result = my_allocating_function();
    ASSERT(result == NULL);  // Should handle failure gracefully

    set_malloc_failure_mode(0);
}

// Test for memory leaks
TEST(no_memory_leaks) {
    int alloc_before = get_allocation_count();

    MyObject* obj = myobject_create();
    ASSERT(obj != NULL);
    myobject_destroy(obj);

    int alloc_after = get_allocation_count();
    ASSERT_EQ(alloc_before, alloc_after);
}
\end{lstlisting}

\section{Memory Leak Detection}

Track all allocations in debug builds:

\begin{lstlisting}
#ifdef DEBUG_MEMORY

typedef struct MemEntry {
    void* ptr;
    size_t size;
    const char* file;
    int line;
    struct MemEntry* next;
} MemEntry;

static MemEntry* mem_list = NULL;
static int alloc_count = 0;
static int free_count = 0;
static size_t bytes_allocated = 0;

void* debug_malloc(size_t size, const char* file, int line) {
    void* ptr = malloc(size);
    if (ptr) {
        MemEntry* entry = malloc(sizeof(MemEntry));
        entry->ptr = ptr;
        entry->size = size;
        entry->file = file;
        entry->line = line;
        entry->next = mem_list;
        mem_list = entry;

        alloc_count++;
        bytes_allocated += size;

        printf("[ALLOC] %p (%zu bytes) at %s:%d\n",
               ptr, size, file, line);
    }
    return ptr;
}

void debug_free(void* ptr, const char* file, int line) {
    if (!ptr) return;

    MemEntry** entry = &mem_list;
    while (*entry) {
        if ((*entry)->ptr == ptr) {
            MemEntry* to_free = *entry;
            *entry = (*entry)->next;

            printf("[FREE] %p at %s:%d\n", ptr, file, line);

            free_count++;
            bytes_allocated -= to_free->size;
            free(to_free);
            free(ptr);
            return;
        }
        entry = &(*entry)->next;
    }

    fprintf(stderr, "[ERROR] Freeing untracked pointer %p at %s:%d\n",
            ptr, file, line);
    free(ptr);
}

void debug_report_leaks(void) {
    printf("\n=== Memory Report ===\n");
    printf("Allocations: %d\n", alloc_count);
    printf("Frees: %d\n", free_count);
    printf("Leaks: %d\n", alloc_count - free_count);
    printf("Bytes still allocated: %zu\n", bytes_allocated);

    if (mem_list) {
        printf("\nLeak details:\n");
        for (MemEntry* e = mem_list; e; e = e->next) {
            printf("  %p: %zu bytes allocated at %s:%d\n",
                   e->ptr, e->size, e->file, e->line);
        }
    }
}

#define malloc(size) debug_malloc(size, __FILE__, __LINE__)
#define free(ptr) debug_free(ptr, __FILE__, __LINE__)

// Call at program exit
atexit(debug_report_leaks);

#endif // DEBUG_MEMORY
\end{lstlisting}

\section{Testing with Mocks and Stubs}

Replace dependencies for isolated testing:

\begin{lstlisting}
// Production code
typedef struct {
    int (*read)(void* handle);
    int (*write)(void* handle, int data);
} IOInterface;

int process_data(IOInterface* io, void* handle) {
    int data = io->read(handle);
    if (data < 0) return -1;

    data *= 2;  // Process

    return io->write(handle, data);
}

// Mock implementation for testing
static int mock_read_value = 42;
static int mock_write_called = 0;
static int mock_write_last_value = 0;

int mock_read(void* handle) {
    return mock_read_value;
}

int mock_write(void* handle, int data) {
    mock_write_called++;
    mock_write_last_value = data;
    return 0;
}

// Test using mocks
TEST(process_data_doubles_value) {
    IOInterface mock_io = {
        .read = mock_read,
        .write = mock_write
    };

    mock_read_value = 21;
    mock_write_called = 0;

    int result = process_data(&mock_io, NULL);

    ASSERT_EQ(result, 0);
    ASSERT_EQ(mock_write_called, 1);
    ASSERT_EQ(mock_write_last_value, 42);
}
\end{lstlisting}

\section{Assertion Patterns}

Use assertions to catch bugs early:

\begin{lstlisting}
#include <assert.h>

// Debug-only assertions (disabled with NDEBUG)
void process_array(int* arr, size_t len) {
    assert(arr != NULL);
    assert(len > 0);

    // Process array...
}

// Always-on assertions for critical checks
#define REQUIRE(cond) do { \
    if (!(cond)) { \
        fprintf(stderr, "Requirement failed: %s\n", #cond); \
        fprintf(stderr, "  at %s:%d in %s\n", \
                __FILE__, __LINE__, __func__); \
        abort(); \
    } \
} while(0)

// Compile-time assertions
#define STATIC_ASSERT(cond, msg) \
    typedef char static_assertion_##msg[(cond) ? 1 : -1]

STATIC_ASSERT(sizeof(int) == 4, int_must_be_4_bytes);
STATIC_ASSERT(sizeof(void*) == 8, need_64bit_platform);

// C11 static assert
_Static_assert(sizeof(int) >= 4, "int too small");
\end{lstlisting}

\section{Debugging Print Utilities}

Make debugging easier with helper functions:

\begin{lstlisting}
// Hexdump for binary data
void hexdump(const void* data, size_t size) {
    const unsigned char* bytes = data;

    for (size_t i = 0; i < size; i++) {
        if (i % 16 == 0) {
            printf("\n%04zx: ", i);
        }
        printf("%02x ", bytes[i]);

        if ((i + 1) % 16 == 0 || i == size - 1) {
            // Print ASCII
            size_t start = i - (i % 16);
            size_t end = i + 1;
            printf(" ");
            for (size_t j = start; j < end; j++) {
                char c = bytes[j];
                printf("%c", (c >= 32 && c < 127) ? c : '.');
            }
        }
    }
    printf("\n");
}

// Print binary representation
void print_binary(unsigned int n) {
    for (int i = 31; i >= 0; i--) {
        printf("%d", (n >> i) & 1);
        if (i % 8 == 0) printf(" ");
    }
    printf("\n");
}

// Dump struct bytes
#define DUMP_STRUCT(s) do { \
    printf("%s = {\n", #s); \
    unsigned char* bytes = (unsigned char*)&(s); \
    for (size_t i = 0; i < sizeof(s); i++) { \
        printf("  [%zu] = 0x%02x", i, bytes[i]); \
        if (i % 8 == 7) printf("\n"); \
    } \
    printf("\n}\n"); \
} while(0)

// Stack trace (GCC/Clang)
#include <execinfo.h>

void print_stack_trace(void) {
    void* array[10];
    size_t size = backtrace(array, 10);
    char** strings = backtrace_symbols(array, size);

    printf("Stack trace:\n");
    for (size_t i = 0; i < size; i++) {
        printf("  [%zu] %s\n", i, strings[i]);
    }
    free(strings);
}
\end{lstlisting}

\section{Debugging with GDB}

Essential GDB commands and patterns:

\begin{lstlisting}
// Compile with debug symbols
// gcc -g -O0 program.c -o program

// Common GDB commands:
// gdb ./program
// (gdb) break main
// (gdb) run
// (gdb) next          # Step over
// (gdb) step          # Step into
// (gdb) continue      # Continue execution
// (gdb) print var     # Print variable
// (gdb) backtrace     # Stack trace
// (gdb) frame 2       # Switch to stack frame
// (gdb) info locals   # Show local variables
// (gdb) watch var     # Break when var changes
// (gdb) quit

// Conditional breakpoint
// (gdb) break myfile.c:42 if x > 100

// Print macro expansions
// (gdb) macro expand MY_MACRO(x)
\end{lstlisting}

\subsection{GDB Helper Functions}

\begin{lstlisting}
// Add to ~/.gdbinit

define plist
    set var $n = $arg0
    while $n
        print *$n
        set var $n = $n->next
    end
end
document plist
Print linked list starting from node.
Usage: plist head_node
end

define parray
    set var $i = 0
    while $i < $arg1
        print $arg0[$i]
        set var $i = $i + 1
    end
end
document parray
Print array elements.
Usage: parray array_name count
end
\end{lstlisting}

\section{Sanitizers}

Modern compilers include powerful bug detectors:

\subsection{AddressSanitizer (ASan)}

\begin{lstlisting}
// Compile with:
// gcc -fsanitize=address -g program.c -o program

// Detects:
// - Buffer overflows
// - Use after free
// - Memory leaks
// - Use after return

// Example bug it catches:
int* create_array(void) {
    int arr[10];
    return arr;  // ASan catches use-after-return!
}
\end{lstlisting}

\subsection{UndefinedBehaviorSanitizer (UBSan)}

\begin{lstlisting}
// Compile with:
// gcc -fsanitize=undefined -g program.c -o program

// Detects:
// - Integer overflow
// - Division by zero
// - NULL pointer dereference
// - Misaligned access

// Example:
int x = INT_MAX;
x++;  // UBSan catches overflow!
\end{lstlisting}

\subsection{MemorySanitizer (MSan)}

\begin{lstlisting}
// Compile with:
// clang -fsanitize=memory -g program.c -o program

// Detects uninitialized memory reads:
int x;
if (x > 0) {  // MSan catches uninitialized read!
    printf("Positive\n");
}
\end{lstlisting}

\section{Valgrind}

The classic memory debugger:

\begin{lstlisting}
// Run with Valgrind:
// valgrind --leak-check=full ./program

// Example output for memory leak:
// ==12345== 100 bytes in 1 blocks are definitely lost
// ==12345==    at 0x4C2FB0F: malloc
// ==12345==    by 0x10918E: main (program.c:42)

// Common Valgrind options:
// --leak-check=full        # Detailed leak info
// --show-leak-kinds=all    # Show all leak types
// --track-origins=yes      # Track uninitialized values
// --verbose                # More information
\end{lstlisting}

\section{Fuzz Testing}

Find bugs by throwing random inputs:

\begin{lstlisting}
// Simple fuzzer
void fuzz_test_parser(void) {
    for (int i = 0; i < 10000; i++) {
        // Generate random input
        size_t len = rand() % 1000;
        char* input = malloc(len + 1);

        for (size_t j = 0; j < len; j++) {
            input[j] = rand() % 256;
        }
        input[len] = '\0';

        // Test parser - should not crash
        parse_input(input);

        free(input);
    }
}

// Using AFL (American Fuzzy Lop)
// afl-gcc program.c -o program
// afl-fuzz -i input_dir -o output_dir ./program @@
\end{lstlisting}

\section{Test-Driven Development in C}

Write tests first:

\begin{lstlisting}
// 1. Write test first
TEST(parse_integer) {
    int result;
    ASSERT_EQ(parse_int("123", &result), 0);
    ASSERT_EQ(result, 123);

    ASSERT_EQ(parse_int("-456", &result), 0);
    ASSERT_EQ(result, -456);

    ASSERT_EQ(parse_int("abc", &result), -1);  // Should fail
}

// 2. Watch it fail (no implementation yet)

// 3. Implement minimal code to pass
int parse_int(const char* str, int* out) {
    char* end;
    long val = strtol(str, &end, 10);

    if (end == str || *end != '\0') {
        return -1;
    }

    *out = (int)val;
    return 0;
}

// 4. Test passes - refactor if needed
\end{lstlisting}

\section{Coverage Testing}

Ensure tests exercise all code:

\begin{lstlisting}
// Compile with coverage:
// gcc -fprofile-arcs -ftest-coverage program.c -o program

// Run tests:
// ./program

// Generate coverage report:
// gcov program.c
// lcov --capture --directory . --output-file coverage.info
// genhtml coverage.info --output-directory coverage_html

// View coverage_html/index.html in browser
\end{lstlisting}

\section{Integration Testing}

Test components working together:

\begin{lstlisting}
TEST(full_system_test) {
    // Setup
    Database* db = database_create(":memory:");
    Server* srv = server_create(8080);

    // Test actual workflow
    database_insert(db, "key1", "value1");

    Request* req = request_create("GET", "/key1");
    Response* resp = server_handle(srv, req);

    ASSERT_EQ(resp->status, 200);
    ASSERT_STR_EQ(resp->body, "value1");

    // Cleanup
    response_destroy(resp);
    request_destroy(req);
    server_destroy(srv);
    database_destroy(db);
}
\end{lstlisting}

\section{Summary}

Testing and debugging in C:

\begin{itemize}
    \item Build your own simple test framework
    \item Track memory allocations in debug builds
    \item Use mocks to isolate components
    \item Add assertions liberally
    \item Use sanitizers during development
    \item Run Valgrind regularly
    \item Learn GDB thoroughly
    \item Write tests first (TDD)
    \item Measure test coverage
    \item Fuzz test parsers and inputs
\end{itemize}

Testing is not optional in C---it's the difference between working code and disaster!

\chapter{Build Patterns and Systems}

\section{The Real-World Build Problem}

Let me show you EXACTLY what happens when you build a real C project, step by step, and WHY each piece exists.

\textbf{Why real-world builds are confusing}:

When you clone a real C project and see \texttt{./configure \&\& make \&\& make install}, you're witnessing decades of evolved build practices. There's \texttt{configure.ac}, \texttt{Makefile.am}, \texttt{config.h.in}, \texttt{m4} macros, shell scripts, and generated files everywhere. It's overwhelming because each piece solves a specific historical problem:

\begin{itemize}
    \item \textbf{1970s}: Just compile all .c files
    \item \textbf{1980s}: Make tracks dependencies
    \item \textbf{1990s}: Autotools handles portability (different Unix flavors)
    \item \textbf{2000s}: pkg-config manages library dependencies
    \item \textbf{2010s}: CMake/Meson provide modern alternatives
    \item \textbf{2020s}: Containers and CI/CD automation
\end{itemize}

Each layer adds complexity, but also solves real problems. This chapter explains \textit{everything}---from the configure script you run to the installation paths hardcoded in the binary.

\subsection{The Problem: Write Once, Run Anywhere}

You write a C program on your Linux laptop. You want other people to compile it on:
\begin{itemize}
    \item Red Hat Enterprise Linux 7 (has GCC 4.8)
    \item Ubuntu 22.04 (has GCC 11)
    \item macOS 12 (has Clang)
    \item FreeBSD 13 (has Clang, different paths)
    \item Alpine Linux (has musl libc instead of glibc)
\end{itemize}

\textbf{Problems you'll hit}:

\begin{enumerate}
    \item \textbf{Different compilers}: GCC vs Clang, different versions, different flags
    \item \textbf{Different library locations}: OpenSSL might be in /usr/lib or /usr/local/lib or /opt/openssl
    \item \textbf{Different header locations}: headers in /usr/include or /usr/local/include
    \item \textbf{Missing functions}: Some systems have \texttt{strlcpy}, others don't
    \item \textbf{Different system calls}: Linux has \texttt{epoll}, macOS has \texttt{kqueue}, BSD has both
    \item \textbf{Feature availability}: Does this system have threading? IPv6? 64-bit file support?
\end{enumerate}

\textbf{The solution}: A configure script that TESTS the system and generates appropriate Makefiles. This is why every serious C project has \texttt{./configure}.

\subsection{Let's Build a Real Project: curl}

I'll show you exactly what happens when you build curl (the command-line HTTP client used by millions). We'll trace EVERY step.

\begin{lstlisting}
# Clone curl
git clone https://github.com/curl/curl.git
cd curl

# What files do you see?
ls -la

# You'll see:
configure.ac        # INPUT: autoconf reads this
Makefile.am         # INPUT: automake reads this
m4/                 # Directory of autoconf macros
src/                # Source code
lib/                # libcurl library
include/            # Headers
docs/               # Documentation
buildconf           # Script to generate configure
configure           # Generated by autoconf (if present)
Makefile.in         # Generated by automake (if present)
\end{lstlisting}

\textbf{Key insight}: Files ending in \texttt{.ac}, \texttt{.am}, \texttt{.in} are INPUTS. The \texttt{configure} script and \texttt{Makefile} are OUTPUTS.

\subsection{Step 1: Generate the Configure Script (Developer Only)}

If you cloned from git, there's no \texttt{configure} script yet. It must be generated:

\begin{lstlisting}
# Run buildconf (or autogen.sh in other projects)
./buildconf

# What does this do? Let's trace it:
# 1. Runs libtoolize (for building shared libraries)
# 2. Runs aclocal (collects m4 macros)
# 3. Runs autoconf (generates configure from configure.ac)
# 4. Runs automake (generates Makefile.in from Makefile.am)

# After this, you have:
configure           # Shell script (~40,000 lines!)
Makefile.in         # Makefile template
aclocal.m4          # Collected macros
config.h.in         # Config header template

# Users who download curl-7.x.tar.gz DON'T run this
# They get the configure script already generated
\end{lstlisting}

\textbf{Why this step?} The configure script is 40,000 lines of shell code. You don't write it by hand. autoconf generates it from configure.ac (which is ~3,000 lines of M4 macros).

\subsection{Step 2: Run Configure - The Magic Happens}

Now let's run configure and see EXACTLY what it does:

\begin{lstlisting}
# Run configure with verbose output
./configure --prefix=/usr/local 2>&1 | tee configure.log

# It prints:
checking for gcc... gcc
checking whether the C compiler works... yes
checking for C compiler default output file name... a.out
checking for suffix of executables...
checking whether we are cross compiling... no
checking for suffix of object files... o
checking whether the compiler supports GNU C... yes
checking for openssl/ssl.h... yes
checking for SSL_connect in -lssl... yes
checking for libz... yes
checking for pthread_create in -lpthread... yes
...hundreds more checks...
configure: creating ./config.status
config.status: creating Makefile
config.status: creating lib/Makefile
config.status: creating src/Makefile
config.status: creating lib/curl_config.h
configure: Configured to build curl/libcurl:
  Host system:     x86_64-pc-linux-gnu
  SSL support:     enabled (OpenSSL)
  SSH support:     enabled
  IPv6 support:    enabled
  Protocol:        HTTP/HTTPS/FTP/FTPS/SCP/SFTP...
\end{lstlisting}

\textbf{What just happened?} Let's break down each check:

\subsubsection{Check 1: Finding the C Compiler}

\begin{lstlisting}
# configure runs:
gcc -v
# Checks exit code. If success, GCC exists.

# Then it compiles a test program:
cat > conftest.c << EOF
int main(void) { return 0; }
EOF

gcc conftest.c -o conftest
# If this works, compiler is functional
rm -f conftest conftest.c

# Result: CC=gcc is set in the Makefile
\end{lstlisting}

\subsubsection{Check 2: Finding OpenSSL}

\begin{lstlisting}
# configure tries multiple methods:

# Method 1: Check if pkg-config knows about openssl
pkg-config --exists openssl
if [ $? -eq 0 ]; then
    OPENSSL_CFLAGS=$(pkg-config --cflags openssl)
    OPENSSL_LIBS=$(pkg-config --libs openssl)
fi

# Method 2: Try to compile a test program
cat > conftest.c << EOF
#include <openssl/ssl.h>
int main(void) { SSL_library_init(); return 0; }
EOF

# Try with default paths
gcc conftest.c -lssl -lcrypto -o conftest 2>/dev/null
if [ $? -eq 0 ]; then
    HAVE_OPENSSL=yes
fi

# Method 3: Search common locations
for dir in /usr /usr/local /opt/openssl; do
    if [ -f $dir/include/openssl/ssl.h ]; then
        OPENSSL_CFLAGS="-I$dir/include"
        OPENSSL_LIBS="-L$dir/lib -lssl -lcrypto"
        HAVE_OPENSSL=yes
        break
    fi
done

# Result: Either HAVE_OPENSSL=yes or HAVE_OPENSSL=no
# This gets written to config.h:
# #define HAVE_OPENSSL 1
\end{lstlisting}

\subsubsection{Check 3: Function Availability}

\begin{lstlisting}
# Check if strlcpy exists (BSD function not in glibc)
cat > conftest.c << EOF
#include <string.h>
int main(void) {
    char buf[10];
    strlcpy(buf, "test", sizeof(buf));
    return 0;
}
EOF

gcc conftest.c -o conftest 2>/dev/null
if [ $? -eq 0 ]; then
    # Function exists
    echo "#define HAVE_STRLCPY 1" >> config.h
else
    # Function missing - use fallback
    echo "/* #undef HAVE_STRLCPY */" >> config.h
fi

# In code, you use:
#ifdef HAVE_STRLCPY
    strlcpy(dest, src, size);
#else
    strncpy(dest, src, size - 1);
    dest[size - 1] = '\0';
#endif
\end{lstlisting}

\subsubsection{The Generated Files}

After configure finishes, you have:

\begin{lstlisting}
# config.h - System capabilities
#define HAVE_OPENSSL 1
#define HAVE_PTHREAD 1
#define HAVE_STRLCPY 1
#define HAVE_SYS_SELECT_H 1
/* #undef HAVE_KQUEUE */
#define SIZEOF_INT 4
#define SIZEOF_LONG 8

# Makefile - Build instructions
CC = gcc
CFLAGS = -O2 -Wall
LDFLAGS = -lssl -lcrypto -lpthread -lz
prefix = /usr/local
bindir = ${prefix}/bin
libdir = ${prefix}/lib

curl: src/main.o lib/libcurl.a
	$(CC) -o curl src/main.o -Llib -lcurl $(LDFLAGS)

src/main.o: src/main.c
	$(CC) $(CFLAGS) -Iinclude -c src/main.c -o src/main.o
\end{lstlisting}

\textbf{The key insight}: configure is a giant detection script. It doesn't compile anything---it just TESTS what's available and generates Makefiles tailored to YOUR system.

\subsection{Step 3: Run Make - Actual Compilation}

Now that we have a configured Makefile, let's compile:

\begin{lstlisting}
# Run make with verbose output
make V=1

# You'll see actual commands:
gcc -O2 -Wall -Iinclude -Ilib -c src/main.c -o src/main.o
gcc -O2 -Wall -Iinclude -Ilib -c src/tool_operate.c -o src/tool_operate.o
gcc -O2 -Wall -Iinclude -Ilib -c src/tool_urlglob.c -o src/tool_urlglob.o
...
gcc -O2 -Wall -Iinclude -c lib/url.c -o lib/url.o
gcc -O2 -Wall -Iinclude -c lib/http.c -o lib/http.o
...
ar rcs lib/libcurl.a lib/url.o lib/http.o lib/ftp.o ...
gcc -o src/curl src/main.o src/tool_operate.o ... -Llib -lcurl -lssl -lcrypto -lz -lpthread
\end{lstlisting}

\textbf{What just happened?}

\begin{enumerate}
    \item \textbf{Compiled source files to .o}: Each .c file becomes a .o (object file)
    \item \textbf{Built library}: All lib/*.o files bundled into libcurl.a with \texttt{ar}
    \item \textbf{Linked executable}: src/*.o files linked with libcurl.a and system libraries
\end{enumerate}

\textbf{Try changing one file}:

\begin{lstlisting}
# Modify one source file
echo "// comment" >> src/main.c

# Run make again
make

# Only compiles changed file and relinks:
gcc -O2 -Wall -Iinclude -Ilib -c src/main.c -o src/main.o
gcc -o src/curl src/main.o ... -Llib -lcurl -lssl -lcrypto -lz

# Make tracks dependencies with timestamps!
# main.o is newer than main.c? Skip compilation
# curl is older than main.o? Relink
\end{lstlisting}

\subsection{Step 4: Run Make Install - System Installation}

\begin{lstlisting}
# Install to /usr/local (requires root)
sudo make install

# What it does:
install -d /usr/local/bin
install -m 755 src/curl /usr/local/bin/curl
install -d /usr/local/lib
install -m 644 lib/libcurl.a /usr/local/lib/libcurl.a
install -m 755 lib/libcurl.so.4.8.0 /usr/local/lib/
ln -sf libcurl.so.4.8.0 /usr/local/lib/libcurl.so.4
ln -sf libcurl.so.4 /usr/local/lib/libcurl.so
install -d /usr/local/include/curl
install -m 644 include/curl/curl.h /usr/local/include/curl/
install -d /usr/local/lib/pkgconfig
install -m 644 libcurl.pc /usr/local/lib/pkgconfig/
install -d /usr/local/share/man/man1
install -m 644 docs/curl.1 /usr/local/share/man/man1/

# Update linker cache (Linux)
ldconfig
\end{lstlisting}

\textbf{Why these paths?}

\begin{itemize}
    \item \texttt{/usr/local/bin}: User-installed executables
    \item \texttt{/usr/local/lib}: User-installed libraries
    \item \texttt{/usr/local/include}: User-installed headers
    \item \texttt{/usr/local/share}: Data files (docs, man pages)
\end{itemize}

System packages use \texttt{/usr} (managed by package manager). You use \texttt{/usr/local} (manual installs).

\section{Why This System Exists - The Historical Problem}

\subsection{The Portability Nightmare (Pre-Autotools)}

In the 1980s-90s, Unix variants were incompatible:

\begin{lstlisting}
# Your program on different systems:

# SunOS (Sun Microsystems)
cc -I/usr/openwin/include -L/usr/openwin/lib -lX11 main.c

# AIX (IBM)
xlc -I/usr/lpp/X11/include -L/usr/lpp/X11/lib -lX11 main.c

# HP-UX (Hewlett-Packard)
cc -I/usr/include/X11R5 -L/usr/lib/X11R5 -lX11 main.c

# IRIX (SGI)
cc -I/usr/include/X11 -L/usr/lib32 -lX11 main.c
\end{lstlisting}

Every system had:
\begin{itemize}
    \item Different compiler names (cc, gcc, xlc, acc)
    \item Different library locations
    \item Different available functions
    \item Different system calls
\end{itemize}

\textbf{Solutions developers tried}:

\subsubsection{Attempt 1: Platform-Specific Makefiles}

\begin{lstlisting}
# Makefile.sunos
CC = cc
CFLAGS = -I/usr/openwin/include
LDFLAGS = -L/usr/openwin/lib -lX11

# Makefile.aix
CC = xlc
CFLAGS = -I/usr/lpp/X11/include
LDFLAGS = -L/usr/lpp/X11/lib -lX11

# Users had to choose:
make -f Makefile.sunos  # On SunOS
make -f Makefile.aix    # On AIX
\end{lstlisting}

\textbf{Problem}: Unmaintainable. 10 Unix variants = 10 Makefiles to maintain.

\subsubsection{Attempt 2: Imake (X11's Solution)}

\begin{lstlisting}
# Imakefile - High-level description
SRCS = main.c utils.c
OBJS = main.o utils.o
ComplexProgramTarget(myprogram)

# Run imake to generate Makefile
imake -DUseInstalled -I/usr/lib/X11/config

# Problem: Required X11 infrastructure everywhere
# Even if your program didn't use X11!
\end{lstlisting}

\subsubsection{Attempt 3: Autotools (The Winner)}

GNU invented autotools in the 1990s. Key insight:

\begin{quote}
Don't hardcode paths. TEST the system and generate appropriate Makefiles.
\end{quote}

\begin{lstlisting}
# User experience becomes universal:
./configure
make
make install

# Works on ANY Unix, even ones that didn't exist yet!
\end{lstlisting}

\section{Makefile Fundamentals}

Make is the oldest and most universal build tool. Every C programmer must know Make.

\subsection{Basic Makefile Structure}

\begin{lstlisting}
# Target: dependencies
#     commands (must be indented with TAB)

# Build executable
program: main.o utils.o
	gcc -o program main.o utils.o

# Compile main.c
main.o: main.c utils.h
	gcc -c main.c

# Compile utils.c
utils.o: utils.c utils.h
	gcc -c utils.c

# Clean build artifacts
clean:
	rm -f *.o program
\end{lstlisting}

\textbf{How Make works}:

\begin{enumerate}
    \item Make reads the Makefile
    \item You run \texttt{make target} (or just \texttt{make} for first target)
    \item Make checks if target exists and if dependencies are newer
    \item If target is out of date, Make runs the commands
    \item Make recursively builds dependencies first
\end{enumerate}

\subsection{Variables and Automatic Variables}

\begin{lstlisting}
# Variables
CC = gcc
CFLAGS = -Wall -Wextra -O2 -g
LDFLAGS = -lm -lpthread
OBJECTS = main.o utils.o parser.o

# Use variables
program: $(OBJECTS)
	$(CC) -o $@ $(OBJECTS) $(LDFLAGS)

# Pattern rule with automatic variables
%.o: %.c
	$(CC) $(CFLAGS) -c $< -o $@

# Automatic variables:
# $@ = target name
# $< = first dependency
# $^ = all dependencies
# $* = stem of pattern match

# Example usage
main.o: main.c utils.h config.h
	$(CC) $(CFLAGS) -c $< -o $@
	# $< is main.c
	# $@ is main.o
	# $^ is main.c utils.h config.h
\end{lstlisting}

\subsection{Phony Targets}

\begin{lstlisting}
# Phony targets don't correspond to files
.PHONY: all clean install test

all: program

clean:
	rm -f $(OBJECTS) program

install: program
	install -m 755 program /usr/local/bin/

test: program
	./program --test
	./run_tests.sh

# Without .PHONY, if a file named "clean" exists,
# make would think target is up to date and skip it!
\end{lstlisting}

\subsection{Real-World Makefile}

\begin{lstlisting}
# Project configuration
PROJECT = myapp
VERSION = 1.0.0
PREFIX = /usr/local

# Compiler and flags
CC = gcc
CFLAGS = -Wall -Wextra -std=c11 -O2 -g
CFLAGS += -D_POSIX_C_SOURCE=200809L
CFLAGS += -DVERSION=\"$(VERSION)\"
LDFLAGS = -lm -lpthread

# Directories
SRCDIR = src
INCDIR = include
BUILDDIR = build
BINDIR = bin

# Source files
SOURCES = $(wildcard $(SRCDIR)/*.c)
OBJECTS = $(SOURCES:$(SRCDIR)/%.c=$(BUILDDIR)/%.o)
DEPENDS = $(OBJECTS:.o=.d)

# Main target
$(BINDIR)/$(PROJECT): $(OBJECTS) | $(BINDIR)
	$(CC) -o $@ $^ $(LDFLAGS)

# Compile with dependency generation
$(BUILDDIR)/%.o: $(SRCDIR)/%.c | $(BUILDDIR)
	$(CC) $(CFLAGS) -I$(INCDIR) -MMD -MP -c $< -o $@

# Create directories
$(BUILDDIR) $(BINDIR):
	mkdir -p $@

# Include auto-generated dependencies
-include $(DEPENDS)

# Phony targets
.PHONY: all clean install uninstall debug release test

all: $(BINDIR)/$(PROJECT)

clean:
	rm -rf $(BUILDDIR) $(BINDIR)

install: $(BINDIR)/$(PROJECT)
	install -d $(PREFIX)/bin
	install -m 755 $(BINDIR)/$(PROJECT) $(PREFIX)/bin/

uninstall:
	rm -f $(PREFIX)/bin/$(PROJECT)

debug: CFLAGS += -DDEBUG -O0 -g3
debug: clean all

release: CFLAGS += -DNDEBUG -O3 -march=native
release: LDFLAGS += -s
release: clean all

test: $(BINDIR)/$(PROJECT)
	$(BINDIR)/$(PROJECT) --run-tests

# Show configuration
info:
	@echo "Project: $(PROJECT) v$(VERSION)"
	@echo "CC: $(CC)"
	@echo "CFLAGS: $(CFLAGS)"
	@echo "LDFLAGS: $(LDFLAGS)"
	@echo "Sources: $(SOURCES)"
\end{lstlisting}

\textbf{Key features}:

\begin{itemize}
    \item Automatic dependency generation with \texttt{-MMD -MP}
    \item Separate source/build/bin directories
    \item Debug and release configurations
    \item Installation support
    \item Configurable prefix for different install locations
\end{itemize}

\section{Dependency Generation}

The hardest part of makefiles: tracking header dependencies automatically.

\subsection{The Problem}

\begin{lstlisting}
// main.c includes utils.h
#include "utils.h"

int main(void) {
    utility_function();
}

// If utils.h changes, main.c must be recompiled
// But how does Make know about this dependency?
\end{lstlisting}

\subsection{Manual Dependencies (Don't Do This)}

\begin{lstlisting}
# Manually list all header dependencies
main.o: main.c utils.h config.h types.h error.h

# Problem: Easy to get out of sync
# Add a new #include? Must update Makefile
# Remove an #include? Makefile still wrong
\end{lstlisting}

\subsection{Automatic Dependencies (The Right Way)}

\begin{lstlisting}
# Generate .d dependency files
%.o: %.c
	$(CC) $(CFLAGS) -MMD -MP -c $< -o $@

# Include them
-include $(OBJECTS:.o=.d)

# What this does:
# -MMD: Generate .d file with dependencies
# -MP: Add phony targets for headers (avoid errors if header deleted)

# Example generated main.d:
# main.o: main.c utils.h config.h
# utils.h:
# config.h:

# The phony targets prevent errors if you delete a header
\end{lstlisting}

\textbf{How it works}:

\begin{enumerate}
    \item First build: no .d files exist, so all .c files compile
    \item Compilation creates .d files listing each .o's dependencies
    \item Next build: Make includes .d files and knows the full dependency graph
    \item Change a header: Make recompiles all files that include it
\end{enumerate}

\section{Static and Dynamic Libraries}

Libraries package reusable code. Understanding them is crucial.

\subsection{Static Libraries (.a files)}

Linked into executable at compile time---becomes part of the binary.

\begin{lstlisting}
# Create static library
# Step 1: Compile to object files
gcc -c utils.c -o utils.o
gcc -c string_utils.c -o string_utils.o
gcc -c math_utils.c -o math_utils.o

# Step 2: Create archive
ar rcs libmyutils.a utils.o string_utils.o math_utils.o
# r = insert/replace
# c = create archive
# s = create symbol index

# Use static library
gcc main.c -L. -lmyutils -o program
# -L. = look for libraries in current directory
# -lmyutils = link libmyutils.a

# In Makefile:
libmyutils.a: utils.o string_utils.o math_utils.o
	ar rcs $@ $^

program: main.o libmyutils.a
	$(CC) -o $@ main.o -L. -lmyutils
\end{lstlisting}

\textbf{Static library advantages}:
\begin{itemize}
    \item No runtime dependencies---program is self-contained
    \item Slightly faster (no dynamic linking overhead)
    \item Easier deployment (one file)
\end{itemize}

\textbf{Disadvantages}:
\begin{itemize}
    \item Larger executables (library code copied in)
    \item No shared memory (each program has own copy)
    \item Must recompile programs to update library
\end{itemize}

\subsection{Dynamic Libraries (.so on Linux, .dylib on macOS, .dll on Windows)}

Loaded at runtime---shared between programs.

\begin{lstlisting}
# Create shared library
gcc -fPIC -c utils.c -o utils.o
gcc -fPIC -c string_utils.c -o string_utils.o
# -fPIC = Position Independent Code (required for shared libraries)

gcc -shared -o libmyutils.so utils.o string_utils.o
# -shared = create shared library

# Use shared library
gcc main.c -L. -lmyutils -o program

# Run program
LD_LIBRARY_PATH=. ./program
# LD_LIBRARY_PATH tells loader where to find .so files

# Or install to system location
sudo cp libmyutils.so /usr/local/lib/
sudo ldconfig  # Update linker cache

# In Makefile:
libmyutils.so: utils.o string_utils.o
	$(CC) -shared -o $@ $^

%.o: %.c
	$(CC) $(CFLAGS) -fPIC -c $< -o $@

program: main.o libmyutils.so
	$(CC) -o $@ main.o -L. -lmyutils -Wl,-rpath,'$$ORIGIN'
	# -Wl,-rpath,'$$ORIGIN' = look for .so in same directory as executable
\end{lstlisting}

\textbf{Dynamic library advantages}:
\begin{itemize}
    \item Smaller executables
    \item Shared memory (one copy in RAM for all programs)
    \item Update library without recompiling programs
    \item Plugin systems possible
\end{itemize}

\textbf{Disadvantages}:
\begin{itemize}
    \item Runtime dependencies---must have .so installed
    \item Slightly slower (dynamic linking overhead)
    \item Version conflicts ("DLL hell")
    \item More complex deployment
\end{itemize}

\subsection{Symbol Visibility}

Control what symbols are exported from libraries:

\begin{lstlisting}
// Library header - myutils.h
#ifndef MYUTILS_H
#define MYUTILS_H

// Public API - visible to users
#ifdef _WIN32
    #define API_EXPORT __declspec(dllexport)
#else
    #define API_EXPORT __attribute__((visibility("default")))
#endif

API_EXPORT int public_function(int x);

// Private function - not visible outside library
int internal_function(int x);

#endif

// Implementation - myutils.c
#include "myutils.h"

// This is exported
int public_function(int x) {
    return internal_function(x) * 2;
}

// This is hidden
static int internal_function(int x) {
    return x + 1;
}

// Compile with hidden visibility by default
gcc -fPIC -fvisibility=hidden -c myutils.c

// Only functions marked API_EXPORT are visible
\end{lstlisting}

\textbf{Why hide symbols?}

\begin{itemize}
    \item Smaller library size
    \item Faster loading
    \item Avoid symbol conflicts
    \item Clear API boundary
\end{itemize}

\section{Compiler Flags Deep Dive}

Flags dramatically affect your program's behavior and performance.

\subsection{Warning Flags (Always Use These)}

\begin{lstlisting}
# Essential warnings
CFLAGS = -Wall -Wextra -Werror
# -Wall = enable common warnings
# -Wextra = enable extra warnings
# -Werror = treat warnings as errors

# More warnings (stricter)
CFLAGS += -Wpedantic         # Strict ISO C
CFLAGS += -Wshadow           # Variable shadowing
CFLAGS += -Wconversion       # Implicit conversions
CFLAGS += -Wcast-align       # Pointer casts increase alignment
CFLAGS += -Wstrict-prototypes  # Functions without prototypes
CFLAGS += -Wmissing-prototypes # Global functions without prototypes
CFLAGS += -Wformat=2         # printf format checking
CFLAGS += -Wunused           # Unused variables/functions

# Paranoid mode (for critical code)
CFLAGS += -Wcast-qual        # Cast away const
CFLAGS += -Wwrite-strings    # String literals are const
CFLAGS += -Wundef            # Undefined macros in #if
CFLAGS += -Wredundant-decls  # Redundant declarations
CFLAGS += -Wdouble-promotion # Float promoted to double
\end{lstlisting}

\subsection{Optimization Flags}

\begin{lstlisting}
# Optimization levels
-O0  # No optimization (default) - fastest compile
-O1  # Basic optimization
-O2  # Recommended for release - good speed, reasonable compile time
-O3  # Aggressive optimization - may increase code size
-Os  # Optimize for size
-Og  # Optimize for debugging experience

# Debug build
CFLAGS_DEBUG = -O0 -g3 -DDEBUG
# -g3 = maximum debug info (includes macros)

# Release build
CFLAGS_RELEASE = -O2 -DNDEBUG
# -DNDEBUG disables assert()

# Maximum performance (benchmark carefully!)
CFLAGS_FAST = -O3 -march=native -flto
# -march=native = use all CPU features available
# -flto = Link Time Optimization

# Size optimization (embedded systems)
CFLAGS_SMALL = -Os -ffunction-sections -fdata-sections
LDFLAGS_SMALL = -Wl,--gc-sections
# Separate each function/data into sections
# Linker removes unused sections
\end{lstlisting}

\textbf{Warning}: \texttt{-O3} and \texttt{-march=native} can break code that relies on undefined behavior. Always test thoroughly!

\subsection{Architecture and Platform Flags}

\begin{lstlisting}
# Target specific architecture
-m32                    # 32-bit x86
-m64                    # 64-bit x86-64
-march=armv7-a          # ARMv7
-march=native           # Optimize for build machine's CPU

# Position Independent Code (required for shared libraries)
-fPIC                   # Position Independent Code
-fPIE                   # Position Independent Executable (for ASLR)

# Platform defines
-D_POSIX_C_SOURCE=200809L  # POSIX 2008
-D_GNU_SOURCE             # GNU extensions
-D_BSD_SOURCE             # BSD extensions
-D_DEFAULT_SOURCE         # Default features

# Threading
-pthread                  # Enable pthread support
\end{lstlisting}

\subsection{Security Flags}

\begin{lstlisting}
# Security hardening
CFLAGS_SECURE = -fstack-protector-strong  # Stack smashing protection
CFLAGS_SECURE += -D_FORTIFY_SOURCE=2      # Buffer overflow detection
CFLAGS_SECURE += -fPIE                    # Position independent executable

LDFLAGS_SECURE = -Wl,-z,relro             # Read-only relocations
LDFLAGS_SECURE += -Wl,-z,now              # Resolve all symbols at startup
LDFLAGS_SECURE += -pie                    # Create PIE executable

# Full security (Debian/Ubuntu style)
CFLAGS += $(CFLAGS_SECURE)
LDFLAGS += $(LDFLAGS_SECURE)
\end{lstlisting}

\section{Cross-Compilation}

Compiling for a different platform (e.g., compiling for ARM on x86).

\subsection{Cross-Compiler Setup}

\begin{lstlisting}
# Install cross-compiler
sudo apt-get install gcc-arm-linux-gnueabihf

# Cross-compile for ARM
CC = arm-linux-gnueabihf-gcc
AR = arm-linux-gnueabihf-ar
STRIP = arm-linux-gnueabihf-strip

# Build for ARM
arm-linux-gnueabihf-gcc -o program main.c

# Makefile for cross-compilation
ifeq ($(TARGET),arm)
    CC = arm-linux-gnueabihf-gcc
    CFLAGS += -march=armv7-a
else ifeq ($(TARGET),win32)
    CC = i686-w64-mingw32-gcc
    EXE = .exe
else
    CC = gcc
    EXE =
endif

program$(EXE): main.c
	$(CC) $(CFLAGS) -o $@ $<

# Usage:
# make                # Native build
# make TARGET=arm     # ARM build
# make TARGET=win32   # Windows build
\end{lstlisting}

\subsection{Toolchain Files}

For complex cross-compilation, use a toolchain file:

\begin{lstlisting}
# arm-toolchain.cmake (for CMake)
set(CMAKE_SYSTEM_NAME Linux)
set(CMAKE_SYSTEM_PROCESSOR arm)

set(CMAKE_C_COMPILER arm-linux-gnueabihf-gcc)
set(CMAKE_CXX_COMPILER arm-linux-gnueabihf-g++)

set(CMAKE_FIND_ROOT_PATH /usr/arm-linux-gnueabihf)
set(CMAKE_FIND_ROOT_PATH_MODE_PROGRAM NEVER)
set(CMAKE_FIND_ROOT_PATH_MODE_LIBRARY ONLY)
set(CMAKE_FIND_ROOT_PATH_MODE_INCLUDE ONLY)

# Usage: cmake -DCMAKE_TOOLCHAIN_FILE=arm-toolchain.cmake ..
\end{lstlisting}

\section{Multi-Directory Projects}

Real projects have multiple subdirectories.

\subsection{Recursive Make}

\begin{lstlisting}
# Top-level Makefile
SUBDIRS = src lib tests

.PHONY: all clean $(SUBDIRS)

all: $(SUBDIRS)

$(SUBDIRS):
	$(MAKE) -C $@

clean:
	for dir in $(SUBDIRS); do \
		$(MAKE) -C $$dir clean; \
	done

# src/Makefile
SOURCES = main.c utils.c
OBJECTS = $(SOURCES:.c=.o)

all: program

program: $(OBJECTS)
	$(CC) -o $@ $^ -L../lib -lmylib

clean:
	rm -f $(OBJECTS) program
\end{lstlisting}

\textbf{Problem}: Recursive make is slow---each subdirectory is a separate make invocation, can't parallelize efficiently.

\subsection{Non-Recursive Make (Better)}

\begin{lstlisting}
# Single Makefile for entire project
SRCDIR = src
LIBDIR = lib
TESTDIR = tests

SOURCES = $(SRCDIR)/main.c $(SRCDIR)/utils.c
LIB_SOURCES = $(LIBDIR)/mylib.c
TEST_SOURCES = $(TESTDIR)/test_main.c

OBJECTS = $(SOURCES:.c=.o)
LIB_OBJECTS = $(LIB_SOURCES:.c=.o)
TEST_OBJECTS = $(TEST_SOURCES:.c=.o)

all: $(SRCDIR)/program

$(SRCDIR)/program: $(OBJECTS) $(LIBDIR)/libmylib.a
	$(CC) -o $@ $(OBJECTS) -L$(LIBDIR) -lmylib

$(LIBDIR)/libmylib.a: $(LIB_OBJECTS)
	ar rcs $@ $^

$(TESTDIR)/tests: $(TEST_OBJECTS) $(LIBDIR)/libmylib.a
	$(CC) -o $@ $(TEST_OBJECTS) -L$(LIBDIR) -lmylib

# Pattern rules work across all directories
%.o: %.c
	$(CC) $(CFLAGS) -c $< -o $@

# Parallel build works: make -j8
\end{lstlisting}

\section{Build System Generators}

Hand-written Makefiles are tedious. Build generators simplify multi-platform builds.

\subsection{CMake}

The most popular C build system generator:

\begin{lstlisting}
# CMakeLists.txt
cmake_minimum_required(VERSION 3.10)
project(MyProject VERSION 1.0.0 LANGUAGES C)

# Set C standard
set(CMAKE_C_STANDARD 11)
set(CMAKE_C_STANDARD_REQUIRED ON)

# Compiler flags
add_compile_options(-Wall -Wextra)

# Find dependencies
find_package(Threads REQUIRED)

# Library
add_library(myutils STATIC
    src/utils.c
    src/string_utils.c
)
target_include_directories(myutils PUBLIC include)

# Executable
add_executable(program
    src/main.c
)
target_link_libraries(program PRIVATE myutils Threads::Threads)

# Install
install(TARGETS program DESTINATION bin)
install(TARGETS myutils DESTINATION lib)
install(DIRECTORY include/ DESTINATION include)

# Tests
enable_testing()
add_executable(tests test/test_main.c)
target_link_libraries(tests PRIVATE myutils)
add_test(NAME MainTests COMMAND tests)

# Build:
# mkdir build
# cd build
# cmake ..
# make
# make test
# make install
\end{lstlisting}

\subsection{Meson (Modern Alternative)}

\begin{lstlisting}
# meson.build
project('myproject', 'c',
  version: '1.0.0',
  default_options: ['c_std=c11', 'warning_level=3'])

# Library
myutils_lib = static_library('myutils',
  'src/utils.c',
  'src/string_utils.c',
  include_directories: include_directories('include'))

# Executable
executable('program',
  'src/main.c',
  link_with: myutils_lib,
  dependencies: dependency('threads'),
  install: true)

# Tests
test_exe = executable('tests',
  'test/test_main.c',
  link_with: myutils_lib)

test('main tests', test_exe)

# Build:
# meson setup build
# cd build
# ninja
# ninja test
# ninja install
\end{lstlisting}

\section{Practical Build Patterns}

\subsection{Out-of-Tree Builds}

Never build in source directory---keeps source tree clean:

\begin{lstlisting}
# Bad: build in source tree
$ make
# Creates .o files mixed with .c files

# Good: separate build directory
$ mkdir build
$ cd build
$ cmake ..
$ make

# Or with plain make:
BUILDDIR = build
SRCDIR = src

$(BUILDDIR)/%.o: $(SRCDIR)/%.c | $(BUILDDIR)
	$(CC) $(CFLAGS) -c $< -o $@

$(BUILDDIR):
	mkdir -p $@

clean:
	rm -rf $(BUILDDIR)
\end{lstlisting}

\subsection{Multiple Configurations}

\begin{lstlisting}
# Makefile supporting debug/release configs
CONFIG ?= release

ifeq ($(CONFIG),debug)
    CFLAGS = -O0 -g3 -DDEBUG
    OUTDIR = build/debug
else ifeq ($(CONFIG),release)
    CFLAGS = -O2 -DNDEBUG
    OUTDIR = build/release
else
    $(error Unknown configuration: $(CONFIG))
endif

$(OUTDIR)/program: $(OUTDIR)/main.o
	$(CC) -o $@ $^

$(OUTDIR)/%.o: src/%.c | $(OUTDIR)
	$(CC) $(CFLAGS) -c $< -o $@

$(OUTDIR):
	mkdir -p $@

# Usage:
# make CONFIG=debug
# make CONFIG=release
\end{lstlisting}

\subsection{Parallel Builds}

\begin{lstlisting}
# Automatic parallel builds
MAKEFLAGS += -j$(shell nproc)

# Or manually:
make -j8  # Use 8 parallel jobs

# CMake parallel builds
cmake --build . -j8

# In Makefile, ensure proper dependencies!
# Bad: race condition
all:
	gcc -c main.c
	gcc -c utils.c
	gcc -o program main.o utils.o  # May run before .o files ready!

# Good: explicit dependencies
all: program

program: main.o utils.o
	gcc -o program main.o utils.o

main.o: main.c
	gcc -c main.c

utils.o: utils.c
	gcc -c utils.c
\end{lstlisting}

\subsection{Dependency Vendoring}

Include third-party libraries in your source tree:

\begin{lstlisting}
# Project structure:
myproject/
+-- src/
+-- include/
+-- vendor/          # Third-party code
    +-- sqlite/
        +-- sqlite3.c
        +-- sqlite3.h
    +-- zlib/
        +-- zlib.c
        +-- zlib.h
+-- Makefile

# Makefile includes vendor code
VENDOR_SOURCES = vendor/sqlite/sqlite3.c vendor/zlib/zlib.c
SOURCES = src/main.c src/utils.c
ALL_SOURCES = $(SOURCES) $(VENDOR_SOURCES)

program: $(ALL_SOURCES:.c=.o)
	$(CC) -o $@ $^

# Advantages:
# - No external dependencies
# - Controlled versions
# - Easy to patch
# - Reproducible builds

# Disadvantages:
# - Larger repository
# - Must manually update vendor code
\end{lstlisting}

\section{Package Configuration}

Use pkg-config for finding libraries:

\begin{lstlisting}
# Find library flags
CFLAGS += $(shell pkg-config --cflags gtk+-3.0)
LDFLAGS += $(shell pkg-config --libs gtk+-3.0)

# Check if package exists
ifeq ($(shell pkg-config --exists openssl && echo yes),yes)
    CFLAGS += $(shell pkg-config --cflags openssl) -DHAVE_OPENSSL
    LDFLAGS += $(shell pkg-config --libs openssl)
endif

# In CMake:
find_package(PkgConfig REQUIRED)
pkg_check_modules(GTK3 REQUIRED gtk+-3.0)

target_include_directories(program PRIVATE ${GTK3_INCLUDE_DIRS})
target_link_libraries(program PRIVATE ${GTK3_LIBRARIES})

# Create .pc file for your library
# mylib.pc.in
prefix=@PREFIX@
libdir=${prefix}/lib
includedir=${prefix}/include

Name: mylib
Description: My utility library
Version: @VERSION@
Libs: -L${libdir} -lmylib
Cflags: -I${includedir}
\end{lstlisting}

\section{Build Optimization Techniques}

\subsection{Precompiled Headers}

Speed up compilation by precompiling common headers:

\begin{lstlisting}
# Create precompiled header
gcc -c common.h -o common.h.gch

# Use it (automatic if common.h.gch exists)
gcc -include common.h main.c

# In Makefile:
PCH = include/common.h.gch

$(PCH): include/common.h
	$(CC) $(CFLAGS) -c $< -o $@

%.o: %.c $(PCH)
	$(CC) $(CFLAGS) -include include/common.h -c $< -o $@

# Can save 20-50% compile time for large projects
\end{lstlisting}

\subsection{Unity Builds}

Compile all sources as one translation unit:

\begin{lstlisting}
# unity.c - includes all sources
#include "src/main.c"
#include "src/utils.c"
#include "src/parser.c"
#include "src/database.c"

# Compile everything at once
gcc -O2 unity.c -o program

# Much faster compilation (but loses incremental builds)
# Enables better optimization across translation units
# Use for release builds, not development
\end{lstlisting}

\subsection{Ccache - Compiler Cache}

Cache compilation results:

\begin{lstlisting}
# Install ccache
sudo apt-get install ccache

# Use it
CC = ccache gcc

# Or set in CMake
find_program(CCACHE_PROGRAM ccache)
if(CCACHE_PROGRAM)
    set(CMAKE_C_COMPILER_LAUNCHER "${CCACHE_PROGRAM}")
endif()

# First build: normal speed
# Rebuild after 'make clean': instant (from cache)
# Saves huge amounts of time in CI/CD
\end{lstlisting}

\section{Continuous Integration}

Automate builds and tests:

\subsection{GitHub Actions}

\begin{lstlisting}
# .github/workflows/build.yml
name: Build

on: [push, pull_request]

jobs:
  build:
    runs-on: ubuntu-latest
    steps:
    - uses: actions/checkout@v2

    - name: Install dependencies
      run: sudo apt-get install -y libssl-dev

    - name: Build
      run: |
        mkdir build
        cd build
        cmake ..
        make

    - name: Test
      run: cd build && make test

    - name: Upload artifacts
      uses: actions/upload-artifact@v2
      with:
        name: program
        path: build/program
\end{lstlisting}

\subsection{Docker Builds}

Reproducible build environments:

\begin{lstlisting}
# Dockerfile
FROM gcc:11

WORKDIR /app
COPY . .

RUN apt-get update && apt-get install -y cmake

RUN mkdir build && cd build && cmake .. && make

CMD ["./build/program"]

# Build in Docker
docker build -t myprogram .
docker run myprogram
\end{lstlisting}

\section{Troubleshooting Build Problems}

\subsection{Verbose Builds}

\begin{lstlisting}
# See actual commands
make V=1

# CMake verbose
make VERBOSE=1

# Or set in CMakeLists.txt
set(CMAKE_VERBOSE_MAKEFILE ON)
\end{lstlisting}

\subsection{Common Linker Errors}

\begin{lstlisting}
# Undefined reference
# Problem: Missing implementation or library
main.c:10: undefined reference to `foo'

# Solutions:
# 1. Missing .o file
gcc main.o foo.o  # Include foo.o

# 2. Missing library
gcc main.o -lfoo  # Link libfoo.a or libfoo.so

# 3. Wrong link order (static libraries)
gcc main.o -lbar -lfoo  # Wrong!
gcc main.o -lfoo -lbar  # Correct - foo depends on bar

# Multiple definitions
# Problem: Same symbol defined in multiple .o files
foo.o: multiple definition of `global_var'
bar.o: first defined here

# Solution: Use 'extern' or 'static'

# Library not found
/usr/bin/ld: cannot find -lfoo

# Solutions:
# 1. Add library path
gcc main.o -L/path/to/lib -lfoo

# 2. Install library
sudo apt-get install libfoo-dev

# 3. Set LD_LIBRARY_PATH
export LD_LIBRARY_PATH=/path/to/lib
\end{lstlisting}

\section{Deep Dive: Reading a Real configure.ac}

Let's read curl's actual configure.ac line by line and understand EVERY part.

\subsection{The Header Section}

\begin{lstlisting}
# configure.ac (first 50 lines)

AC_PREREQ(2.57)
# Requires autoconf version 2.57 or later

AC_INIT([curl], [7.88.0], [curl-bug@haxx.se])
# Sets package name, version, bug report email

AC_CONFIG_SRCDIR([lib/curl.c])
# Sanity check - this file must exist (we're in right directory)

AC_CONFIG_HEADERS([lib/curl_config.h])
# Generate lib/curl_config.h from lib/curl_config.h.in

AM_INIT_AUTOMAKE([foreign no-define])
# Initialize automake
# "foreign" = don't require GNU files (NEWS, AUTHORS, etc.)
# "no-define" = don't add -DPACKAGE -DVERSION to every compile

AC_PROG_CC
# Find the C compiler (tries gcc, cc, clang in order)
# Sets $(CC) variable

AC_PROG_INSTALL
# Find install program
# Sets $(INSTALL) variable

AC_PROG_LN_S
# Find ln -s command (or fallback on Windows)
# Sets $(LN_S) variable
\end{lstlisting}

\subsection{System Detection}

\begin{lstlisting}
AC_CANONICAL_HOST
# Detects the system triplet: CPU-VENDOR-OS
# Examples:
#   x86_64-pc-linux-gnu
#   x86_64-apple-darwin21
#   aarch64-unknown-linux-gnu
#   i686-w64-mingw32

# Now we can check the OS:
case $host_os in
  linux*)
    # Linux-specific code
    AC_DEFINE([OS_LINUX], [1], [Linux])
    ;;
  darwin*)
    # macOS-specific code
    AC_DEFINE([OS_DARWIN], [1], [macOS])
    # macOS uses different SSL library
    LIBS="$LIBS -framework CoreFoundation -framework Security"
    ;;
  mingw*)
    # Windows-specific code
    AC_DEFINE([OS_WINDOWS], [1], [Windows])
    LIBS="$LIBS -lws2_32"  # Windows sockets
    ;;
esac
\end{lstlisting}

\subsection{Feature Detection - The Heart of Configure}

\begin{lstlisting}
# Check for header files
AC_CHECK_HEADERS([
  sys/socket.h
  netinet/in.h
  arpa/inet.h
  sys/select.h
  sys/epoll.h
  sys/event.h
  windows.h
])

# What this does:
# For each header, generates and compiles:
cat > conftest.c << EOF
#include <sys/socket.h>
int main(void) { return 0; }
EOF
gcc -c conftest.c 2>/dev/null
# If success: #define HAVE_SYS_SOCKET_H 1
# If failure: /* #undef HAVE_SYS_SOCKET_H */

# Check for functions
AC_CHECK_FUNCS([
  socket
  select
  poll
  epoll_create
  kqueue
  strlcpy
  strlcat
  getaddrinfo
  gethostbyname
])

# For each function:
cat > conftest.c << EOF
int main(void) {
  void *p = (void*)socket;
  return (int)(long)p;
}
EOF
gcc conftest.c -o conftest 2>/dev/null
# If links successfully: #define HAVE_SOCKET 1

# Check for libraries
AC_CHECK_LIB([z], [inflate], [
  AC_DEFINE([HAVE_ZLIB], [1], [zlib available])
  LIBS="$LIBS -lz"
], [
  AC_MSG_WARN([zlib not found - compression disabled])
])

# What this does:
cat > conftest.c << EOF
extern int inflate();
int main(void) { inflate(); return 0; }
EOF
gcc conftest.c -lz -o conftest 2>/dev/null
# If links: HAVE_ZLIB=1, add -lz to LIBS
\end{lstlisting}

\subsection{Optional Features (--enable / --disable)}

\begin{lstlisting}
# Add a --enable-debug option
AC_ARG_ENABLE([debug],
  [AS_HELP_STRING([--enable-debug],
    [Enable debug build (default: no)])],
  [enable_debug=$enableval],
  [enable_debug=no])

# This creates:
# ./configure --enable-debug    -> enable_debug=yes
# ./configure --disable-debug   -> enable_debug=no
# ./configure (no flag)         -> enable_debug=no (default)

# Use the option:
if test "x$enable_debug" = "xyes"; then
  CFLAGS="$CFLAGS -g -O0 -DDEBUG"
  AC_DEFINE([DEBUG_BUILD], [1], [Debug build])
else
  CFLAGS="$CFLAGS -O2 -DNDEBUG"
fi

# Real example from curl - IPv6 support:
AC_ARG_ENABLE([ipv6],
  [AS_HELP_STRING([--enable-ipv6],
    [Enable IPv6 support (default: auto)])],
  [enable_ipv6=$enableval],
  [enable_ipv6=auto])

if test "x$enable_ipv6" != "xno"; then
  # Try to compile IPv6 test program
  AC_MSG_CHECKING([for IPv6 support])
  AC_COMPILE_IFELSE([AC_LANG_PROGRAM([[
    #include <sys/socket.h>
    #include <netinet/in.h>
  ]], [[
    struct sockaddr_in6 sa;
    sa.sin6_family = AF_INET6;
  ]])], [
    AC_MSG_RESULT([yes])
    AC_DEFINE([ENABLE_IPV6], [1], [IPv6 enabled])
    have_ipv6=yes
  ], [
    AC_MSG_RESULT([no])
    if test "x$enable_ipv6" = "xyes"; then
      AC_MSG_ERROR([IPv6 requested but not available])
    fi
    have_ipv6=no
  ])
fi
\end{lstlisting}

\subsection{External Dependencies (--with / --without)}

\begin{lstlisting}
# SSL library selection
AC_ARG_WITH([ssl],
  [AS_HELP_STRING([--with-ssl=PATH],
    [Use OpenSSL (in PATH)])],
  [want_ssl=$withval],
  [want_ssl=yes])

if test "x$want_ssl" != "xno"; then
  # If path specified, look there first
  if test "x$want_ssl" != "xyes"; then
    CPPFLAGS="$CPPFLAGS -I$want_ssl/include"
    LDFLAGS="$LDFLAGS -L$want_ssl/lib"
  fi

  # Try pkg-config first
  PKG_CHECK_MODULES([OPENSSL], [openssl >= 1.0.0], [
    AC_DEFINE([HAVE_OPENSSL], [1], [OpenSSL available])
    LIBS="$LIBS $OPENSSL_LIBS"
    CFLAGS="$CFLAGS $OPENSSL_CFLAGS"
    have_ssl=yes
  ], [
    # pkg-config failed, try manual detection
    AC_CHECK_HEADERS([openssl/ssl.h], [
      AC_CHECK_LIB([ssl], [SSL_connect], [
        AC_DEFINE([HAVE_OPENSSL], [1])
        LIBS="$LIBS -lssl -lcrypto"
        have_ssl=yes
      ], [
        have_ssl=no
      ], [-lcrypto])
    ], [
      have_ssl=no
    ])
  ])

  # If required but not found, error
  if test "x$have_ssl" = "xno" && test "x$want_ssl" = "xyes"; then
    AC_MSG_ERROR([
      OpenSSL not found. Install libssl-dev or use:
      --without-ssl     (disable SSL support)
      --with-ssl=PATH   (specify OpenSSL location)
    ])
  fi
fi
\end{lstlisting}

\subsection{Generating Output Files}

\begin{lstlisting}
# List all files to generate
AC_CONFIG_FILES([
  Makefile
  lib/Makefile
  src/Makefile
  tests/Makefile
  docs/Makefile
  libcurl.pc
])

# Actually generate them
AC_OUTPUT

# At end, configure prints summary:
echo ""
echo "configure: Configured to build curl/libcurl:"
echo ""
echo "  Host setup:       $host"
echo "  Install prefix:   $prefix"
echo "  Compiler:         $CC"
echo "  CFLAGS:          $CFLAGS"
echo "  LDFLAGS:         $LDFLAGS"
echo "  LIBS:            $LIBS"
echo ""
echo "  SSL support:     $have_ssl"
echo "  IPv6 support:    $have_ipv6"
echo "  HTTP support:    yes"
echo "  HTTPS support:   $have_ssl"
echo ""
\end{lstlisting}

\textbf{The complete flow}:

\begin{enumerate}
    \item Developer writes configure.ac (~3,000 lines of M4 macros)
    \item \texttt{autoconf} reads configure.ac
    \item \texttt{autoconf} generates configure (~40,000 lines of shell script)
    \item Developer distributes configure (users don't need autoconf!)
    \item User runs \texttt{./configure}
    \item configure tests the system
    \item configure generates Makefile and config.h
    \item User runs \texttt{make}
\end{enumerate}

\subsection{Autotools: The Full Pipeline}

Autotools is actually three tools that work together:

\begin{lstlisting}
# 1. autoconf - generates configure script
autoconf
# Reads: configure.ac
# Generates: configure

# 2. automake - generates Makefile.in templates
automake --add-missing
# Reads: Makefile.am
# Generates: Makefile.in

# 3. configure - generates actual Makefiles
./configure
# Reads: Makefile.in, config.h.in
# Generates: Makefile, config.h

# The developer workflow:
# Write configure.ac and Makefile.am
# Run: autoreconf -i (runs autoconf + automake)
# Distribute: configure script (users don't need autotools!)
# Users run: ./configure && make
\end{lstlisting}

\subsection{Simple configure.ac Example}

This is what project maintainers write:

\begin{lstlisting}
# configure.ac - Input for autoconf
AC_INIT([myproject], [1.0.0], [bug-report@example.com])
AM_INIT_AUTOMAKE([-Wall -Werror foreign])
AC_PROG_CC
AC_CONFIG_HEADERS([config.h])
AC_CONFIG_FILES([Makefile src/Makefile])

# Check for required headers
AC_CHECK_HEADERS([stdlib.h string.h unistd.h])

# Check for required functions
AC_CHECK_FUNCS([malloc realloc memset])

# Check for libraries
AC_CHECK_LIB([pthread], [pthread_create])
AC_CHECK_LIB([m], [sqrt])

# Optional features
AC_ARG_ENABLE([debug],
    AS_HELP_STRING([--enable-debug], [Enable debug mode]),
    [enable_debug=yes],
    [enable_debug=no])

AS_IF([test "x$enable_debug" = "xyes"], [
    AC_DEFINE([DEBUG], [1], [Debug mode enabled])
    CFLAGS="$CFLAGS -g -O0"
], [
    CFLAGS="$CFLAGS -O2 -DNDEBUG"
])

# Optional dependencies
AC_ARG_WITH([openssl],
    AS_HELP_STRING([--with-openssl], [Build with OpenSSL support]),
    [],
    [with_openssl=check])

AS_IF([test "x$with_openssl" != "xno"], [
    PKG_CHECK_MODULES([OPENSSL], [openssl >= 1.1.0], [
        AC_DEFINE([HAVE_OPENSSL], [1], [OpenSSL available])
        have_openssl=yes
    ], [
        AS_IF([test "x$with_openssl" = "xyes"], [
            AC_MSG_ERROR([OpenSSL requested but not found])
        ])
        have_openssl=no
    ])
])

AC_OUTPUT

# Summary message
echo ""
echo "Configuration summary:"
echo "  Prefix: $prefix"
echo "  Debug mode: $enable_debug"
echo "  OpenSSL: $have_openssl"
echo ""
\end{lstlisting}

\subsection{Makefile.am - Automake Input}

Much simpler than raw Makefiles:

\begin{lstlisting}
# Makefile.am - High-level description
bin_PROGRAMS = myprogram
myprogram_SOURCES = main.c utils.c parser.c
myprogram_CFLAGS = $(OPENSSL_CFLAGS)
myprogram_LDADD = $(OPENSSL_LIBS) -lpthread

# Build a library
lib_LTLIBRARIES = libmylib.la
libmylib_la_SOURCES = lib.c helper.c
libmylib_la_LDFLAGS = -version-info 1:0:0

# Install headers
include_HEADERS = mylib.h

# Subdirectories
SUBDIRS = src tests docs

# Extra files to distribute
EXTRA_DIST = README.md LICENSE example.conf

# Tests
TESTS = tests/test_basic tests/test_advanced
check_PROGRAMS = $(TESTS)
\end{lstlisting}

\textbf{Automake magic variables}:
\begin{itemize}
    \item \texttt{bin\_PROGRAMS}: Executables installed to \$prefix/bin
    \item \texttt{lib\_LTLIBRARIES}: Libraries (libtool handles portability)
    \item \texttt{include\_HEADERS}: Headers installed to \$prefix/include
    \item \texttt{\_SOURCES}: Source files
    \item \texttt{\_CFLAGS}: Additional compiler flags
    \item \texttt{\_LDADD}: Libraries to link
\end{itemize}

\subsection{Generated config.h}

Configuration results go into config.h:

\begin{lstlisting}
/* config.h.in - Template */
#undef HAVE_STDLIB_H
#undef HAVE_PTHREAD
#undef HAVE_OPENSSL
#undef DEBUG
#define VERSION "@VERSION@"
#define PACKAGE "@PACKAGE@"

/* config.h - Generated by configure */
#define HAVE_STDLIB_H 1
#define HAVE_PTHREAD 1
#define HAVE_OPENSSL 1
/* #undef DEBUG */
#define VERSION "1.0.0"
#define PACKAGE "myproject"

/* Usage in code */
#include "config.h"

#ifdef HAVE_OPENSSL
    #include <openssl/ssl.h>
    // Use OpenSSL
#endif

#ifdef DEBUG
    #define LOG(fmt, ...) fprintf(stderr, fmt, ##__VA_ARGS__)
#else
    #define LOG(fmt, ...)
#endif
\end{lstlisting}

\section{pkg-config Deep Dive}

pkg-config solves the "where are the libraries?" problem. Every library installs a .pc file describing itself.

\subsection{Understanding .pc Files}

\begin{lstlisting}
# /usr/lib/pkgconfig/openssl.pc
prefix=/usr
exec_prefix=${prefix}
libdir=${exec_prefix}/lib
includedir=${prefix}/include

Name: OpenSSL
Description: Secure Sockets Layer and cryptography libraries
Version: 1.1.1
Requires: libcrypto libssl
Libs: -L${libdir} -lssl -lcrypto
Libs.private: -ldl -lpthread
Cflags: -I${includedir}

# Query it:
pkg-config --cflags openssl
# Output: -I/usr/include

pkg-config --libs openssl
# Output: -L/usr/lib -lssl -lcrypto

pkg-config --libs --static openssl
# Output: -L/usr/lib -lssl -lcrypto -ldl -lpthread

pkg-config --modversion openssl
# Output: 1.1.1

pkg-config --exists openssl && echo "Found"
# Output: Found
\end{lstlisting}

\subsection{Using pkg-config in Makefiles}

\begin{lstlisting}
# Find packages
PKG_CONFIG ?= pkg-config

# Check if package exists
ifeq ($(shell $(PKG_CONFIG) --exists gtk+-3.0 && echo yes),yes)
    HAS_GTK = 1
    GTK_CFLAGS = $(shell $(PKG_CONFIG) --cflags gtk+-3.0)
    GTK_LIBS = $(shell $(PKG_CONFIG) --libs gtk+-3.0)
else
    HAS_GTK = 0
    GTK_CFLAGS =
    GTK_LIBS =
endif

# Use the flags
program: main.c
	$(CC) $(CFLAGS) $(GTK_CFLAGS) main.c -o program $(GTK_LIBS)

# Require minimum version
REQUIRED_VERSION = 3.20
ifeq ($(shell $(PKG_CONFIG) --atleast-version=$(REQUIRED_VERSION) gtk+-3.0 && echo yes),yes)
    $(info GTK+ version OK)
else
    $(error GTK+ >= $(REQUIRED_VERSION) required)
endif
\end{lstlisting}

\subsection{Creating Your Own .pc File}

When building a library, install a .pc file:

\begin{lstlisting}
# mylib.pc.in - Template
prefix=@prefix@
exec_prefix=@exec_prefix@
libdir=@libdir@
includedir=@includedir@

Name: MyLib
Description: My utility library
URL: https://example.com/mylib
Version: @VERSION@
Requires: zlib >= 1.2.0
Requires.private: openssl
Libs: -L${libdir} -lmylib
Libs.private: -lm
Cflags: -I${includedir}

# configure.ac substitutes @variables@
AC_CONFIG_FILES([mylib.pc])

# Makefile.am installs it
pkgconfigdir = $(libdir)/pkgconfig
pkgconfig_DATA = mylib.pc

# After installation, users can:
pkg-config --cflags --libs mylib
\end{lstlisting}

\section{Real Project Structure Explained}

Let's dissect a typical open-source C project:

\begin{lstlisting}
project/
+-- autogen.sh           # Bootstrap script (runs autotools)
+-- configure.ac         # Autoconf input
+-- Makefile.am          # Top-level Automake input
+-- config.h.in          # Config header template
+-- m4/                  # Custom autoconf macros
    +-- my_checks.m4
+-- src/
    +-- Makefile.am      # Source directory Automake input
    +-- main.c
    +-- utils.c
+-- include/
    +-- myproject.h.in   # Header template (version substitution)
+-- lib/                 # Library code
    +-- Makefile.am
    +-- libmylib.c
+-- tests/
    +-- Makefile.am
    +-- test_main.c
+-- docs/
    +-- Makefile.am
    +-- manual.md
+-- scripts/             # Helper scripts
    +-- build.sh         # Convenience build script
    +-- install-deps.sh  # Install dependencies
+-- .github/
    +-- workflows/
        +-- ci.yml       # GitHub Actions CI
+-- README.md
+-- LICENSE
+-- NEWS                 # Changelog (autotools convention)
+-- AUTHORS              # Contributors
+-- INSTALL              # Installation instructions

# Generated files (not in git):
+-- configure            # Generated by autoconf
+-- Makefile.in          # Generated by automake
+-- config.status        # Records configuration
+-- config.log           # Detailed test log
+-- Makefile             # Generated by configure
+-- config.h             # Generated by configure
+-- build/               # Out-of-tree build directory
+-- .deps/               # Dependency files
\end{lstlisting}

\subsection{The autogen.sh Bootstrap Script}

Many projects have autogen.sh to regenerate autotools files:

\begin{lstlisting}
#!/bin/sh
# autogen.sh - Regenerate autotools files

set -e  # Exit on error

echo "Bootstrapping build system..."

# Check for required tools
for tool in autoconf automake libtool; do
    if ! command -v $tool >/dev/null 2>&1; then
        echo "Error: $tool not found"
        exit 1
    fi
done

# Create m4 directory if needed
mkdir -p m4

# Copy auxiliary files
echo "Running libtoolize..."
libtoolize --copy --force

echo "Running aclocal..."
aclocal -I m4

echo "Running autoheader..."
autoheader

echo "Running automake..."
automake --add-missing --copy --foreign

echo "Running autoconf..."
autoconf

echo ""
echo "Bootstrap complete. Now run:"
echo "  ./configure"
echo "  make"
\end{lstlisting}

\subsection{The build.sh Convenience Script}

\begin{lstlisting}
#!/bin/bash
# build.sh - One-command build

set -e

# Configuration
PREFIX=${PREFIX:-/usr/local}
BUILD_TYPE=${BUILD_TYPE:-release}

# Colors for output
RED='\033[0;31m'
GREEN='\033[0;32m'
YELLOW='\033[1;33m'
NC='\033[0m' # No Color

info() {
    echo -e "${GREEN}[INFO]${NC} $*"
}

error() {
    echo -e "${RED}[ERROR]${NC} $*"
    exit 1
}

warn() {
    echo -e "${YELLOW}[WARN]${NC} $*"
}

# Check dependencies
info "Checking dependencies..."
for pkg in openssl zlib; do
    if ! pkg-config --exists $pkg; then
        error "Required package not found: $pkg"
    fi
done

# Clean if requested
if [ "$1" = "clean" ]; then
    info "Cleaning build artifacts..."
    make clean 2>/dev/null || true
    rm -rf build/
    info "Clean complete"
    exit 0
fi

# Bootstrap if needed
if [ ! -f configure ]; then
    info "Running autogen.sh..."
    ./autogen.sh
fi

# Create build directory
BUILD_DIR="build-$BUILD_TYPE"
mkdir -p "$BUILD_DIR"
cd "$BUILD_DIR"

# Configure
info "Configuring..."
CONFIG_FLAGS="--prefix=$PREFIX"

case $BUILD_TYPE in
    debug)
        CONFIG_FLAGS="$CONFIG_FLAGS --enable-debug"
        ;;
    release)
        CONFIG_FLAGS="$CONFIG_FLAGS --disable-debug"
        ;;
    *)
        error "Unknown build type: $BUILD_TYPE"
        ;;
esac

../configure $CONFIG_FLAGS

# Build
info "Building with $(nproc) parallel jobs..."
make -j$(nproc)

# Test
info "Running tests..."
make check

info "Build successful!"
echo ""
echo "To install:"
echo "  cd $BUILD_DIR && sudo make install"
\end{lstlisting}

\section{Conditional Compilation Patterns}

Real projects compile differently based on OS, architecture, and features.

\subsection{Platform Detection}

\begin{lstlisting}
# In configure.ac
AC_CANONICAL_HOST

case $host_os in
    linux*)
        AC_DEFINE([OS_LINUX], [1], [Linux OS])
        PLATFORM=linux
        ;;
    darwin*)
        AC_DEFINE([OS_MACOS], [1], [macOS])
        PLATFORM=macos
        ;;
    mingw* | msys*)
        AC_DEFINE([OS_WINDOWS], [1], [Windows])
        PLATFORM=windows
        ;;
    *)
        AC_MSG_ERROR([Unsupported OS: $host_os])
        ;;
esac

AC_SUBST([PLATFORM])

# In code (config.h defines these)
#ifdef OS_LINUX
    #include <linux/version.h>
    // Linux-specific code
#elif defined(OS_MACOS)
    #include <TargetConditionals.h>
    // macOS-specific code
#elif defined(OS_WINDOWS)
    #include <windows.h>
    // Windows-specific code
#endif
\end{lstlisting}

\subsection{Feature Detection}

\begin{lstlisting}
# configure.ac - Test if functions exist
AC_CHECK_FUNCS([clock_gettime])
AC_CHECK_FUNCS([pthread_setname_np])
AC_CHECK_FUNCS([strdup strndup])

# Check if struct has member
AC_CHECK_MEMBER([struct stat.st_mtim],
    [AC_DEFINE([HAVE_STAT_MTIM], [1], [struct stat has st_mtim])],
    [],
    [#include <sys/stat.h>])

# Test code compilation
AC_MSG_CHECKING([for C11 _Thread_local])
AC_COMPILE_IFELSE([AC_LANG_PROGRAM([[
    _Thread_local int x;
]], [[
    x = 42;
]])], [
    AC_MSG_RESULT([yes])
    AC_DEFINE([HAVE_THREAD_LOCAL], [1], [C11 thread_local available])
], [
    AC_MSG_RESULT([no])
])

# Usage in code
#ifdef HAVE_CLOCK_GETTIME
    struct timespec ts;
    clock_gettime(CLOCK_MONOTONIC, &ts);
#else
    // Fallback implementation
    struct timeval tv;
    gettimeofday(&tv, NULL);
#endif
\end{lstlisting}

\subsection{Conditional Source Compilation}

\begin{lstlisting}
# Makefile.am - Conditional sources
myprogram_SOURCES = main.c utils.c

if HAVE_OPENSSL
myprogram_SOURCES += crypto.c
endif

if OS_LINUX
myprogram_SOURCES += linux_specific.c
endif

if OS_WINDOWS
myprogram_SOURCES += windows_specific.c
endif

# In configure.ac
AM_CONDITIONAL([HAVE_OPENSSL], [test "x$have_openssl" = "xyes"])
AM_CONDITIONAL([OS_LINUX], [test "x$PLATFORM" = "xlinux"])
AM_CONDITIONAL([OS_WINDOWS], [test "x$PLATFORM" = "xwindows"])
\end{lstlisting}

\section{Installation and DESTDIR}

Understanding how \texttt{make install} works is crucial.

\subsection{Standard Installation Directories}

\begin{lstlisting}
# configure --prefix=/usr/local (default)
# Creates these directories:

$prefix/bin              # Executables
$prefix/lib              # Libraries
$prefix/include          # Headers
$prefix/share            # Data files
$prefix/share/man        # Man pages
$prefix/share/doc        # Documentation
$prefix/etc              # Configuration
$prefix/var              # Variable data

# Real paths after ./configure --prefix=/usr/local:
# /usr/local/bin/myprogram
# /usr/local/lib/libmylib.so
# /usr/local/include/mylib.h
# /usr/local/share/myproject/data.txt
\end{lstlisting}

\subsection{DESTDIR for Package Building}

Package builders (RPM, DEB) need to install to a temporary directory:

\begin{lstlisting}
# Normal install:
./configure --prefix=/usr
make
sudo make install
# Installs to /usr/bin/program

# Package building:
./configure --prefix=/usr
make
make install DESTDIR=/tmp/package-root
# Installs to /tmp/package-root/usr/bin/program

# Then package manager creates .deb/.rpm from /tmp/package-root

# In Makefile:
install: all
	install -d $(DESTDIR)$(bindir)
	install -m 755 program $(DESTDIR)$(bindir)/
	install -d $(DESTDIR)$(libdir)
	install -m 644 libmylib.a $(DESTDIR)$(libdir)/
	install -d $(DESTDIR)$(includedir)
	install -m 644 mylib.h $(DESTDIR)$(includedir)/

# Variables:
# bindir = $(prefix)/bin
# libdir = $(prefix)/lib
# includedir = $(prefix)/include
# DESTDIR is prepended to everything
\end{lstlisting}

\subsection{Uninstall Target}

\begin{lstlisting}
# Makefile - Proper uninstall
uninstall:
	rm -f $(DESTDIR)$(bindir)/program
	rm -f $(DESTDIR)$(libdir)/libmylib.a
	rm -f $(DESTDIR)$(includedir)/mylib.h
	rm -rf $(DESTDIR)$(datadir)/myproject

# Automake generates this automatically from install rules
\end{lstlisting}

\section{Embedded Version Information}

Real projects embed version info in binaries.

\begin{lstlisting}
# configure.ac
AC_INIT([myproject], [1.2.3])
AC_SUBST([VERSION], [1.2.3])

# Generate version header
AC_CONFIG_FILES([include/version.h])

# version.h.in
#ifndef VERSION_H
#define VERSION_H

#define PROJECT_VERSION "@VERSION@"
#define VERSION_MAJOR @VERSION_MAJOR@
#define VERSION_MINOR @VERSION_MINOR@
#define VERSION_PATCH @VERSION_PATCH@

// Git commit (if building from git)
#define GIT_COMMIT "@GIT_COMMIT@"

#endif

# Makefile.am - Extract version components
VERSION_MAJOR = $(shell echo $(VERSION) | cut -d. -f1)
VERSION_MINOR = $(shell echo $(VERSION) | cut -d. -f2)
VERSION_PATCH = $(shell echo $(VERSION) | cut -d. -f3)

# Get git commit
GIT_COMMIT = $(shell git rev-parse --short HEAD 2>/dev/null || echo unknown)

# Usage in code
#include "version.h"

void print_version(void) {
    printf("%s version %s (git: %s)\n",
           PROJECT_NAME, PROJECT_VERSION, GIT_COMMIT);
}
\end{lstlisting}

\section{Build Variants}

Real projects support multiple build configurations simultaneously.

\begin{lstlisting}
# Build multiple variants
./configure --prefix=/usr --enable-debug
make
mv src/program src/program-debug

make clean
./configure --prefix=/usr --disable-debug --enable-optimizations
make
mv src/program src/program-release

# Better: use build directories
mkdir build-debug
cd build-debug
../configure --enable-debug
make

cd ..
mkdir build-release
cd build-release
../configure --disable-debug
make

# Now you have both:
# build-debug/src/program
# build-release/src/program
\end{lstlisting}

\section{Common Real-World Patterns}

\subsection{Checking for Optional Features}

\begin{lstlisting}
# Check for readline (for interactive programs)
AC_CHECK_HEADERS([readline/readline.h])
AC_CHECK_LIB([readline], [readline], [
    HAVE_READLINE=yes
    READLINE_LIBS=-lreadline
], [
    HAVE_READLINE=no
    READLINE_LIBS=
])
AC_SUBST([READLINE_LIBS])

# Use in code
#ifdef HAVE_READLINE_READLINE_H
    #include <readline/readline.h>
    char* input = readline("prompt> ");
#else
    char input[256];
    printf("prompt> ");
    fgets(input, sizeof(input), stdin);
#endif
\end{lstlisting}

\subsection{Custom Configure Options}

\begin{lstlisting}
# Add custom configuration options
AC_ARG_ENABLE([profiling],
    AS_HELP_STRING([--enable-profiling], [Enable profiling support]),
    [enable_profiling=$enableval],
    [enable_profiling=no])

AC_ARG_WITH([custom-allocator],
    AS_HELP_STRING([--with-custom-allocator], [Use custom allocator]),
    [use_custom_allocator=yes],
    [use_custom_allocator=no])

AC_ARG_VAR([MAX_THREADS], [Maximum number of threads (default: 16)])
if test -z "$MAX_THREADS"; then
    MAX_THREADS=16
fi

AC_DEFINE_UNQUOTED([MAX_THREADS], [$MAX_THREADS], [Maximum threads])

# Usage:
./configure --enable-profiling --with-custom-allocator MAX_THREADS=32
\end{lstlisting}

\section{Real Project Examples}

Let me show you exactly how different popular C projects handle building.

\subsection{Example 1: Redis (Simple Makefile)}

Redis deliberately avoids autotools for simplicity:

\begin{lstlisting}
# Clone Redis
git clone https://github.com/redis/redis.git
cd redis

# No configure script! Just:
make

# Why? Redis's Makefile is smart:
# redis/Makefile

# Detect OS
uname_S := $(shell uname -s)

# Platform-specific settings
ifeq ($(uname_S),Linux)
    CFLAGS += -DHAVE_EPOLL
    LDFLAGS += -ldl -pthread
endif
ifeq ($(uname_S),Darwin)
    CFLAGS += -DHAVE_KQUEUE
endif
ifeq ($(uname_S),FreeBSD)
    CFLAGS += -DHAVE_KQUEUE
    LDFLAGS += -lpthread
endif

# Auto-detect dependencies
ifeq ($(shell pkg-config --exists openssl && echo yes),yes)
    CFLAGS += $(shell pkg-config --cflags openssl)
    LDFLAGS += $(shell pkg-config --libs openssl)
endif

# Build
redis-server: redis.o networking.o ...
	$(CC) -o $@ $^ $(LDFLAGS)

# Simple and works!
# Trade-off: Less portable than autotools
# Works for Redis because they control dependencies
\end{lstlisting}

\subsection{Example 2: SQLite (Amalgamation Build)}

SQLite uses a clever trick---ship all code in ONE file:

\begin{lstlisting}
# Download SQLite
wget https://sqlite.org/2023/sqlite-amalgamation-3400000.zip
unzip sqlite-amalgamation-3400000.zip
cd sqlite-amalgamation-3400000

# Contents:
ls
sqlite3.c    # ALL SQLite code in one file (240,000 lines!)
sqlite3.h    # Public header
shell.c      # Command-line tool

# Build is trivial:
gcc -O2 -o sqlite3 shell.c sqlite3.c -lpthread -ldl

# Why this works:
# - No build system needed
# - No dependencies
# - Compiles everywhere
# - Users can't mess up the build

# The "amalgamation" is generated from 100+ source files:
# (developers work on separate files, release as one file)
\end{lstlisting}

\subsection{Example 3: Git (Autoconf Optional)}

Git supports both autotools AND manual configuration:

\begin{lstlisting}
# Clone Git
git clone https://github.com/git/git.git
cd git

# Method 1: Manual configuration
make configure
./configure
make

# Method 2: Direct make (tries to auto-detect)
make

# How? Git's Makefile detects features:
# Makefile
ifeq ($(shell echo '\#include <openssl/ssl.h>' | gcc -E - 2>/dev/null | grep -c ssl.h),1)
    OPENSSL_AVAIL = YesPlease
endif

ifdef OPENSSL_AVAIL
    BASIC_CFLAGS += -DHAVE_OPENSSL
    EXTLIBS += -lssl -lcrypto
endif

# Clever: Works without configure, but configure available if needed
\end{lstlisting}

\subsection{Example 4: nginx (Hand-Written Configure)}

nginx has a custom configure script (NOT autotools):

\begin{lstlisting}
# Clone nginx
git clone https://github.com/nginx/nginx.git
cd nginx

# Configure with custom script:
./auto/configure \
    --prefix=/usr/local/nginx \
    --with-http_ssl_module \
    --with-pcre

# What's different from autotools?
# auto/configure is a HAND-WRITTEN shell script
# Specifically tailored for nginx
# Simpler than autotools but less portable

# Why nginx does this:
# - Full control over build process
# - Optimized for web server needs
# - Handles module system elegantly
# - Simpler for nginx developers

# Inside auto/configure:
#!/bin/sh

# Detect compiler
if [ -n "$CC" ]; then
    echo "using $CC compiler"
else
    if [ -x /usr/bin/gcc ]; then
        CC=gcc
    elif [ -x /usr/bin/cc ]; then
        CC=cc
    fi
fi

# Check for OpenSSL
if [ -f /usr/include/openssl/ssl.h ]; then
    OPENSSL_FOUND=YES
    OPENSSL_CFLAGS="-I/usr/include"
    OPENSSL_LIBS="-lssl -lcrypto"
fi

# Generate Makefile
cat > Makefile << END
CC = $CC
CFLAGS = $CFLAGS $OPENSSL_CFLAGS
LIBS = $LIBS $OPENSSL_LIBS

nginx: ngx_main.o ngx_event.o ...
	\$(CC) -o nginx \$^ \$(LIBS)
END
\end{lstlisting}

\section{The Packaging Perspective}

When distributions (Debian, Fedora, Arch) package your software, they need:

\subsection{Debian Package Build}

\begin{lstlisting}
# How Debian builds curl package:

# 1. Download source
wget https://curl.se/download/curl-7.88.0.tar.gz
tar xzf curl-7.88.0.tar.gz
cd curl-7.88.0

# 2. Configure for Debian's standards
./configure \
    --prefix=/usr \
    --sysconfdir=/etc \
    --localstatedir=/var \
    --mandir=/usr/share/man \
    --enable-shared \
    --disable-static \
    --with-openssl \
    --with-ca-bundle=/etc/ssl/certs/ca-certificates.crt

# 3. Build
make -j$(nproc)

# 4. Install to temporary directory
make install DESTDIR=$PWD/debian/tmp

# 5. Create .deb package
dpkg-deb --build debian/tmp curl_7.88.0-1_amd64.deb

# Now users can:
apt install ./curl_7.88.0-1_amd64.deb
\end{lstlisting}

\subsection{Why DESTDIR Matters}

\begin{lstlisting}
# Without DESTDIR (WRONG for packaging):
./configure --prefix=/usr
make
make install
# Installs directly to /usr/bin/curl
# Can't build packages this way!

# With DESTDIR (RIGHT for packaging):
./configure --prefix=/usr
make
make install DESTDIR=/tmp/package-root
# Installs to /tmp/package-root/usr/bin/curl
# Package manager packages /tmp/package-root/*

# In Makefile, this works because:
install: all
	install -d $(DESTDIR)$(bindir)
	install -m 755 curl $(DESTDIR)$(bindir)/

# bindir = /usr/bin
# DESTDIR = /tmp/package-root
# Full path: /tmp/package-root/usr/bin/curl
\end{lstlisting}

\section{Troubleshooting Real Build Problems}

\subsection{Problem 1: "configure: error: OpenSSL not found"}

\begin{lstlisting}
# Error during configure:
./configure
checking for openssl/ssl.h... no
configure: error: OpenSSL development files not found

# Why? Missing development headers
# Solution depends on distro:

# Ubuntu/Debian:
sudo apt-get install libssl-dev

# Fedora/RHEL:
sudo dnf install openssl-devel

# macOS:
brew install openssl
# macOS keeps OpenSSL in non-standard location:
./configure --with-ssl=$(brew --prefix openssl)

# Now configure finds it:
checking for openssl/ssl.h... yes
checking for SSL_connect in -lssl... yes
\end{lstlisting}

\subsection{Problem 2: "undefined reference to `pthread\_create'"}

\begin{lstlisting}
# Error during linking:
gcc -o program main.o -lssl -lcrypto
main.o: undefined reference to `pthread_create'

# Why? Missing -lpthread

# Solution: Add to LDFLAGS
./configure LDFLAGS="-lpthread"

# Or in Makefile:
LDFLAGS += -lpthread
\end{lstlisting}

\subsection{Problem 3: "cannot find -lz"}

\begin{lstlisting}
# Error:
/usr/bin/ld: cannot find -lz

# Why? libz.so not in standard path

# Find it:
find /usr -name "libz.so*"
# Found: /usr/local/lib/libz.so

# Solution 1: Tell linker where to look
./configure LDFLAGS="-L/usr/local/lib"

# Solution 2: Add to library path
export LD_LIBRARY_PATH=/usr/local/lib
./configure

# Solution 3: Install system package
sudo apt-get install zlib1g-dev
\end{lstlisting}

\section{Summary}

Build systems are crucial for productive C development. Now you understand:

\begin{itemize}
    \item \textbf{Why configure exists}: Solves Unix portability nightmare
    \item \textbf{What configure does}: Tests system, generates Makefiles
    \item \textbf{configure.ac -> configure}: autoconf generates 40K line shell script
    \item \textbf{Makefile.in -> Makefile}: configure fills in variables
    \item \textbf{config.h}: Stores feature detection results
    \item \textbf{Make}: Tracks dependencies, rebuilds what changed
    \item \textbf{pkg-config}: Finds libraries via .pc files
    \item \textbf{DESTDIR}: Enables package building
    \item \textbf{Installation paths}: prefix, bindir, libdir, etc.
\end{itemize}

\textbf{The complete picture}:

\begin{verbatim}
DEVELOPER WORKFLOW:
1. Write configure.ac (3,000 lines of M4 macros)
2. Write Makefile.am (high-level build description)
3. Run autoconf -> generates configure (40,000 lines of shell)
4. Run automake -> generates Makefile.in (template)
5. Commit configure to git (users don't need autotools)
6. Create tarball: tar czf project-1.0.tar.gz ...

USER WORKFLOW:
1. Download tarball
2. tar xzf project-1.0.tar.gz
3. cd project-1.0
4. ./configure (tests system, generates Makefile)
5. make (compiles code)
6. make check (runs tests)
7. sudo make install (copies to /usr/local)

PACKAGER WORKFLOW (Debian/Fedora):
1. ./configure --prefix=/usr
2. make
3. make install DESTDIR=/tmp/staging
4. Package /tmp/staging/* into .deb/.rpm
5. Users install via: apt/dnf install package
\end{verbatim}

\textbf{Why each step matters}:

\begin{itemize}
    \item \textbf{configure}: Can't hardcode paths---every system is different
    \item \textbf{config.h}: Runtime checks for missing functions
    \item \textbf{Makefile generation}: Different flags per system
    \item \textbf{pkg-config}: Libraries install in different locations
    \item \textbf{DESTDIR}: Can't install to /usr during package build
    \item \textbf{Dependency tracking}: Don't recompile unchanged files
\end{itemize}

\textbf{Alternative approaches}:

\begin{itemize}
    \item \textbf{Simple Makefile} (Redis): If you control dependencies
    \item \textbf{Amalgamation} (SQLite): Ship as single .c file
    \item \textbf{Custom configure} (nginx): Hand-written for your needs
    \item \textbf{CMake/Meson}: Modern alternatives to autotools
\end{itemize}

\textbf{When you see confusing builds now}:

\begin{lstlisting}
# curl
./buildconf     # Generate configure (developer only)
./configure     # Test system
make            # Build
make install    # Install

# Why buildconf? Generates configure from configure.ac
# Why configure? Detects OpenSSL, zlib, platform differences
# Why make? Compiles with detected settings
\end{lstlisting}

\textbf{Key insight}: Real C projects aren't complex to be difficult---they're complex because they solve REAL problems (portability across dozens of Unix variants, optional dependencies, graceful degradation when features missing).

Every confusing part has a reason:
\begin{itemize}
    \item configure.ac exists because hardcoded paths break
    \item config.h exists because functions differ per system
    \item Makefile.in is a template because flags vary per platform
    \item pkg-config exists because library locations vary
    \item DESTDIR exists because packagers need staging directories
\end{itemize}

Now when you clone a C project, you understand the build system isn't arbitrary complexity---it's battle-tested solutions to decades of portability problems!

\chapter{Performance Patterns: 50 Years of C Optimization Tricks}

\section{Introduction: The Pursuit of Speed}

C has been the language of choice for performance-critical systems for over 50 years. During this time, programmers have discovered countless tricks, idioms, and patterns to squeeze every last cycle out of the hardware. This chapter collects the wisdom of generations of C programmers—from the early days of PDP-11s to modern multi-core processors with complex memory hierarchies.

\begin{tipbox}
\textbf{The Golden Rule:} Profile first, optimize second. Measure everything. Your intuition about performance is probably wrong.
\end{tipbox}

\section{Understanding Modern CPU Architecture}

Before diving into tricks, understand what makes modern CPUs fast:

\begin{lstlisting}
// CPU speed hierarchy (approximate latencies):
// Register access:     0-1 cycles
// L1 cache:            4 cycles
// L2 cache:            12 cycles
// L3 cache:            38 cycles
// Main RAM:            100-300 cycles
// SSD:                 50,000-150,000 cycles
// Network (LAN):       millions of cycles

// This means: cache misses kill performance!
// A single cache miss can cost 100+ instructions worth of time
\end{lstlisting}

Modern CPUs have:
\begin{itemize}
    \item \textbf{Pipelining:} Multiple instructions in flight simultaneously
    \item \textbf{Branch prediction:} Guesses which way branches go
    \item \textbf{Out-of-order execution:} Runs instructions when data is ready, not in order
    \item \textbf{Speculative execution:} Executes both paths of a branch
    \item \textbf{SIMD:} Single Instruction Multiple Data parallelism
    \item \textbf{Prefetching:} Loads data before it's needed
\end{itemize}

\section{Cache-Friendly Programming}

\subsection{The Power of Sequential Access}

\begin{lstlisting}
// Example: Processing 1 million integers
// Sequential access: ~3ms
// Random access: ~300ms (100x slower!)

// Bad: Pointer chasing (cache miss every access)
typedef struct Node {
    int data;
    struct Node* next;
} Node;

void sum_list(Node* head) {
    long sum = 0;
    for (Node* n = head; n; n = n->next) {
        sum += n->data;  // Each access is likely a cache miss
    }
}

// Good: Array (stays in cache)
void sum_array(int* arr, size_t len) {
    long sum = 0;
    for (size_t i = 0; i < len; i++) {
        sum += arr[i];  // Prefetcher loads next cache line
    }
}
\end{lstlisting}

\subsection{Array of Structs vs Struct of Arrays}

This is one of the most important optimization patterns:

\begin{lstlisting}
// Array of Structs (AoS) - typical object-oriented layout
typedef struct {
    float x, y, z;     // Position: 12 bytes
    float r, g, b, a;  // Color: 16 bytes
    float nx, ny, nz;  // Normal: 12 bytes
    float u, v;        // Texture coords: 8 bytes
} Vertex;  // Total: 48 bytes

Vertex vertices[10000];

// Process only positions - loads ALL 48 bytes per vertex!
for (int i = 0; i < 10000; i++) {
    vertices[i].x += 1.0f;
    // Also loads color, normal, UV (wasted bandwidth)
}

// Struct of Arrays (SoA) - data-oriented layout
typedef struct {
    float* x;
    float* y;
    float* z;
    float* r;
    float* g;
    float* b;
    float* a;
    float* nx;
    float* ny;
    float* nz;
    float* u;
    float* v;
    size_t count;
} VertexArray;

// Initialize SoA
VertexArray* create_vertices(size_t count) {
    VertexArray* va = malloc(sizeof(VertexArray));
    va->count = count;
    va->x = malloc(count * sizeof(float));
    va->y = malloc(count * sizeof(float));
    // ... allocate other fields
    return va;
}

// Process only positions - loads ONLY position data!
for (size_t i = 0; i < va->count; i++) {
    va->x[i] += 1.0f;
    // Perfect cache utilization
}

// Hybrid approach: "Chunked" SoA
#define CHUNK_SIZE 64
typedef struct {
    float x[CHUNK_SIZE];
    float y[CHUNK_SIZE];
    float z[CHUNK_SIZE];
} PositionChunk;

typedef struct {
    float r[CHUNK_SIZE];
    float g[CHUNK_SIZE];
    float b[CHUNK_SIZE];
} ColorChunk;

// Now positions and colors are separate, but each is contiguous
// Good cache locality + reasonable memory layout
\end{lstlisting}

\subsection{Cache Line Alignment and False Sharing}

\begin{lstlisting}
// Cache lines are typically 64 bytes
#define CACHE_LINE_SIZE 64

// False sharing: Different threads accessing different variables
// in the same cache line causes cache thrashing
typedef struct {
    int counter1;  // Thread 1 updates this
    int counter2;  // Thread 2 updates this
} BadCounters;  // Both in same cache line - constant invalidation!

// Fix: Align each counter to its own cache line
typedef struct {
    alignas(64) int counter1;
    char pad1[CACHE_LINE_SIZE - sizeof(int)];
    alignas(64) int counter2;
    char pad2[CACHE_LINE_SIZE - sizeof(int)];
} GoodCounters;

// Or use compiler attribute
typedef struct {
    int counter1;
} __attribute__((aligned(64))) AlignedCounter;

// Prefetch next cache line in advance
for (size_t i = 0; i < n; i++) {
    __builtin_prefetch(&data[i + 8], 0, 3);  // Prefetch 8 ahead
    process(data[i]);
}
\end{lstlisting}

\subsection{Loop Blocking (Tiling) for Cache}

Classic technique from BLAS/LAPACK libraries:

\begin{lstlisting}
// Matrix multiplication: naive version
// Poor cache usage for large matrices
void matmul_naive(float** A, float** B, float** C, int n) {
    for (int i = 0; i < n; i++) {
        for (int j = 0; j < n; j++) {
            float sum = 0;
            for (int k = 0; k < n; k++) {
                sum += A[i][k] * B[k][j];  // B accessed non-sequentially
            }
            C[i][j] = sum;
        }
    }
}

// Blocked version: process in cache-sized tiles
#define BLOCK_SIZE 32  // Tune for your cache size

void matmul_blocked(float** A, float** B, float** C, int n) {
    // Zero output
    for (int i = 0; i < n; i++)
        for (int j = 0; j < n; j++)
            C[i][j] = 0;

    // Process in blocks
    for (int ii = 0; ii < n; ii += BLOCK_SIZE) {
        for (int jj = 0; jj < n; jj += BLOCK_SIZE) {
            for (int kk = 0; kk < n; kk += BLOCK_SIZE) {
                // Multiply block
                int i_max = (ii + BLOCK_SIZE < n) ? ii + BLOCK_SIZE : n;
                int j_max = (jj + BLOCK_SIZE < n) ? jj + BLOCK_SIZE : n;
                int k_max = (kk + BLOCK_SIZE < n) ? kk + BLOCK_SIZE : n;

                for (int i = ii; i < i_max; i++) {
                    for (int j = jj; j < j_max; j++) {
                        float sum = C[i][j];
                        for (int k = kk; k < k_max; k++) {
                            sum += A[i][k] * B[k][j];
                        }
                        C[i][j] = sum;
                    }
                }
            }
        }
    }
}
// Speedup: 5-10x for large matrices!
\end{lstlisting}

\section{Branch Prediction and Control Flow}

\subsection{Likely/Unlikely Hints}

\begin{lstlisting}
// Branch prediction helps, but you can guide the CPU
#define likely(x)   __builtin_expect(!!(x), 1)
#define unlikely(x) __builtin_expect(!!(x), 0)

// Use for error handling
if (unlikely(ptr == NULL)) {
    // Rare error path
    handle_error();
    return -1;
}
// Common path continues here

// Critical hot loop
while (likely(has_more_data())) {
    process_next();
}

// Real example: Linux kernel uses this everywhere
int copy_from_user(void* to, const void* from, size_t n) {
    if (unlikely(!access_ok(from, n)))
        return -EFAULT;
    return __copy_from_user(to, from, n);
}
\end{lstlisting}

\subsection{Branchless Code}

Sometimes eliminating branches is faster than predicting them:

\begin{lstlisting}
// With branch
int max_with_branch(int a, int b) {
    if (a > b)
        return a;
    else
        return b;
}

// Branchless using ternary (compiler often optimizes this)
int max_branchless(int a, int b) {
    return (a > b) ? a : b;
}

// Branchless using bit tricks
int max_bitwise(int a, int b) {
    int diff = a - b;
    int sign = diff >> 31;  // -1 if a < b, 0 if a >= b
    return a - (diff & sign);
}

// Branchless absolute value
int abs_branch(int x) {
    return x < 0 ? -x : x;  // Branch
}

int abs_branchless(int x) {
    int mask = x >> 31;  // All 1s if negative, all 0s if positive
    return (x + mask) ^ mask;
}

// Branchless min/max for floats (using CMOV instruction)
float fmax_branchless(float a, float b) {
    return a > b ? a : b;  // Compiles to MAXSS on x86
}

// Branchless selection
int select(int condition, int true_val, int false_val) {
    // If condition is 0 or 1
    return false_val + (condition & (true_val - false_val));
}

// Copy if condition is true (branchless)
void conditional_copy(int* dst, int* src, int condition) {
    int mask = -condition;  // 0xFFFFFFFF if true, 0 if false
    *dst = (*dst & ~mask) | (*src & mask);
}
\end{lstlisting}

\subsection{Computed Goto (GCC Extension)}

Much faster than switch for interpreters and VMs:

\begin{lstlisting}
// Traditional switch-based interpreter
enum OpCode { OP_ADD, OP_SUB, OP_MUL, OP_DIV, OP_HALT };

void interpret_switch(uint8_t* bytecode) {
    int pc = 0;
    int stack[256];
    int sp = 0;

    while (1) {
        switch (bytecode[pc++]) {
            case OP_ADD:
                stack[sp - 2] = stack[sp - 2] + stack[sp - 1];
                sp--;
                break;
            case OP_SUB:
                stack[sp - 2] = stack[sp - 2] - stack[sp - 1];
                sp--;
                break;
            case OP_MUL:
                stack[sp - 2] = stack[sp - 2] * stack[sp - 1];
                sp--;
                break;
            case OP_DIV:
                stack[sp - 2] = stack[sp - 2] / stack[sp - 1];
                sp--;
                break;
            case OP_HALT:
                return;
        }
    }
}

// Computed goto version (much faster!)
void interpret_goto(uint8_t* bytecode) {
    static void* dispatch_table[] = {
        &&op_add, &&op_sub, &&op_mul, &&op_div, &&op_halt
    };

    int pc = 0;
    int stack[256];
    int sp = 0;

    #define DISPATCH() goto *dispatch_table[bytecode[pc++]]

    DISPATCH();

op_add:
    stack[sp - 2] = stack[sp - 2] + stack[sp - 1];
    sp--;
    DISPATCH();

op_sub:
    stack[sp - 2] = stack[sp - 2] - stack[sp - 1];
    sp--;
    DISPATCH();

op_mul:
    stack[sp - 2] = stack[sp - 2] * stack[sp - 1];
    sp--;
    DISPATCH();

op_div:
    stack[sp - 2] = stack[sp - 2] / stack[sp - 1];
    sp--;
    DISPATCH();

op_halt:
    return;
}
// Speedup: 20-30% for interpreter dispatch!
// Used by: Python, Ruby, Lua VMs
\end{lstlisting}

\section{Loop Optimization Techniques}

\subsection{Duff's Device}

The most famous loop optimization in C history:

\begin{lstlisting}
// Standard loop to copy n bytes
void copy_standard(char* to, char* from, size_t count) {
    for (size_t i = 0; i < count; i++) {
        *to++ = *from++;
    }
}

// Duff's Device: loop unrolling with switch fallthrough
void copy_duff(char* to, char* from, size_t count) {
    size_t n = (count + 7) / 8;  // Number of 8-byte chunks
    switch (count % 8) {
        case 0: do { *to++ = *from++;
        case 7:      *to++ = *from++;
        case 6:      *to++ = *from++;
        case 5:      *to++ = *from++;
        case 4:      *to++ = *from++;
        case 3:      *to++ = *from++;
        case 2:      *to++ = *from++;
        case 1:      *to++ = *from++;
                } while (--n > 0);
    }
}
// Handles remainder and main loop in one construct!

// Modern version: use memcpy for bulk copies
// But Duff's device shows the principle of unrolling
\end{lstlisting}

\subsection{Loop Unrolling}

\begin{lstlisting}
// Basic loop
void scale_array(float* arr, float factor, size_t n) {
    for (size_t i = 0; i < n; i++) {
        arr[i] *= factor;
    }
}

// Manual unroll by 4
void scale_array_unroll4(float* arr, float factor, size_t n) {
    size_t i = 0;

    // Process 4 elements at a time
    for (; i + 4 <= n; i += 4) {
        arr[i + 0] *= factor;
        arr[i + 1] *= factor;
        arr[i + 2] *= factor;
        arr[i + 3] *= factor;
    }

    // Handle remainder
    for (; i < n; i++) {
        arr[i] *= factor;
    }
}

// Unroll with independent operations (better ILP)
void scale_array_unroll_ilp(float* arr, float factor, size_t n) {
    size_t i = 0;

    for (; i + 4 <= n; i += 4) {
        float a0 = arr[i + 0] * factor;
        float a1 = arr[i + 1] * factor;
        float a2 = arr[i + 2] * factor;
        float a3 = arr[i + 3] * factor;

        arr[i + 0] = a0;
        arr[i + 1] = a1;
        arr[i + 2] = a2;
        arr[i + 3] = a3;
    }

    for (; i < n; i++) {
        arr[i] *= factor;
    }
}

// Pragma for compiler unrolling
void scale_array_pragma(float* arr, float factor, size_t n) {
    #pragma GCC unroll 8
    for (size_t i = 0; i < n; i++) {
        arr[i] *= factor;
    }
}
\end{lstlisting}

\subsection{Loop Fusion and Fission}

\begin{lstlisting}
// Loop fission: split one loop into multiple
// Good when operations can't be pipelined together

// Original: poor instruction-level parallelism
for (int i = 0; i < n; i++) {
    a[i] = b[i] + c[i];
    d[i] = a[i] * 2;  // Depends on previous line
    e[i] = d[i] + 1;  // Depends on previous line
}

// Fissioned: better for some CPUs
for (int i = 0; i < n; i++) {
    a[i] = b[i] + c[i];
}
for (int i = 0; i < n; i++) {
    d[i] = a[i] * 2;
}
for (int i = 0; i < n; i++) {
    e[i] = d[i] + 1;
}

// Loop fusion: combine multiple loops
// Good for cache locality

// Original: multiple passes over data
for (int i = 0; i < n; i++) {
    a[i] = b[i] + 1;
}
for (int i = 0; i < n; i++) {
    c[i] = a[i] * 2;
}
for (int i = 0; i < n; i++) {
    d[i] = c[i] + a[i];
}

// Fused: one pass, better cache usage
for (int i = 0; i < n; i++) {
    a[i] = b[i] + 1;
    c[i] = a[i] * 2;
    d[i] = c[i] + a[i];
}
\end{lstlisting}

\subsection{Loop Interchange}

Change loop order for better cache performance:

\begin{lstlisting}
// Bad: column-major access in row-major array
for (int j = 0; j < N; j++) {
    for (int i = 0; i < M; i++) {
        matrix[i][j] = 0;  // Strided access, cache-unfriendly
    }
}

// Good: row-major access
for (int i = 0; i < M; i++) {
    for (int j = 0; j < N; j++) {
        matrix[i][j] = 0;  // Sequential access, cache-friendly
    }
}

// Matrix transpose: blocked version
void transpose_blocked(float** A, float** B, int n) {
    const int BLOCK = 16;
    for (int i = 0; i < n; i += BLOCK) {
        for (int j = 0; j < n; j += BLOCK) {
            // Transpose block
            for (int ii = i; ii < i + BLOCK && ii < n; ii++) {
                for (int jj = j; jj < j + BLOCK && jj < n; jj++) {
                    B[jj][ii] = A[ii][jj];
                }
            }
        }
    }
}
\end{lstlisting}

\subsection{Loop Invariant Code Motion}

\begin{lstlisting}
// Bad: recalculates invariant every iteration
for (int i = 0; i < n; i++) {
    for (int j = 0; j < m; j++) {
        arr[i][j] = sqrt(x * x + y * y) + z;  // x, y, z don't change!
    }
}

// Good: calculate invariant once
double dist = sqrt(x * x + y * y) + z;
for (int i = 0; i < n; i++) {
    for (int j = 0; j < m; j++) {
        arr[i][j] = dist;
    }
}

// Common mistake: strlen in loop condition
for (int i = 0; i < strlen(str); i++) {  // strlen() called every iteration!
    process(str[i]);
}

// Fix: cache the length
size_t len = strlen(str);
for (size_t i = 0; i < len; i++) {
    process(str[i]);
}
\end{lstlisting}

\subsection{Strength Reduction}

Replace expensive operations with cheaper ones:

\begin{lstlisting}
// Multiplication to addition
for (int i = 0; i < n; i++) {
    arr[i * 4] = value;  // Multiply every iteration
}

// Better: use pointer arithmetic or addition
int offset = 0;
for (int i = 0; i < n; i++) {
    arr[offset] = value;
    offset += 4;  // Addition is faster than multiplication
}

// Division to multiplication (for constants)
for (int i = 0; i < n; i++) {
    result[i] = data[i] / 255;  // Division is slow
}

// Better: multiply by reciprocal
float inv = 1.0f / 255.0f;
for (int i = 0; i < n; i++) {
    result[i] = data[i] * inv;  // Multiplication is fast
}

// Integer division by power of 2
int div = x / 8;  // Division instruction
int div = x >> 3; // Right shift (faster)

// Modulo by power of 2
int mod = x % 32;   // Division instruction
int mod = x & 31;   // AND operation (much faster)

// General power-of-2 check
int is_power_of_2(unsigned int x) {
    return x && !(x & (x - 1));
}
\end{lstlisting}

\section{SIMD: Single Instruction Multiple Data}

Process multiple values simultaneously:

\begin{lstlisting}
#include <immintrin.h>  // Intel intrinsics
#include <arm_neon.h>   // ARM NEON intrinsics

// Scalar version: process one float at a time
void add_arrays_scalar(float* a, float* b, float* c, size_t n) {
    for (size_t i = 0; i < n; i++) {
        c[i] = a[i] + b[i];
    }
}

// SSE version: 4 floats at a time (128-bit)
void add_arrays_sse(float* a, float* b, float* c, size_t n) {
    size_t i = 0;

    // Process 4 floats at a time
    for (; i + 4 <= n; i += 4) {
        __m128 va = _mm_load_ps(&a[i]);
        __m128 vb = _mm_load_ps(&b[i]);
        __m128 vc = _mm_add_ps(va, vb);
        _mm_store_ps(&c[i], vc);
    }

    // Handle remainder
    for (; i < n; i++) {
        c[i] = a[i] + b[i];
    }
}

// AVX version: 8 floats at a time (256-bit)
void add_arrays_avx(float* a, float* b, float* c, size_t n) {
    size_t i = 0;

    for (; i + 8 <= n; i += 8) {
        __m256 va = _mm256_load_ps(&a[i]);
        __m256 vb = _mm256_load_ps(&b[i]);
        __m256 vc = _mm256_add_ps(va, vb);
        _mm256_store_ps(&c[i], vc);
    }

    for (; i < n; i++) {
        c[i] = a[i] + b[i];
    }
}

// AVX-512: 16 floats at a time (512-bit)
void add_arrays_avx512(float* a, float* b, float* c, size_t n) {
    size_t i = 0;

    for (; i + 16 <= n; i += 16) {
        __m512 va = _mm512_load_ps(&a[i]);
        __m512 vb = _mm512_load_ps(&b[i]);
        __m512 vc = _mm512_add_ps(va, vb);
        _mm512_store_ps(&c[i], vc);
    }

    for (; i < n; i++) {
        c[i] = a[i] + b[i];
    }
}

// Auto-vectorization: let compiler do it
void add_arrays_auto(float* restrict a,
                      float* restrict b,
                      float* restrict c,
                      size_t n) {
    // Tell compiler there's no aliasing
    #pragma GCC ivdep  // ignore vector dependencies
    for (size_t i = 0; i < n; i++) {
        c[i] = a[i] + b[i];
    }
}

// Horizontal sum using SIMD
float sum_array_simd(float* arr, size_t n) {
    __m256 sum_vec = _mm256_setzero_ps();
    size_t i = 0;

    for (; i + 8 <= n; i += 8) {
        __m256 v = _mm256_load_ps(&arr[i]);
        sum_vec = _mm256_add_ps(sum_vec, v);
    }

    // Horizontal add
    __m128 sum_high = _mm256_extractf128_ps(sum_vec, 1);
    __m128 sum_low = _mm256_castps256_ps128(sum_vec);
    __m128 sum128 = _mm_add_ps(sum_low, sum_high);

    float result[4];
    _mm_store_ps(result, sum128);
    float sum = result[0] + result[1] + result[2] + result[3];

    // Add remainder
    for (; i < n; i++) {
        sum += arr[i];
    }

    return sum;
}

// Detect CPU features at runtime
#include <cpuid.h>

int has_avx2(void) {
    unsigned int eax, ebx, ecx, edx;
    if (!__get_cpuid(7, &eax, &ebx, &ecx, &edx))
        return 0;
    return (ebx & bit_AVX2) != 0;
}

// Function pointer dispatch based on CPU features
void (*add_arrays)(float*, float*, float*, size_t) = add_arrays_scalar;

void init_simd(void) {
    if (has_avx2()) {
        add_arrays = add_arrays_avx;
    } else {
        add_arrays = add_arrays_sse;
    }
}
\end{lstlisting}

\section{Memory Management Patterns}

\subsection{Memory Pooling}

Pre-allocate memory to avoid malloc overhead:

\begin{lstlisting}
// Simple pool allocator
typedef struct {
    void* memory;
    size_t size;
    size_t used;
} MemPool;

MemPool* pool_create(size_t size) {
    MemPool* pool = malloc(sizeof(MemPool));
    pool->memory = malloc(size);
    pool->size = size;
    pool->used = 0;
    return pool;
}

void* pool_alloc(MemPool* pool, size_t size) {
    // Align to 8 bytes
    size = (size + 7) & ~7;

    if (pool->used + size > pool->size)
        return NULL;

    void* ptr = (char*)pool->memory + pool->used;
    pool->used += size;
    return ptr;
}

void pool_reset(MemPool* pool) {
    pool->used = 0;  // Reset pointer, reuse memory
}

void pool_destroy(MemPool* pool) {
    free(pool->memory);
    free(pool);
}

// Usage: perfect for per-frame allocations in games
MemPool* frame_pool = pool_create(1024 * 1024);  // 1 MB

void render_frame(void) {
    // Allocate temporary data
    float* temp = pool_alloc(frame_pool, 1000 * sizeof(float));

    // Use temp...

    // End of frame: reset pool (no free() calls needed!)
    pool_reset(frame_pool);
}
\end{lstlisting}

\subsection{Arena Allocator}

\begin{lstlisting}
// Arena: allocate in chunks, free all at once
typedef struct ArenaBlock {
    struct ArenaBlock* next;
    size_t size;
    size_t used;
    char data[];  // Flexible array member
} ArenaBlock;

typedef struct {
    ArenaBlock* current;
    size_t block_size;
} Arena;

Arena* arena_create(size_t block_size) {
    Arena* arena = malloc(sizeof(Arena));
    arena->block_size = block_size;
    arena->current = NULL;
    return arena;
}

void* arena_alloc(Arena* arena, size_t size) {
    // Align to 8 bytes
    size = (size + 7) & ~7;

    // Need new block?
    if (!arena->current || arena->current->used + size > arena->current->size) {
        size_t block_size = (size > arena->block_size) ? size : arena->block_size;
        ArenaBlock* block = malloc(sizeof(ArenaBlock) + block_size);
        block->size = block_size;
        block->used = 0;
        block->next = arena->current;
        arena->current = block;
    }

    void* ptr = arena->current->data + arena->current->used;
    arena->current->used += size;
    return ptr;
}

void arena_destroy(Arena* arena) {
    ArenaBlock* block = arena->current;
    while (block) {
        ArenaBlock* next = block->next;
        free(block);
        block = next;
    }
    free(arena);
}

// Usage: parse file, build data structures, process, free all at once
Arena* arena = arena_create(4096);

void process_file(const char* filename) {
    // Parse file into arena-allocated structures
    Node* tree = parse_file(filename, arena);

    // Process the tree...

    // Done: free everything at once
    arena_destroy(arena);
}
\end{lstlisting}

\subsection{Object Pools for Fixed-Size Allocations}

\begin{lstlisting}
// Free list for objects of the same size
typedef struct FreeNode {
    struct FreeNode* next;
} FreeNode;

typedef struct {
    void* memory;
    FreeNode* free_list;
    size_t obj_size;
    size_t capacity;
} ObjectPool;

ObjectPool* objpool_create(size_t obj_size, size_t capacity) {
    ObjectPool* pool = malloc(sizeof(ObjectPool));
    pool->obj_size = obj_size;
    pool->capacity = capacity;
    pool->memory = malloc(obj_size * capacity);

    // Build free list
    pool->free_list = NULL;
    for (size_t i = 0; i < capacity; i++) {
        void* obj = (char*)pool->memory + i * obj_size;
        FreeNode* node = (FreeNode*)obj;
        node->next = pool->free_list;
        pool->free_list = node;
    }

    return pool;
}

void* objpool_alloc(ObjectPool* pool) {
    if (!pool->free_list)
        return NULL;  // Pool exhausted

    void* obj = pool->free_list;
    pool->free_list = pool->free_list->next;
    return obj;
}

void objpool_free(ObjectPool* pool, void* obj) {
    FreeNode* node = (FreeNode*)obj;
    node->next = pool->free_list;
    pool->free_list = node;
}

// Usage: game entities
ObjectPool* entity_pool = objpool_create(sizeof(Entity), 10000);

Entity* spawn_entity(void) {
    Entity* e = objpool_alloc(entity_pool);
    if (e) {
        init_entity(e);
    }
    return e;
}

void despawn_entity(Entity* e) {
    objpool_free(entity_pool, e);
}
\end{lstlisting}

\subsection{Small String Optimization (SSO)}

\begin{lstlisting}
// Store short strings inline, allocate for long strings
#define SSO_SIZE 23

typedef struct {
    union {
        struct {
            char* ptr;
            size_t len;
            size_t cap;
        } heap;
        struct {
            char buf[SSO_SIZE];
            unsigned char len;  // High bit = is_heap
        } sso;
    };
} String;

int string_is_heap(String* s) {
    return s->sso.len & 0x80;
}

void string_init(String* s, const char* str) {
    size_t len = strlen(str);

    if (len < SSO_SIZE) {
        // Short string: store inline
        memcpy(s->sso.buf, str, len + 1);
        s->sso.len = len;
    } else {
        // Long string: allocate on heap
        s->heap.ptr = malloc(len + 1);
        memcpy(s->heap.ptr, str, len + 1);
        s->heap.len = len | 0x80;  // Set heap flag
        s->heap.cap = len + 1;
    }
}

const char* string_cstr(String* s) {
    return string_is_heap(s) ? s->heap.ptr : s->sso.buf;
}

void string_free(String* s) {
    if (string_is_heap(s)) {
        free(s->heap.ptr);
    }
}

// Most strings are short - SSO avoids malloc for them!
// Used by: std::string in C++, many high-performance C libraries
\end{lstlisting}

\subsection{Slab Allocator (Linux Kernel Pattern)}

\begin{lstlisting}
// Slab allocator: pre-allocated objects with constructor
typedef struct Slab {
    struct Slab* next;
    size_t obj_size;
    size_t capacity;
    size_t used;
    void* objects;
} Slab;

typedef void (*ctor_fn)(void*);
typedef void (*dtor_fn)(void*);

typedef struct {
    Slab* slabs;
    size_t obj_size;
    size_t slab_size;
    ctor_fn ctor;
    dtor_fn dtor;
} SlabCache;

SlabCache* slab_create(size_t obj_size, size_t slab_size,
                       ctor_fn ctor, dtor_fn dtor) {
    SlabCache* cache = malloc(sizeof(SlabCache));
    cache->obj_size = obj_size;
    cache->slab_size = slab_size;
    cache->ctor = ctor;
    cache->dtor = dtor;
    cache->slabs = NULL;
    return cache;
}

void* slab_alloc(SlabCache* cache) {
    // Find slab with space, or create new one
    // ... implementation similar to object pool ...

    void* obj = /* allocate from slab */;

    // Call constructor
    if (cache->ctor) {
        cache->ctor(obj);
    }

    return obj;
}

void slab_free(SlabCache* cache, void* obj) {
    // Call destructor
    if (cache->dtor) {
        cache->dtor(obj);
    }

    // Return to slab free list
    // ...
}

// Constructor: initialize object to ready state
void task_ctor(void* ptr) {
    Task* task = ptr;
    task->state = TASK_READY;
    task->priority = 0;
    // Initialize other fields...
}

// Cache of pre-constructed tasks
SlabCache* task_cache = slab_create(sizeof(Task), 4096,
                                    task_ctor, NULL);
\end{lstlisting}

\section{Bit Manipulation Tricks}

\subsection{Classic Bit Hacks}

\begin{lstlisting}
// Check if power of 2
int is_power_of_2(unsigned int x) {
    return x && !(x & (x - 1));
}

// Round up to next power of 2
unsigned int next_power_of_2(unsigned int x) {
    x--;
    x |= x >> 1;
    x |= x >> 2;
    x |= x >> 4;
    x |= x >> 8;
    x |= x >> 16;
    return x + 1;
}

// Count trailing zeros (CTZ)
int count_trailing_zeros(unsigned int x) {
    return __builtin_ctz(x);  // Compiles to single instruction
}

// Count leading zeros (CLZ)
int count_leading_zeros(unsigned int x) {
    return __builtin_clz(x);
}

// Count set bits (population count)
int popcount(unsigned int x) {
    return __builtin_popcount(x);
}

// Manually (Brian Kernighan's algorithm)
int popcount_manual(unsigned int x) {
    int count = 0;
    while (x) {
        x &= x - 1;  // Clear lowest set bit
        count++;
    }
    return count;
}

// Find lowest set bit
unsigned int lowest_bit(unsigned int x) {
    return x & -x;
}

// Swap two values without temporary
void swap_xor(int* a, int* b) {
    *a ^= *b;
    *b ^= *a;
    *a ^= *b;
}

// Absolute value without branch
int abs_bitwise(int x) {
    int mask = x >> 31;  // All 1s if negative, all 0s if positive
    return (x + mask) ^ mask;
}

// Min/max without branch
int min_bitwise(int x, int y) {
    return y ^ ((x ^ y) & -(x < y));
}

int max_bitwise(int x, int y) {
    return x ^ ((x ^ y) & -(x < y));
}

// Sign of integer (-1, 0, 1)
int sign(int x) {
    return (x > 0) - (x < 0);
}

// Check if signs differ
int opposite_signs(int x, int y) {
    return (x ^ y) < 0;
}

// Reverse bits
unsigned int reverse_bits(unsigned int x) {
    x = ((x & 0xAAAAAAAA) >> 1) | ((x & 0x55555555) << 1);
    x = ((x & 0xCCCCCCCC) >> 2) | ((x & 0x33333333) << 2);
    x = ((x & 0xF0F0F0F0) >> 4) | ((x & 0x0F0F0F0F) << 4);
    x = ((x & 0xFF00FF00) >> 8) | ((x & 0x00FF00FF) << 8);
    return (x >> 16) | (x << 16);
}

// Byte swap (endianness conversion)
uint32_t bswap32(uint32_t x) {
    return __builtin_bswap32(x);  // Single instruction
}

// Parity (even number of set bits?)
int parity(unsigned int x) {
    return __builtin_parity(x);
}
\end{lstlisting}

\subsection{Bit Fields for Flags}

\begin{lstlisting}
// Using bit fields for compact flag storage
typedef struct {
    unsigned int is_active : 1;
    unsigned int is_visible : 1;
    unsigned int has_physics : 1;
    unsigned int is_static : 1;
    unsigned int layer : 4;  // 0-15
    unsigned int unused : 24;
} EntityFlags;

// Or use explicit bit operations
#define FLAG_ACTIVE   (1 << 0)
#define FLAG_VISIBLE  (1 << 1)
#define FLAG_PHYSICS  (1 << 2)
#define FLAG_STATIC   (1 << 3)

typedef struct {
    uint32_t flags;
} Entity;

void entity_set_flag(Entity* e, uint32_t flag) {
    e->flags |= flag;
}

void entity_clear_flag(Entity* e, uint32_t flag) {
    e->flags &= ~flag;
}

int entity_has_flag(Entity* e, uint32_t flag) {
    return (e->flags & flag) != 0;
}

void entity_toggle_flag(Entity* e, uint32_t flag) {
    e->flags ^= flag;
}

// Bit set operations
typedef struct {
    uint64_t bits[16];  // 1024 bits
} BitSet;

void bitset_set(BitSet* bs, int index) {
    bs->bits[index / 64] |= (1ULL << (index % 64));
}

void bitset_clear(BitSet* bs, int index) {
    bs->bits[index / 64] &= ~(1ULL << (index % 64));
}

int bitset_test(BitSet* bs, int index) {
    return (bs->bits[index / 64] & (1ULL << (index % 64))) != 0;
}
\end{lstlisting}

\subsection{Morton Codes (Z-Order Curve)}

Encode 2D coordinates in a cache-friendly way:

\begin{lstlisting}
// Interleave bits of x and y coordinates
uint32_t morton_encode(uint16_t x, uint16_t y) {
    uint32_t result = 0;
    for (int i = 0; i < 16; i++) {
        result |= ((x & (1 << i)) << i) | ((y & (1 << i)) << (i + 1));
    }
    return result;
}

// Fast version using magic numbers
uint32_t morton_encode_fast(uint16_t x, uint16_t y) {
    uint32_t xx = x;
    uint32_t yy = y;

    xx = (xx | (xx << 8)) & 0x00FF00FF;
    xx = (xx | (xx << 4)) & 0x0F0F0F0F;
    xx = (xx | (xx << 2)) & 0x33333333;
    xx = (xx | (xx << 1)) & 0x55555555;

    yy = (yy | (yy << 8)) & 0x00FF00FF;
    yy = (yy | (yy << 4)) & 0x0F0F0F0F;
    yy = (yy | (yy << 2)) & 0x33333333;
    yy = (yy | (yy << 1)) & 0x55555555;

    return xx | (yy << 1);
}

// Decode
void morton_decode(uint32_t code, uint16_t* x, uint16_t* y) {
    uint32_t xx = code & 0x55555555;
    uint32_t yy = (code >> 1) & 0x55555555;

    xx = (xx | (xx >> 1)) & 0x33333333;
    xx = (xx | (xx >> 2)) & 0x0F0F0F0F;
    xx = (xx | (xx >> 4)) & 0x00FF00FF;
    xx = (xx | (xx >> 8)) & 0x0000FFFF;

    yy = (yy | (yy >> 1)) & 0x33333333;
    yy = (yy | (yy >> 2)) & 0x0F0F0F0F;
    yy = (yy | (yy >> 4)) & 0x00FF00FF;
    yy = (yy >> 8)) & 0x0000FFFF;

    *x = xx;
    *y = yy;
}

// Use for spatial data structures
// Objects near in 2D space have nearby Morton codes
// -> better cache locality when iterating
\end{lstlisting}

\section{Function Call Optimization}

\subsection{Inline Functions}

\begin{lstlisting}
// Small helper functions should be inline
static inline int min(int a, int b) {
    return a < b ? a : b;
}

static inline int max(int a, int b) {
    return a > b ? a : b;
}

static inline int clamp(int x, int low, int high) {
    return min(max(x, low), high);
}

// Force inline for critical functions
__attribute__((always_inline))
static inline void critical_function(void) {
    // Must be inlined for performance
}

// Prevent inline (for debugging or code size)
__attribute__((noinline))
void debug_function(void) {
    // Keep as function call
}

// Hot/cold function hints
__attribute__((hot))
void frequently_called(void) {
    // Compiler optimizes aggressively
}

__attribute__((cold))
void error_handler(void) {
    // Optimize for size, not speed
}

// Pure function (no side effects, same output for same input)
__attribute__((pure))
int compute_value(int x, int y) {
    return x * x + y * y;
}

// Const function (pure + doesn't read memory)
__attribute__((const))
int add(int a, int b) {
    return a + b;
}
\end{lstlisting}

\subsection{Tail Call Optimization}

\begin{lstlisting}
// Non-tail recursive (uses stack space)
int factorial(int n) {
    if (n <= 1)
        return 1;
    return n * factorial(n - 1);  // Can't optimize: multiply after call
}

// Tail recursive (can be optimized to loop)
int factorial_tail(int n, int acc) {
    if (n <= 1)
        return acc;
    return factorial_tail(n - 1, n * acc);  // Last operation is call
}

// Compiler can optimize tail call to:
int factorial_loop(int n, int acc) {
    while (n > 1) {
        acc = n * acc;
        n = n - 1;
    }
    return acc;
}

// Use -O2 or -O3 to enable tail call optimization
// Or __attribute__((optimize("O2")))
\end{lstlisting}

\subsection{Function Pointer Overhead}

\begin{lstlisting}
// Indirect calls prevent inlining and CPU prediction
void process_indirect(void (*func)(int), int* data, size_t n) {
    for (size_t i = 0; i < n; i++) {
        func(data[i]);  // Indirect call, expensive
    }
}

// Direct calls can be inlined
static inline void process_func(int x) {
    // Do something
}

void process_direct(int* data, size_t n) {
    for (size_t i = 0; i < n; i++) {
        process_func(data[i]);  // Direct call, can inline
    }
}

// If you need function pointers, batch the calls
void process_batched(void (*func)(int*, size_t), int* data, size_t n) {
    const size_t BATCH = 1000;
    for (size_t i = 0; i < n; i += BATCH) {
        size_t count = (i + BATCH <= n) ? BATCH : (n - i);
        func(&data[i], count);  // One indirect call per 1000 items
    }
}
\end{lstlisting}

\section{Algorithm-Level Optimizations}

\subsection{Fast Path for Common Case}

\begin{lstlisting}
// Optimize for the common case
int parse_int(const char* str) {
    // Fast path: single digit (very common)
    if (str[0] >= '0' && str[0] <= '9' && str[1] == '\0') {
        return str[0] - '0';
    }

    // Slow path: general case
    return atoi(str);
}

// Fast path for ASCII strings (common case)
int string_length(const char* str) {
    // Fast path: ASCII (no multibyte characters)
    if ((*str & 0x80) == 0) {
        return strlen(str);
    }

    // Slow path: UTF-8 (multibyte characters)
    return utf8_length(str);
}

// Early exit optimization
int find_element(int* arr, size_t n, int target) {
    // Check first element (often finds it immediately)
    if (n > 0 && arr[0] == target)
        return 0;

    // Check last element
    if (n > 1 && arr[n-1] == target)
        return n - 1;

    // General search
    for (size_t i = 1; i < n - 1; i++) {
        if (arr[i] == target)
            return i;
    }

    return -1;
}
\end{lstlisting}

\subsection{Lookup Tables}

\begin{lstlisting}
// Compute once, lookup many times

// Example: character classification
static const unsigned char char_table[256] = {
    ['0'] = 1, ['1'] = 1, ['2'] = 1, ['3'] = 1, ['4'] = 1,
    ['5'] = 1, ['6'] = 1, ['7'] = 1, ['8'] = 1, ['9'] = 1,
    ['a'] = 2, ['b'] = 2, ['c'] = 2, ['d'] = 2, ['e'] = 2, ['f'] = 2,
    ['A'] = 2, ['B'] = 2, ['C'] = 2, ['D'] = 2, ['E'] = 2, ['F'] = 2,
    // ... rest are 0
};

int is_digit(char c) {
    return char_table[(unsigned char)c] == 1;
}

int is_hex_digit(char c) {
    return char_table[(unsigned char)c] != 0;
}

// Precomputed trig table (classic game dev trick)
#define TRIG_TABLE_SIZE 360

float sin_table[TRIG_TABLE_SIZE];
float cos_table[TRIG_TABLE_SIZE];

void init_trig_table(void) {
    for (int i = 0; i < TRIG_TABLE_SIZE; i++) {
        float angle = i * (M_PI / 180.0f);
        sin_table[i] = sinf(angle);
        cos_table[i] = cosf(angle);
    }
}

float fast_sin(int degrees) {
    degrees = degrees % 360;
    if (degrees < 0) degrees += 360;
    return sin_table[degrees];
}

// Square root approximation table
float sqrt_table[1000];

void init_sqrt_table(void) {
    for (int i = 0; i < 1000; i++) {
        sqrt_table[i] = sqrtf(i);
    }
}

float fast_sqrt(float x) {
    if (x < 1000) {
        int index = (int)x;
        float frac = x - index;
        return sqrt_table[index] + frac * (sqrt_table[index+1] - sqrt_table[index]);
    }
    return sqrtf(x);
}
\end{lstlisting}

\subsection{Lazy Evaluation and Caching}

\begin{lstlisting}
// Compute expensive values only when needed

typedef struct {
    float x, y, z;
    float length;  // Cached
    int length_valid;
} Vector;

float vector_length(Vector* v) {
    if (!v->length_valid) {
        v->length = sqrtf(v->x * v->x + v->y * v->y + v->z * v->z);
        v->length_valid = 1;
    }
    return v->length;
}

void vector_set(Vector* v, float x, float y, float z) {
    v->x = x;
    v->y = y;
    v->z = z;
    v->length_valid = 0;  // Invalidate cache
}

// Memoization for recursive functions
typedef struct {
    int n;
    int result;
} FibCache;

FibCache fib_cache[100];
int fib_cache_size = 0;

int fibonacci(int n) {
    // Check cache
    for (int i = 0; i < fib_cache_size; i++) {
        if (fib_cache[i].n == n)
            return fib_cache[i].result;
    }

    // Compute
    int result;
    if (n <= 1)
        result = n;
    else
        result = fibonacci(n - 1) + fibonacci(n - 2);

    // Store in cache
    if (fib_cache_size < 100) {
        fib_cache[fib_cache_size].n = n;
        fib_cache[fib_cache_size].result = result;
        fib_cache_size++;
    }

    return result;
}
\end{lstlisting}

\subsection{Sentinel Values}

Eliminate loop bound checks:

\begin{lstlisting}
// Without sentinel: two comparisons per iteration
int find_linear(int* arr, int n, int target) {
    for (int i = 0; i < n; i++) {  // Check i < n
        if (arr[i] == target)      // Check value
            return i;
    }
    return -1;
}

// With sentinel: one comparison per iteration
int find_sentinel(int* arr, int n, int target) {
    int last = arr[n - 1];  // Save last element
    arr[n - 1] = target;    // Place sentinel

    int i = 0;
    while (arr[i] != target)  // Only one check!
        i++;

    arr[n - 1] = last;  // Restore last element

    if (i < n - 1 || last == target)
        return i;
    return -1;
}

// Sentinel in linked list
typedef struct Node {
    int data;
    struct Node* next;
} Node;

// Add sentinel at end
Node sentinel;
sentinel.data = target;
sentinel.next = NULL;

Node* find_list(Node* head, int target) {
    // No need to check for NULL!
    while (head->data != target)
        head = head->next;

    return (head != &sentinel) ? head : NULL;
}
\end{lstlisting}

\section{String Optimization}

\subsection{Avoiding strlen in Loops}

\begin{lstlisting}
// Bad: O(n^2) due to strlen calls
void process_bad(char* str) {
    for (int i = 0; i < strlen(str); i++) {  // strlen is O(n)!
        process_char(str[i]);
    }
}

// Good: O(n) - cache length
void process_good(char* str) {
    size_t len = strlen(str);
    for (size_t i = 0; i < len; i++) {
        process_char(str[i]);
    }
}

// Best: iterate to null terminator
void process_best(char* str) {
    for (char* p = str; *p; p++) {
        process_char(*p);
    }
}
\end{lstlisting}

\subsection{String Building}

\begin{lstlisting}
// Bad: repeated reallocation
char* build_string_bad(char** words, int count) {
    char* result = strdup("");
    for (int i = 0; i < count; i++) {
        char* temp = malloc(strlen(result) + strlen(words[i]) + 2);
        sprintf(temp, "%s %s", result, words[i]);
        free(result);
        result = temp;
    }
    return result;
}

// Good: pre-calculate size
char* build_string_good(char** words, int count) {
    // Calculate total size
    size_t total = 0;
    for (int i = 0; i < count; i++) {
        total += strlen(words[i]) + 1;  // +1 for space
    }

    // Allocate once
    char* result = malloc(total);
    char* p = result;

    // Copy strings
    for (int i = 0; i < count; i++) {
        if (i > 0) *p++ = ' ';
        size_t len = strlen(words[i]);
        memcpy(p, words[i], len);
        p += len;
    }
    *p = '\0';

    return result;
}

// String builder with growth strategy
typedef struct {
    char* data;
    size_t len;
    size_t cap;
} StringBuilder;

void sb_append(StringBuilder* sb, const char* str) {
    size_t str_len = strlen(str);

    // Grow if needed
    if (sb->len + str_len >= sb->cap) {
        sb->cap = (sb->cap + str_len) * 2;
        sb->data = realloc(sb->data, sb->cap);
    }

    memcpy(sb->data + sb->len, str, str_len);
    sb->len += str_len;
    sb->data[sb->len] = '\0';
}
\end{lstlisting}

\subsection{Fast String Comparison}

\begin{lstlisting}
// strcmp is optimized, but you can do better for special cases

// Compare first, they're often different
int string_equal_fast(const char* a, const char* b) {
    // Quick checks
    if (a == b) return 1;
    if (*a != *b) return 0;  // First char different

    return strcmp(a, b) == 0;
}

// Known-length comparison
int string_equal_n(const char* a, const char* b, size_t len) {
    return memcmp(a, b, len) == 0;  // memcmp is fast
}

// Compare 8 bytes at a time
int string_equal_fast8(const char* a, const char* b, size_t len) {
    const uint64_t* a64 = (const uint64_t*)a;
    const uint64_t* b64 = (const uint64_t*)b;

    size_t i = 0;
    for (; i + 8 <= len; i += 8) {
        if (*a64++ != *b64++)
            return 0;
    }

    // Handle remainder
    for (; i < len; i++) {
        if (a[i] != b[i])
            return 0;
    }

    return 1;
}
\end{lstlisting}

\section{I/O Optimization}

\subsection{Buffering}

\begin{lstlisting}
// Unbuffered: system call per byte (extremely slow)
void write_unbuffered(int fd, const char* data, size_t n) {
    for (size_t i = 0; i < n; i++) {
        write(fd, &data[i], 1);  // 1 byte at a time!
    }
}

// Buffered: accumulate data, write in chunks
#define BUFFER_SIZE 4096

typedef struct {
    int fd;
    char buffer[BUFFER_SIZE];
    size_t pos;
} BufferedWriter;

void bw_write(BufferedWriter* bw, const char* data, size_t n) {
    for (size_t i = 0; i < n; i++) {
        bw->buffer[bw->pos++] = data[i];

        if (bw->pos == BUFFER_SIZE) {
            write(bw->fd, bw->buffer, BUFFER_SIZE);
            bw->pos = 0;
        }
    }
}

void bw_flush(BufferedWriter* bw) {
    if (bw->pos > 0) {
        write(bw->fd, bw->buffer, bw->pos);
        bw->pos = 0;
    }
}

// Use stdio (already buffered)
FILE* f = fopen("file.txt", "w");
setvbuf(f, NULL, _IOFBF, BUFFER_SIZE);  // Full buffering
\end{lstlisting}

\subsection{Memory-Mapped Files}

For large files, memory mapping is faster:

\begin{lstlisting}
#include <sys/mman.h>
#include <sys/stat.h>
#include <fcntl.h>

// Traditional read: copy from kernel to user space
void process_file_read(const char* filename) {
    int fd = open(filename, O_RDONLY);
    struct stat st;
    fstat(fd, &st);

    char* buffer = malloc(st.st_size);
    read(fd, buffer, st.st_size);  // Copy!

    // Process buffer...

    free(buffer);
    close(fd);
}

// Memory-mapped: direct access to file data
void process_file_mmap(const char* filename) {
    int fd = open(filename, O_RDONLY);
    struct stat st;
    fstat(fd, &st);

    // Map file into memory
    char* data = mmap(NULL, st.st_size, PROT_READ, MAP_PRIVATE, fd, 0);

    // Access data directly (no copy!)
    // OS handles paging automatically

    // Advise kernel about access pattern
    madvise(data, st.st_size, MADV_SEQUENTIAL);

    // Process data...

    munmap(data, st.st_size);
    close(fd);
}

// Write with mmap (for random access)
void update_file_mmap(const char* filename, size_t offset, const void* data, size_t len) {
    int fd = open(filename, O_RDWR);
    struct stat st;
    fstat(fd, &st);

    char* mapped = mmap(NULL, st.st_size, PROT_READ | PROT_WRITE, MAP_SHARED, fd, 0);

    memcpy(mapped + offset, data, len);  // Direct write

    munmap(mapped, st.st_size);
    close(fd);
}
\end{lstlisting}

\subsection{Vectored I/O}

Gather/scatter I/O for non-contiguous buffers:

\begin{lstlisting}
#include <sys/uio.h>

// Write multiple buffers in one system call
void write_vectored(int fd, char* header, size_t hlen,
                    char* body, size_t blen,
                    char* footer, size_t flen) {
    struct iovec iov[3];

    iov[0].iov_base = header;
    iov[0].iov_len = hlen;
    iov[1].iov_base = body;
    iov[1].iov_len = blen;
    iov[2].iov_base = footer;
    iov[2].iov_len = flen;

    writev(fd, iov, 3);  // One system call instead of three!
}
\end{lstlisting}

\section{Compiler Optimizations}

\subsection{Understanding Optimization Levels}

\begin{lstlisting}
// Compiler flags:
// -O0: No optimization (default, debugging)
// -O1: Basic optimizations (constant folding, dead code elimination)
// -O2: Recommended (adds inlining, CSE, loop optimizations)
// -O3: Aggressive (vectorization, unrolling, more inlining)
// -Os: Optimize for size
// -Ofast: -O3 + fast math (may break IEEE 754 compliance)

// Example: compile with maximum optimization
// gcc -O3 -march=native -flto program.c -o program

// -march=native: use all CPU instructions available
// -flto: link-time optimization
\end{lstlisting}

\subsection{Link-Time Optimization (LTO)}

\begin{lstlisting}
// Traditionally: optimize each file separately
// gcc -O3 -c file1.c -o file1.o
// gcc -O3 -c file2.c -o file2.o
// gcc file1.o file2.o -o program

// With LTO: optimize across all files
// gcc -O3 -flto -c file1.c -o file1.o
// gcc -O3 -flto -c file2.c -o file2.o
// gcc -flto file1.o file2.o -o program

// Benefits:
// - Inline functions across files
// - Dead code elimination across files
// - Better optimization of function calls
// - Typically 5-15% speedup
\end{lstlisting}

\subsection{Profile-Guided Optimization (PGO)}

\begin{lstlisting}
// Step 1: Compile with profiling instrumentation
// gcc -O2 -fprofile-generate program.c -o program

// Step 2: Run with typical workload
// ./program < typical_input.txt
// This creates .gcda files with profile data

// Step 3: Recompile using profile data
// gcc -O2 -fprofile-use program.c -o program

// Compiler now knows:
// - Which branches are taken most often
// - Which functions are called most
// - Which loops iterate many times
// Result: 10-30% speedup for branch-heavy code!

// Real example: used by Firefox, Chrome, GCC itself
\end{lstlisting}

\subsection{Function Attributes}

\begin{lstlisting}
// Tell compiler about function properties

// Pure: same inputs always produce same output, no side effects
__attribute__((pure))
int compute_hash(const char* str) {
    // Only reads memory, doesn't modify
    int hash = 0;
    while (*str) hash = hash * 31 + *str++;
    return hash;
}

// Const: like pure, but doesn't even read memory
__attribute__((const))
int add(int a, int b) {
    return a + b;  // Only depends on arguments
}

// Compiler can cache results of pure/const functions!

// Malloc: returns new memory
__attribute__((malloc))
void* my_alloc(size_t size) {
    return malloc(size);
}

// Returns non-null
__attribute__((returns_nonnull))
void* must_succeed_alloc(size_t size) {
    void* ptr = malloc(size);
    if (!ptr) abort();
    return ptr;
}

// Warn if result is ignored
__attribute__((warn_unused_result))
int important_function(void) {
    return 42;
}

// Function doesn't return
__attribute__((noreturn))
void fatal_error(const char* msg) {
    fprintf(stderr, "Fatal: %s\n", msg);
    exit(1);
}

// Alias: two names for same function
int foo(int x) { return x * 2; }
int bar(int x) __attribute__((alias("foo")));
\end{lstlisting}

\subsection{Restrict Keyword}

Tell compiler about pointer aliasing:

\begin{lstlisting}
// Without restrict: compiler assumes pointers might overlap
void copy(int* dst, int* src, size_t n) {
    for (size_t i = 0; i < n; i++) {
        dst[i] = src[i];
        // Compiler must reload src[i] each time
        // (dst[i] write might have changed src[i+1])
    }
}

// With restrict: pointers don't overlap
void copy_fast(int* restrict dst, int* restrict src, size_t n) {
    for (size_t i = 0; i < n; i++) {
        dst[i] = src[i];
        // Compiler can vectorize, reorder, optimize
    }
}

// Real impact: memcpy uses restrict internally
void* memcpy(void* restrict dst, const void* restrict src, size_t n);

// Example: vector operations
void vector_add(float* restrict out,
                const float* restrict a,
                const float* restrict b,
                size_t n) {
    for (size_t i = 0; i < n; i++) {
        out[i] = a[i] + b[i];
    }
}
// With restrict: compiler auto-vectorizes with SIMD
// Without restrict: scalar code only
\end{lstlisting}

\section{Assembly and Low-Level Tricks}

\subsection{Inline Assembly}

\begin{lstlisting}
// For truly critical code, use assembly

// Read CPU timestamp counter
static inline uint64_t rdtsc(void) {
    uint32_t lo, hi;
    __asm__ __volatile__ ("rdtsc" : "=a"(lo), "=d"(hi));
    return ((uint64_t)hi << 32) | lo;
}

// Atomic compare-and-swap
static inline int cas(int* ptr, int old_val, int new_val) {
    int result;
    __asm__ __volatile__ (
        "lock cmpxchgl %2, %1"
        : "=a"(result), "+m"(*ptr)
        : "r"(new_val), "0"(old_val)
        : "memory"
    );
    return result == old_val;
}

// CPU pause instruction (for spin loops)
static inline void cpu_pause(void) {
    __asm__ __volatile__ ("pause" ::: "memory");
}

// Memory fence
static inline void memory_barrier(void) {
    __asm__ __volatile__ ("mfence" ::: "memory");
}
\end{lstlisting}

\subsection{Reading Compiler Output}

\begin{lstlisting}
// Generate assembly to see what compiler does
// gcc -S -O3 file.c -o file.s

// Or use online tools: godbolt.org (Compiler Explorer)

// Example: check if loop vectorized
void scale(float* arr, float factor, int n) {
    for (int i = 0; i < n; i++) {
        arr[i] *= factor;
    }
}

// Look for SIMD instructions in assembly:
// movaps, mulps (SSE)
// vmovaps, vmulps (AVX)
// vmovups, vmulps (AVX-512)

// If you see only scalar: movss, mulss
// -> loop didn't vectorize, investigate why!
\end{lstlisting}

\section{Profiling and Measurement}

\subsection{Timing Code}

\begin{lstlisting}
#include <time.h>

// Simple timing with clock()
double time_function(void (*func)(void)) {
    clock_t start = clock();
    func();
    clock_t end = clock();
    return (double)(end - start) / CLOCKS_PER_SEC;
}

// High-resolution timing
#include <sys/time.h>

double get_time_usec(void) {
    struct timeval tv;
    gettimeofday(&tv, NULL);
    return tv.tv_sec * 1e6 + tv.tv_usec;
}

// Modern: clock_gettime (nanosecond resolution)
#include <time.h>

uint64_t get_time_nsec(void) {
    struct timespec ts;
    clock_gettime(CLOCK_MONOTONIC, &ts);
    return ts.tv_sec * 1000000000ULL + ts.tv_nsec;
}

// Benchmark with warm-up and multiple iterations
double benchmark(void (*func)(void), int iterations) {
    // Warm up (fill caches)
    func();
    func();

    uint64_t start = get_time_nsec();

    for (int i = 0; i < iterations; i++) {
        func();
    }

    uint64_t end = get_time_nsec();
    return (double)(end - start) / iterations;
}

// Use CPU timestamp counter for cycle-accurate timing
uint64_t measure_cycles(void (*func)(void)) {
    uint64_t start = rdtsc();
    func();
    uint64_t end = rdtsc();
    return end - start;
}
\end{lstlisting}

\subsection{Using gprof}

\begin{lstlisting}
// Step 1: Compile with profiling
// gcc -pg -O2 program.c -o program

// Step 2: Run program
// ./program

// Step 3: Analyze profile
// gprof program gmon.out > profile.txt

// Shows:
// - Flat profile: time spent in each function
// - Call graph: who calls whom, how often
// - Annotated source: time per line

// Example output:
//   %   cumulative   self              self     total
//  time   seconds   seconds    calls  ms/call  ms/call  name
//  60.00      0.60     0.60   100000     0.01     0.01  process_data
//  30.00      0.90     0.30        1   300.00   900.00  main
//  10.00      1.00     0.10    10000     0.01     0.01  helper_func
\end{lstlisting}

\subsection{Using perf (Linux)}

\begin{lstlisting}
// Record performance data
// perf record -g ./program

// View report
// perf report

// Sample specific events
// perf stat -e cache-misses,cache-references,branches,branch-misses ./program

// Example output:
//  Performance counter stats for './program':
//
//         1,234,567      cache-misses
//        10,234,567      cache-references    # 12.06 % cache miss rate
//       100,234,567      branches
//         2,345,678      branch-misses       #  2.34% of all branches
//
//       1.234567890 seconds time elapsed

// Profile specific CPU events
// perf stat -e L1-dcache-load-misses,L1-dcache-loads ./program

// Annotate source with profile data
// perf annotate function_name

// See which lines cause cache misses
// perf record -e cache-misses ./program
// perf annotate
\end{lstlisting}

\subsection{Using Valgrind Cachegrind}

\begin{lstlisting}
// Profile cache behavior
// valgrind --tool=cachegrind ./program

// Output file: cachegrind.out.PID

// Analyze
// cg_annotate cachegrind.out.12345

// Shows:
// - L1 instruction cache misses
// - L1 data cache misses
// - L3/LLC cache misses
// - Per-function and per-line statistics

// Visualize
// kcachegrind cachegrind.out.12345
\end{lstlisting}

\section{Platform-Specific Optimizations}

\subsection{CPU Feature Detection}

\begin{lstlisting}
#include <cpuid.h>

typedef struct {
    int has_sse;
    int has_sse2;
    int has_sse3;
    int has_ssse3;
    int has_sse4_1;
    int has_sse4_2;
    int has_avx;
    int has_avx2;
    int has_avx512;
    int has_popcnt;
    int has_bmi1;
    int has_bmi2;
} CpuFeatures;

void detect_cpu_features(CpuFeatures* feat) {
    unsigned int eax, ebx, ecx, edx;

    // CPUID function 1
    __get_cpuid(1, &eax, &ebx, &ecx, &edx);

    feat->has_sse = (edx & bit_SSE) != 0;
    feat->has_sse2 = (edx & bit_SSE2) != 0;
    feat->has_sse3 = (ecx & bit_SSE3) != 0;
    feat->has_ssse3 = (ecx & bit_SSSE3) != 0;
    feat->has_sse4_1 = (ecx & bit_SSE4_1) != 0;
    feat->has_sse4_2 = (ecx & bit_SSE4_2) != 0;
    feat->has_avx = (ecx & bit_AVX) != 0;
    feat->has_popcnt = (ecx & bit_POPCNT) != 0;

    // CPUID function 7
    __get_cpuid_count(7, 0, &eax, &ebx, &ecx, &edx);

    feat->has_avx2 = (ebx & bit_AVX2) != 0;
    feat->has_bmi1 = (ebx & bit_BMI) != 0;
    feat->has_bmi2 = (ebx & bit_BMI2) != 0;
    feat->has_avx512 = (ebx & bit_AVX512F) != 0;
}

// Function pointers for runtime dispatch
void (*process_array)(float*, size_t);

void init_dispatch(void) {
    CpuFeatures feat;
    detect_cpu_features(&feat);

    if (feat.has_avx2) {
        process_array = process_array_avx2;
    } else if (feat.has_sse4_2) {
        process_array = process_array_sse4;
    } else {
        process_array = process_array_scalar;
    }
}
\end{lstlisting}

\subsection{Huge Pages}

\begin{lstlisting}
// Huge pages reduce TLB misses for large allocations

#include <sys/mman.h>

// Allocate with huge pages (Linux)
void* alloc_huge(size_t size) {
    void* ptr = mmap(NULL, size, PROT_READ | PROT_WRITE,
                     MAP_PRIVATE | MAP_ANONYMOUS | MAP_HUGETLB,
                     -1, 0);
    if (ptr == MAP_FAILED) {
        // Fallback to normal pages
        ptr = mmap(NULL, size, PROT_READ | PROT_WRITE,
                   MAP_PRIVATE | MAP_ANONYMOUS, -1, 0);
    }
    return ptr;
}

// Or use transparent huge pages (automatic)
// Check: cat /sys/kernel/mm/transparent_hugepage/enabled

// Benefit: 2MB pages instead of 4KB
// -> 512x fewer TLB entries needed
// -> Significant speedup for large data structures
\end{lstlisting}

\section{Real-World Performance Patterns}

\subsection{SQLite Optimizations}

\begin{lstlisting}
// Lessons from SQLite (one of the most optimized C codebases):

// 1. Small string optimization
typedef struct Mem {
    union {
        double r;      // Real value
        i64 i;         // Integer value
        char* z;       // String (heap-allocated)
        struct {       // Small string (inline)
            char buf[32];
            u8 len;
        } sso;
    } u;
    u16 flags;
} Mem;

// 2. Custom allocators for each subsystem
// - Lookaside allocator for small objects
// - Scratch allocator for temporary data
// - Page cache for database pages

// 3. Computed goto for bytecode interpreter (see earlier section)

// 4. Careful use of likely/unlikely
if (likely(p->nAlloc >= p->nChar + N)) {
    // Fast path
} else {
    // Slow reallocation path
}

// 5. Minimize cache misses with locality
// Store frequently-accessed fields first in structs
\end{lstlisting}

\subsection{Redis Optimizations}

\begin{lstlisting}
// Lessons from Redis (in-memory database):

// 1. SDS: Simple Dynamic String with header
typedef struct {
    uint8_t len;      // Current length (1 byte for short strings)
    uint8_t alloc;    // Allocated capacity
    char buf[];       // Inline string data
} sdshdr8;

// 2. Ziplist: compact list for small lists
// Instead of linked list of objects, use:
// [total-bytes][tail-offset][len][entry1][entry2]...[end]
// Saves pointer overhead, better cache locality

// 3. Lazy free: don't block on large deletes
// Mark for deletion, free in background thread

// 4. Event loop optimization
// Avoid syscalls: batch socket reads/writes

// 5. Memory-efficient encodings
// Small integers: store as immediate values, not pointers
// Small strings: embed in object, not allocated
\end{lstlisting}

\subsection{Linux Kernel Patterns}

\begin{lstlisting}
// Lessons from Linux kernel:

// 1. Likely/unlikely everywhere
if (unlikely(error))
    goto out;

// 2. Per-CPU data structures (avoid cache bouncing)
DEFINE_PER_CPU(struct mystruct, myvar);
get_cpu_var(myvar);  // Access on current CPU
put_cpu_var(myvar);

// 3. RCU (Read-Copy-Update) for scalable reads
// Readers never block, writers copy-modify-replace

// 4. Object pools (slab allocator)
// Pre-constructed objects, cache-aligned

// 5. Inline assembly for critical paths
static __always_inline void spin_lock(spinlock_t *lock) {
    asm volatile(
        "1: lock; decb %0\n\t"
        "jns 3f\n"
        "2: pause\n\t"
        "cmpb $0, %0\n\t"
        "jle 2b\n\t"
        "jmp 1b\n"
        "3:"
        : "+m" (lock->slock)
        :: "memory"
    );
}
\end{lstlisting}

\section{Anti-Patterns: What NOT to Do}

\begin{lstlisting}
// 1. Premature optimization
// Don't optimize before profiling!

// 2. Micro-optimizing cold code
void load_config(void) {
    // This runs once at startup
    // Don't waste time optimizing it!
}

// 3. Sacrificing readability
// Bad: unreadable "optimization"
int x = (a & 0x80) ? ~((a ^ 0xFF) + 1) : a;
// Good: clear code (compiler optimizes anyway)
int x = abs(a);

// 4. Ignoring the profiler
// Your intuition is wrong. Measure first!

// 5. Optimizing the wrong thing
// 90% of time is in one function?
// Optimize that function, not the other 100!

// 6. Breaking portability for tiny gains
// Don't use inline assembly unless you measured 10%+ improvement

// 7. Over-engineering
// Simple O(n) might beat complex O(log n) for small n

// 8. Copying "optimizations" without understanding
// Every codebase is different. Profile YOUR code!
\end{lstlisting}

\section{Checklist: Making Code Fast}

\begin{enumerate}
    \item \textbf{Profile first}: Use gprof, perf, or valgrind
    \item \textbf{Fix algorithms}: O(n log n) beats optimized O(n²)
    \item \textbf{Cache locality}: Sequential access, avoid pointer chasing
    \item \textbf{Reduce allocations}: Pool, arena, or stack allocation
    \item \textbf{Compiler flags}: -O3 -march=native -flto
    \item \textbf{Minimize branches}: Use branchless code or likely/unlikely
    \item \textbf{Vectorize}: Help compiler with restrict, or use SIMD intrinsics
    \item \textbf{Inline hot functions}: Small, frequently called functions
    \item \textbf{Lookup tables}: Precompute expensive operations
    \item \textbf{Fast paths}: Optimize the common case
    \item \textbf{Profile again}: Verify your optimizations worked!
\end{enumerate}

\section{Summary}

Performance optimization in C is about understanding the hardware and helping the compiler help you. The key principles:

\begin{itemize}
    \item \textbf{Cache is king}: Sequential access beats random by 100x
    \item \textbf{Profile everything}: Measure, don't guess
    \item \textbf{Algorithm matters most}: O(n) beats optimized O(n²)
    \item \textbf{Help the compiler}: Use restrict, const, inline, pure
    \item \textbf{Minimize allocations}: Use pools, arenas, stack
    \item \textbf{Branch prediction}: Use likely/unlikely, or go branchless
    \item \textbf{SIMD for parallelism}: 4-16x speedup for data-parallel code
    \item \textbf{Optimize hot paths only}: 90/10 rule—90\% of time in 10\% of code
    \item \textbf{Maintain readability}: Clear code that's 95\% fast beats unreadable code that's 100\% fast
\end{itemize}

The best C programmers don't just write fast code—they write correct, maintainable code that's fast where it matters. Profile first, optimize second, and measure everything!

\begin{tipbox}
\textbf{Remember:} ``Premature optimization is the root of all evil.'' — Donald Knuth

But also: ``We should forget about small efficiencies, say about 97\% of the time... yet we should not pass up our opportunities in that critical 3\%.'' — The rest of that quote!
\end{tipbox}

\section{Legendary Optimizations from History}

These are real stories of performance optimizations that made history. Each one demonstrates a specific technique that led to massive speedups in production systems.

\subsection{Legend 1: id Software's Quake - Fast Inverse Square Root}

One of the most famous optimizations in gaming history:

\begin{lstlisting}
// Quake III Arena (1999)
// Calculate 1/sqrt(x) without division or sqrt()
// Used for vector normalization in 3D graphics

float Q_rsqrt(float number) {
    long i;
    float x2, y;
    const float threehalfs = 1.5F;

    x2 = number * 0.5F;
    y  = number;
    i  = * ( long * ) &y;                       // Evil floating point bit hack
    i  = 0x5f3759df - ( i >> 1 );               // What the fuck?
    y  = * ( float * ) &i;
    y  = y * ( threehalfs - ( x2 * y * y ) );   // 1st Newton-Raphson iteration
//  y  = y * ( threehalfs - ( x2 * y * y ) );   // 2nd iteration (optional)

    return y;
}

// Traditional method: ~30-40 cycles
// This method: ~10 cycles
// Speedup: 3-4x faster!

// Why it worked:
// - Clever bit manipulation estimates sqrt
// - One Newton-Raphson iteration refines it
// - Good enough for graphics (99.8% accurate)

// Impact: Enabled smooth 3D graphics on 1990s hardware
// Used in: Quake III, Unreal, countless games

// Modern note: CPUs now have RSQRTSS instruction (even faster)
// But this remains a legendary example of creative optimization
\end{lstlisting}

\subsection{Legend 2: Linux Kernel - RCU (Read-Copy-Update)}

Revolutionary synchronization mechanism (2002):

\begin{lstlisting}
// Problem: rwlock causes cache-line bouncing
// Readers acquire lock -> cache line invalidated on all CPUs

// Traditional rwlock
struct data {
    int value;
    rwlock_t lock;
};

void reader(struct data* d) {
    read_lock(&d->lock);     // Cache miss on every read!
    int v = d->value;
    read_unlock(&d->lock);   // Another cache miss!
}

void writer(struct data* d, int new_val) {
    write_lock(&d->lock);
    d->value = new_val;
    write_unlock(&d->lock);
}

// RCU: readers never block, no cache bouncing
struct data {
    int value;
};

void reader_rcu(struct data* d) {
    rcu_read_lock();         // No atomic ops, just barrier
    struct data* p = rcu_dereference(d);
    int v = p->value;        // Direct read, no cache ping-pong
    rcu_read_unlock();       // Just a barrier
}

void writer_rcu(struct data** d, int new_val) {
    struct data* new = malloc(sizeof(struct data));
    new->value = new_val;
    rcu_assign_pointer(*d, new);  // Atomic pointer swap
    synchronize_rcu();             // Wait for readers
    free(old);                     // Safe to free
}

// Performance impact:
// - Readers: 10-100x faster (no cache bouncing)
// - Scales to 100+ CPUs
// - Network stack speedup: 40%
// - VFS (filesystem) speedup: 60%

// Used in: Linux kernel (routing tables, file descriptor lookup)
// Invented by: Paul McKenney, IBM
// Impact: Made Linux scale to massive servers
\end{lstlisting}

\subsection{Legend 3: zlib - Huffman Coding Table Optimization}

Mark Adler's zlib optimization (1995):

\begin{lstlisting}
// Original: decode one bit at a time
int decode_symbol_slow(bitstream* bs, tree* t) {
    node* n = t->root;
    while (!n->is_leaf) {
        int bit = read_bit(bs);
        n = bit ? n->right : n->left;
    }
    return n->symbol;
}

// Optimized: decode multiple bits at once with lookup table
#define LOOKUP_BITS 9  // Decode 9 bits at once

int decode_symbol_fast(bitstream* bs, tree* t) {
    // Peek 9 bits
    int bits = peek_bits(bs, LOOKUP_BITS);

    // Table lookup gives symbol + bit count
    int entry = t->table[bits];
    int len = entry & 0xF;
    int symbol = entry >> 4;

    consume_bits(bs, len);
    return symbol;
}

// Build the table (done once at startup)
void build_table(tree* t) {
    // For every possible 9-bit sequence
    for (int i = 0; i < 512; i++) {
        // Trace through tree to find symbol
        // Store symbol and path length
        t->table[i] = (symbol << 4) | len;
    }
}

// Performance:
// Before: 8-15 cycles per symbol
// After: 2-4 cycles per symbol
// Speedup: 3-5x faster decompression

// Memory cost: 512 * 2 bytes = 1 KB
// Trade: 1 KB memory for 3-5x speed (worth it!)

// Impact: zlib became the standard (gzip, PNG, HTTP compression)
// Used by: Billions of devices, every web browser, Linux kernel
\end{lstlisting}

\subsection{Legend 4: SQLite - Virtual Database Engine}

D. Richard Hipp's bytecode VM (2001):

\begin{lstlisting}
// Traditional SQL engine: interpret AST
// Problem: Function call overhead, branch mispredictions

// SQLite: compile SQL to bytecode, use computed goto
enum OpCode {
    OP_Column, OP_Add, OP_Eq, OP_Goto, OP_Return, ...
};

// Computed goto dispatch (see earlier section)
static void* opcodes[] = {
    &&op_column, &&op_add, &&op_eq, &&op_goto, &&op_return
};

#define DISPATCH() goto *opcodes[pc++->opcode]

void execute(Instruction* program) {
    Instruction* pc = program;
    DISPATCH();

op_column:
    // Fetch column from current row
    stack[++sp] = cursor->column[pc->p1];
    DISPATCH();

op_add:
    // Pop two values, add, push result
    stack[sp-1] = stack[sp-1] + stack[sp];
    sp--;
    DISPATCH();

// ... other opcodes ...
}

// Additional optimization: instruction specialization
// Instead of generic "Op_Column", generate:
// - Op_Column_Int (for integer columns)
// - Op_Column_Text (for text columns)
// - Op_Column_Null (for NULL)

// Performance impact:
// vs. MySQL (interpreted AST): 2-3x faster
// vs. PostgreSQL (similar): competitive
// Code size: smaller (bytecode is compact)

// Why successful:
// 1. Computed goto eliminates indirect call overhead
// 2. Specialized opcodes reduce branching
// 3. Stack-based VM is cache-friendly
// 4. Instruction stream is sequential (good prefetch)

// Impact: SQLite runs on 4+ billion devices
// Used in: Every smartphone, web browser, car, IoT device
\end{lstlisting}

\subsection{Legend 5: DOOM's Visplane Optimization}

John Carmack's renderer optimization (1993):

\begin{lstlisting}
// Original: draw floors/ceilings pixel-by-pixel
void draw_floor_slow(int x, int y1, int y2) {
    for (int y = y1; y < y2; y++) {
        // Calculate texture coordinates for each pixel
        float dist = calculate_distance(x, y);
        float u = x / dist;
        float v = y / dist;
        int color = texture_lookup(u, v);
        put_pixel(x, y, color);
    }
}

// Optimized: "visplane" - track floor spans
typedef struct {
    int y;
    int x1, x2;  // Start and end X
} Span;

Span spans[MAX_SPANS];
int span_count;

void collect_span(int y, int x1, int x2) {
    spans[span_count].y = y;
    spans[span_count].x1 = x1;
    spans[span_count].x2 = x2;
    span_count++;
}

void draw_spans_optimized(void) {
    // Sort spans by Y coordinate
    qsort(spans, span_count, sizeof(Span), compare_y);

    // Draw horizontal spans (cache-friendly!)
    for (int i = 0; i < span_count; i++) {
        Span* s = &spans[i];
        float dist_start = calculate_distance(s->x1, s->y);
        float dist_end = calculate_distance(s->x2, s->y);

        // Linear interpolation across span
        for (int x = s->x1; x <= s->x2; x++) {
            float t = (float)(x - s->x1) / (s->x2 - s->x1);
            float dist = lerp(dist_start, dist_end, t);
            // ... rest of texture mapping
        }
    }
}

// Key insights:
// 1. Batch spans together
// 2. Process horizontally (cache-friendly)
// 3. Linear interpolation is cheaper than per-pixel calculation
// 4. Reduce division operations (expensive on 486)

// Performance:
// 486DX/33 MHz: 15 FPS -> 35 FPS
// Speedup: 2.3x

// Impact: Made DOOM possible on low-end PCs
// Enabled: The entire FPS genre
\end{lstlisting}

\subsection{Legend 6: Git's Pack File Format}

Linus Torvalds' delta compression (2005):

\begin{lstlisting}
// Problem: Linux kernel repo = 500 MB of files
// Traditional: store each version separately
// Git: delta compression

// Store base object + deltas
typedef struct {
    uint8_t type;
    uint32_t base_offset;  // Points to base version
    uint32_t delta_size;
    uint8_t* delta_data;   // "Insert X bytes at offset Y"
} PackedObject;

// Delta encoding
void create_delta(uint8_t* base, size_t base_len,
                  uint8_t* target, size_t target_len,
                  uint8_t** delta, size_t* delta_len) {
    // Use sliding window to find matching blocks
    for (size_t i = 0; i < target_len; ) {
        // Find longest match in base
        Match m = find_match(base, base_len, target + i, target_len - i);

        if (m.len > 8) {
            // Copy from base
            emit_copy(delta, m.base_offset, m.len);
            i += m.len;
        } else {
            // Insert literal byte
            emit_insert(delta, target[i]);
            i++;
        }
    }
}

// Pack file structure:
// [Header][Obj1][Obj2][Delta1][Delta2]...
// Delta1 -> "Based on Obj1, insert/copy to make version 2"

// Performance impact:
// - Linux kernel: 500 MB -> 150 MB (3.3x compression)
// - Clone speed: 3x faster (less data to transfer)
// - Disk usage: 3x smaller

// Innovation: Delta against ANY previous object, not just parent
// Traditional VCS: Delta only against direct parent
// Git: Delta against any similar object (even in different directory!)

// Additional optimization: delta chains
// Version 1 -> Version 2 -> Version 3
// But limit chain depth to avoid decompression overhead

// Impact: Made distributed version control practical
// Enabled: GitHub, millions of repositories
\end{lstlisting}

\subsection{Legend 7: Redis's Ziplist}

Salvatore Sanfilippo's memory-efficient list (2009):

\begin{lstlisting}
// Traditional linked list
typedef struct Node {
    void* data;
    struct Node* next;
    struct Node* prev;
} Node;  // 24 bytes overhead per node (on 64-bit)!

// For list of 100 small integers: 2,400 bytes overhead

// Redis ziplist: compact encoding
// [total_bytes][tail_offset][count][entry1][entry2]...[end]
//
// Each entry:
// [prev_len][encoding][data]
//
// prev_len: 1 or 5 bytes (small or large)
// encoding: 1 byte (says if int or string, and size)
// data: actual data

// Example: list [1, 2, 3, 127, 128]
// Traditional: 24*5 = 120 bytes overhead + data
// Ziplist: 11 bytes header + 1 byte per small int = 16 bytes total!
// Savings: 87%!

typedef struct {
    uint32_t total_bytes;
    uint32_t tail_offset;
    uint16_t count;
    uint8_t entries[];  // Flexible array
} Ziplist;

// Encoding tricks
#define ENCODING_INT8   0xFE   // 1-byte integer
#define ENCODING_INT16  0xC0   // 2-byte integer
#define ENCODING_INT32  0xD0   // 4-byte integer
#define ENCODING_STR    0x00   // String (length follows)

uint8_t* ziplist_push(Ziplist* zl, int value) {
    // Choose smallest encoding
    uint8_t encoding;
    int len;

    if (value >= -128 && value <= 127) {
        encoding = ENCODING_INT8;
        len = 1;
    } else if (value >= -32768 && value <= 32767) {
        encoding = ENCODING_INT16;
        len = 2;
    } else {
        encoding = ENCODING_INT32;
        len = 4;
    }

    // Grow ziplist, insert entry
    // ...
}

// Performance:
// Memory: 70-95% less than linked list
// Cache: Much better (contiguous memory)
// Speed: Iteration is 10x faster (no pointer chasing)
// Trade-off: Insertions can be O(n) due to reallocation

// When to use:
// - Small lists (< 512 entries)
// - Mostly append/read workload
// - Memory is more important than insertion speed

// Impact: Redis uses 30-50% less memory for typical workloads
// Used for: Small hashes, lists, sorted sets
// Result: Can fit 2x more data in same RAM
\end{lstlisting}

\subsection{Legend 8: LuaJIT's Trace Compiler}

Mike Pall's JIT compiler (2005-2013):

\begin{lstlisting}
// Problem: Dynamic languages are slow (10-100x slower than C)
// Traditional JIT: Compile whole functions

// LuaJIT innovation: Trace compilation

// 1. Interpreter runs and records "hot" loops
// 2. When loop runs 50+ times, start tracing
// 3. Record actual operations executed (with types!)
// 4. Compile trace to machine code
// 5. Execute compiled trace

// Example Lua code:
// function sum(n)
//     local s = 0
//     for i = 1, n do
//         s = s + i
//     end
//     return s
// end

// Trace recorded (with type information):
// i:int = 1
// s:int = 0
// loop:
//   if i > n:int then goto exit
//   s:int = s:int + i:int  // Types known!
//   i:int = i:int + 1
//   goto loop
// exit:
//   return s:int

// Compiled to x86:
//   xor eax, eax        ; s = 0
//   mov ecx, 1          ; i = 1
// loop:
//   cmp ecx, [n]
//   jg exit
//   add eax, ecx        ; s += i (integer add, not slow Lua add!)
//   inc ecx
//   jmp loop
// exit:
//   ret

// Key innovations:
// 1. Type specialization (integer add, not generic add)
// 2. Traces are linear (good for CPU prediction)
// 3. SSA form optimization
// 4. SIMD for array operations
// 5. FFI (call C without overhead)

// Performance vs. Standard Lua:
// - Numeric code: 50-100x faster
// - Overall: 10-50x faster
// - Sometimes matches C speed!

// Comparison:
// Python (CPython): 1x
// Python (PyPy): 5x
// Lua (standard): 1x
// Lua (LuaJIT): 20-50x

// Impact: Made Lua viable for performance-critical applications
// Used in: Game engines (World of Warcraft, Roblox), nginx,
//          network appliances, embedded systems
\end{lstlisting}

\subsection{Legend 9: nginx's Event-Driven Architecture}

Igor Sysoev's async I/O server (2004):

\begin{lstlisting}
// Apache model: one process/thread per connection
// Problem: 10,000 connections = 10,000 threads
// Each thread: 1-8 MB stack
// Total: 10-80 GB RAM just for stacks!

void apache_handle_request(int socket) {
    char buffer[8192];

    // Blocking reads - thread sleeps here
    int n = read(socket, buffer, sizeof(buffer));

    // Process request
    process_request(buffer, n);

    // Blocking write - thread sleeps here
    write(socket, response, response_len);

    close(socket);
}

// nginx model: event loop with non-blocking I/O
// One worker per CPU core
// Each worker handles thousands of connections

typedef struct {
    int fd;
    int state;  // READING, PROCESSING, WRITING
    char* buffer;
    size_t buffer_size;
    void (*handler)(struct connection*);
} Connection;

void nginx_worker(void) {
    Connection connections[10000];
    int epoll_fd = epoll_create(10000);

    while (1) {
        // Wait for events (non-blocking)
        struct epoll_event events[100];
        int n = epoll_wait(epoll_fd, events, 100, -1);

        for (int i = 0; i < n; i++) {
            Connection* conn = events[i].data.ptr;

            switch (conn->state) {
            case READING:
                // Read what's available (non-blocking)
                int n = read(conn->fd, conn->buffer, conn->buffer_size);
                if (n > 0) {
                    conn->state = PROCESSING;
                    conn->handler(conn);
                }
                break;

            case WRITING:
                // Write what's possible (non-blocking)
                write(conn->fd, conn->response, conn->response_len);
                conn->state = DONE;
                break;
            }
        }
    }
}

// Performance comparison:
// Apache (10,000 connections):
// - Memory: 10 GB (threads)
// - Context switches: constant
// - Throughput: 2,000 req/sec

// nginx (10,000 connections):
// - Memory: 100 MB
// - Context switches: minimal
// - Throughput: 50,000 req/sec

// Speedup: 25x more throughput, 100x less memory!

// Additional optimizations:
// 1. Cache-friendly data structures
// 2. Memory pools (no malloc in hot path)
// 3. Zero-copy sendfile()
// 4. TCP_CORK for optimal packet size

// Impact: Powers 30%+ of top websites
// Used by: Netflix, Cloudflare, WordPress.com
// Inspired: Node.js, Tornado, many others
\end{lstlisting}

\subsection{Legend 10: Doom 3's Reverse-Z Depth Buffer}

Fabian Giesen's floating-point depth trick (2004):

\begin{lstlisting}
// Traditional depth buffer: 0.0 (near) to 1.0 (far)
// Problem: floating-point precision is non-uniform

// IEEE 754 float:
// - Near 0: very precise (0.0000001 spacing)
// - Near 1: less precise (0.000001 spacing)
// - Far objects get "z-fighting" artifacts

// Depth calculation (traditional)
float depth_traditional(float z_near, float z_far, float z) {
    // Maps [z_near, z_far] to [0.0, 1.0]
    return (z_far * (z - z_near)) / (z * (z_far - z_near));
}

// Near plane (z=1): depth = 0.000001 precision
// Far plane (z=1000): depth = 0.0001 precision
// Precision loss: 100x!

// Reverse-Z: 1.0 (near) to 0.0 (far)
float depth_reverse_z(float z_near, float z_far, float z) {
    // Maps [z_near, z_far] to [1.0, 0.0]
    return (z_near * (z_far - z)) / (z * (z_far - z_near));
}

// Near plane (z=1): depth = 0.9999999 -> high precision near 1.0
// Far plane (z=1000): depth = 0.001 -> still good precision near 0.0
// Precision: More uniform across depth range!

// Why it works:
// Float has more precision near 0
// By putting FAR at 0, we use more bits for distant objects
// By putting NEAR at 1, we still have precision for close objects

// Performance impact:
// - Eliminates z-fighting at distance
// - Can use larger view distances
// - No extra computation (just flip comparison)

// Setup:
// 1. Reverse projection matrix
// 2. Change depth test: GREATER instead of LESS
// 3. Clear depth buffer to 0.0 instead of 1.0

glDepthFunc(GL_GREATER);  // Reverse comparison
glClearDepth(0.0);        // Clear to 0 instead of 1

// Quality improvement:
// Traditional 24-bit depth:
// - Usable range: 1 - 1000 units
// - Z-fighting beyond 500 units

// Reverse-Z 24-bit depth:
// - Usable range: 1 - 1,000,000 units!
// - Clean rendering even at extreme distances

// 32-bit float depth with reverse-Z:
// - Practically infinite precision
// - Can use z_near = 0.01, z_far = infinity

// Impact: Now standard in modern engines
// Used by: Unreal Engine 4+, Unity, Frostbite, id Tech
// Solved: 20-year-old depth precision problem
\end{lstlisting}

\subsection{Bonus Legend: Michael Abrash's Mode X}

VGA programming trick (1991):

\begin{lstlisting}
// Standard VGA mode 13h: 320x200, 256 colors
// Problem: linear frame buffer, slow writes

// Mode X: Undocumented VGA mode
// Innovation: Planar memory access

// Standard mode: write each pixel
void draw_line_mode13(int y, int x1, int x2, uint8_t color) {
    uint8_t* screen = (uint8_t*)0xA0000;
    for (int x = x1; x <= x2; x++) {
        screen[y * 320 + x] = color;  // One byte at a time
    }
}

// Mode X: write 4 pixels at once
void draw_line_modeX(int y, int x1, int x2, uint8_t color) {
    // VGA has 4 planes, can write to all at once
    uint8_t* screen = (uint8_t*)0xA0000;

    // Set write mode: all 4 planes
    outportb(0x3C4, 0x02);
    outportb(0x3C5, 0x0F);  // Enable all 4 planes

    // One write updates 4 pixels!
    int offset = (y * 320 + x1) / 4;
    for (int x = x1; x <= x2; x += 4) {
        screen[offset++] = color;  // 4 pixels in one write!
    }
}

// Performance:
// Mode 13h: 320x200 clear = 64,000 writes
// Mode X: 320x200 clear = 16,000 writes (4x faster!)

// Additional benefits:
// - Page flipping (double buffering)
// - Hardware scrolling
// - 320x240 resolution (instead of 200)

// Impact:
// - Enabled smooth animation on 386/486
// - Used by: Doom, Duke Nukem 3D, hundreds of DOS games
// - Explained in: Michael Abrash's Graphics Programming Black Book
// - Became required knowledge for DOS game programmers

// This optimization required understanding VGA hardware
// deeply - true systems programming
\end{lstlisting}

\subsection{Lessons from the Legends}

What these legendary optimizations teach us:

\begin{enumerate}
    \item \textbf{Understand your hardware}: Quake's fast inverse sqrt, Mode X
    \item \textbf{Question assumptions}: Reverse-Z (put far at 0, not near)
    \item \textbf{Trade memory for speed}: zlib lookup tables (1KB → 3x faster)
    \item \textbf{Trade CPU for memory}: Redis ziplist (slower insert, 90\% less memory)
    \item \textbf{Specialize for common case}: LuaJIT traces with type info
    \item \textbf{Eliminate synchronization}: Linux RCU (readers never block)
    \item \textbf{Batch operations}: DOOM visplanes (process spans together)
    \item \textbf{Compression wins}: Git delta encoding (3x less data)
    \item \textbf{Event-driven scales}: nginx (25x more throughput)
    \item \textbf{Measure and profile}: All of them profiled first!
\end{enumerate}

These aren't just performance tricks—they're \textbf{paradigm shifts} that changed what was possible in software. Study them, understand the principles, and apply those principles to your own code!

\chapter{Platform-Specific Code: The Complete Cross-Platform Survival Guide}

\section{Introduction: The Portability Nightmare and How to Tame It}

Writing portable C code isn't just about using \texttt{\#ifdef}---it's about understanding fundamental differences between Windows, Linux, macOS, and hybrid environments like MSYS2/MinGW/Cygwin. This chapter covers \textbf{everything} real projects like cURL, FFmpeg, Git, CMake, Python, and libuv deal with to compile and run everywhere.

\subsection{Why This Chapter Exists}

If you've ever tried to compile a Linux program on Windows, or vice versa, you've discovered a harsh truth: \textbf{C is not automatically portable}. The language itself is standardized, but real programs need to:

\begin{itemize}
    \item Open files and directories
    \item Create network connections
    \item Spawn processes
    \item Handle keyboard interrupts
    \item Display colored terminal output
    \item Load plugins dynamically
    \item Measure time accurately
\end{itemize}

None of these have a standard C solution that works everywhere. Each requires platform-specific code.

\subsection{What You'll Learn}

This chapter teaches you to write C code that actually works across platforms by:

\begin{enumerate}
    \item \textbf{Understanding the platforms}: Windows is not Unix with a GUI. We'll explain the fundamental architectural differences.

    \item \textbf{Detecting your environment correctly}: Not all Windows builds are the same. MinGW, MSYS2, Cygwin, and MSVC all behave differently.

    \item \textbf{Abstracting platform differences}: Learn to create thin wrapper layers that hide platform-specific APIs behind clean, uniform interfaces.

    \item \textbf{Avoiding common pitfalls}: We'll show you the gotchas that bite everyone (like forgetting \texttt{WSAStartup()} on Windows).

    \item \textbf{Testing properly}: Your code compiling on Linux doesn't mean it works on Windows, even if you used \texttt{\#ifdef}.
\end{enumerate}

\begin{warningbox}
\textbf{Reality Check:} True portability is hard. Major projects have 30-50\% of their code dedicated to platform abstraction. This isn't a flaw---it's necessity. The good news? Most of that code follows established patterns you can learn.
\end{warningbox}

\subsection{Chapter Roadmap}

We'll systematically cover every major platform difference:

\begin{itemize}
    \item \textbf{Platform Detection}: How to reliably identify your OS, compiler, and build environment (including MSYS2/Cygwin traps)

    \item \textbf{Networking}: Winsock vs BSD sockets---they look similar but are fundamentally different. We'll show you how to write code that works with both.

    \item \textbf{Console/Terminal}: Colors, raw mode, Unicode output---all different across platforms.

    \item \textbf{Character Encoding}: Windows uses UTF-16, everyone else uses UTF-8. This affects everything.

    \item \textbf{File System}: Path separators, case sensitivity, length limits, and permissions all vary.

    \item \textbf{Process Management}: \texttt{fork()} doesn't exist on Windows. Learn the portable alternatives.

    \item \textbf{Line Endings}: CRLF vs LF matters more than you think, especially in binary protocols.

    \item \textbf{Dynamic Libraries}: .dll vs .so vs .dylib---different extensions, different APIs, different calling conventions.

    \item \textbf{Signals and Events}: Unix signals vs Windows console events---completely different models.

    \item \textbf{Time Functions}: \texttt{sleep()}, \texttt{Sleep()}, \texttt{nanosleep()}---which to use when?

    \item \textbf{Environment Variables}: Even this simple thing has platform quirks.
\end{itemize}

By the end of this chapter, you'll understand why projects like SQLite, cURL, and Git have dedicated platform abstraction layers---and you'll know how to build your own.

\begin{tipbox}
\textbf{Pro Tip:} Don't try to memorize everything here. Use this chapter as a reference. When you encounter a platform-specific problem, come back and find the relevant section. Over time, these patterns will become second nature.
\end{tipbox}

\section{Platform and Environment Detection: The Complete Matrix}

\subsection{Understanding the Windows Build Environments}

Before we detect anything, you need to understand what you're dealing with. This is crucial because many developers assume "Windows" is one thing, when it's actually four completely different C development environments:

\begin{lstlisting}
// Windows has FOUR different C development environments:
//
// 1. Native Windows (MSVC / Visual Studio):
//    - Microsoft's compiler (cl.exe)
//    - Windows API exclusively
//    - wchar_t for Unicode (UTF-16)
//    - Winsock2 for networking
//    - Windows threads (CreateThread)
//    - Example: Most commercial Windows software
//
// 2. MinGW (Minimalist GNU for Windows):
//    - GCC compiler targeting native Windows
//    - Windows API (CreateFile, etc.)
//    - Some POSIX wrappers (open() wraps CreateFile)
//    - Winsock2 for networking
//    - Can mix Windows and limited POSIX
//    - Example: GCC-compiled Windows executables
//
// 3. MSYS2:
//    - Build environment with Unix tools
//    - Uses MinGW-w64 compiler
//    - Programs still use Windows API at runtime
//    - Bash shell for building
//    - Example: Building Unix projects on Windows
//
// 4. Cygwin:
//    - Full POSIX compatibility layer
//    - cygwin1.dll translates POSIX to Windows
//    - BSD sockets (not Winsock)
//    - fork() works!
//    - Programs depend on cygwin1.dll
//    - Example: Running Unix programs on Windows
//
// Key insight: MSYS2 is a BUILD environment.
// Your program still runs as native Windows!
\end{lstlisting}

\subsection{Comprehensive Platform Detection}

\begin{lstlisting}
// platform.h - Industrial-strength platform detection
#ifndef PLATFORM_H
#define PLATFORM_H

/* ===== STEP 1: Compiler Detection (FIRST!) ===== */

#if defined(_MSC_VER)
    #define COMPILER_MSVC
    #define COMPILER_VERSION _MSC_VER
    #define COMPILER_NAME "MSVC"

    // MSVC implies native Windows
    #define PLATFORM_WINDOWS
    #define NATIVE_WINDOWS

#elif defined(__MINGW32__) || defined(__MINGW64__)
    #define COMPILER_MINGW
    #define COMPILER_NAME "MinGW"
    #define PLATFORM_WINDOWS
    #define MINGW_WINDOWS

    #ifdef __MINGW64__
        #define MINGW64
    #else
        #define MINGW32
    #endif

    // MinGW can be detected as GCC too
    #if defined(__GNUC__)
        #define COMPILER_GCC_COMPATIBLE
    #endif

#elif defined(__CYGWIN__)
    #define COMPILER_GCC
    #define COMPILER_NAME "GCC (Cygwin)"
    #define PLATFORM_CYGWIN
    #define POSIX_ON_WINDOWS

    // Cygwin is POSIX-like despite being on Windows
    #define HAVE_POSIX

#elif defined(__clang__)
    #define COMPILER_CLANG
    #define COMPILER_NAME "Clang"
    #define COMPILER_VERSION \
        (__clang_major__ * 10000 + __clang_minor__ * 100 + __clang_patchlevel__)
    #define COMPILER_GCC_COMPATIBLE

#elif defined(__GNUC__)
    #define COMPILER_GCC
    #define COMPILER_NAME "GCC"
    #define COMPILER_VERSION \
        (__GNUC__ * 10000 + __GNUC_MINOR__ * 100 + __GNUC_PATCHLEVEL__)
    #define COMPILER_GCC_COMPATIBLE

#elif defined(__INTEL_COMPILER) || defined(__ICC)
    #define COMPILER_INTEL
    #define COMPILER_NAME "Intel C"

#elif defined(__PGI)
    #define COMPILER_PGI
    #define COMPILER_NAME "PGI"

#else
    #define COMPILER_UNKNOWN
    #define COMPILER_NAME "Unknown"
#endif

/* ===== STEP 2: Operating System Detection ===== */

#ifndef PLATFORM_WINDOWS
    #if defined(_WIN32) || defined(_WIN64) || defined(__WIN32__) || \
        defined(__TOS_WIN__) || defined(__WINDOWS__)
        #define PLATFORM_WINDOWS
        #define NATIVE_WINDOWS
    #endif
#endif

#if defined(__APPLE__) && defined(__MACH__)
    #include <TargetConditionals.h>
    #define PLATFORM_APPLE
    #define PLATFORM_BSD_LIKE

    #if TARGET_OS_IPHONE || TARGET_IPHONE_SIMULATOR
        #define PLATFORM_IOS
    #elif TARGET_OS_MAC
        #define PLATFORM_MACOS
        #define PLATFORM_MACOS_DESKTOP
    #elif TARGET_OS_TV
        #define PLATFORM_TVOS
    #elif TARGET_OS_WATCH
        #define PLATFORM_WATCHOS
    #endif

#elif defined(__linux__)
    #define PLATFORM_LINUX
    #define PLATFORM_UNIX_LIKE

    #ifdef __ANDROID__
        #define PLATFORM_ANDROID
    #endif

    // Detect Linux distributions (best effort)
    #if defined(__GLIBC__)
        #define LIBC_GLIBC
    #elif defined(__MUSL__)
        #define LIBC_MUSL
    #elif defined(__UCLIBC__)
        #define LIBC_UCLIBC
    #endif

#elif defined(__FreeBSD__)
    #define PLATFORM_FREEBSD
    #define PLATFORM_BSD_LIKE
    #define PLATFORM_UNIX_LIKE

#elif defined(__OpenBSD__)
    #define PLATFORM_OPENBSD
    #define PLATFORM_BSD_LIKE
    #define PLATFORM_UNIX_LIKE

#elif defined(__NetBSD__)
    #define PLATFORM_NETBSD
    #define PLATFORM_BSD_LIKE
    #define PLATFORM_UNIX_LIKE

#elif defined(__DragonFly__)
    #define PLATFORM_DRAGONFLY
    #define PLATFORM_BSD_LIKE
    #define PLATFORM_UNIX_LIKE

#elif defined(__unix__) || defined(__unix)
    #define PLATFORM_UNIX
    #define PLATFORM_UNIX_LIKE

#elif defined(__sun)
    #if defined(__SVR4) || defined(__svr4__)
        #define PLATFORM_SOLARIS
    #else
        #define PLATFORM_SUNOS
    #endif
    #define PLATFORM_UNIX_LIKE

#elif defined(__hpux) || defined(_hpux)
    #define PLATFORM_HPUX
    #define PLATFORM_UNIX_LIKE

#elif defined(_AIX)
    #define PLATFORM_AIX
    #define PLATFORM_UNIX_LIKE

#elif defined(__QNX__) || defined(__QNXNTO__)
    #define PLATFORM_QNX
    #define PLATFORM_UNIX_LIKE

#elif defined(__HAIKU__)
    #define PLATFORM_HAIKU
    #define PLATFORM_UNIX_LIKE

#else
    #define PLATFORM_UNKNOWN
#endif

/* ===== STEP 3: POSIX Feature Detection ===== */

#if defined(PLATFORM_LINUX) || defined(PLATFORM_BSD_LIKE) || \
    defined(PLATFORM_CYGWIN) || defined(PLATFORM_MACOS) || \
    defined(PLATFORM_UNIX_LIKE)
    #define HAVE_POSIX
#endif

/* ===== STEP 4: Architecture Detection ===== */

#if defined(__x86_64__) || defined(_M_X64) || defined(__amd64__)
    #define ARCH_X86_64
    #define ARCH_64BIT
    #define ARCH_NAME "x86_64"

#elif defined(__i386__) || defined(_M_IX86) || defined(__i386) || \
      defined(__i486__) || defined(__i586__) || defined(__i686__)
    #define ARCH_X86
    #define ARCH_32BIT
    #define ARCH_NAME "x86"

#elif defined(__aarch64__) || defined(_M_ARM64) || defined(__arm64__)
    #define ARCH_ARM64
    #define ARCH_64BIT
    #define ARCH_NAME "arm64"

#elif defined(__arm__) || defined(_M_ARM) || defined(__arm)
    #define ARCH_ARM
    #define ARCH_32BIT
    #define ARCH_NAME "arm"

#elif defined(__riscv)
    #if __riscv_xlen == 64
        #define ARCH_RISCV64
        #define ARCH_64BIT
        #define ARCH_NAME "riscv64"
    #else
        #define ARCH_RISCV32
        #define ARCH_32BIT
        #define ARCH_NAME "riscv32"
    #endif

#elif defined(__powerpc64__) || defined(__ppc64__) || defined(__PPC64__)
    #define ARCH_PPC64
    #define ARCH_64BIT
    #define ARCH_NAME "ppc64"

#elif defined(__powerpc__) || defined(__ppc__) || defined(__PPC__)
    #define ARCH_PPC
    #define ARCH_32BIT
    #define ARCH_NAME "ppc"

#elif defined(__mips64)
    #define ARCH_MIPS64
    #define ARCH_64BIT
    #define ARCH_NAME "mips64"

#elif defined(__mips__)
    #define ARCH_MIPS
    #define ARCH_32BIT
    #define ARCH_NAME "mips"

#elif defined(__s390x__)
    #define ARCH_S390X
    #define ARCH_64BIT
    #define ARCH_NAME "s390x"

#elif defined(__sparc64__)
    #define ARCH_SPARC64
    #define ARCH_64BIT
    #define ARCH_NAME "sparc64"

#elif defined(__sparc__)
    #define ARCH_SPARC
    #define ARCH_32BIT
    #define ARCH_NAME "sparc"

#elif defined(__ia64__) || defined(_M_IA64)
    #define ARCH_IA64
    #define ARCH_64BIT
    #define ARCH_NAME "ia64"

#else
    #define ARCH_UNKNOWN
    #define ARCH_NAME "unknown"
#endif

/* ===== STEP 5: Pointer Size and Data Model ===== */

#if defined(_WIN64) || defined(__LP64__) || defined(_LP64) || \
    defined(__x86_64__) || defined(__aarch64__)
    #define PLATFORM_64BIT
    typedef unsigned long long uintptr_sized_t;
#else
    #define PLATFORM_32BIT
    typedef unsigned int uintptr_sized_t;
#endif

/* ===== STEP 6: Endianness Detection ===== */

#if defined(__BYTE_ORDER__) && defined(__ORDER_LITTLE_ENDIAN__) && \
    __BYTE_ORDER__ == __ORDER_LITTLE_ENDIAN__
    #define PLATFORM_LITTLE_ENDIAN
#elif defined(__BYTE_ORDER__) && defined(__ORDER_BIG_ENDIAN__) && \
      __BYTE_ORDER__ == __ORDER_BIG_ENDIAN__
    #define PLATFORM_BIG_ENDIAN
#elif defined(__i386__) || defined(__x86_64__) || defined(_M_IX86) || \
      defined(_M_X64)
    // x86 is always little-endian
    #define PLATFORM_LITTLE_ENDIAN
#elif defined(__ARMEB__)
    #define PLATFORM_BIG_ENDIAN
#elif defined(__ARMEL__)
    #define PLATFORM_LITTLE_ENDIAN
#else
    // Runtime detection needed
    #define PLATFORM_ENDIAN_UNKNOWN
#endif

/* ===== STEP 7: API Selection ===== */

#if defined(PLATFORM_WINDOWS) && !defined(PLATFORM_CYGWIN)
    #define USE_WINDOWS_API
    #define USE_WINSOCK
#else
    #define USE_POSIX_API
    #define USE_BSD_SOCKETS
#endif

#endif // PLATFORM_H
\end{lstlisting}

\begin{notebox}
\textbf{Why so complex?} Because \texttt{\#ifdef \_WIN32} isn't enough! Cygwin defines \texttt{\_WIN32} but doesn't use Windows APIs. MinGW uses Windows APIs but with GCC. MSVC has different intrinsics than GCC. Real portability requires understanding all these nuances.

\textbf{How to use this:} Include this header in your project, then use the defined macros throughout your code. For example:
\begin{itemize}
    \item Use \texttt{PLATFORM\_WINDOWS} to detect any Windows build
    \item Use \texttt{NATIVE\_WINDOWS} to detect native Windows (not Cygwin)
    \item Use \texttt{HAVE\_POSIX} to check for POSIX API availability
    \item Use \texttt{COMPILER\_MSVC} for MSVC-specific code
\end{itemize}

This approach scales much better than scattered \texttt{\#ifdef} checks throughout your codebase.
\end{notebox}


\section{Networking: The Winsock vs BSD Sockets Nightmare}

\subsection{Why Networking is Different on Windows}

The socket API on Unix (BSD sockets) and Windows (Winsock) look similar but have critical differences that will break your code if you're not careful. Here's what makes Windows networking special:

\begin{enumerate}
    \item \textbf{Initialization Required}: On Windows, you MUST call \texttt{WSAStartup()} before using any socket functions. Forget this, and all socket operations silently fail or return cryptic errors.

    \item \textbf{Different Types}: Unix uses \texttt{int} for socket descriptors. Windows uses \texttt{SOCKET}, which is an unsigned type. This matters for error checking.

    \item \textbf{Different Error Codes}: Unix sets \texttt{errno}. Windows requires \texttt{WSAGetLastError()}. They're not compatible.

    \item \textbf{Cleanup Required}: Windows requires \texttt{WSACleanup()} when done. Unix doesn't need this.

    \item \textbf{Different Close Function}: Unix uses \texttt{close()}. Windows uses \texttt{closesocket()}. Using the wrong one fails.

    \item \textbf{Blocking Behavior}: Setting non-blocking mode uses different functions and flags on each platform.
\end{enumerate}

The good news? With proper abstractions, you can write networking code once and have it work everywhere. Let's build those abstractions.

Windows doesn't use BSD sockets directly---it uses Winsock, which is \textit{inspired by} BSD sockets but has critical differences:


\subsection{The Problem in Detail}

Here's what happens if you naively write cross-platform socket code:

\begin{lstlisting}
// Key differences:
//
// 1. Initialization: Windows requires WSAStartup()
// 2. Cleanup: Windows requires WSACleanup()
// 3. Socket type: Windows uses SOCKET (unsigned), Unix uses int
// 4. Invalid socket: Windows uses INVALID_SOCKET, Unix uses -1
// 5. Error codes: Windows uses WSAGetLastError(), Unix uses errno
// 6. Error constants: WSAEWOULDBLOCK vs EWOULDBLOCK
// 7. Close function: Windows uses closesocket(), Unix uses close()
// 8. ioctl: Windows uses ioctlsocket(), Unix uses ioctl() or fcntl()
// 9. socklen_t: Windows uses int, POSIX uses socklen_t
// 10. poll: Windows has WSAPoll, Unix has poll (different bugs!)
\end{lstlisting}

\subsection{Socket Initialization}

\begin{lstlisting}
// sockets.h - Portable socket initialization
#ifndef SOCKETS_H
#define SOCKETS_H

#ifdef USE_WINSOCK
    // Windows networking
    #include <winsock2.h>
    #include <ws2tcpip.h>

    // Need to link: -lws2_32
    #pragma comment(lib, "ws2_32.lib")

    typedef SOCKET socket_t;
    #define INVALID_SOCKET_FD INVALID_SOCKET
    #define SOCKET_ERROR_CODE WSAGetLastError()

    // Error code compatibility
    #define EWOULDBLOCK_COMPAT WSAEWOULDBLOCK
    #define EINPROGRESS_COMPAT WSAEINPROGRESS
    #define ECONNREFUSED_COMPAT WSAECONNREFUSED
    #define ETIMEDOUT_COMPAT WSAETIMEDOUT
    #define EADDRINUSE_COMPAT WSAEADDRINUSE

#else
    // Unix/POSIX networking
    #include <sys/socket.h>
    #include <sys/types.h>
    #include <netinet/in.h>
    #include <netinet/tcp.h>
    #include <arpa/inet.h>
    #include <netdb.h>
    #include <unistd.h>
    #include <fcntl.h>
    #include <errno.h>

    typedef int socket_t;
    #define INVALID_SOCKET_FD (-1)
    #define SOCKET_ERROR (-1)
    #define SOCKET_ERROR_CODE errno

    // Error codes are directly from errno
    #define EWOULDBLOCK_COMPAT EWOULDBLOCK
    #define EINPROGRESS_COMPAT EINPROGRESS
    #define ECONNREFUSED_COMPAT ECONNREFUSED
    #define ETIMEDOUT_COMPAT ETIMEDOUT
    #define EADDRINUSE_COMPAT EADDRINUSE
#endif

// Initialize networking subsystem
static inline int net_init(void) {
#ifdef USE_WINSOCK
    WSADATA wsa_data;
    int result = WSAStartup(MAKEWORD(2, 2), &wsa_data);
    if (result != 0) {
        return -1;
    }

    // Verify Winsock 2.2 is available
    if (LOBYTE(wsa_data.wVersion) != 2 ||
        HIBYTE(wsa_data.wVersion) != 2) {
        WSACleanup();
        return -1;
    }
    return 0;
#else
    // Unix doesn't need initialization
    return 0;
#endif
}

// Cleanup networking subsystem
static inline void net_cleanup(void) {
#ifdef USE_WINSOCK
    WSACleanup();
#endif
}

// Close a socket
static inline int net_close(socket_t sock) {
#ifdef USE_WINSOCK
    return closesocket(sock);
#else
    return close(sock);
#endif
}

// Set socket to non-blocking mode
static inline int net_set_nonblocking(socket_t sock) {
#ifdef USE_WINSOCK
    u_long mode = 1;
    return ioctlsocket(sock, FIONBIO, &mode);
#else
    int flags = fcntl(sock, F_GETFL, 0);
    if (flags == -1) return -1;
    return fcntl(sock, F_SETFL, flags | O_NONBLOCK);
#endif
}

// Set socket to blocking mode
static inline int net_set_blocking(socket_t sock) {
#ifdef USE_WINSOCK
    u_long mode = 0;
    return ioctlsocket(sock, FIONBIO, &mode);
#else
    int flags = fcntl(sock, F_GETFL, 0);
    if (flags == -1) return -1;
    return fcntl(sock, F_SETFL, flags & ~O_NONBLOCK);
#endif
}

// Check if error is "would block"
static inline int net_would_block(void) {
    int err = SOCKET_ERROR_CODE;
#ifdef USE_WINSOCK
    return err == WSAEWOULDBLOCK || err == WSAEINPROGRESS;
#else
    return err == EWOULDBLOCK || err == EAGAIN || err == EINPROGRESS;
#endif
}

// Portable socket options
static inline int net_set_reuseaddr(socket_t sock, int enable) {
#ifdef USE_WINSOCK
    BOOL opt = enable ? TRUE : FALSE;
    return setsockopt(sock, SOL_SOCKET, SO_REUSEADDR,
                     (const char*)&opt, sizeof(opt));
#else
    int opt = enable ? 1 : 0;
    return setsockopt(sock, SOL_SOCKET, SO_REUSEADDR,
                     &opt, sizeof(opt));
#endif
}

static inline int net_set_nodelay(socket_t sock, int enable) {
#ifdef USE_WINSOCK
    BOOL opt = enable ? TRUE : FALSE;
    return setsockopt(sock, IPPROTO_TCP, TCP_NODELAY,
                     (const char*)&opt, sizeof(opt));
#else
    int opt = enable ? 1 : 0;
    return setsockopt(sock, IPPROTO_TCP, TCP_NODELAY,
                     &opt, sizeof(opt));
#endif
}

static inline int net_set_keepalive(socket_t sock, int enable) {
#ifdef USE_WINSOCK
    BOOL opt = enable ? TRUE : FALSE;
    return setsockopt(sock, SOL_SOCKET, SO_KEEPALIVE,
                     (const char*)&opt, sizeof(opt));
#else
    int opt = enable ? 1 : 0;
    return setsockopt(sock, SOL_SOCKET, SO_KEEPALIVE,
                     &opt, sizeof(opt));
#endif
}

// Get last socket error as string
static inline const char* net_strerror(int err) {
#ifdef USE_WINSOCK
    static char buf[256];
    FormatMessageA(FORMAT_MESSAGE_FROM_SYSTEM |
                  FORMAT_MESSAGE_IGNORE_INSERTS,
                  NULL, err, 0, buf, sizeof(buf), NULL);
    return buf;
#else
    return strerror(err);
#endif
}

#endif // SOCKETS_H
\end{lstlisting}

\subsection{Complete Socket Example}

\begin{lstlisting}
#include "sockets.h"
#include <stdio.h>
#include <string.h>

int main(void) {
    // Initialize network subsystem
    if (net_init() != 0) {
        fprintf(stderr, "Failed to initialize networking\n");
        return 1;
    }

    // Create socket (same on all platforms)
    socket_t sock = socket(AF_INET, SOCK_STREAM, IPPROTO_TCP);
    if (sock == INVALID_SOCKET_FD) {
        fprintf(stderr, "socket() failed: %s\n",
                net_strerror(SOCKET_ERROR_CODE));
        net_cleanup();
        return 1;
    }

    // Enable address reuse
    net_set_reuseaddr(sock, 1);

    // Bind to port 8080
    struct sockaddr_in addr;
    memset(&addr, 0, sizeof(addr));
    addr.sin_family = AF_INET;
    addr.sin_addr.s_addr = INADDR_ANY;
    addr.sin_port = htons(8080);

    if (bind(sock, (struct sockaddr*)&addr, sizeof(addr)) == SOCKET_ERROR) {
        fprintf(stderr, "bind() failed: %s\n",
                net_strerror(SOCKET_ERROR_CODE));
        net_close(sock);
        net_cleanup();
        return 1;
    }

    // Listen
    if (listen(sock, 5) == SOCKET_ERROR) {
        fprintf(stderr, "listen() failed: %s\n",
                net_strerror(SOCKET_ERROR_CODE));
        net_close(sock);
        net_cleanup();
        return 1;
    }

    printf("Listening on port 8080...\n");

    // Accept connection
    socket_t client = accept(sock, NULL, NULL);
    if (client == INVALID_SOCKET_FD) {
        fprintf(stderr, "accept() failed: %s\n",
                net_strerror(SOCKET_ERROR_CODE));
    } else {
        printf("Client connected!\n");
        const char* msg = "Hello from portable C!\n";
        send(client, msg, (int)strlen(msg), 0);
        net_close(client);
    }

    // Cleanup
    net_close(sock);
    net_cleanup();
    return 0;
}
\end{lstlisting}

\begin{warningbox}
\textbf{MinGW Gotcha:} MinGW provides some POSIX functions like \texttt{read()} and \texttt{write()}, but they DON'T work on sockets! You must use \texttt{send()} and \texttt{recv()}. Cygwin doesn't have this problem.
\end{warningbox}

\section{Console and Terminal Handling}

Unix terminals and Windows consoles have completely different APIs for:
\begin{itemize}
    \item \textbf{Color output}: ANSI escape codes vs Windows Console API
    \item \textbf{Raw mode}: \texttt{termios} vs \texttt{SetConsoleMode()}
    \item \textbf{Size detection}: \texttt{ioctl()} vs \texttt{GetConsoleScreenBufferInfo()}
    \item \textbf{Unicode}: UTF-8 vs UTF-16 (again!)
\end{itemize}

Let's make it portable.

\subsection{Terminal Colors and Formatting}

\begin{lstlisting}
// console.h - Portable console output with colors
#ifndef CONSOLE_H
#define CONSOLE_H

#include <stdio.h>

#ifdef PLATFORM_WINDOWS
    #include <windows.h>
    #include <io.h>
    #define isatty _isatty
    #define fileno _fileno
#else
    #include <unistd.h>
#endif

// ANSI color codes
#define ANSI_RESET   "\033[0m"
#define ANSI_BOLD    "\033[1m"
#define ANSI_RED     "\033[31m"
#define ANSI_GREEN   "\033[32m"
#define ANSI_YELLOW  "\033[33m"
#define ANSI_BLUE    "\033[34m"
#define ANSI_MAGENTA "\033[35m"
#define ANSI_CYAN    "\033[36m"
#define ANSI_WHITE   "\033[37m"

static int console_color_enabled = -1;

// Initialize console (enable colors on Windows 10+)
static inline void console_init(void) {
#ifdef PLATFORM_WINDOWS
    // Windows 10 supports ANSI escape codes
    HANDLE hOut = GetStdHandle(STD_OUTPUT_HANDLE);
    if (hOut != INVALID_HANDLE_VALUE) {
        DWORD mode = 0;
        if (GetConsoleMode(hOut, &mode)) {
            mode |= ENABLE_VIRTUAL_TERMINAL_PROCESSING;
            SetConsoleMode(hOut, mode);
        }
    }
#endif

    // Check if stdout is a terminal
    console_color_enabled = isatty(fileno(stdout));
}

// Print with color (only if terminal supports it)
static inline void console_print(const char* color, const char* text) {
    if (console_color_enabled == -1) console_init();

    if (console_color_enabled) {
        printf("%s%s%s", color, text, ANSI_RESET);
    } else {
        printf("%s", text);
    }
}

// Convenience functions
#define print_error(msg)   console_print(ANSI_RED, msg)
#define print_success(msg) console_print(ANSI_GREEN, msg)
#define print_warning(msg) console_print(ANSI_YELLOW, msg)
#define print_info(msg)    console_print(ANSI_CYAN, msg)

// Clear screen
static inline void console_clear(void) {
#ifdef PLATFORM_WINDOWS
    system("cls");
#else
    system("clear");
    // Or: printf("\033[2J\033[H");
#endif
}

// Get terminal size
static inline int console_get_size(int* width, int* height) {
#ifdef PLATFORM_WINDOWS
    CONSOLE_SCREEN_BUFFER_INFO csbi;
    if (GetConsoleScreenBufferInfo(GetStdHandle(STD_OUTPUT_HANDLE), &csbi)) {
        *width = csbi.srWindow.Right - csbi.srWindow.Left + 1;
        *height = csbi.srWindow.Bottom - csbi.srWindow.Top + 1;
        return 0;
    }
    return -1;
#else
    #include <sys/ioctl.h>
    struct winsize w;
    if (ioctl(STDOUT_FILENO, TIOCGWINSZ, &w) == 0) {
        *width = w.ws_col;
        *height = w.ws_row;
        return 0;
    }
    return -1;
#endif
}

#endif // CONSOLE_H
\end{lstlisting}

\subsection{Raw Terminal Mode (No Echo, No Buffering)}

\begin{lstlisting}
// terminal.h - Raw terminal input
#ifndef TERMINAL_H
#define TERMINAL_H

#ifdef PLATFORM_WINDOWS
    #include <windows.h>
    #include <conio.h>

    static DWORD original_mode = 0;

    static inline int terminal_raw_mode_enable(void) {
        HANDLE hIn = GetStdHandle(STD_INPUT_HANDLE);
        if (hIn == INVALID_HANDLE_VALUE) return -1;

        if (!GetConsoleMode(hIn, &original_mode)) return -1;

        DWORD mode = original_mode;
        mode &= ~(ENABLE_ECHO_INPUT | ENABLE_LINE_INPUT);
        mode |= ENABLE_PROCESSED_INPUT;

        if (!SetConsoleMode(hIn, mode)) return -1;
        return 0;
    }

    static inline int terminal_raw_mode_disable(void) {
        HANDLE hIn = GetStdHandle(STD_INPUT_HANDLE);
        if (hIn == INVALID_HANDLE_VALUE) return -1;
        return SetConsoleMode(hIn, original_mode) ? 0 : -1;
    }

    static inline int terminal_getchar_nonblock(void) {
        if (_kbhit()) {
            return _getch();
        }
        return -1;
    }

#else
    #include <termios.h>
    #include <unistd.h>
    #include <fcntl.h>

    static struct termios original_termios;
    static int termios_saved = 0;

    static inline int terminal_raw_mode_enable(void) {
        if (tcgetattr(STDIN_FILENO, &original_termios) == -1) {
            return -1;
        }
        termios_saved = 1;

        struct termios raw = original_termios;
        raw.c_lflag &= ~(ECHO | ICANON | IEXTEN | ISIG);
        raw.c_iflag &= ~(BRKINT | ICRNL | INPCK | ISTRIP | IXON);
        raw.c_cflag |= CS8;
        raw.c_oflag &= ~(OPOST);
        raw.c_cc[VMIN] = 0;
        raw.c_cc[VTIME] = 1;

        if (tcsetattr(STDIN_FILENO, TCSAFLUSH, &raw) == -1) {
            return -1;
        }
        return 0;
    }

    static inline int terminal_raw_mode_disable(void) {
        if (!termios_saved) return 0;
        if (tcsetattr(STDIN_FILENO, TCSAFLUSH, &original_termios) == -1) {
            return -1;
        }
        return 0;
    }

    static inline int terminal_getchar_nonblock(void) {
        int flags = fcntl(STDIN_FILENO, F_GETFL, 0);
        fcntl(STDIN_FILENO, F_SETFL, flags | O_NONBLOCK);

        char c;
        ssize_t n = read(STDIN_FILENO, &c, 1);

        fcntl(STDIN_FILENO, F_SETFL, flags);

        if (n == 1) return (unsigned char)c;
        return -1;
    }
#endif

// Get a single character with timeout
static inline int terminal_getch_timeout(int timeout_ms) {
#ifdef PLATFORM_WINDOWS
    DWORD start = GetTickCount();
    while (GetTickCount() - start < (DWORD)timeout_ms) {
        if (_kbhit()) return _getch();
        Sleep(10);
    }
    return -1;
#else
    fd_set readfds;
    struct timeval tv;

    FD_ZERO(&readfds);
    FD_SET(STDIN_FILENO, &readfds);

    tv.tv_sec = timeout_ms / 1000;
    tv.tv_usec = (timeout_ms % 1000) * 1000;

    if (select(STDIN_FILENO + 1, &readfds, NULL, NULL, &tv) > 0) {
        char c;
        if (read(STDIN_FILENO, &c, 1) == 1) {
            return (unsigned char)c;
        }
    }
    return -1;
#endif
}

#endif // TERMINAL_H
\end{lstlisting}

\section{Character Encoding: UTF-8 vs UTF-16}

\subsection{The Windows Unicode Problem}

This is one of the most frustrating platform differences. Here's the situation:

\begin{itemize}
    \item \textbf{Unix/Linux/macOS}: Use UTF-8 everywhere. Strings are \texttt{char*}. Everything is simple.

    \item \textbf{Windows}: The "ANSI" API (\texttt{CreateFileA}, \texttt{GetFileAttributesA}) uses the system codepage (often Windows-1252), which can't handle international characters reliably.

    \item \textbf{Windows Unicode API}: Uses UTF-16 with \texttt{wchar\_t*} (\texttt{CreateFileW}, \texttt{GetFileAttributesW}). This is the only reliable way to handle Unicode on Windows.
\end{itemize}

The problem: Your portable code should use UTF-8 internally, but Windows APIs need UTF-16. Solution: Convert at the boundary.

\begin{lstlisting}
// Windows internally uses UTF-16 (wchar_t)
// Unix uses UTF-8 (char)
// This affects EVERY string API

#ifdef PLATFORM_WINDOWS
    // Windows: wmain for Unicode
    #include <windows.h>
    #include <wchar.h>

    // Convert UTF-8 to UTF-16 (for Windows APIs)
    wchar_t* utf8_to_utf16(const char* utf8) {
        if (!utf8) return NULL;

        int len = MultiByteToWideChar(CP_UTF8, 0, utf8, -1, NULL, 0);
        if (len <= 0) return NULL;

        wchar_t* utf16 = malloc(len * sizeof(wchar_t));
        if (!utf16) return NULL;

        MultiByteToWideChar(CP_UTF8, 0, utf8, -1, utf16, len);
        return utf16;
    }

    // Convert UTF-16 to UTF-8
    char* utf16_to_utf8(const wchar_t* utf16) {
        if (!utf16) return NULL;

        int len = WideCharToMultiByte(CP_UTF8, 0, utf16, -1,
                                     NULL, 0, NULL, NULL);
        if (len <= 0) return NULL;

        char* utf8 = malloc(len);
        if (!utf8) return NULL;

        WideCharToMultiByte(CP_UTF8, 0, utf16, -1,
                           utf8, len, NULL, NULL);
        return utf8;
    }

    // Use wmain and convert arguments to UTF-8
    int wmain(int argc, wchar_t* argv[]) {
        // Convert all arguments to UTF-8
        char** argv_utf8 = malloc(argc * sizeof(char*));
        for (int i = 0; i < argc; i++) {
            argv_utf8[i] = utf16_to_utf8(argv[i]);
        }

        // Call your real main
        extern int utf8_main(int argc, char** argv);
        int ret = utf8_main(argc, argv_utf8);

        // Cleanup
        for (int i = 0; i < argc; i++) {
            free(argv_utf8[i]);
        }
        free(argv_utf8);
        return ret;
    }

    // Your actual main function
    int utf8_main(int argc, char** argv) {
        // All strings are UTF-8 now!
        printf("UTF-8 argument: %s\n", argv[0]);
        return 0;
    }

#else
    // Unix: already UTF-8
    int main(int argc, char** argv) {
        printf("UTF-8 argument: %s\n", argv[0]);
        return 0;
    }
#endif
\end{lstlisting}

\begin{tipbox}
\textbf{Pro Tip:} Always use UTF-8 internally in your program. Convert to UTF-16 only when calling Windows APIs. This makes your code portable and avoids wchar\_t madness.
\end{tipbox}

\section{File System Differences}

File systems vary dramatically across platforms:

\begin{itemize}
    \item \textbf{Path separators}: Windows uses backslash (\texttt{\textbackslash}), Unix uses forward slash (\texttt{/})
    \item \textbf{Case sensitivity}: Unix file systems are case-sensitive (\texttt{file.txt} $\neq$ \texttt{File.txt}). Windows usually isn't.
    \item \textbf{Path length limits}: Windows has a notorious 260-character limit (MAX\_PATH). Unix typically allows 4096.
    \item \textbf{Reserved names}: Windows forbids \texttt{CON}, \texttt{PRN}, \texttt{AUX}, \texttt{NUL}, etc. as filenames.
    \item \textbf{Absolute paths}: Windows uses drive letters (\texttt{C:\textbackslash}). Unix uses root (\texttt{/}).
    \item \textbf{Permissions}: Unix has \texttt{chmod}/\texttt{stat}. Windows has ACLs (Access Control Lists).
\end{itemize}

\subsection{Path Handling}

Let's create a portable path manipulation library:

\begin{lstlisting}
// path.h - Portable path manipulation
#ifndef PATH_H
#define PATH_H

#include <string.h>
#include <stdlib.h>

#ifdef PLATFORM_WINDOWS
    #define PATH_SEPARATOR '\\'
    #define PATH_SEPARATOR_STR "\\"
    #define PATH_LIST_SEPARATOR ';'
    #define IS_PATH_SEPARATOR(c) ((c) == '\\' || (c) == '/')
    #define MAX_PATH_LEN 260  // Windows limitation
#else
    #define PATH_SEPARATOR '/'
    #define PATH_SEPARATOR_STR "/"
    #define PATH_LIST_SEPARATOR ':'
    #define IS_PATH_SEPARATOR(c) ((c) == '/')
    #define MAX_PATH_LEN 4096  // PATH_MAX on most Unix
#endif

// Normalize path separators
static inline void path_normalize(char* path) {
    if (!path) return;

    for (char* p = path; *p; p++) {
        if (IS_PATH_SEPARATOR(*p)) {
            *p = PATH_SEPARATOR;
        }
    }

#ifdef PLATFORM_WINDOWS
    // Windows paths are case-insensitive
    // Optional: convert to lowercase
#endif
}

// Join two paths
static inline char* path_join(const char* dir, const char* file) {
    if (!dir || !file) return NULL;

    size_t dir_len = strlen(dir);
    size_t file_len = strlen(file);

    // Check if dir already ends with separator
    int need_sep = (dir_len > 0 && !IS_PATH_SEPARATOR(dir[dir_len - 1]));

    size_t total_len = dir_len + file_len + (need_sep ? 1 : 0) + 1;
    char* result = malloc(total_len);
    if (!result) return NULL;

    strcpy(result, dir);
    if (need_sep) {
        result[dir_len] = PATH_SEPARATOR;
        result[dir_len + 1] = '\0';
    }
    strcat(result, file);

    path_normalize(result);
    return result;
}

// Get filename from path
static inline const char* path_filename(const char* path) {
    if (!path) return NULL;

    const char* last_sep = strrchr(path, PATH_SEPARATOR);
#ifdef PLATFORM_WINDOWS
    // Windows accepts both / and \
    const char* last_fwd = strrchr(path, '/');
    if (last_fwd && (!last_sep || last_fwd > last_sep)) {
        last_sep = last_fwd;
    }
#endif

    return last_sep ? last_sep + 1 : path;
}

// Get directory from path (modifies path!)
static inline char* path_dirname(char* path) {
    if (!path || !*path) return ".";

    char* last_sep = strrchr(path, PATH_SEPARATOR);
#ifdef PLATFORM_WINDOWS
    char* last_fwd = strrchr(path, '/');
    if (last_fwd && (!last_sep || last_fwd > last_sep)) {
        last_sep = last_fwd;
    }
#endif

    if (!last_sep) return ".";

    *last_sep = '\0';
    return path;
}

// Check if path is absolute
static inline int path_is_absolute(const char* path) {
    if (!path || !*path) return 0;

#ifdef PLATFORM_WINDOWS
    // C:\ or \\server\share
    if (path[1] == ':' && IS_PATH_SEPARATOR(path[2])) return 1;
    if (IS_PATH_SEPARATOR(path[0]) && IS_PATH_SEPARATOR(path[1])) return 1;
    return 0;
#else
    return path[0] == '/';
#endif
}

// Get file extension
static inline const char* path_extension(const char* path) {
    const char* filename = path_filename(path);
    const char* dot = strrchr(filename, '.');
    return dot ? dot + 1 : "";
}

// Check if path exists
static inline int path_exists(const char* path) {
#ifdef PLATFORM_WINDOWS
    DWORD attr = GetFileAttributesA(path);
    return attr != INVALID_FILE_ATTRIBUTES;
#else
    return access(path, F_OK) == 0;
#endif
}

// Check if path is a directory
static inline int path_is_directory(const char* path) {
#ifdef PLATFORM_WINDOWS
    DWORD attr = GetFileAttributesA(path);
    return (attr != INVALID_FILE_ATTRIBUTES) &&
           (attr & FILE_ATTRIBUTE_DIRECTORY);
#else
    struct stat st;
    return (stat(path, &st) == 0) && S_ISDIR(st.st_mode);
#endif
}

#endif // PATH_H
\end{lstlisting}

\section{Process Management}

Process creation is fundamentally different across platforms:

\begin{itemize}
    \item \textbf{Unix}: Uses \texttt{fork()} to clone the current process, then \texttt{exec()} to replace it with a new program. Simple and elegant.

    \item \textbf{Windows}: No \texttt{fork()}! Must use \texttt{CreateProcess()} which creates a new process directly. Completely different model.
\end{itemize}

Why no \texttt{fork()} on Windows? Because Windows doesn't have copy-on-write process memory like Unix. Cloning a process would require copying all memory, which is prohibitively expensive.

\subsection{Process Creation (No fork on Windows!)}

\begin{lstlisting}
// process.h - Portable process creation
#ifndef PROCESS_H
#define PROCESS_H

#ifdef PLATFORM_WINDOWS
    #include <windows.h>
    #include <process.h>

    typedef HANDLE process_t;
    #define INVALID_PROCESS NULL

    // Execute command and wait
    static inline int process_execute(const char* cmd) {
        return system(cmd);
    }

    // Spawn process (like fork + exec on Unix)
    static inline process_t process_spawn(const char* path,
                                         char* const argv[]) {
        // Build command line
        char cmdline[8192] = {0};
        int pos = 0;

        for (int i = 0; argv[i]; i++) {
            if (i > 0) cmdline[pos++] = ' ';

            // Quote arguments with spaces
            int needs_quote = strchr(argv[i], ' ') != NULL;
            if (needs_quote) cmdline[pos++] = '"';

            strcpy(cmdline + pos, argv[i]);
            pos += strlen(argv[i]);

            if (needs_quote) cmdline[pos++] = '"';
        }

        STARTUPINFOA si = {0};
        PROCESS_INFORMATION pi = {0};
        si.cb = sizeof(si);

        if (!CreateProcessA(path, cmdline, NULL, NULL, FALSE,
                           0, NULL, NULL, &si, &pi)) {
            return INVALID_PROCESS;
        }

        CloseHandle(pi.hThread);
        return pi.hProcess;
    }

    // Wait for process to finish
    static inline int process_wait(process_t proc) {
        if (proc == INVALID_PROCESS) return -1;

        WaitForSingleObject(proc, INFINITE);

        DWORD exitcode;
        GetExitCodeProcess(proc, &exitcode);
        CloseHandle(proc);

        return (int)exitcode;
    }

#else
    #include <sys/types.h>
    #include <sys/wait.h>
    #include <unistd.h>

    typedef pid_t process_t;
    #define INVALID_PROCESS (-1)

    // Execute command and wait
    static inline int process_execute(const char* cmd) {
        return system(cmd);
    }

    // Spawn process using fork + exec
    static inline process_t process_spawn(const char* path,
                                         char* const argv[]) {
        pid_t pid = fork();

        if (pid == -1) {
            return INVALID_PROCESS;
        }

        if (pid == 0) {
            // Child process
            execv(path, argv);
            // If execv returns, it failed
            _exit(127);
        }

        // Parent process
        return pid;
    }

    // Wait for process to finish
    static inline int process_wait(process_t proc) {
        if (proc == INVALID_PROCESS) return -1;

        int status;
        if (waitpid(proc, &status, 0) == -1) {
            return -1;
        }

        if (WIFEXITED(status)) {
            return WEXITSTATUS(status);
        }

        return -1;
    }
#endif

// Get current process ID
static inline int process_getpid(void) {
#ifdef PLATFORM_WINDOWS
    return (int)GetCurrentProcessId();
#else
    return (int)getpid();
#endif
}

// Kill a process
static inline int process_kill(process_t proc) {
#ifdef PLATFORM_WINDOWS
    return TerminateProcess(proc, 1) ? 0 : -1;
#else
    return kill(proc, SIGKILL);
#endif
}

#endif // PROCESS_H
\end{lstlisting}

\begin{warningbox}
\textbf{fork() doesn't exist on Windows!} You must use CreateProcess or \_spawn functions. This is one of the biggest portability challenges---Unix code using fork() needs complete rewriting for Windows.
\end{warningbox}

\section{Line Endings: CRLF vs LF}

This seems trivial but causes real bugs:

\begin{itemize}
    \item \textbf{Unix/Linux/macOS}: Use LF (\texttt{\textbackslash n}) for line endings
    \item \textbf{Windows}: Use CRLF (\texttt{\textbackslash r\textbackslash n}) for line endings
    \item \textbf{Old Mac}: Used CR (\texttt{\textbackslash r}), but not since OS X
\end{itemize}

\textbf{Why it matters}: When you open a file in text mode on Windows, the C runtime automatically converts \texttt{\textbackslash n} to \texttt{\textbackslash r\textbackslash n} on write and vice versa on read. This is GREAT for text files but DISASTROUS for binary data.

The fix: Always use binary mode (\texttt{"rb"}, \texttt{"wb"}) for precise control.

\begin{lstlisting}
// Line ending differences cause SO many bugs

// Windows text files: \r\n (CRLF, 0x0D 0x0A)
// Unix text files: \n (LF, 0x0A)
// Old Mac: \r (CR, 0x0D)

// When reading files:
#ifdef PLATFORM_WINDOWS
    // Windows fopen in text mode converts \r\n to \n automatically
    FILE* f = fopen("file.txt", "r");  // Text mode

    // Binary mode preserves \r\n
    FILE* f = fopen("file.txt", "rb");  // Binary mode
#else
    // Unix: no conversion happens, \n is just \n
    FILE* f = fopen("file.txt", "r");
#endif

// Portable approach: always use binary mode
FILE* f = fopen("file.txt", "rb");

// Then normalize line endings manually if needed
void normalize_line_endings(char* text) {
    char* read = text;
    char* write = text;

    while (*read) {
        if (*read == '\r' && *(read + 1) == '\n') {
            // CRLF -> LF
            *write++ = '\n';
            read += 2;
        } else if (*read == '\r') {
            // CR -> LF
            *write++ = '\n';
            read++;
        } else {
            *write++ = *read++;
        }
    }
    *write = '\0';
}

// When writing: be explicit
#ifdef PLATFORM_WINDOWS
    fprintf(f, "Line 1\r\n");  // Native Windows format
#else
    fprintf(f, "Line 1\n");    // Unix format
#endif

// Or use a macro
#ifdef PLATFORM_WINDOWS
    #define EOL "\r\n"
#else
    #define EOL "\n"
#endif

fprintf(f, "Line 1" EOL);
\end{lstlisting}

\begin{tipbox}
\textbf{Pro Tip:} Git handles this with \texttt{core.autocrlf}. Your editor might too. But in C code dealing with binary protocols or exact file formats, you need to handle it yourself!
\end{tipbox}

\section{Dynamic Libraries: .dll vs .so vs .dylib}

Dynamic libraries (shared libraries) allow code to be loaded at runtime, enabling plugins and reducing memory usage. But the implementation varies wildly:

\begin{itemize}
    \item \textbf{Windows}: Uses \texttt{.dll} files with \texttt{LoadLibrary()}/\texttt{GetProcAddress()}
    \item \textbf{Linux/BSD}: Uses \texttt{.so} files with \texttt{dlopen()}/\texttt{dlsym()}
    \item \textbf{macOS}: Uses \texttt{.dylib} files (also supports \texttt{dlopen()}/\texttt{dlsym()})
\end{itemize}

\subsection{Naming and Extensions}

\begin{lstlisting}
// Library naming conventions differ:
//
// Windows (MSVC/MinGW):
//   mylib.dll           (dynamic library)
//   mylib.lib           (import library for DLL)
//   mylib.a             (static library, MinGW)
//
// Linux:
//   libmylib.so         (shared object)
//   libmylib.so.1       (with version)
//   libmylib.so.1.2.3   (full version)
//   libmylib.a          (static library)
//
// macOS:
//   libmylib.dylib      (dynamic library)
//   libmylib.1.dylib    (with version)
//   libmylib.a          (static library)

// When loading dynamically:
#ifdef PLATFORM_WINDOWS
    #define LIB_PREFIX ""
    #define LIB_SUFFIX ".dll"
#elif defined(PLATFORM_MACOS)
    #define LIB_PREFIX "lib"
    #define LIB_SUFFIX ".dylib"
#else
    #define LIB_PREFIX "lib"
    #define LIB_SUFFIX ".so"
#endif

// Build library name
char libname[256];
snprintf(libname, sizeof(libname), "%s%s%s",
         LIB_PREFIX, "mylib", LIB_SUFFIX);
// Result: "mylib.dll" on Windows, "libmylib.so" on Linux
\end{lstlisting}

\subsection{Symbol Export/Import}

\begin{lstlisting}
// mylib.h - Exporting symbols from DLL/shared library
#ifndef MYLIB_H
#define MYLIB_H

// Windows requires explicit DLL export/import
#ifdef PLATFORM_WINDOWS
    #ifdef MYLIB_BUILDING
        // Building the DLL
        #define MYLIB_API __declspec(dllexport)
    #else
        // Using the DLL
        #define MYLIB_API __declspec(dllimport)
    #endif
#else
    // Unix: symbols are exported by default
    // But you can control visibility
    #if defined(__GNUC__) && __GNUC__ >= 4
        #define MYLIB_API __attribute__((visibility("default")))
    #else
        #define MYLIB_API
    #endif
#endif

// Now mark your API functions
MYLIB_API int mylib_init(void);
MYLIB_API void mylib_cleanup(void);
MYLIB_API int mylib_do_something(int x);

#endif // MYLIB_H

// When compiling the library:
// gcc -DMYLIB_BUILDING -shared -o libmylib.so mylib.c
// cl /DMYLIB_BUILDING /LD mylib.c  (creates mylib.dll and mylib.lib)
\end{lstlisting}

\subsection{Dynamic Loading}

\begin{lstlisting}
// dynload.h - Portable dynamic library loading
#ifndef DYNLOAD_H
#define DYNLOAD_H

#ifdef PLATFORM_WINDOWS
    #include <windows.h>

    typedef HMODULE dynlib_handle_t;

    static inline dynlib_handle_t dynlib_open(const char* path) {
        return LoadLibraryA(path);
    }

    static inline void* dynlib_symbol(dynlib_handle_t lib, const char* name) {
        return (void*)GetProcAddress(lib, name);
    }

    static inline void dynlib_close(dynlib_handle_t lib) {
        FreeLibrary(lib);
    }

    static inline const char* dynlib_error(void) {
        static char buf[512];
        DWORD err = GetLastError();
        FormatMessageA(FORMAT_MESSAGE_FROM_SYSTEM, NULL, err,
                      0, buf, sizeof(buf), NULL);
        return buf;
    }

#else
    #include <dlfcn.h>

    typedef void* dynlib_handle_t;

    static inline dynlib_handle_t dynlib_open(const char* path) {
        return dlopen(path, RTLD_NOW | RTLD_LOCAL);
    }

    static inline void* dynlib_symbol(dynlib_handle_t lib, const char* name) {
        return dlsym(lib, name);
    }

    static inline void dynlib_close(dynlib_handle_t lib) {
        dlclose(lib);
    }

    static inline const char* dynlib_error(void) {
        return dlerror();
    }
#endif

// Example: Load plugin
typedef int (*plugin_init_func)(void);

int load_plugin(const char* name) {
    char path[512];
    snprintf(path, sizeof(path), "%s%s%s",
             LIB_PREFIX, name, LIB_SUFFIX);

    dynlib_handle_t lib = dynlib_open(path);
    if (!lib) {
        fprintf(stderr, "Failed to load %s: %s\n",
                path, dynlib_error());
        return -1;
    }

    plugin_init_func init = (plugin_init_func)
        dynlib_symbol(lib, "plugin_init");

    if (!init) {
        fprintf(stderr, "Plugin missing init function\n");
        dynlib_close(lib);
        return -1;
    }

    return init();
}

#endif // DYNLOAD_H
\end{lstlisting}

\section{Signal Handling vs Windows Events}

Handling Ctrl+C and other interrupts requires platform-specific code:

\begin{itemize}
    \item \textbf{Unix}: Uses signals (\texttt{SIGINT}, \texttt{SIGTERM}, etc.) with \texttt{signal()} or \texttt{sigaction()}
    \item \textbf{Windows}: Uses console control handlers with \texttt{SetConsoleCtrlHandler()}
\end{itemize}

The concepts are similar but the APIs are completely different.

\subsection{Portable Signal/Interrupt Handling}

\begin{lstlisting}
// signals.h - Portable signal handling
#ifndef SIGNALS_H
#define SIGNALS_H

#include <signal.h>

#ifdef PLATFORM_WINDOWS
    #include <windows.h>

    static volatile int signal_received = 0;

    // Windows console control handler
    static BOOL WINAPI console_ctrl_handler(DWORD signal) {
        switch (signal) {
            case CTRL_C_EVENT:
            case CTRL_BREAK_EVENT:
            case CTRL_CLOSE_EVENT:
                signal_received = 1;
                return TRUE;
            default:
                return FALSE;
        }
    }

    static inline void signal_setup(void) {
        SetConsoleCtrlHandler(console_ctrl_handler, TRUE);
    }

    static inline int signal_check(void) {
        return signal_received;
    }

#else
    // Unix signal handling
    static volatile sig_atomic_t signal_received = 0;

    static void signal_handler(int signum) {
        (void)signum;
        signal_received = 1;
    }

    static inline void signal_setup(void) {
        struct sigaction sa;
        sa.sa_handler = signal_handler;
        sigemptyset(&sa.sa_mask);
        sa.sa_flags = 0;

        sigaction(SIGINT, &sa, NULL);   // Ctrl+C
        sigaction(SIGTERM, &sa, NULL);  // Termination request

        #ifndef PLATFORM_MACOS
        // Ignore SIGPIPE (broken pipe)
        signal(SIGPIPE, SIG_IGN);
        #endif
    }

    static inline int signal_check(void) {
        return signal_received;
    }
#endif

#endif // SIGNALS_H
\end{lstlisting}

\section{Time and Sleep Functions}

Even basic timing functions differ:

\begin{itemize}
    \item \textbf{Sleep duration}: Unix uses \texttt{sleep()} (seconds) or \texttt{usleep()} (microseconds). Windows uses \texttt{Sleep()} (milliseconds).
    \item \textbf{High-resolution time}: Unix has \texttt{clock\_gettime()}. Windows has \texttt{QueryPerformanceCounter()}.
    \item \textbf{Function names}: Note the capital 'S' in Windows \texttt{Sleep()} vs lowercase in Unix \texttt{sleep()}.
\end{itemize}

\subsection{Portable Timing}

\begin{lstlisting}
// timing.h - Portable high-resolution timing
#ifndef TIMING_H
#define TIMING_H

#include <stdint.h>

#ifdef PLATFORM_WINDOWS
    #include <windows.h>

    typedef struct {
        LARGE_INTEGER freq;
        LARGE_INTEGER start;
    } timer_t;

    static inline void timer_init(timer_t* t) {
        QueryPerformanceFrequency(&t->freq);
    }

    static inline void timer_start(timer_t* t) {
        QueryPerformanceCounter(&t->start);
    }

    static inline double timer_elapsed_seconds(timer_t* t) {
        LARGE_INTEGER end;
        QueryPerformanceCounter(&end);
        return (double)(end.QuadPart - t->start.QuadPart) /
               t->freq.QuadPart;
    }

    static inline uint64_t timer_elapsed_ms(timer_t* t) {
        LARGE_INTEGER end;
        QueryPerformanceCounter(&end);
        return (uint64_t)(end.QuadPart - t->start.QuadPart) * 1000 /
               t->freq.QuadPart;
    }

    // Sleep functions
    static inline void sleep_ms(unsigned int ms) {
        Sleep(ms);
    }

    static inline void sleep_seconds(unsigned int seconds) {
        Sleep(seconds * 1000);
    }

#else
    #include <time.h>
    #include <unistd.h>

    typedef struct {
        struct timespec start;
    } timer_t;

    static inline void timer_init(timer_t* t) {
        (void)t;
    }

    static inline void timer_start(timer_t* t) {
        clock_gettime(CLOCK_MONOTONIC, &t->start);
    }

    static inline double timer_elapsed_seconds(timer_t* t) {
        struct timespec end;
        clock_gettime(CLOCK_MONOTONIC, &end);
        return (end.tv_sec - t->start.tv_sec) +
               (end.tv_nsec - t->start.tv_nsec) / 1e9;
    }

    static inline uint64_t timer_elapsed_ms(timer_t* t) {
        struct timespec end;
        clock_gettime(CLOCK_MONOTONIC, &end);
        return (uint64_t)(end.tv_sec - t->start.tv_sec) * 1000 +
               (end.tv_nsec - t->start.tv_nsec) / 1000000;
    }

    // Sleep functions
    static inline void sleep_ms(unsigned int ms) {
        usleep(ms * 1000);
    }

    static inline void sleep_seconds(unsigned int seconds) {
        sleep(seconds);
    }
#endif

#endif // TIMING_H
\end{lstlisting}

\section{Environment Variables}

Even environment variables have platform quirks:

\begin{itemize}
    \item \textbf{Unix}: Case-sensitive (\texttt{PATH} $\neq$ \texttt{Path})
    \item \textbf{Windows}: Case-insensitive (\texttt{PATH} == \texttt{Path})
    \item \textbf{Thread safety}: \texttt{getenv()} is not thread-safe on any platform
\end{itemize}

\subsection{Safe Environment Access}

\begin{lstlisting}
// env.h - Portable environment variable access
#ifndef ENV_H
#define ENV_H

#include <stdlib.h>
#include <string.h>

#ifdef PLATFORM_WINDOWS
    #include <windows.h>

    // Get environment variable (returns newly allocated string)
    static inline char* env_get(const char* name) {
        DWORD size = GetEnvironmentVariableA(name, NULL, 0);
        if (size == 0) return NULL;

        char* value = malloc(size);
        if (!value) return NULL;

        GetEnvironmentVariableA(name, value, size);
        return value;
    }

    // Set environment variable
    static inline int env_set(const char* name, const char* value) {
        return SetEnvironmentVariableA(name, value) ? 0 : -1;
    }

    // Unset environment variable
    static inline int env_unset(const char* name) {
        return SetEnvironmentVariableA(name, NULL) ? 0 : -1;
    }

#else
    // Unix uses standard getenv/setenv/unsetenv

    static inline char* env_get(const char* name) {
        const char* value = getenv(name);
        return value ? strdup(value) : NULL;
    }

    static inline int env_set(const char* name, const char* value) {
        return setenv(name, value, 1);
    }

    static inline int env_unset(const char* name) {
        return unsetenv(name);
    }
#endif

#endif // ENV_H
\end{lstlisting}

\section{Complete Practical Example: A Cross-Platform HTTP Server}

Let's put everything together. Here's a minimal HTTP server that works on Windows, Linux, and macOS, demonstrating all the concepts from this chapter:

\begin{lstlisting}
// server.c - Cross-platform HTTP server
#include "platform.h"  // From earlier in chapter
#include "sockets.h"   // From networking section
#include "signals.h"   // From signals section

#include <stdio.h>
#include <string.h>
#include <time.h>

#define PORT 8080
#define BUFFER_SIZE 4096

// Platform-specific sleep
void portable_sleep_ms(int milliseconds) {
#ifdef PLATFORM_WINDOWS
    Sleep(milliseconds);
#else
    usleep(milliseconds * 1000);
#endif
}

// Get current timestamp
void get_timestamp(char* buffer, size_t size) {
    time_t now = time(NULL);
    struct tm* tm_info = localtime(&now);
    strftime(buffer, size, "%Y-%m-%d %H:%M:%S", tm_info);
}

// Send HTTP response
void send_response(socket_t client, const char* status,
                   const char* content_type, const char* body) {
    char response[BUFFER_SIZE];
    int len = snprintf(response, sizeof(response),
        "HTTP/1.1 %s\r\n"
        "Content-Type: %s\r\n"
        "Content-Length: %zu\r\n"
        "Connection: close\r\n"
        "\r\n"
        "%s",
        status, content_type, strlen(body), body);

    send(client, response, len, 0);
}

// Handle HTTP request
void handle_request(socket_t client) {
    char buffer[BUFFER_SIZE];
    int bytes = recv(client, buffer, sizeof(buffer) - 1, 0);

    if (bytes <= 0) {
        return;
    }

    buffer[bytes] = '\0';

    // Parse request line
    char method[16], path[256];
    sscanf(buffer, "%15s %255s", method, path);

    // Generate response
    char timestamp[64];
    get_timestamp(timestamp, sizeof(timestamp));

    char body[1024];
    snprintf(body, sizeof(body),
        "<html>\n"
        "<head><title>Cross-Platform Server</title></head>\n"
        "<body>\n"
        "<h1>Hello from portable C!</h1>\n"
        "<p>Method: %s</p>\n"
        "<p>Path: %s</p>\n"
        "<p>Time: %s</p>\n"
        "<p>Platform: %s</p>\n"
        "<p>Architecture: %s</p>\n"
        "</body>\n"
        "</html>\n",
        method, path, timestamp,
#ifdef PLATFORM_WINDOWS
        "Windows",
#elif defined(PLATFORM_LINUX)
        "Linux",
#elif defined(PLATFORM_MACOS)
        "macOS",
#else
        "Unknown",
#endif
        ARCH_NAME);

    send_response(client, "200 OK", "text/html", body);
}

int main(void) {
    printf("Cross-Platform HTTP Server\n");
    printf("Platform: ");
#ifdef PLATFORM_WINDOWS
    printf("Windows");
#elif defined(PLATFORM_LINUX)
    printf("Linux");
#elif defined(PLATFORM_MACOS)
    printf("macOS");
#else
    printf("Unknown");
#endif
    printf(" (%s)\n", ARCH_NAME);

    // Initialize networking
    if (net_init() != 0) {
        fprintf(stderr, "Failed to initialize networking\n");
        return 1;
    }

    // Setup signal handling
    signal_setup();

    // Create server socket
    socket_t server = socket(AF_INET, SOCK_STREAM, IPPROTO_TCP);
    if (server == INVALID_SOCKET_FD) {
        fprintf(stderr, "socket() failed: %s\n",
                net_strerror(SOCKET_ERROR_CODE));
        net_cleanup();
        return 1;
    }

    // Set socket options
    net_set_reuseaddr(server, 1);

    // Bind to port
    struct sockaddr_in addr;
    memset(&addr, 0, sizeof(addr));
    addr.sin_family = AF_INET;
    addr.sin_addr.s_addr = INADDR_ANY;
    addr.sin_port = htons(PORT);

    if (bind(server, (struct sockaddr*)&addr, sizeof(addr))
        == SOCKET_ERROR) {
        fprintf(stderr, "bind() failed: %s\n",
                net_strerror(SOCKET_ERROR_CODE));
        net_close(server);
        net_cleanup();
        return 1;
    }

    // Listen
    if (listen(server, 5) == SOCKET_ERROR) {
        fprintf(stderr, "listen() failed: %s\n",
                net_strerror(SOCKET_ERROR_CODE));
        net_close(server);
        net_cleanup();
        return 1;
    }

    printf("Server listening on port %d\n", PORT);
    printf("Press Ctrl+C to stop\n\n");

    // Accept loop
    while (!signal_check()) {
        // Set short timeout for accept so we can check signals
        net_set_blocking(server, 0);

        socket_t client = accept(server, NULL, NULL);

        if (client == INVALID_SOCKET_FD) {
            // Would block - check for signal and continue
            portable_sleep_ms(100);
            continue;
        }

        printf("Client connected\n");
        handle_request(client);
        net_close(client);
        printf("Client disconnected\n");
    }

    printf("\nShutting down...\n");
    net_close(server);
    net_cleanup();
    return 0;
}
\end{lstlisting}

\subsection{Building the Example}

\textbf{On Linux/macOS:}
\begin{lstlisting}[language=bash]
gcc -o server server.c -DPLATFORM_LINUX
./server
\end{lstlisting}

\textbf{On Windows (MinGW):}
\begin{lstlisting}[language=bash]
gcc -o server.exe server.c -lws2_32 -DPLATFORM_WINDOWS
server.exe
\end{lstlisting}

\textbf{On Windows (MSVC):}
\begin{lstlisting}[language=bash]
cl server.c /DPLATFORM_WINDOWS ws2_32.lib
server.exe
\end{lstlisting}

\subsection{What This Example Demonstrates}

\begin{enumerate}
    \item \textbf{Platform Detection}: Uses macros from the platform.h header
    \item \textbf{Network Abstraction}: Uses the sockets.h wrapper for portability
    \item \textbf{Signal Handling}: Graceful shutdown on Ctrl+C (both platforms)
    \item \textbf{Conditional Compilation}: Different code paths for Windows vs Unix
    \item \textbf{Proper Cleanup}: WSACleanup on Windows, nothing needed on Unix
    \item \textbf{Error Handling}: Platform-appropriate error messages
    \item \textbf{Non-blocking I/O}: Timeout handling for signal checking
\end{enumerate}

This is the pattern used by real projects: abstract the differences, provide uniform interfaces, handle errors properly, test on all platforms.

\section{Best Practices Summary}

\subsection{The Golden Rules}

\begin{enumerate}
    \item \textbf{Isolate Platform Code}: Create thin abstraction layers (like our sockets.h)
    \item \textbf{Test Everywhere}: Compilation success doesn't mean it works
    \item \textbf{Use Feature Detection}: Not just platform detection
    \item \textbf{Handle Errors Properly}: Error codes differ across platforms
    \item \textbf{Mind the Encodings}: UTF-8 internally, convert at boundaries
    \item \textbf{Respect Line Endings}: Use binary mode for exact control
    \item \textbf{Abstract System Calls}: Never use raw Windows/POSIX APIs directly in business logic
    \item \textbf{Document Platform Assumptions}: Be explicit about requirements
\end{enumerate}

\subsection{Common Pitfalls (and How to Avoid Them)}

\begin{warningbox}
\textbf{Watch Out For:}
\begin{itemize}
    \item \textbf{Assuming \texttt{int} is 32-bit}: Use \texttt{int32\_t} from \texttt{<stdint.h>}

    \item \textbf{Ignoring endianness}: Use \texttt{htons()}/\texttt{ntohs()} for network protocols

    \item \textbf{Mixing I/O functions}: Don't use \texttt{read()}/\texttt{write()} on sockets on Windows. Always use \texttt{send()}/\texttt{recv()} for sockets.

    \item \textbf{Forgetting \texttt{WSAStartup()}}: On Windows, EVERY socket program needs this. Create a wrapper that calls it automatically.

    \item \textbf{Using \texttt{fork()}}: Doesn't exist on Windows. Use \texttt{CreateProcess()} or better yet, use threads or a process abstraction library.

    \item \textbf{Case sensitivity}: Write tests that verify behavior on case-sensitive file systems even if you develop on case-insensitive ones.

    \item \textbf{Not handling EINTR}: On Unix, system calls can be interrupted by signals. Check for \texttt{errno == EINTR} and retry.

    \item \textbf{MAX\_PATH limitations}: Windows paths limited to 260 chars by default. Use the \texttt{\textbackslash\textbackslash?\textbackslash} prefix for longer paths.

    \item \textbf{UTF-16 vs UTF-8}: Keep UTF-8 internally. Convert to UTF-16 only when calling Windows W functions.

    \item \textbf{Line endings in binary mode}: Always use \texttt{"rb"}/\texttt{"wb"} for binary files. Text mode converts line endings unpredictably.
\end{itemize}
\end{warningbox}

\subsection{Testing Strategy}

You cannot trust cross-platform code without testing it. Here's a practical strategy:

\begin{enumerate}
    \item \textbf{Test on actual platforms}: Virtual machines or CI/CD services (GitHub Actions, AppVeyor)

    \item \textbf{Test corner cases}:
    \begin{itemize}
        \item Files with Unicode characters in names
        \item Paths longer than 260 characters
        \item Network errors and timeouts
        \item Large files (>2GB) to catch 32-bit integer issues
    \end{itemize}

    \item \textbf{Test with different compilers}: GCC, Clang, MSVC all have quirks

    \item \textbf{Test in different locales}: Set LANG and see if your program breaks

    \item \textbf{Use static analysis}: Tools like cppcheck catch platform-specific bugs
\end{enumerate}

\section{Conclusion}

Writing truly portable C code is challenging but achievable. You've now learned the complete landscape of platform differences and how to handle them professionally.

\subsection{Key Takeaways}

The fundamental insights from this chapter:

\begin{itemize}
    \item \textbf{Windows is fundamentally different}: Not just Unix with a GUI. It has different process models, different networking initialization, different string encoding, and different system APIs. Accept this and abstract it.

    \item \textbf{Four Windows environments}: MSYS2 $\neq$ Cygwin $\neq$ MinGW $\neq$ MSVC. Each behaves differently. Your \texttt{\#ifdef} checks must account for all of them.

    \item \textbf{Networking requires abstraction}: Winsock and BSD sockets look similar but are incompatible. Always wrap them in a uniform API like our sockets.h example.

    \item \textbf{Character encoding is critical}: UTF-16 on Windows APIs, UTF-8 everywhere else. Keep UTF-8 internally and convert at Windows API boundaries.

    \item \textbf{No fork() on Windows}: Process creation is completely different. Use threads or process abstractions instead of relying on Unix-specific \texttt{fork()}.

    \item \textbf{Line endings affect binary data}: CRLF vs LF matters even for binary protocols. Always use binary mode (\texttt{"rb"}/\texttt{"wb"}) for precise control.

    \item \textbf{File system quirks are everywhere}: Path separators, case sensitivity, length limits, reserved names---all differ. Use path manipulation functions, don't parse paths manually.

    \item \textbf{Testing is non-negotiable}: Code that compiles on Linux might not work on Windows, and vice versa. Test on actual platforms, not just in theory.
\end{itemize}

\subsection{The Three-Layer Architecture}

Successful cross-platform projects follow this pattern:

\begin{enumerate}
    \item \textbf{Platform Detection Layer}: Headers that define what platform you're on (\texttt{platform.h})

    \item \textbf{Abstraction Layer}: Wrappers that provide uniform APIs (\texttt{sockets.h}, \texttt{path.h}, \texttt{signals.h})

    \item \textbf{Application Layer}: Your actual code, which uses the abstractions and rarely needs \texttt{\#ifdef}
\end{enumerate}

This architecture is used by SQLite, cURL, FFmpeg, libuv, and virtually every successful cross-platform C project. It works.

\subsection{Your Learning Path}

Don't try to learn everything at once. Here's a practical progression:

\begin{enumerate}
    \item \textbf{Start simple}: Write code for one platform first. Get it working.

    \item \textbf{Add platform detection}: Include the platform.h header. Understand what macros are defined.

    \item \textbf{Port gradually}: Pick one feature (like networking) and make it cross-platform. Test it.

    \item \textbf{Build abstractions}: Create wrappers for system-specific APIs. Make them look uniform.

    \item \textbf{Test continuously}: Don't wait until "the end" to test on other platforms. Test early and often.

    \item \textbf{Study real code}: Read how cURL handles networking, how SQLite handles file I/O, how Git handles processes. Learn from battle-tested code.
\end{enumerate}

\subsection{When Things Go Wrong}

They will. Here's how to debug cross-platform issues:

\begin{enumerate}
    \item \textbf{Isolate the platform}: Does it fail on all platforms or just one? This tells you if it's platform-specific or a general bug.

    \item \textbf{Check initialization}: On Windows, did you call \texttt{WSAStartup()}? Did you open files in binary mode?

    \item \textbf{Verify types}: Socket types, error codes, file descriptors---all differ. Make sure you're using the right types.

    \item \textbf{Print everything}: Use \texttt{printf} debugging liberally. Print error codes, return values, buffer contents.

    \item \textbf{Read the actual error}: Use \texttt{net\_strerror()} or similar to get human-readable errors. Don't guess.

    \item \textbf{Consult documentation}: Microsoft's docs for Windows APIs, POSIX specs for Unix. Know what the APIs actually guarantee.
\end{enumerate}

\subsection{The Reality Check}

Cross-platform C development teaches you humility. You'll discover that:

\begin{itemize}
    \item Code that works perfectly on Linux will mysteriously fail on Windows
    \item Tests passing in CI don't guarantee real-world compatibility
    \item ``It works on my machine'' becomes your most-used phrase
    \item Platform-specific bugs appear only in production, never during testing
    \item File paths that work everywhere will break on that one customer's system
\end{itemize}

\subsection{War Stories: Real Platform Gotchas}

\textbf{The Case-Sensitive Filename Mystery:}
A developer wrote \texttt{\#include "Config.h"} but the actual file was \texttt{config.h}. Worked fine on Windows and macOS (case-insensitive), failed spectacularly on Linux CI servers.

\textbf{The Socket That Wasn't:}
Forgot \texttt{WSAStartup()} on Windows. Program worked perfectly in Wine on Linux (which doesn't require it), failed on actual Windows machines. Months of confusion ensued.

\textbf{The Line Ending Horror:}
Binary protocol accidentally used text mode file operations. Windows CRLF translation corrupted every packet. Debugger showed correct data before \texttt{fwrite()}, garbage after. The answer? \texttt{O\_BINARY}.

\textbf{The MAX\_PATH Surprise:}
Deep directory nesting in tests worked everywhere until a Windows user hit the 260-character path limit. Solution required Windows-specific long path prefix \texttt{\\textbackslash\textbackslash?\textbackslash}.

\subsection{Final Wisdom}

Platform-specific code isn't a failure---it's a reality. The goal isn't to avoid it entirely, but to:

\begin{enumerate}
    \item \textbf{Isolate it}: Keep platform code in clearly marked sections
    \item \textbf{Abstract it}: Provide uniform interfaces above platform differences
    \item \textbf{Test it}: Run on actual platforms, not just cross-compilers
    \item \textbf{Document it}: Explain why platform-specific code exists
    \item \textbf{Maintain it}: Platform APIs evolve; your abstractions must too
\end{enumerate}

Remember: Perfect portability is a myth. Good portability is achievable. Great portability means your platform-specific code is so well-organized that adding new platforms is straightforward.

The best cross-platform code isn't code that magically works everywhere---it's code where platform differences are explicit, manageable, and testable. You've now seen how to write it.

Now go forth and conquer all the platforms!

\begin{tipbox}
\textbf{Pro Tip:} Study how successful projects do it. Look at:
\begin{itemize}
    \item \textbf{cURL}: Networking abstraction done right
    \item \textbf{SDL}: Graphics, input, and audio across all platforms
    \item \textbf{SQLite}: Single-file database that runs everywhere
    \item \textbf{libuv}: Cross-platform async I/O (powers Node.js)
\end{itemize}
\end{tipbox}

\chapter{Advanced Patterns: The Deep Magic}

\section{The Power of X-Macros Revisited}

X-Macros are one of C's most powerful meta-programming tools. Let's explore advanced uses:

\begin{lstlisting}
// Define a complete subsystem with one list
#define COMMANDS \
    X(quit,   "q",  "Exit program",        cmd_quit) \
    X(help,   "h",  "Show help",           cmd_help) \
    X(save,   "s",  "Save current state",  cmd_save) \
    X(load,   "l",  "Load saved state",    cmd_load) \
    X(list,   "ls", "List items",          cmd_list) \
    X(add,    "a",  "Add new item",        cmd_add)

// Generate enum
#define X(name, short_cmd, desc, func) CMD_##name,
typedef enum {
    COMMANDS
    CMD_COUNT
} Command;
#undef X

// Generate function prototypes
#define X(name, short_cmd, desc, func) \
    void func(const char* args);
COMMANDS
#undef X

// Generate dispatch table
#define X(name, short_cmd, desc, func) \
    {#name, short_cmd, desc, func},
typedef struct {
    const char* name;
    const char* short_name;
    const char* description;
    void (*handler)(const char*);
} CommandEntry;

CommandEntry command_table[] = {
    COMMANDS
};
#undef X

// Generate help text
void print_help(void) {
    printf("Available commands:\n");
#define X(name, short_cmd, desc, func) \
    printf("  %-10s (%-3s) - %s\n", #name, short_cmd, desc);
    COMMANDS
#undef X
}

// Generate command name lookup
const char* command_name(Command cmd) {
#define X(name, short_cmd, desc, func) #name,
    static const char* names[] = { COMMANDS };
#undef X
    return names[cmd];
}
\end{lstlisting}

\section{Coroutines in C}

\noindent\rule{\textwidth}{0.4pt}

Coroutines provide cooperative multitasking without the overhead of threads or the complexity of callbacks. They allow functions to suspend execution and resume later, maintaining their local state between invocations. While C lacks native coroutine support, several clever techniques simulate this behavior.

\begin{notebox}
\textbf{What You'll Learn:} This section explores stackless coroutine implementations in C, from Simon Tatham's elegant macro-based approach to explicit state machines. You'll see practical applications in protocol parsing, generators, and async I/O.
\end{notebox}

\subsection{Understanding Coroutines}

Before diving into implementation, we must understand what makes coroutines fundamentally different from ordinary functions. A normal function has a simple lifecycle: it begins execution, runs to completion, and returns. All local variables are destroyed when the function exits. Each call starts fresh with no memory of previous invocations.

Coroutines break this model entirely. They introduce the concept of \textit{suspendable execution}---a function that can pause in the middle, return control to its caller, and later resume exactly where it left off. This seemingly simple change has profound implications for how we structure code.

Coroutines differ from regular functions in key ways:

\begin{description}[style=nextline,leftmargin=0pt]
    \item[\textbf{Suspendable}] Can pause execution and return control to caller. Unlike a normal \texttt{return}, which destroys the function's context, a coroutine yield preserves everything. The instruction pointer, local variables, and execution state remain alive but dormant.

    \item[\textbf{Resumable}] Can continue from where they left off. When called again, the coroutine doesn't start from the beginning. Instead, it resumes immediately after the last yield point, as if it never stopped.

    \item[\textbf{State Preservation}] Maintain local variables across invocations. This is the key challenge in C. Languages with native coroutine support handle this automatically, but in C, we must explicitly preserve state between calls using static variables or context structures.

    \item[\textbf{Cooperative}] Explicitly yield control (unlike preemptive threads). The coroutine decides when to suspend. This eliminates race conditions and the need for locks, but requires careful design to avoid one coroutine monopolizing CPU time.
\end{description}

The power of coroutines becomes apparent when dealing with complex state machines, parsers, or any algorithm that naturally involves multiple stages. Instead of writing explicit state tracking code with switch statements and state variables, the coroutine's execution flow itself represents the state. This makes code more readable and maintainable.

\subsection{Simon Tatham's Coroutine Macros}

\begin{tipbox}
\textbf{The Elegant Solution:} Simon Tatham's coroutine macros represent one of the most clever uses of C preprocessor magic. By combining Duff's Device with \texttt{\_\_LINE\_\_}, they create automatic state machines with minimal boilerplate.
\end{tipbox}

\vspace{0.3cm}

The most elegant stackless coroutine implementation uses Duff's Device and the \texttt{\_\_LINE\_\_} macro. This technique, devised by Simon Tatham, is a masterpiece of macro engineering. It exploits an obscure interaction between C's switch statement and the preprocessor to create automatic state machines.

\textbf{The Fundamental Insight:} C allows case labels anywhere within a switch statement, even nested inside other constructs like loops. Combined with the \texttt{\_\_LINE\_\_} macro (which expands to the current source line number), we can create unique state identifiers automatically. Each yield point gets a different line number, providing a natural way to track where execution should resume.

\begin{lstlisting}
// Coroutine macros using line numbers for state
#define crBegin static int state=0; switch(state) { case 0:
#define crReturn(x) do { state=__LINE__; return x; \
                         case __LINE__:; } while(0)
#define crFinish }

// Simple example: Generate Fibonacci numbers
int fibonacci(void) {
    static int a = 0, b = 1;

    crBegin;

    while(1) {
        crReturn(a);
        int temp = a;
        a = b;
        b = temp + b;
    }

    crFinish;
    return 0;
}

// Usage
for(int i = 0; i < 10; i++) {
    printf("%d ", fibonacci());  // 0 1 1 2 3 5 8 13 21 34
}
\end{lstlisting}

\vspace{0.3cm}
\noindent\textbf{Understanding the Parser:}

This parser demonstrates several important coroutine patterns. First, notice how the control flow reads naturally from top to bottom, just like you'd describe the protocol in English: ``read the header until you find a blank line, extract the content length, allocate a buffer, read the body, then process the request.''

The \texttt{crReturn} calls represent points where we need more data. In a traditional blocking implementation, these would be blocking reads. In a callback-based implementation, each would be a separate function. Here, they're simple yield points---the function pauses, returns control to the caller (who presumably will provide more data), and resumes when called again.

The static variables preserve all state: where we are in the header, how much body we've read, what the content length is. This is essential because each call to the parser might provide only one byte of data. The coroutine accumulates this data incrementally, maintaining perfect knowledge of its progress through the protocol.

Error handling becomes more natural too. Instead of propagating error codes through multiple callback functions, we can simply return an ERROR state and reset. The sequential flow makes it easy to see the happy path and the error conditions.

\begin{warningbox}
\textbf{Memory Management Caveat:} Notice that we allocate \texttt{body} with malloc and must remember to free it. In a more robust implementation, you'd want cleanup logic that runs even if the parser is abandoned mid-stream. This is one area where stackless coroutines show their limitations---you can't rely on automatic cleanup like you would with scope-based resource management.
\end{warningbox}

\vspace{0.3cm}
\noindent\textbf{\large How It Works: A Deep Dive}

\vspace{0.2cm}
\noindent Let's dissect this mechanism in detail, because understanding it requires thinking about C preprocessing and control flow simultaneously.

\begin{enumerate}[leftmargin=*]
    \item \textbf{First call:} When the function first executes, the static variable \texttt{state} is initialized to 0. The \texttt{crBegin} macro expands to declare this variable and open a switch statement with \texttt{case 0:}. Since \texttt{state} is 0, execution begins at this case label and proceeds normally.

    \item \textbf{Yielding:} When \texttt{crReturn} executes, the preprocessor replaces \texttt{\_\_LINE\_\_} with the current source line number. This number is stored in \texttt{state}. The macro then returns from the function with the specified value. Crucially, because \texttt{state} is static, it persists after the function returns.

    \item \textbf{Next call:} On the subsequent call, \texttt{state} still holds the line number from the previous yield. The switch statement now jumps directly to the case label with that line number. Because \texttt{crReturn} places a case label immediately after the return statement, execution resumes right where it left off.

    \item \textbf{Static variables:} All state (like \texttt{a} and \texttt{b} in the Fibonacci example) must be static. This is the price of stackless coroutines---we cannot rely on the normal function call stack. Instead, we explicitly persist everything we need between invocations.
\end{enumerate}

\begin{warningbox}
\textbf{Stackless Trade-off:} This technique is called ``stackless'' because it doesn't manipulate the actual call stack. You gain simplicity and portability, but lose automatic variable preservation. Every piece of state must be explicitly declared as static.
\end{warningbox}

\vspace{0.2cm}
\noindent\textit{The genius of this approach is that the state machine is implicit.} You write code that looks like normal sequential logic, and the macros transform it into a state machine at compile time. The alternative---hand-coding the state machine---is error-prone and obscures the algorithm's logic.

\vspace{0.5cm}
\subsection{Protocol State Machine Example}

\noindent\rule{\textwidth}{0.4pt}
\vspace{0.2cm}

Coroutines excel at implementing complex protocols without callback hell. Traditional callback-based approaches force you to split your logic across multiple functions, each handling one stage of the protocol. State must be passed around in context structures, and the overall flow becomes hard to follow.

\textbf{The Advantage:} With coroutines, the entire protocol implementation lives in one function, written as straightforward sequential code. This is a game-changer for protocol implementations, parsers, and state machines. Let's examine a realistic protocol parser that demonstrates these advantages:

\begin{lstlisting}
typedef enum { WAITING, READING_HEADER, READING_BODY,
               PROCESSING, COMPLETE, ERROR } State;

// HTTP-like protocol parser as coroutine
State http_parser(char* input, int len) {
    static int state = 0;
    static char header[256];
    static int header_pos = 0;
    static int content_length = 0;
    static int body_pos = 0;
    static char* body = NULL;

    crBegin;

    // Read header until blank line
    header_pos = 0;
    while(1) {
        crReturn(READING_HEADER);

        if(input[0] == '\n' && header_pos > 0 &&
           header[header_pos-1] == '\n') {
            header[header_pos] = '\0';
            break;
        }

        if(header_pos < sizeof(header)-1) {
            header[header_pos++] = input[0];
        }
    }

    // Extract content length
    content_length = parse_content_length(header);
    if(content_length <= 0) {
        crReturn(ERROR);
    }

    // Allocate and read body
    body = malloc(content_length);
    body_pos = 0;

    while(body_pos < content_length) {
        crReturn(READING_BODY);
        body[body_pos++] = input[0];
    }

    // Process complete request
    process_request(header, body, content_length);
    free(body);

    crReturn(COMPLETE);

    // Reset for next request
    state = 0;

    crFinish;
    return ERROR;
}
\end{lstlisting}

\vspace{0.3cm}
\noindent\textbf{Analyzing the Implementation:}

This example illustrates the explicit approach's flexibility. The \texttt{CoroContext} structure contains all state: the current position in the state machine (\texttt{state}), loop counters (\texttt{i}, \texttt{j}), accumulated results (\texttt{total}), and buffers.

The state machine has clear stages: initialization (state 0), input collection (state 1), processing (state 2), and output (state 3). Each state does a small amount of work and returns, allowing the caller to interleave multiple coroutines or respond to other events.

Notice the fall-through behavior between some states (using comments to indicate this). State 0 initializes and immediately falls into state 1. This is deliberate---initialization completes instantly, so we don't need to yield. State 3, after outputting results, resets to state 0 for the next cycle.

The processing stage (state 2) demonstrates ``yielding in a loop.'' It processes one character per call, yielding between each. This allows the coroutine to make incremental progress without blocking. In a real application, this might represent a computationally expensive operation that we want to spread over multiple frames or time slices.

The return values (\texttt{CORO\_YIELDED} vs \texttt{CORO\_DONE}) inform the caller about the coroutine's status. This is more explicit than Simon Tatham's approach, where the return value typically carries application data. Here, we separate status from data, making the protocol cleaner.

Multiple instances work naturally: just allocate multiple \texttt{CoroContext} structures. Each maintains independent state. This is perfect for scenarios like handling multiple network connections, where each connection needs its own parser coroutine.

\vspace{0.5cm}
\subsection{Explicit State Structure Approach}

\noindent\rule{\textwidth}{0.4pt}
\vspace{0.2cm}

For more complex scenarios, explicit state management provides better control. While Simon Tatham's macros are elegant for simple cases, they have limitations: all state must be static (preventing multiple coroutine instances), and the macro magic can be hard to debug.

\begin{tipbox}
\textbf{When to Go Explicit:} Use explicit state structures when you need multiple coroutine instances, better debuggability, or fine-grained control over memory management. The trade-off is more boilerplate for transparency and flexibility.
\end{tipbox}

\vspace{0.2cm}
\noindent An alternative approach uses explicit state structures. This is more verbose but offers significant advantages: you can have multiple coroutine instances, the state is visible and debuggable, and you have complete control over memory management and initialization.

This approach effectively hand-codes what the macros generate automatically. You explicitly number your states and write the switch statement yourself.

\begin{lstlisting}
typedef struct {
    int state;
    // Coroutine-specific state
    int i, j;
    int total;
    char buffer[256];
    size_t buffer_pos;
} CoroContext;

typedef enum { CORO_RUNNING, CORO_YIELDED, CORO_DONE } CoroStatus;

// Initialize coroutine
void coro_init(CoroContext* ctx) {
    memset(ctx, 0, sizeof(*ctx));
}

// Multi-stage data processor
CoroStatus data_processor(CoroContext* ctx, char input) {
    switch(ctx->state) {
        case 0:  // Initialization
            ctx->total = 0;
            ctx->buffer_pos = 0;
            ctx->state = 1;
            // Fall through

        case 1:  // Collect input until newline
            if(input == '\n') {
                ctx->buffer[ctx->buffer_pos] = '\0';
                ctx->state = 2;
                ctx->i = 0;
                return CORO_YIELDED;
            }

            if(ctx->buffer_pos < sizeof(ctx->buffer) - 1) {
                ctx->buffer[ctx->buffer_pos++] = input;
            }
            return CORO_YIELDED;

        case 2:  // Process buffer (simulate slow operation)
            // Process one character at a time, yielding between
            while(ctx->i < ctx->buffer_pos) {
                ctx->total += ctx->buffer[ctx->i];
                ctx->i++;
                return CORO_YIELDED;  // Yield after each char
            }
            ctx->state = 3;
            // Fall through

        case 3:  // Output result
            printf("Processed: %s (sum=%d)\n",
                   ctx->buffer, ctx->total);
            ctx->state = 0;  // Reset
            return CORO_DONE;
    }

    return CORO_DONE;
}

// Usage: Process input incrementally
CoroContext ctx;
coro_init(&ctx);

const char* inputs = "Hello\nWorld\n";
for(size_t i = 0; i < strlen(inputs); i++) {
    CoroStatus status = data_processor(&ctx, inputs[i]);
    if(status == CORO_DONE) {
        coro_init(&ctx);  // Start new processing cycle
    }
}
\end{lstlisting}

The prime generator showcases a more sophisticated example. It maintains a growing list of discovered primes, using them to test future candidates. This is a form of the Sieve of Eratosthenes, but implemented as a generator rather than a batch algorithm.

Each call to \texttt{prime\_next} does just enough work to find one prime. The state persists between calls: the current candidate number, all previously discovered primes, and where we are in the testing process. This allows the caller to request primes one at a time, stopping whenever they have enough.

The optimization inside the divisibility check is worth noting. We only test divisors up to the square root of the candidate (checked by \texttt{primes[i] * primes[i] > candidate}). This dramatically reduces the number of divisions needed, especially for large primes.

Memory management is explicit here. The generator allocates and reallocates its internal prime list as needed, using a doubling strategy for amortized O(1) insertions. The caller must call \texttt{prime\_free} when done. This is manual but gives complete control over allocations.

The key advantage over generating all primes upfront is flexibility. If you need the first million primes, a generator produces them incrementally, allowing processing to overlap with generation. If you only need primes until you find one meeting some condition, you can stop early without wasting computation. The generator's state is suspended, ready to continue if needed.

This pattern extends to many scenarios: walking tree structures, generating permutations, producing infinite sequences, or reading large files line-by-line. The coroutine maintains complex traversal state while presenting a simple ``give me the next item'' interface.

\vspace{0.5cm}
\subsection{Generator Pattern}

\noindent\rule{\textwidth}{0.4pt}
\vspace{0.2cm}

Coroutines naturally implement generators---functions that produce a sequence of values over time rather than all at once. Languages like Python and JavaScript have native generator syntax, but C requires manual implementation. Coroutines provide an elegant way to achieve similar behavior.

\begin{notebox}
\textbf{Key Insight:} A generator is just a coroutine that yields values. Instead of yielding to wait for input (like parsers), a generator yields to provide output. Each call produces the next value in the sequence.
\end{notebox}

\vspace{0.2cm}
\noindent This pattern is incredibly useful for iteration, lazy evaluation, and working with sequences too large to fit in memory. Instead of generating an entire array upfront (which might be millions of elements), a generator produces values on demand.

\begin{lstlisting}
// Range generator with step
typedef struct {
    int current;
    int end;
    int step;
    int state;
} RangeGenerator;

void range_init(RangeGenerator* gen, int start, int end, int step) {
    gen->current = start;
    gen->end = end;
    gen->step = step;
    gen->state = 0;
}

int range_next(RangeGenerator* gen, int* value) {
    switch(gen->state) {
        case 0:
            if(gen->current >= gen->end) {
                return 0;  // Done
            }
            *value = gen->current;
            gen->current += gen->step;
            return 1;  // Has value
    }
    return 0;
}

// Primes generator using Sieve approach
typedef struct {
    int state;
    int candidate;
    int* primes;
    size_t prime_count;
    size_t prime_capacity;
} PrimeGenerator;

void prime_init(PrimeGenerator* gen) {
    gen->state = 0;
    gen->candidate = 2;
    gen->prime_count = 0;
    gen->prime_capacity = 16;
    gen->primes = malloc(gen->prime_capacity * sizeof(int));
}

int prime_next(PrimeGenerator* gen, int* value) {
    switch(gen->state) {
        case 0:  // First prime
            *value = 2;
            gen->primes[gen->prime_count++] = 2;
            gen->candidate = 3;
            gen->state = 1;
            return 1;

        case 1:  // Find next prime
            while(1) {
                int is_prime = 1;

                // Check divisibility by known primes
                for(size_t i = 0; i < gen->prime_count; i++) {
                    if(gen->candidate % gen->primes[i] == 0) {
                        is_prime = 0;
                        break;
                    }
                    // Optimization: only check up to sqrt
                    if(gen->primes[i] * gen->primes[i] > gen->candidate) {
                        break;
                    }
                }

                if(is_prime) {
                    *value = gen->candidate;

                    // Store prime for future checks
                    if(gen->prime_count >= gen->prime_capacity) {
                        gen->prime_capacity *= 2;
                        gen->primes = realloc(gen->primes,
                                            gen->prime_capacity * sizeof(int));
                    }
                    gen->primes[gen->prime_count++] = gen->candidate;

                    gen->candidate += 2;  // Skip even numbers
                    return 1;
                }

                gen->candidate += 2;
            }
    }
    return 0;
}

void prime_free(PrimeGenerator* gen) {
    free(gen->primes);
}

// Usage
PrimeGenerator gen;
prime_init(&gen);
int prime;
for(int i = 0; i < 20; i++) {
    if(prime_next(&gen, &prime)) {
        printf("%d ", prime);
    }
}
prime_free(&gen);
\end{lstlisting}

This async file reader demonstrates integrating coroutines with non-blocking I/O. The file is opened with \texttt{O\_NONBLOCK}, meaning \texttt{read()} returns immediately rather than waiting for data. If no data is available, it returns -1 with \texttt{errno} set to \texttt{EAGAIN}.

The state machine handles this explicitly. State 0 opens the file and immediately yields---even though opening might be fast, we yield for consistency. State 1 contains the main reading loop. Each iteration attempts a read. If it would block (\texttt{EAGAIN}), we yield, giving other coroutines a chance to run. The event loop will call us again later, and we'll retry the read.

This is cooperative multitasking in action. Each coroutine does a small amount of work (one read attempt) and yields. No coroutine monopolizes the CPU. The event loop gives each coroutine a chance to make progress.

When we reach EOF (\texttt{bytes\_read == 0}), we transition to the cleanup state. State 2 closes the file, reports statistics, and resets the state machine. Returning \texttt{ASYNC\_COMPLETE} tells the event loop this coroutine is done.

The event loop implementation shows how multiple coroutines run concurrently. It maintains an array of active coroutines and polls each one every iteration. Completed coroutines are removed from the array. This is vastly simpler than traditional select/epoll event loops with callback registration.

The \texttt{usleep(1000)} prevents busy-waiting. In a real implementation, you'd use \texttt{select()}, \texttt{poll()}, or \texttt{epoll()} to sleep until at least one file descriptor has data. The coroutine approach integrates naturally with these mechanisms---each coroutine represents an I/O operation, and the event loop drives them all forward.

This pattern scales to thousands of concurrent operations. Each has its own state machine tracking where it is in the I/O sequence. They all share one thread, eliminating context switch overhead and synchronization complexity. This is how servers like nginx achieve high concurrency---though they often use more sophisticated coroutine libraries rather than hand-rolled state machines.

\vspace{0.5cm}
\subsection{Async I/O Simulation}

\noindent\rule{\textwidth}{0.4pt}
\vspace{0.2cm}

Coroutines can simulate async operations without callbacks, providing a compelling alternative to traditional event-driven I/O. This is one of the most practical applications of coroutines in systems programming.

\begin{tipbox}
\textbf{The Async Advantage:} Coroutines let you write I/O code that looks synchronous but behaves asynchronously. The function appears to block at each I/O operation, but actually yields control. The result is readable, maintainable code with the efficiency of non-blocking I/O.
\end{tipbox}

\vspace{0.2cm}
\noindent\textbf{The Problem with Callbacks:} Traditional async I/O forces you to fragment your logic. Reading a file becomes: start the read, register a callback, return. When data arrives, the callback fires, processes some data, starts another read, registers another callback, and so on. Each callback is a separate function, and you must manually thread state between them.

\begin{lstlisting}
typedef struct {
    int state;
    int fd;
    char buffer[1024];
    size_t bytes_read;
    size_t total_read;
} AsyncReader;

typedef enum { ASYNC_PENDING, ASYNC_COMPLETE, ASYNC_ERROR } AsyncStatus;

AsyncStatus async_read_file(AsyncReader* reader, const char* filename) {
    switch(reader->state) {
        case 0:  // Open file
            reader->fd = open(filename, O_RDONLY | O_NONBLOCK);
            if(reader->fd < 0) {
                return ASYNC_ERROR;
            }
            reader->total_read = 0;
            reader->state = 1;
            return ASYNC_PENDING;

        case 1:  // Read chunk
            reader->bytes_read = read(reader->fd, reader->buffer,
                                     sizeof(reader->buffer));

            if(reader->bytes_read < 0) {
                if(errno == EAGAIN || errno == EWOULDBLOCK) {
                    return ASYNC_PENDING;  // Would block, yield
                }
                close(reader->fd);
                return ASYNC_ERROR;
            }

            if(reader->bytes_read == 0) {
                // EOF
                reader->state = 2;
                return ASYNC_PENDING;
            }

            // Process data
            process_data(reader->buffer, reader->bytes_read);
            reader->total_read += reader->bytes_read;

            // Continue reading
            return ASYNC_PENDING;

        case 2:  // Cleanup
            close(reader->fd);
            printf("Total read: %zu bytes\n", reader->total_read);
            reader->state = 0;
            return ASYNC_COMPLETE;
    }

    return ASYNC_ERROR;
}

// Event loop integration
void event_loop(void) {
    AsyncReader readers[MAX_READERS];
    int active_count = 0;

    // ... initialize readers ...

    while(active_count > 0) {
        for(int i = 0; i < active_count; i++) {
            AsyncStatus status = async_read_file(&readers[i], "file.txt");

            if(status == ASYNC_COMPLETE || status == ASYNC_ERROR) {
                // Remove completed reader
                readers[i] = readers[--active_count];
                i--;
            }
        }

        usleep(1000);  // Sleep briefly to avoid busy-waiting
    }
}
\end{lstlisting}

\vspace{0.5cm}
\subsection{Limitations and Considerations}

\noindent\rule{\textwidth}{0.4pt}
\vspace{0.2cm}

While coroutines are powerful, they come with significant constraints, especially in C's stackless implementations. Understanding these limitations is crucial for deciding when and how to use them.

\vspace{0.3cm}
\noindent\textbf{\large Stackless Limitations}

\vspace{0.2cm}
\noindent The techniques we've explored are ``stackless'' coroutines---they don't manipulate the actual call stack. This simplicity comes at a cost:

\begin{description}[style=nextline,leftmargin=0pt]
    \item[\textbf{Cannot preserve local variables automatically}] Every piece of state must be explicitly stored in static variables or a context structure. This is tedious and error-prone. You can't just declare a local variable and expect it to survive across yields. This is the biggest practical limitation and makes stackless coroutines feel unnatural compared to languages with native support.

    \item[\textbf{Cannot yield from nested function calls}] If your coroutine calls another function, that function cannot yield. Only the top-level coroutine function can yield. This forces you to flatten your code or pass the coroutine context to helper functions so they can modify state without yielding. It prevents the natural decomposition of complex coroutines into smaller helper functions.

    \item[\textbf{All state must be explicit}] There's no hidden magic. Every variable, every counter, every buffer must be declared in your context structure or as a static variable. This makes the state machine visible, which aids debugging, but adds significant boilerplate. You must carefully consider what state needs to persist across yields.

    \item[\textbf{Switch-based approach limits where yields can occur}] The switch statement mechanism requires yield points to be in specific places. You cannot yield inside a function call or from within certain expressions. This sometimes forces awkward code restructuring. Additionally, the technique relies on undefined behavior in some interpretations of the C standard (though it works on all practical compilers).
\end{description}

\vspace{0.4cm}
\noindent\textbf{\large Best Practices}

\begin{warningbox}
\textbf{Critical Guidelines:} Following these practices will save you from subtle bugs, memory leaks, and maintenance nightmares. Coroutines require discipline---cut corners at your peril.
\end{warningbox}

\vspace{0.2cm}

\begin{description}[style=nextline,leftmargin=0pt]
    \item[\textbf{Use for I/O-bound operations, not CPU-bound}] Coroutines shine when waiting for external events---network data, user input, file I/O. They're less useful for pure computation. If your task is CPU-intensive, coroutine overhead provides little benefit over straight-line code. The value is in managing many concurrent I/O operations efficiently.

    \item[\textbf{Keep coroutine state structures small}] Large state structures mean more memory per coroutine instance, limiting how many you can have. This matters when handling thousands of concurrent operations. Consider whether all fields are truly needed or if some can be computed on-demand.

    \item[\textbf{Document yield points clearly}] Comment each yield point explaining why you're yielding and what you expect when resumed. This helps future maintainers understand the control flow. The non-linear execution is coroutines' greatest strength and their greatest source of confusion.

    \item[\textbf{Consider thread safety}] If multiple threads might call the same coroutine, you need synchronization. Static variables in Simon Tatham's approach are particularly problematic here---they're implicitly shared. Context structure approaches are safer because each thread can have its own contexts, but you still need care if contexts are shared.

    \item[\textbf{Free allocated resources in cleanup states}] Memory leaks are easy in coroutines because resources acquired in one state might need cleanup in another. Always include explicit cleanup states, and consider what happens if a coroutine is abandoned mid-execution. In some cases, you might need a separate ``abort'' function that cleans up regardless of current state.

    \item[\textbf{Test state machine transitions thoroughly}] Every state, every transition, every error path needs testing. State machines have many more execution paths than linear code. Use unit tests that exercise all states, and consider property-based testing or state space exploration tools for critical coroutines.
\end{description}

\vspace{0.4cm}
\noindent\textbf{\large When to Use Coroutines}

\vspace{0.2cm}
\noindent Coroutines are the right tool for specific scenarios:

\begin{description}[style=nextline,leftmargin=0pt]
    \item[\textbf{Parsing complex protocols incrementally}] When you must process data as it arrives, byte by byte or packet by packet, coroutines let you write the parser as linear code rather than a tangled web of callbacks. This is perhaps their single best use case.

    \item[\textbf{Implementing generators and iterators}] Any time you need to produce a sequence of values without generating them all upfront, generators (coroutines that yield values) are ideal. This includes tree traversals, combinatorial generation, infinite sequences, and lazy evaluation.

    \item[\textbf{State machines that span multiple function calls}] If your state machine naturally wants to remember where it is across multiple invocations, coroutines are cleaner than manual state tracking. The execution point itself becomes your state.

    \item[\textbf{Cooperative task scheduling}] When you have many tasks that can make incremental progress, coroutines provide lightweight task switching. This is the foundation of many async I/O frameworks and game engines' task systems.

    \item[\textbf{Avoiding callback pyramids in async code}] When traditional callback-based async programming leads to deeply nested, hard-to-follow code, coroutines flatten the control flow. The async/await pattern in modern languages is essentially coroutines with syntactic sugar.
\end{description}

\vspace{0.4cm}
\noindent\textbf{\large Alternatives to Consider}

\begin{notebox}
\textbf{Choose Wisely:} Coroutines aren't always the answer. Each alternative has its place. Match the tool to the problem.
\end{notebox}

\vspace{0.2cm}

\begin{description}[style=nextline,leftmargin=0pt]
    \item[\textbf{Threads}] For true parallelism across CPU cores, threads are necessary. They have higher overhead (each thread needs a stack, context switching is expensive) but actually run simultaneously. Use threads for CPU-bound parallel work, coroutines for I/O-bound concurrent work.

    \item[\textbf{Callbacks}] Sometimes callbacks are simpler. For straightforward event handling with minimal state, callbacks work fine. They become problematic only when you have complex sequences of async operations. Don't reach for coroutines if a few simple callbacks suffice.

    \item[\textbf{ucontext}] POSIX provides \texttt{getcontext}, \texttt{setcontext}, \texttt{makecontext}, and \texttt{swapcontext} for stack-based context switching. These enable true stack-preserving coroutines where local variables work normally. However, this API is deprecated, non-portable (doesn't work on Windows), and tricky to use correctly. It's more powerful than stackless coroutines but fragile.

    \item[\textbf{Assembly}] You can implement coroutines in assembly by manually saving and restoring registers and manipulating stack pointers. This gives maximum control and efficiency but is architecture-specific, hard to maintain, and easy to get wrong. Only consider this for performance-critical systems code where you've exhausted all other options.

    \item[\textbf{Libraries}] Several C libraries implement coroutines: libaco (fast asymmetric coroutines), libcoro (symmetric coroutines), libtask (Plan 9-style task library), and others. These provide more features and better ergonomics than rolling your own. The cost is an external dependency and learning a library-specific API. For production use, libraries are often the right choice.
\end{description}

\vspace{0.4cm}
\begin{tipbox}
\textbf{Final Wisdom:} Coroutines in C require discipline but provide powerful abstraction for complex control flow without the overhead of operating system threads. They represent a middle ground between callbacks (simple but limiting) and threads (powerful but expensive). When you understand their constraints and use them appropriately, they can dramatically simplify systems that manage multiple concurrent operations. The key is recognizing when the benefits of sequential-looking code outweigh the costs of explicit state management.
\end{tipbox}

\vspace{0.5cm}
\noindent\rule{\textwidth}{0.4pt}

\section{Intrusive Data Structures}

Linux kernel-style intrusive containers:

\begin{lstlisting}
// Intrusive list node
typedef struct list_head {
    struct list_head *next, *prev;
} list_head;

// Initialize list
#define LIST_HEAD_INIT(name) { &(name), &(name) }
#define LIST_HEAD(name) \
    list_head name = LIST_HEAD_INIT(name)

static inline void list_init(list_head* list) {
    list->next = list;
    list->prev = list;
}

// Add to list
static inline void list_add(list_head* new_node,
                             list_head* head) {
    head->next->prev = new_node;
    new_node->next = head->next;
    new_node->prev = head;
    head->next = new_node;
}

// Remove from list
static inline void list_del(list_head* entry) {
    entry->next->prev = entry->prev;
    entry->prev->next = entry->next;
}

// Container-of magic
#define container_of(ptr, type, member) \
    ((type*)((char*)(ptr) - offsetof(type, member)))

// Iterate
#define list_for_each(pos, head) \
    for (pos = (head)->next; pos != (head); pos = pos->next)

#define list_entry(ptr, type, member) \
    container_of(ptr, type, member)

// Example usage
typedef struct {
    int id;
    char name[50];
    list_head list;  // Intrusive list node
} Person;

LIST_HEAD(people);

void add_person(int id, const char* name) {
    Person* p = malloc(sizeof(Person));
    p->id = id;
    strncpy(p->name, name, sizeof(p->name));
    list_add(&p->list, &people);
}

void print_all_people(void) {
    list_head* pos;
    list_for_each(pos, &people) {
        Person* p = list_entry(pos, Person, list);
        printf("%d: %s\n", p->id, p->name);
    }
}
\end{lstlisting}

\section{Tagged Unions (Sum Types)}

Type-safe variant types:

\begin{lstlisting}
typedef enum {
    VALUE_INT,
    VALUE_FLOAT,
    VALUE_STRING,
    VALUE_ERROR
} ValueType;

typedef struct {
    ValueType type;
    union {
        int as_int;
        double as_float;
        char* as_string;
        struct {
            int code;
            char message[100];
        } as_error;
    };
} Value;

// Type-safe constructors
Value value_int(int x) {
    return (Value){.type = VALUE_INT, .as_int = x};
}

Value value_float(double x) {
    return (Value){.type = VALUE_FLOAT, .as_float = x};
}

Value value_string(const char* s) {
    return (Value){.type = VALUE_STRING, .as_string = strdup(s)};
}

Value value_error(int code, const char* msg) {
    Value v = {.type = VALUE_ERROR};
    v.as_error.code = code;
    strncpy(v.as_error.message, msg,
            sizeof(v.as_error.message) - 1);
    return v;
}

// Pattern matching with macros
#define MATCH_VALUE(v, INT_CASE, FLOAT_CASE, STR_CASE, ERR_CASE) \
    do { \
        switch((v).type) { \
            case VALUE_INT: { \
                int _val = (v).as_int; \
                INT_CASE(_val); \
            } break; \
            case VALUE_FLOAT: { \
                double _val = (v).as_float; \
                FLOAT_CASE(_val); \
            } break; \
            case VALUE_STRING: { \
                char* _val = (v).as_string; \
                STR_CASE(_val); \
            } break; \
            case VALUE_ERROR: { \
                int _code = (v).as_error.code; \
                char* _msg = (v).as_error.message; \
                ERR_CASE(_code, _msg); \
            } break; \
        } \
    } while(0)

// Usage
Value v = compute_value();
MATCH_VALUE(v,
    INT(x)   -> printf("Int: %d\n", x),
    FLOAT(x) -> printf("Float: %f\n", x),
    STR(x)   -> printf("String: %s\n", x),
    ERR(c,m) -> printf("Error %d: %s\n", c, m)
);
\end{lstlisting}

\section{Generic Programming with Macros}

Type-safe generic containers:

\begin{lstlisting}
// Define a vector for any type
#define DEFINE_VECTOR(T) \
    typedef struct { \
        T* data; \
        size_t size; \
        size_t capacity; \
    } T##_vector; \
    \
    T##_vector* T##_vector_create(void) { \
        T##_vector* v = malloc(sizeof(T##_vector)); \
        v->data = NULL; \
        v->size = 0; \
        v->capacity = 0; \
        return v; \
    } \
    \
    void T##_vector_push(T##_vector* v, T item) { \
        if(v->size >= v->capacity) { \
            v->capacity = v->capacity ? v->capacity * 2 : 8; \
            v->data = realloc(v->data, v->capacity * sizeof(T)); \
        } \
        v->data[v->size++] = item; \
    } \
    \
    T T##_vector_get(T##_vector* v, size_t index) { \
        return v->data[index]; \
    } \
    \
    void T##_vector_destroy(T##_vector* v) { \
        free(v->data); \
        free(v); \
    }

// Generate vectors for different types
DEFINE_VECTOR(int)
DEFINE_VECTOR(float)
DEFINE_VECTOR(double)

// Usage
int_vector* iv = int_vector_create();
int_vector_push(iv, 42);
int_vector_push(iv, 100);
printf("%d\n", int_vector_get(iv, 0));
int_vector_destroy(iv);
\end{lstlisting}

\section{Reflection and Introspection}

Runtime type information in C:

\begin{lstlisting}
// Type descriptor
typedef enum {
    TYPE_INT,
    TYPE_FLOAT,
    TYPE_STRING,
    TYPE_STRUCT
} TypeKind;

typedef struct TypeInfo TypeInfo;

struct TypeInfo {
    TypeKind kind;
    const char* name;
    size_t size;

    // For structs
    struct {
        size_t field_count;
        struct {
            const char* name;
            TypeInfo* type;
            size_t offset;
        } *fields;
    } struct_info;
};

// Example: Describe a struct
typedef struct {
    int x;
    int y;
    char* name;
} Point;

TypeInfo int_type = {TYPE_INT, "int", sizeof(int)};
TypeInfo charptr_type = {TYPE_STRING, "char*", sizeof(char*)};

TypeInfo point_type = {
    .kind = TYPE_STRUCT,
    .name = "Point",
    .size = sizeof(Point),
    .struct_info = {
        .field_count = 3,
        .fields = (struct {const char* name; TypeInfo* type;
                          size_t offset;}[]){
            {"x", &int_type, offsetof(Point, x)},
            {"y", &int_type, offsetof(Point, y)},
            {"name", &charptr_type, offsetof(Point, name)},
        }
    }
};

// Generic serialization using type info
void serialize(void* obj, TypeInfo* type, FILE* f) {
    switch(type->kind) {
        case TYPE_INT:
            fprintf(f, "%d", *(int*)obj);
            break;
        case TYPE_FLOAT:
            fprintf(f, "%f", *(float*)obj);
            break;
        case TYPE_STRING:
            fprintf(f, "\"%s\"", *(char**)obj);
            break;
        case TYPE_STRUCT:
            fprintf(f, "{");
            for(size_t i = 0; i < type->struct_info.field_count; i++) {
                if(i > 0) fprintf(f, ",");
                fprintf(f, "\"%s\":",
                        type->struct_info.fields[i].name);
                void* field_ptr = (char*)obj +
                    type->struct_info.fields[i].offset;
                serialize(field_ptr,
                         type->struct_info.fields[i].type, f);
            }
            fprintf(f, "}");
            break;
    }
}
\end{lstlisting}

\section{Compile-Time Computation}

Push work to compile time:

\begin{lstlisting}
// Compute at compile time with const
static const int fibonacci[] = {
    0, 1, 1, 2, 3, 5, 8, 13, 21, 34, 55, 89, 144
};

// Compile-time assertions
#define COMPILE_TIME_ASSERT(cond) \
    ((void)sizeof(char[1 - 2*!(cond)]))

// Use in code
void check_assumptions(void) {
    COMPILE_TIME_ASSERT(sizeof(int) == 4);
    COMPILE_TIME_ASSERT(sizeof(void*) == 8);
    COMPILE_TIME_ASSERT(sizeof(long) >= sizeof(int));
}

// Constexpr-like behavior (C11)
#define ARRAY_SIZE 100
static const size_t buffer_size = ARRAY_SIZE * sizeof(int);
char buffer[buffer_size]; // Compile-time computation
\end{lstlisting}

\section{Continuation Passing Style}

\begin{lstlisting}
// CPS transforms control flow into data
typedef void (*Continuation)(void* result, void* context);

void async_read_file(const char* path,
                     Continuation cont,
                     void* context) {
    // Start async read
    // When done, call: cont(data, context);
}

void on_file_read(void* result, void* context) {
    char* data = (char*)result;
    printf("File contents: %s\n", data);
    free(data);
}

// Chain continuations
void step1_done(void* result, void* context) {
    printf("Step 1 complete\n");
    async_read_file("file.txt", step2_done, context);
}

void step2_done(void* result, void* context) {
    printf("Step 2 complete\n");
    // Continue chain...
}
\end{lstlisting}

\section{Object System}

Minimal object-oriented system:

\begin{lstlisting}
// Base object with vtable
typedef struct Class Class;
typedef struct Object Object;

struct Class {
    const char* name;
    size_t size;
    void (*constructor)(Object* self);
    void (*destructor)(Object* self);
    char* (*to_string)(Object* self);
};

struct Object {
    Class* class;
    int ref_count;
};

// Object operations
Object* object_new(Class* class) {
    Object* obj = calloc(1, class->size);
    obj->class = class;
    obj->ref_count = 1;
    if(class->constructor) {
        class->constructor(obj);
    }
    return obj;
}

void object_retain(Object* obj) {
    obj->ref_count++;
}

void object_release(Object* obj) {
    if(--obj->ref_count == 0) {
        if(obj->class->destructor) {
            obj->class->destructor(obj);
        }
        free(obj);
    }
}

// Example class
typedef struct {
    Object base;
    int value;
} Integer;

void integer_constructor(Object* self) {
    Integer* i = (Integer*)self;
    i->value = 0;
}

char* integer_to_string(Object* self) {
    Integer* i = (Integer*)self;
    char* str = malloc(20);
    sprintf(str, "%d", i->value);
    return str;
}

Class IntegerClass = {
    .name = "Integer",
    .size = sizeof(Integer),
    .constructor = integer_constructor,
    .destructor = NULL,
    .to_string = integer_to_string
};

// Usage
Integer* num = (Integer*)object_new(&IntegerClass);
num->value = 42;
char* str = num->base.class->to_string((Object*)num);
printf("%s\n", str);
free(str);
object_release((Object*)num);
\end{lstlisting}

\section{Zero-Cost Abstractions}

Macros that compile to optimal code:

\begin{lstlisting}
// Optional type that optimizes away
#define OPTION(T) \
    struct { \
        int has_value; \
        T value; \
    }

#define SOME(x) {1, (x)}
#define NONE {0}

#define IS_SOME(opt) ((opt).has_value)
#define UNWRAP(opt) ((opt).value)

// Usage
OPTION(int) maybe_divide(int a, int b) {
    if(b == 0) {
        OPTION(int) result = NONE;
        return result;
    }
    OPTION(int) result = SOME(a / b);
    return result;
}

OPTION(int) result = maybe_divide(10, 2);
if(IS_SOME(result)) {
    printf("Result: %d\n", UNWRAP(result));
}

// Compiles to simple branch, no overhead!
\end{lstlisting}

\section{Aspect-Oriented Programming}

Cross-cutting concerns with macros:

\begin{lstlisting}
// Automatic logging
#define LOGGED_FUNCTION(ret, name, ...) \
    ret _logged_##name(__VA_ARGS__); \
    ret name(__VA_ARGS__) { \
        printf("[CALL] %s\n", #name); \
        ret result = _logged_##name(__VA_ARGS__); \
        printf("[RETURN] %s\n", #name); \
        return result; \
    } \
    ret _logged_##name(__VA_ARGS__)

// Use it
LOGGED_FUNCTION(int, add, int a, int b) {
    return a + b;
}

// Expands to function with automatic logging
// add(5, 3) prints:
// [CALL] add
// [RETURN] add

// Timing decorator
#define TIMED_FUNCTION(ret, name, ...) \
    ret _timed_##name(__VA_ARGS__); \
    ret name(__VA_ARGS__) { \
        clock_t start = clock(); \
        ret result = _timed_##name(__VA_ARGS__); \
        clock_t end = clock(); \
        printf("%s took %.6f seconds\n", #name, \
               (double)(end - start) / CLOCKS_PER_SEC); \
        return result; \
    } \
    ret _timed_##name(__VA_ARGS__)
\end{lstlisting}

\section{Memory Pools: Custom Allocators}

Sometimes malloc/free are too slow or cause fragmentation. Memory pools to the rescue:

\begin{lstlisting}
// Fixed-size object pool
typedef struct Pool Pool;

struct Pool {
    void* memory;
    size_t object_size;
    size_t capacity;
    size_t count;
    void** free_list;
};

Pool* pool_create(size_t object_size, size_t capacity) {
    Pool* pool = malloc(sizeof(Pool));
    pool->object_size = object_size;
    pool->capacity = capacity;
    pool->count = 0;

    // Allocate memory block
    pool->memory = malloc(object_size * capacity);

    // Build free list
    pool->free_list = malloc(sizeof(void*) * capacity);
    for(size_t i = 0; i < capacity; i++) {
        pool->free_list[i] = (char*)pool->memory +
                              (i * object_size);
    }

    return pool;
}

void* pool_alloc(Pool* pool) {
    if(pool->count >= pool->capacity) {
        return NULL; // Pool exhausted
    }
    return pool->free_list[pool->count++];
}

void pool_free(Pool* pool, void* ptr) {
    if(pool->count == 0) return;
    pool->free_list[--pool->count] = ptr;
}

void pool_destroy(Pool* pool) {
    free(pool->memory);
    free(pool->free_list);
    free(pool);
}

// Usage: Lightning-fast allocation
typedef struct { int x, y, z; } Particle;

Pool* particle_pool = pool_create(sizeof(Particle), 10000);

Particle* p1 = pool_alloc(particle_pool);
Particle* p2 = pool_alloc(particle_pool);
// No malloc overhead!

pool_free(particle_pool, p1);
pool_free(particle_pool, p2);
\end{lstlisting}

\subsection{Arena Allocator: Bulk Deallocation}

\begin{lstlisting}
// Allocate many objects, free all at once
typedef struct {
    char* buffer;
    size_t size;
    size_t used;
} Arena;

Arena* arena_create(size_t size) {
    Arena* arena = malloc(sizeof(Arena));
    arena->buffer = malloc(size);
    arena->size = size;
    arena->used = 0;
    return arena;
}

void* arena_alloc(Arena* arena, size_t size) {
    // Align to 8 bytes
    size = (size + 7) & ~7;

    if(arena->used + size > arena->size) {
        return NULL; // Arena full
    }

    void* ptr = arena->buffer + arena->used;
    arena->used += size;
    return ptr;
}

void arena_reset(Arena* arena) {
    arena->used = 0; // Free everything!
}

void arena_destroy(Arena* arena) {
    free(arena->buffer);
    free(arena);
}

// Perfect for per-request data in servers
Arena* request_arena = arena_create(1024 * 1024); // 1MB

while(handle_request()) {
    // Allocate tons of temporary data
    char* buffer = arena_alloc(request_arena, 4096);
    Node* tree = arena_alloc(request_arena, sizeof(Node));

    // Process request...

    // Free everything instantly!
    arena_reset(request_arena);
}
\end{lstlisting}

\section{Plugin Systems: Dynamic Loading}

Build extensible applications with runtime plugin loading:

\begin{lstlisting}
// Plugin interface
typedef struct {
    const char* name;
    const char* version;
    int (*init)(void);
    void (*shutdown)(void);
    void (*process)(void* data);
} Plugin;

// Plugin loader
#ifdef _WIN32
#include <windows.h>
typedef HMODULE PluginHandle;
#define LOAD_PLUGIN(path) LoadLibrary(path)
#define GET_SYMBOL(handle, name) GetProcAddress(handle, name)
#define CLOSE_PLUGIN(handle) FreeLibrary(handle)
#else
#include <dlfcn.h>
typedef void* PluginHandle;
#define LOAD_PLUGIN(path) dlopen(path, RTLD_LAZY)
#define GET_SYMBOL(handle, name) dlsym(handle, name)
#define CLOSE_PLUGIN(handle) dlclose(handle)
#endif

typedef struct {
    PluginHandle handle;
    Plugin* plugin;
} LoadedPlugin;

LoadedPlugin load_plugin(const char* path) {
    LoadedPlugin loaded = {0};

    loaded.handle = LOAD_PLUGIN(path);
    if(!loaded.handle) {
        fprintf(stderr, "Failed to load plugin: %s\n", path);
        return loaded;
    }

    // Get plugin descriptor
    Plugin* (*get_plugin)(void) = GET_SYMBOL(loaded.handle,
                                              "get_plugin");
    if(!get_plugin) {
        fprintf(stderr, "Plugin missing get_plugin()\n");
        CLOSE_PLUGIN(loaded.handle);
        loaded.handle = NULL;
        return loaded;
    }

    loaded.plugin = get_plugin();

    if(loaded.plugin->init) {
        if(loaded.plugin->init() != 0) {
            fprintf(stderr, "Plugin init failed\n");
            CLOSE_PLUGIN(loaded.handle);
            loaded.handle = NULL;
            return loaded;
        }
    }

    printf("Loaded plugin: %s v%s\n",
           loaded.plugin->name, loaded.plugin->version);

    return loaded;
}

void unload_plugin(LoadedPlugin* loaded) {
    if(loaded->handle) {
        if(loaded->plugin && loaded->plugin->shutdown) {
            loaded->plugin->shutdown();
        }
        CLOSE_PLUGIN(loaded->handle);
        loaded->handle = NULL;
        loaded->plugin = NULL;
    }
}

// Example plugin implementation (in separate .so/.dll)
int my_plugin_init(void) {
    printf("My plugin initializing\n");
    return 0;
}

void my_plugin_shutdown(void) {
    printf("My plugin shutting down\n");
}

void my_plugin_process(void* data) {
    printf("Processing: %s\n", (char*)data);
}

Plugin my_plugin = {
    .name = "MyPlugin",
    .version = "1.0",
    .init = my_plugin_init,
    .shutdown = my_plugin_shutdown,
    .process = my_plugin_process
};

Plugin* get_plugin(void) {
    return &my_plugin;
}
\end{lstlisting}

\section{Domain-Specific Languages (DSLs)}

Create mini-languages for specific tasks:

\begin{lstlisting}
// Simple expression DSL
// Example: "x + y * 2" or "max(a, b + c)"

typedef enum {
    TOKEN_NUMBER,
    TOKEN_IDENT,
    TOKEN_PLUS,
    TOKEN_MINUS,
    TOKEN_STAR,
    TOKEN_SLASH,
    TOKEN_LPAREN,
    TOKEN_RPAREN,
    TOKEN_COMMA,
    TOKEN_EOF
} TokenType;

typedef struct {
    TokenType type;
    union {
        double number;
        char ident[32];
    };
} Token;

// Tokenizer
typedef struct {
    const char* input;
    size_t pos;
    Token current;
} Lexer;

void lexer_init(Lexer* lex, const char* input) {
    lex->input = input;
    lex->pos = 0;
}

void lexer_next(Lexer* lex) {
    // Skip whitespace
    while(isspace(lex->input[lex->pos])) lex->pos++;

    char c = lex->input[lex->pos];

    if(c == '\0') {
        lex->current.type = TOKEN_EOF;
        return;
    }

    if(isdigit(c)) {
        char* end;
        lex->current.type = TOKEN_NUMBER;
        lex->current.number = strtod(lex->input + lex->pos, &end);
        lex->pos = end - lex->input;
        return;
    }

    if(isalpha(c)) {
        lex->current.type = TOKEN_IDENT;
        size_t i = 0;
        while(isalnum(lex->input[lex->pos]) && i < 31) {
            lex->current.ident[i++] = lex->input[lex->pos++];
        }
        lex->current.ident[i] = '\0';
        return;
    }

    // Operators
    lex->pos++;
    switch(c) {
        case '+': lex->current.type = TOKEN_PLUS; break;
        case '-': lex->current.type = TOKEN_MINUS; break;
        case '*': lex->current.type = TOKEN_STAR; break;
        case '/': lex->current.type = TOKEN_SLASH; break;
        case '(': lex->current.type = TOKEN_LPAREN; break;
        case ')': lex->current.type = TOKEN_RPAREN; break;
        case ',': lex->current.type = TOKEN_COMMA; break;
    }
}

// Simple recursive descent parser
typedef struct Expr Expr;

struct Expr {
    enum { EXPR_NUM, EXPR_VAR, EXPR_BINOP, EXPR_CALL } type;
    union {
        double number;
        char var[32];
        struct {
            char op;
            Expr *left, *right;
        } binop;
        struct {
            char func[32];
            Expr** args;
            int arg_count;
        } call;
    };
};

// Parse and evaluate
double eval(Expr* expr, double* vars) {
    switch(expr->type) {
        case EXPR_NUM:
            return expr->number;
        case EXPR_VAR:
            // Look up variable (simplified)
            return vars[expr->var[0] - 'a'];
        case EXPR_BINOP: {
            double left = eval(expr->binop.left, vars);
            double right = eval(expr->binop.right, vars);
            switch(expr->binop.op) {
                case '+': return left + right;
                case '-': return left - right;
                case '*': return left * right;
                case '/': return left / right;
            }
        }
        case EXPR_CALL:
            // Function calls (simplified)
            if(strcmp(expr->call.func, "max") == 0) {
                double a = eval(expr->call.args[0], vars);
                double b = eval(expr->call.args[1], vars);
                return a > b ? a : b;
            }
            break;
    }
    return 0;
}

// Usage
// Parse "x + y * 2" and evaluate with x=5, y=3
// Result: 5 + 3*2 = 11
\end{lstlisting}

\section{Finite State Transducers}

Beyond state machines---transform input to output:

\begin{lstlisting}
// FST: Transform input sequence to output sequence
typedef struct {
    int state;
    int input;
    int output;
    int next_state;
} Transition;

typedef struct {
    Transition* transitions;
    int transition_count;
    int current_state;
} FST;

// Example: Convert "hello" to "HELLO"
Transition uppercase_fst[] = {
    {0, 'h', 'H', 0},
    {0, 'e', 'E', 0},
    {0, 'l', 'L', 0},
    {0, 'o', 'O', 0},
    // ... more transitions
};

int fst_process(FST* fst, int input) {
    for(int i = 0; i < fst->transition_count; i++) {
        Transition* t = &fst->transitions[i];
        if(t->state == fst->current_state &&
           t->input == input) {
            fst->current_state = t->next_state;
            return t->output;
        }
    }
    return -1; // No transition
}

// More complex: Phone number formatter
// Input: "5551234567"
// Output: "(555) 123-4567"
\end{lstlisting}

\section{Visitor Pattern in C}

Object-oriented visitor pattern without classes:

\begin{lstlisting}
// Abstract syntax tree
typedef struct Node Node;

struct Node {
    enum { NODE_NUM, NODE_ADD, NODE_MUL } type;
    union {
        int number;
        struct { Node *left, *right; } binop;
    };
};

// Visitor interface
typedef struct {
    void (*visit_num)(int value, void* context);
    void (*visit_add)(Node* left, Node* right, void* context);
    void (*visit_mul)(Node* left, Node* right, void* context);
} Visitor;

void node_accept(Node* node, Visitor* visitor, void* context) {
    switch(node->type) {
        case NODE_NUM:
            visitor->visit_num(node->number, context);
            break;
        case NODE_ADD:
            node_accept(node->binop.left, visitor, context);
            node_accept(node->binop.right, visitor, context);
            visitor->visit_add(node->binop.left, node->binop.right,
                              context);
            break;
        case NODE_MUL:
            node_accept(node->binop.left, visitor, context);
            node_accept(node->binop.right, visitor, context);
            visitor->visit_mul(node->binop.left, node->binop.right,
                              context);
            break;
    }
}

// Example visitor: Pretty printer
void print_num(int value, void* ctx) {
    printf("%d", value);
}

void print_add(Node* left, Node* right, void* ctx) {
    printf(" + ");
}

void print_mul(Node* left, Node* right, void* ctx) {
    printf(" * ");
}

Visitor printer = {
    .visit_num = print_num,
    .visit_add = print_add,
    .visit_mul = print_mul
};

// Example visitor: Evaluator
void eval_num(int value, void* ctx) {
    int* result = (int*)ctx;
    *result = value;
}

void eval_add(Node* left, Node* right, void* ctx) {
    int left_val, right_val;
    node_accept(left, &evaluator, &left_val);
    node_accept(right, &evaluator, &right_val);
    *(int*)ctx = left_val + right_val;
}
\end{lstlisting}

\section{Summary}

You've now seen the deep magic of C:

\begin{itemize}
    \item \textbf{X-Macros}: Maintainable code generation without external tools
    \item \textbf{Coroutines}: Cooperative multitasking without threads
    \item \textbf{Intrusive Structures}: Linux kernel-style zero-overhead containers
    \item \textbf{Tagged Unions}: Type-safe variant types
    \item \textbf{Generic Programming}: Type-safe generics through macros
    \item \textbf{Reflection}: Runtime type information in C
    \item \textbf{Compile-Time Computation}: Push work to the compiler
    \item \textbf{Continuation Passing}: Transform control flow to data
    \item \textbf{Object Systems}: OOP when you need it
    \item \textbf{Zero-Cost Abstractions}: High-level code, low-level performance
    \item \textbf{Aspect-Oriented}: Cross-cutting concerns through macros
    \item \textbf{Memory Pools}: Custom allocators for performance
    \item \textbf{Plugin Systems}: Runtime extensibility
    \item \textbf{DSLs}: Domain-specific languages embedded in C
    \item \textbf{FSTs}: Finite state transducers for transformations
    \item \textbf{Visitor Pattern}: Object-oriented patterns without objects
\end{itemize}

\subsection{The Art of Advanced C}

These patterns aren't tricks---they're techniques. Each solves real problems:

\begin{itemize}
    \item Use X-Macros when you have parallel data structures
    \item Use intrusive containers when performance matters
    \item Use memory pools for predictable allocation
    \item Use plugins for extensible architectures
    \item Use DSLs when configuration isn't enough
    \item Use visitors when operations vary more than types
\end{itemize}

\subsection{When to Use Advanced Patterns}

\textbf{Always:}
\begin{itemize}
    \item X-Macros for enums with string names
    \item Tagged unions for variant types
    \item Compile-time assertions
\end{itemize}

\textbf{Often:}
\begin{itemize}
    \item Generic programming with macros
    \item Intrusive data structures in performance code
    \item Memory pools in real-time systems
\end{itemize}

\textbf{Sometimes:}
\begin{itemize}
    \item Coroutines for state machines
    \item Reflection for serialization
    \item Plugin systems for extensibility
\end{itemize}

\textbf{Rarely:}
\begin{itemize}
    \item Full object systems (just use C++)
    \item Continuation passing (confusing for most)
    \item DSLs (big maintenance burden)
\end{itemize}

\subsection{Final Thoughts on Advanced Patterns}

C is simple, but not simplistic. It provides just enough to build sophisticated abstractions while staying close to the metal. These patterns show that C can express complex ideas---but should you?

The best C code is:
\begin{enumerate}
    \item \textbf{Obvious}: Someone reading it understands it quickly
    \item \textbf{Efficient}: It doesn't waste resources
    \item \textbf{Maintainable}: Future you can modify it without fear
    \item \textbf{Appropriate}: The complexity matches the problem
\end{enumerate}

Don't use advanced patterns to show off. Use them to solve problems. The cleverest C code isn't the most complex---it's the simplest code that does the job right.

Now you have the full arsenal of C techniques. Use them wisely, use them well, and remember: just because you \textit{can} build a coroutine-based intrusive generic reflection system doesn't mean you \textit{should}.

Master the patterns, but master restraint too. That's the true art of C programming!


\backmatter

% Appendix A: Quick Reference
\chapter{Appendix A: Quick Reference Guide}

\section*{Essential C Idioms at a Glance}

\subsection*{Opaque Pointers}
\begin{lstlisting}
// In header (.h):
typedef struct MyStruct MyStruct;
MyStruct* mystruct_create(void);

// In implementation (.c):
struct MyStruct { int data; };
\end{lstlisting}

\subsection*{Error Handling}
\begin{lstlisting}
// Return -1 on error, 0 on success
int do_something(void) {
    if (error) return -1;
    return 0;
}

// Return NULL on error
void* allocate_something(void) {
    void* ptr = malloc(size);
    if (!ptr) return NULL;
    return ptr;
}
\end{lstlisting}

\subsection*{String Safety}
\begin{lstlisting}
// Safe string copy
strncpy(dest, src, sizeof(dest) - 1);
dest[sizeof(dest) - 1] = '\0';

// Safe string format
snprintf(buf, sizeof(buf), "Value: %d", x);
\end{lstlisting}

\subsection*{Memory Management}
\begin{lstlisting}
// Always check malloc
void* ptr = malloc(size);
if (!ptr) { /* handle error */ }

// Always free and NULL
free(ptr);
ptr = NULL;

// sizeof the variable, not the type
MyStruct* s = malloc(sizeof(*s));
\end{lstlisting}

\subsection*{Struct Initialization}
\begin{lstlisting}
// Zero-initialize
MyStruct s = {0};

// Designated initializers (C99+)
MyStruct s = {.x = 10, .y = 20};
\end{lstlisting}

\subsection*{Common Macros}
\begin{lstlisting}
#define MAX(a, b) ((a) > (b) ? (a) : (b))
#define ARRAY_SIZE(arr) (sizeof(arr) / sizeof((arr)[0]))
#define UNUSED(x) (void)(x)
\end{lstlisting}

\section*{Common Mistakes to Avoid}

\begin{itemize}
    \item \textbf{Using gets()}: Use \texttt{fgets()} instead
    \item \textbf{Using strcpy()}: Use \texttt{strncpy()} or \texttt{strlcpy()}
    \item \textbf{Using sprintf()}: Use \texttt{snprintf()} instead
    \item \textbf{Comparing floats with ==}: Use epsilon comparison
    \item \textbf{Using memcmp on structs}: Padding bytes have undefined values
    \item \textbf{Forgetting to null-terminate strings}: Always add \texttt{'\textbackslash 0'}
    \item \textbf{Dereferencing before checking NULL}: Always check pointers first
    \item \textbf{Returning pointers to local variables}: They're gone after function returns
\end{itemize}

\section*{Must-Know Header Files}

\begin{itemize}
    \item \texttt{<stdio.h>}: File I/O, printf, scanf
    \item \texttt{<stdlib.h>}: malloc, free, exit, atoi
    \item \texttt{<string.h>}: String operations, memcpy, memset
    \item \texttt{<stdint.h>}: Fixed-width integer types (int32\_t, uint64\_t, etc.)
    \item \texttt{<stdbool.h>}: bool, true, false (C99+)
    \item \texttt{<stddef.h>}: size\_t, NULL, offsetof
    \item \texttt{<assert.h>}: Runtime assertions for debugging
    \item \texttt{<errno.h>}: Error numbers for system calls
\end{itemize}

\section*{Compiler Flags You Should Use}

\textbf{GCC/Clang:}
\begin{verbatim}
gcc -Wall -Wextra -Werror -std=c99 -O2 -g program.c
\end{verbatim}

\textbf{MSVC:}
\begin{verbatim}
cl /W4 /WX /std:c11 /O2 program.c
\end{verbatim}

\section*{Debugging Tools}

\begin{itemize}
    \item \textbf{valgrind}: Memory leak detection and profiling
    \item \textbf{gdb}: GNU debugger
    \item \textbf{lldb}: LLVM debugger
    \item \textbf{AddressSanitizer}: Detects memory errors at runtime
    \item \textbf{cppcheck}: Static analysis for C/C++
    \item \textbf{clang-tidy}: Linter and static analyzer
\end{itemize}

% Appendix B: Resources
\chapter{Appendix B: Recommended Resources}

\section*{Essential Books}

\begin{itemize}
    \item \textbf{The C Programming Language} by Kernighan \& Ritchie — The classic. Read it.
    \item \textbf{Expert C Programming} by Peter van der Linden — Deep insights and war stories
    \item \textbf{C Interfaces and Implementations} by David Hanson — Real-world design patterns
    \item \textbf{Modern C} by Jens Gustedt — C11/C17 features and modern practices
    \item \textbf{21st Century C} by Ben Klemens — Contemporary C development
\end{itemize}

\section*{Online Resources}

\begin{itemize}
    \item \textbf{cppreference.com}: Comprehensive C and C++ reference
    \item \textbf{C FAQ}: \url{http://c-faq.com/} — Answers to common questions
    \item \textbf{Stack Overflow}: Tag [c] — Community Q\&A
    \item \textbf{CERT C Coding Standard}: Security-focused guidelines
    \item \textbf{SEI CERT C}: Safe, secure, and reliable C
\end{itemize}

\section*{Code to Study}

\begin{itemize}
    \item \textbf{SQLite}: \url{https://sqlite.org/} — Best-written C code in existence
    \item \textbf{Redis}: \url{https://redis.io/} — Clean, readable systems code
    \item \textbf{Git}: \url{https://git-scm.com/} — Real-world C project structure
    \item \textbf{Nginx}: \url{https://nginx.org/} — High-performance network server
    \item \textbf{Linux Kernel}: \url{https://kernel.org/} — Ultimate C codebase (advanced)
\end{itemize}

\section*{Development Tools}

\begin{itemize}
    \item \textbf{Compilers}: GCC, Clang, MSVC
    \item \textbf{Build Systems}: Make, CMake, Meson
    \item \textbf{Version Control}: Git (obviously)
    \item \textbf{Editors/IDEs}: VS Code, CLion, Vim/Emacs
    \item \textbf{Documentation}: Doxygen, Sphinx
\end{itemize}

\section*{Communities}

\begin{itemize}
    \item \textbf{r/C\_Programming}: Reddit community
    \item \textbf{comp.lang.c}: Usenet newsgroup (still active!)
    \item \textbf{Freenode \#c}: IRC channel
    \item \textbf{C Discord servers}: Real-time chat communities
\end{itemize}

\section*{Academic Papers and Standards}

\begin{itemize}
    \item \textbf{ISO/IEC 9899:2018}: The official C17/C18 standard document
    \item \textbf{ISO/IEC 9899:2011}: The C11 standard (previous version)
    \item \textbf{ISO/IEC 9899:1999}: The C99 standard
    \item \textbf{MISRA C}: Guidelines for the use of C in critical systems
    \item \textbf{SEI CERT C Coding Standard}: Secure coding practices
\end{itemize}

% Bibliography
\chapter{Bibliography and References}

\section*{Foundational Books}

\begin{enumerate}
    \item Kernighan, Brian W., and Dennis M. Ritchie. \textit{The C Programming Language}. 2nd ed. Prentice Hall, 1988.

    \textit{The definitive C book by the language's creators. Every C programmer should read this at least once.}

    \item van der Linden, Peter. \textit{Expert C Programming: Deep C Secrets}. Prentice Hall, 1994.

    \textit{Filled with insights, war stories, and explanations of C's quirks. Entertaining and educational.}

    \item Hanson, David R. \textit{C Interfaces and Implementations: Techniques for Creating Reusable Software}. Addison-Wesley, 1997.

    \textit{Shows how to design professional C libraries with clean interfaces. Essential for API design.}

    \item Gustedt, Jens. \textit{Modern C}. Manning Publications, 2019.

    \textit{Focuses on C11 and C17 features. Shows modern approaches to C programming.}

    \item Klemens, Ben. \textit{21st Century C: C Tips from the New School}. 2nd ed. O'Reilly Media, 2014.

    \textit{Contemporary C development practices, build systems, and tooling.}

    \item Seacord, Robert C. \textit{Effective C: An Introduction to Professional C Programming}. No Starch Press, 2020.

    \textit{Modern best practices and secure coding techniques for C.}

    \item Prinz, Peter, and Tony Crawford. \textit{C in a Nutshell}. 2nd ed. O'Reilly Media, 2015.

    \textit{Comprehensive reference covering C11 standard library and features.}
\end{enumerate}

\section*{Systems Programming and Design}

\begin{enumerate}
    \item Stevens, W. Richard, and Stephen A. Rago. \textit{Advanced Programming in the UNIX Environment}. 3rd ed. Addison-Wesley, 2013.

    \textit{The bible of Unix systems programming. Essential for understanding POSIX APIs.}

    \item Kerrisk, Michael. \textit{The Linux Programming Interface}. No Starch Press, 2010.

    \textit{Comprehensive guide to Linux and UNIX system programming. Over 1500 pages of detailed information.}

    \item Love, Robert. \textit{Linux System Programming}. 2nd ed. O'Reilly Media, 2013.

    \textit{System calls, I/O, process management, and threading on Linux.}

    \item Bryant, Randal E., and David R. O'Hallaron. \textit{Computer Systems: A Programmer's Perspective}. 3rd ed. Pearson, 2015.

    \textit{How C maps to hardware. Essential for understanding performance and optimization.}
\end{enumerate}

\section*{Data Structures and Algorithms}

\begin{enumerate}
    \item Cormen, Thomas H., et al. \textit{Introduction to Algorithms}. 4th ed. MIT Press, 2022.

    \textit{The comprehensive algorithms textbook. Many examples are in pseudocode but applicable to C.}

    \item Sedgewick, Robert. \textit{Algorithms in C}. 3rd ed. Addison-Wesley, 1997-2002. (Parts 1-5)

    \textit{Classic algorithms book with all code in C. Practical implementations.}

    \item Loudon, Kyle. \textit{Mastering Algorithms with C}. O'Reilly Media, 1999.

    \textit{Practical data structures and algorithms specifically in C.}
\end{enumerate}

\section*{Memory Management and Performance}

\begin{enumerate}
    \item Hagar, Jon. \textit{Software Test Attacks to Break Mobile and Embedded Devices}. CRC Press, 2013.

    \textit{Memory management and testing techniques for embedded systems.}

    \item Gerber, Richard, et al. \textit{Software Optimization Cookbook}. 2nd ed. Intel Press, 2006.

    \textit{Performance optimization techniques for C/C++ on Intel architectures.}

    \item Fog, Agner. \textit{Optimizing Software in C++}. 2023. Available online at \url{https://agner.org/optimize/}

    \textit{Deep dive into CPU optimization, though focused on C++, many techniques apply to C.}
\end{enumerate}

\section*{Secure Coding and Safety}

\begin{enumerate}
    \item Seacord, Robert C. \textit{Secure Coding in C and C++}. 2nd ed. Addison-Wesley, 2013.

    \textit{Security vulnerabilities in C and how to prevent them. Essential for production code.}

    \item Wheeler, David A. \textit{Secure Programming HOWTO}. Available at \url{https://dwheeler.com/secure-programs/}

    \textit{Free online guide to writing secure programs in C/C++.}

    \item CERT Division. \textit{SEI CERT C Coding Standard}. Software Engineering Institute, Carnegie Mellon University. Available at \url{https://wiki.sei.cmu.edu/confluence/display/c/}

    \textit{Rules and recommendations for secure C programming.}
\end{enumerate}

\section*{Notable Open Source Codebases}

\textit{(These are not books but essential reading for learning professional C patterns)}

\begin{enumerate}
    \item \textbf{SQLite} — \url{https://sqlite.org/}

    \textit{Arguably the best-written C code in existence. Extensively tested, well-documented, and demonstrates professional practices throughout.}

    \item \textbf{Redis} — \url{https://redis.io/}

    \textit{Clean, readable C code. Excellent use of data structures. The codebase is approachable and well-commented.}

    \item \textbf{Git} — \url{https://git-scm.com/}

    \textit{Real-world project structure, cross-platform support, and practical C idioms. Shows how to organize a large C project.}

    \item \textbf{Nginx} — \url{https://nginx.org/}

    \textit{High-performance network programming. Event-driven architecture and optimization techniques.}

    \item \textbf{cURL} — \url{https://curl.se/}

    \textit{Cross-platform HTTP library. Shows extensive platform abstraction and API design.}

    \item \textbf{Linux Kernel} — \url{https://kernel.org/}

    \textit{The ultimate C codebase. Advanced patterns, intrusive data structures, and systems programming at scale.}

    \item \textbf{FFmpeg} — \url{https://ffmpeg.org/}

    \textit{Multimedia processing. Complex algorithms, performance optimization, and extensive hardware support.}

    \item \textbf{libuv} — \url{https://libuv.org/}

    \textit{Cross-platform asynchronous I/O library. Powers Node.js. Excellent cross-platform abstraction.}
\end{enumerate}

\section*{Online Documentation and Standards}

\begin{enumerate}
    \item \textbf{cppreference.com} — \url{https://en.cppreference.com/}

    \textit{Comprehensive C and C++ reference with examples. Community-maintained and highly accurate.}

    \item \textbf{The C FAQ} — \url{http://c-faq.com/}

    \textit{Frequently Asked Questions about C, compiled by Steve Summit. Answers to common pitfalls.}

    \item \textbf{ISO C Standard} (ISO/IEC 9899)

    \textit{The official language specification. Available for purchase or as drafts online.}

    \item \textbf{POSIX Standard} (IEEE Std 1003.1)

    \textit{Portable Operating System Interface specification. Defines standard Unix APIs.}

    \item \textbf{GNU C Library Documentation} — \url{https://www.gnu.org/software/libc/manual/}

    \textit{Complete reference for glibc, the C standard library on GNU/Linux systems.}
\end{enumerate}

\section*{Tools Documentation}

\begin{enumerate}
    \item \textbf{GCC Manual} — \url{https://gcc.gnu.org/onlinedocs/}

    \textit{Complete documentation for the GNU Compiler Collection.}

    \item \textbf{Clang Documentation} — \url{https://clang.llvm.org/docs/}

    \textit{LLVM C/C++ compiler documentation and usage guides.}

    \item \textbf{Valgrind User Manual} — \url{https://valgrind.org/docs/manual/}

    \textit{Memory debugging and profiling tools documentation.}

    \item \textbf{GDB Manual} — \url{https://sourceware.org/gdb/documentation/}

    \textit{The GNU Debugger comprehensive documentation.}

    \item \textbf{CMake Documentation} — \url{https://cmake.org/documentation/}

    \textit{Cross-platform build system widely used for C projects.}
\end{enumerate}

\section*{Historical and Background Reading}

\begin{enumerate}
    \item Ritchie, Dennis M. ``The Development of the C Language.'' \textit{ACM SIGPLAN Notices} 28.3 (1993): 201-208.

    \textit{Dennis Ritchie's own account of how C was created. Essential historical context.}

    \item Kernighan, Brian W. ``The C Programming Language and Its Impact.'' Various talks and papers.

    \textit{Reflections on C's design philosophy and influence.}

    \item Pike, Rob. ``Notes on Programming in C.'' February 21, 1989.

    \textit{Programming style guidelines from one of Unix's creators. Still relevant today.}

    \item Thompson, Ken. ``Reflections on Trusting Trust.'' \textit{Communications of the ACM} 27.8 (1984): 761-763.

    \textit{Famous Turing Award lecture on security and compilers. Classic paper.}
\end{enumerate}

\section*{Style Guides and Conventions}

\begin{enumerate}
    \item \textbf{Linux Kernel Coding Style} — \url{https://www.kernel.org/doc/html/latest/process/coding-style.html}

    \textit{Linus Torvalds' style guide for kernel code. Opinionated but influential.}

    \item \textbf{GNU Coding Standards} — \url{https://www.gnu.org/prep/standards/}

    \textit{Style and design conventions for GNU project software.}

    \item \textbf{Google C++ Style Guide} (applicable to C in many ways)

    \textit{Modern corporate coding standards with rationale for each rule.}

    \item \textbf{MISRA C Guidelines}

    \textit{Coding standards for safety-critical systems. Restrictive but important for embedded/automotive.}
\end{enumerate}

\section*{Communities and Forums}

\begin{itemize}
    \item \textbf{Stack Overflow} — Tag [c] — \url{https://stackoverflow.com/questions/tagged/c}
    \item \textbf{Reddit r/C\_Programming} — \url{https://reddit.com/r/C_Programming}
    \item \textbf{comp.lang.c} — Usenet newsgroup (via Google Groups or news reader)
    \item \textbf{Freenode \#c} — IRC channel for C programming discussions
    \item \textbf{C Discord servers} — Various real-time chat communities
\end{itemize}

\vspace{2em}

\textit{Note: URLs and availability of resources are current as of 2025. Some resources may move or become unavailable over time. Use search engines to locate current versions if links are broken.}

% Errata
\chapter{Errata and Updates}

\section*{How to Report Errors}

Despite careful review, technical books inevitably contain errors. If you find mistakes in this book—whether technical errors, typos, or unclear explanations—please report them.

\textbf{Reporting Issues:}

\begin{itemize}
    \item \textbf{Repository}: \url{https://codeberg.org/_a/C_Idioms_And_Patterns/issues}
    \item \textbf{Include}: Page number, section title, description of the issue, and suggested correction if possible
\end{itemize}

\section*{Known Errata}

\textit{This section will be updated with confirmed errors and corrections. Check the repository for the most current errata list.}

\subsection*{First Edition (2025)}

\textit{No confirmed errata at time of publication.}

\vspace{1em}

\textbf{Major Corrections:}

\textit{(None yet)}

\vspace{1em}

\textbf{Minor Corrections:}

\textit{(None yet)}

\vspace{1em}

\textbf{Clarifications:}

\textit{(None yet)}

\section*{Online Resources}

\textbf{Book Repository}: \url{https://codeberg.org/_a/C_Idioms_And_Patterns}

All code examples from this book are available in the repository, organized by chapter. The repository includes:
\begin{itemize}
    \item Complete, compilable versions of all examples
    \item Additional examples not in the book
    \item Makefiles for easy compilation
    \item Test cases and validation scripts
    \item Solutions to exercises (if applicable)
\end{itemize}

\section*{Updates and New Editions}

C evolves slowly, but tools, practices, and platforms change. Major updates or corrections will be documented here and on the book's website.

For significant changes that warrant a new edition:
\begin{itemize}
    \item Major new C standards (C2x when finalized)
    \item Significant platform changes (new OS versions, compiler updates)
    \item Discovery of major technical errors
    \item Reader feedback indicating sections need rewriting
\end{itemize}

\textbf{Current Edition}: First Edition, 2025

\section*{Contributing}

If you'd like to contribute:
\begin{itemize}
    \item \textbf{Corrections}: Submit via repository issues
    \item \textbf{Suggestions}: Ideas for additional content or improvements
    \item \textbf{Examples}: Better code examples or real-world use cases
    \item \textbf{Translations}: Contact via repository if interested in translating
\end{itemize}

Thank you to everyone who helps improve this book!

% Glossary
\chapter{Glossary of C Terms}

\section*{A}

\textbf{Address Sanitizer (ASan)}: A runtime memory error detector that finds bugs like buffer overflows, use-after-free, and memory leaks.

\textbf{Alignment}: The requirement that data be stored at memory addresses divisible by certain values (e.g., 4-byte ints at addresses divisible by 4).

\textbf{Arena Allocator}: A memory allocation strategy where you allocate a large block once, then sub-allocate from it quickly, and free everything at once.

\textbf{Arithmetic Overflow}: When an arithmetic operation produces a result larger than the maximum value the type can hold.

\section*{B}

\textbf{Binary Search Tree (BST)}: A tree data structure where each node has at most two children, with left < parent < right.

\textbf{Bit Field}: A struct member that occupies only a specified number of bits, used to pack multiple small values.

\textbf{Bloom Filter}: A probabilistic data structure that tests set membership with possible false positives but no false negatives.

\textbf{Buffer Overflow}: Writing data beyond the allocated bounds of a buffer, causing undefined behavior and security vulnerabilities.

\section*{C}

\textbf{Callback}: A function pointer passed to another function, which the other function calls at appropriate times.

\textbf{Circular Buffer}: A fixed-size buffer that wraps around when full, useful for queues and streaming data.

\textbf{Compound Literal}: A C99 feature that creates unnamed objects: \texttt{(struct Point)\{10, 20\}}.

\textbf{Const Correctness}: Using \texttt{const} appropriately to indicate which data should not be modified.

\section*{D}

\textbf{Dangling Pointer}: A pointer that points to memory that has been freed or is otherwise invalid.

\textbf{Designated Initializer}: C99 syntax for initializing specific struct members: \texttt{\{.x = 10, .y = 20\}}.

\textbf{Double Free}: Calling \texttt{free()} twice on the same pointer, causing undefined behavior.

\textbf{Dynamic Array}: A resizable array that grows automatically, implemented with \texttt{realloc()}.

\section*{E}

\textbf{Endianness}: The byte order in which multi-byte numbers are stored (big-endian vs little-endian).

\textbf{errno}: A global variable set by system calls to indicate the type of error that occurred.

\section*{F}

\textbf{Flexible Array Member}: A C99 feature where the last struct member can be an array of unspecified size.

\textbf{Forward Declaration}: Declaring something before defining it, allowing references before the full definition.

\textbf{Function Pointer}: A pointer that points to a function, enabling callbacks and runtime polymorphism.

\section*{G}

\textbf{Generic Programming}: Writing code that works with multiple types, typically using macros or void pointers in C.

\section*{H}

\textbf{Hash Function}: A function that converts data into a fixed-size hash value, used by hash tables.

\textbf{Hash Table}: A data structure providing O(1) average-case lookup by mapping keys to array indices via hashing.

\textbf{Header Guard}: Preprocessor directives preventing multiple inclusion: \texttt{\#ifndef}, \texttt{\#define}, \texttt{\#endif}.

\section*{I}

\textbf{Intrusive Data Structure}: A data structure where link pointers are embedded directly in the objects being linked.

\textbf{Implementation-Defined Behavior}: Behavior that varies by compiler but must be documented (e.g., size of \texttt{int}).

\section*{L}

\textbf{Linkage}: The visibility of identifiers across translation units (external, internal, or no linkage).

\section*{M}

\textbf{Memory Leak}: Allocated memory that is never freed, causing memory consumption to grow over time.

\textbf{Memory Pool}: Pre-allocated memory divided into fixed-size chunks for fast allocation.

\section*{N}

\textbf{Null Pointer}: A pointer with value NULL (or 0), indicating it doesn't point to valid memory.

\section*{O}

\textbf{Opaque Pointer}: A pointer to an incomplete type, hiding implementation details from users.

\textbf{Overflow}: See Arithmetic Overflow or Buffer Overflow.

\section*{P}

\textbf{Padding}: Extra bytes inserted by the compiler between struct members to satisfy alignment requirements.

\textbf{Pointer Arithmetic}: Performing addition/subtraction on pointers, moving by multiples of the pointed-to type's size.

\section*{S}

\textbf{Segmentation Fault}: A runtime error occurring when a program tries to access memory it doesn't have permission to access.

\textbf{Stack}: The memory region where local variables and function call information are stored.

\textbf{Static}: Keyword with multiple meanings: file-scope linkage, persistent local variables, or compile-time allocation.

\section*{U}

\textbf{Undefined Behavior (UB)}: Code whose behavior is not defined by the C standard, potentially causing crashes or unpredictable results.

\textbf{Union}: A data type where all members share the same memory location.

\section*{V}

\textbf{Volatile}: Keyword indicating a variable may be modified by external sources, preventing compiler optimizations.

\textbf{VTable}: Virtual function table, a pattern for implementing runtime polymorphism in C using function pointers.

\end{document}
